\section{GraphBLAS Math}
\label{sec:math}

Consider a graph represented as an $n$-by-$n$ adjacency matrix $\matrix{A}$,
where $A_{ij}$ is the weight of the edge from vertex $i$ to vertex $j$,
and a second $k$-by-$n$ matrix $\matrix{B}$ representing a subset (of size k) of the vertices
in the graph, such that $B_{ji}$ is $1$ if the $j$th element of the subset is vertex $i$
(and all other elements of $\matrix{B}$ are 0).  The traditional matrix
product $\matrix{B} \times \matrix{A}$ over real arithmetic of these two matrices returns 
the cost based on the edge weights of reaching the set of vertices
adjacent to the vertices in $\matrix{B}$.  This fundamental operation can be
used to construct a wide range of graph algorithms.

We extend the range of graph operations by keeping the basic
pattern of a matrix-matrix multiplication, but varying
the operators and the interpretation of the values in the matrices (the \emph{domain}).
By carefully choosing operators and the domain, we control the
relation between matrix operations familiar in linear algebra and graph operations, thereby enabling
composable graph algorithms.

This generalized matrix multiplication is performed on an algebraic semiring.   
A semiring is an algebraic
structure over a domain $D$ with two binary operators $\oplus$ and $\otimes$.
The \emph{addition} operator, $\oplus$, is a commutative monoid with an identity 
element $\mathbf{0}$ (not necessarily the number 0)
while the \emph{multiplication} operator, $\otimes$, is a commutative monoid with an 
identity element $\mathbf{1}$ (not necessarily the number 1).  The additive 
identity is also an annihilator for the multiplication 
operator ($\otimes$), and multiplication distributes over addition.  The most 
common semirings used in the graph algorithms community are 
shown in Table~\ref{Tab:semirings}.
  
\begin{table}[h]
\hrule
\begin{center}
\caption{Common semirings used with graph algorithms.}
\label{Tab:semirings}
\begin{tabular}{lcclcc}
{\sf Semiring}           & \multicolumn{2}{c}{operators} & domain                             & $\mathbf{0}$  & $\mathbf{1}$ \\
                         & $\oplus$     & $\otimes$      & \\    
\hline
Standard arithmetic      & $ + $        & $ \times $     & $\mathbb{R}$                       & $0$           & $1$ \\
max-plus algebras        & $ \max $     & $ + $          & $\{-\infty \cup  \mathbb{R} \}$    & $-\infty$     & $0$ \\
min-max algebras         & $ \min $     & $ \max $       & $\infty \cup  \mathbb{R}_{\geq 0}$ & $\infty$      & $0$ \\
Galois fields (\eg, GF2) & $ \xor $     & $ \mbox{and}$  & $\{0, 1\}$                         & $0$           & $1$ \\
Power set algebras       & $ \cup $     & $ \cap $       & $\mathcal{P}(\mathbb{Z})$          & $\varnothing$ & $U$ \\
\end{tabular}
\end{center}
\hrule
\end{table}

It is often convenient to change the semiring applied
to a  matrix.  This means we must represent the matrix and the semiring 
separately,
and the two come together only when an operation is performed.
Mathematically, the ability to change semirings 
when moving from one GraphBLAS operation to the next impacts the meaning of 
the \emph{implied}  \emph{zero} in a sparse representation of the matrix.
This element in real arithmetic is the number zero ($0$), which is the 
identity of the addition operator and the annihilator of
the multiplication operator.   As the semiring changes, this 
implied zero changes to the identity of 
the addition operator and the annihilator of the multiplication 
operator for the new semiring.   Nothing changes in the
stored matrix, but the implied values within the sparse matrix change
with respect to a particular operation.  

This feature has significant impact on the definitions of GraphBLAS operations.   
Consider matrix multiplication over the domain $\mathbb{S}$ 
with semiring operators 
$\oplus$ and $\otimes$:
 $$
   \mathbf{C} = \mathbf{A} {\oplus}.{\otimes} \mathbf{B} = \mathbf{A} \mathbf{B}.
$$
Using index notation familiar in linear algebra
  $$
   {\bf C}(i,j) = \bigoplus_{k=1}^l {\bf A}(i,k) \otimes {\bf B}(k,j)
  $$
for matrices with dimensions
$$
  {\bf A} : \mathbb{S}^{m \times l} ~~~~~
  {\bf B} : \mathbb{S}^{l \times n} ~~~~~
  {\bf C} : \mathbb{S}^{m \times n}
$$
The summation notation only works, however, if we redefine the implied zero of the 
sparse matrices as we change the semiring (to the corresponding additive identity).   
Depending on the domains associated with the
matrix elements and the operations, this can lead to awkward definitions of the
operations involving the implied zeros.  A cleaner approach based on set notation
avoids this problem.  For example, we can define the previous matrix multiplication
as  
$$
\mathbf{C}(i,j)
= \bigoplus_{k \in \mathbf{ind}(\matrix{{A}}(i,:)) \cap
\mathbf{ind}(\matrix{{B}}(:,j))} (\matrix{{A}}(i,k)
\otimes \matrix{{B}}(k,j)),
$$ 
where $\mathbf{ind}(\matrix{{A}}(i,:))$ is the set of the column indices of the 
elements that are stored in row $i$ of matrix $\mathbf{A}$, and
$\mathbf{ind}(\matrix{{B}}(:,j))$ is the set of the row indices of the 
elements that are stored in column $j$ of matrix $\matrix{B}$.

In other words, the binary operation $\otimes$ is applied to the elements in the intersection of the 
two sets $\mathbf{ind}(\matrix{{A}}(i,:))$ and $\mathbf{ind}(\matrix{{B}}(:,j))$, 
and the results of this operation are accumulated using the $\oplus$ operator.
These notations are equivalent. By defining  pairwise operations over
set intersections, however, we avoid needing to define how the semiring's additive identity 
interacts with the matrix's implied zeros.

In addition to matrix multiplication, the GraphBLAS math specification defines
a range of additional operations over matrices and vectors.  These are summarized in Table~\ref{Tab:GraphBLASOps}.
%In addition to matrices, these operations also manipulate \emph{vectors}, which are
%one-dimensional structures (as opposed to the two-dimensions matrices).


\begin{table}[h]
\hrule
\begin{center}
\caption{A mathematical overview of the fundamental GraphBLAS operations supported
in this specification. $\matrix{A}$, $\matrix{B}$, and $\matrix{C}$ are GraphBLAS matrices; 
$\vector{u}$, $\vector{v}$, and $\vector{w}$ are GraphBLAS vectors; $i$ and $j$ are single indices;
$\mathbf{i}$ and $\mathbf{j}$ are arrays of indices;
$\oplus$ and $\otimes$ are arbitrary element-wise operators; the element-wise $\odot$
operator is used for the optional accumulation with the output GraphBLAS object where 
$x~\odot\hspace{-0.11cm}= y$ implies $x = x \odot y$; and $F_u()$ is a unary function.
Although not shown here, the input 
matrices $\matrix{A}$ and $\matrix{B}$ may be selected for transposition prior to 
the operation, and masks can be used to control which values are written to the output GraphBLAS object.}
\label{Tab:GraphBLASOps}
\newcommand{\odotequals}{~\odot\hspace{-0.09cm}=\hspace{-0.2cm}}
\begin{tabular}{l|rl}
{\sf Operation name} & \multicolumn{2}{c}{Mathematical description}  \\
\hline
{\sf mxm}          & $\matrix{C}$    $\odotequals$ & $\matrix{A} \oplus.\otimes \matrix{B}$ \\
{\sf mxv}          & $\vector{w}$    $\odotequals$ & $\matrix{A} \oplus.\otimes \vector{v}$ \\
{\sf vxm}          & $\vector{w}^T$  $\odotequals$ & $\vector{v}^T \oplus.\otimes \matrix{A}$  \\
{\sf eWiseMult}    & $\matrix{C}$    $\odotequals$ & $\matrix{A} \otimes \matrix{B}$ \\
                   & $\vector{w}$    $\odotequals$ & $\vector{u} \otimes \vector{v}$ \\
{\sf eWiseAdd}     & $\matrix{C}$    $\odotequals$ & $\matrix{A} \oplus \matrix{B}$ \\
                   & $\vector{w}$    $\odotequals$ & $\vector{u} \oplus \vector{v}$ \\
{\sf reduce} (row) & $\vector{w}$    $\odotequals$ & $\bigoplus_j\matrix{A}(:,j)$  \\
{\sf apply}        & $\matrix{C}$    $\odotequals$ & $F_u(\matrix{A})$ \\
                   & $\vector{w}$    $\odotequals$ & $F_u(\vector{u})$ \\
{\sf transpose}    & $\matrix{C}$    $\odotequals$ & $\matrix{A}^T$ \\
{\sf extract}      & $\matrix{C}$    $\odotequals$ & $\matrix{A}(\vector{i},\vector{j})$ \\
                   & $\vector{w}$    $\odotequals$ & $\vector{u}(\vector{i})$ \\
{\sf assign}       & $\matrix{C}(\vector{i},\vector{j})$  $\odotequals$ &  $\matrix{A}$ \\
                   & $\vector{w}(\vector{i})$  $\odotequals$ & $\vector{u}$ 
\end{tabular}

\end{center}
\hrule
\end{table}

