\section{Introduction}
\label{sec:intro}

Graphs are a fundamental abstraction in computer science.  They represent
relationships between a finite collection of objects.   The objects, or
\emph{vertices}, in a graph are connected by \emph{edges}.  This leads
to the common view of a graph as two sets: a set of vertices and a set
of edges.

Graphs can also be represented as matrices.   For example, the
\emph{adjacency matrix} for a graph is constructed by labeling rows and
columns of the matrix by the vertices of the graph.  The elements of
the matrix denote the edges in the graph with matrix element $A_{ij}$
defining the edge from vertex $i$ to vertex $j$.  Most
vertices in a large graph, such as those arising in social networks,
are not connected to each other so the matrices used for graphs tend to
be very sparse.

Many graph algorithms have been defined in the ``language of linear
algebra''~\cite{kepner2011graph}.  Mapping sparse linear algebra algorithms 
onto modern architectures is well understood; a fact several 
groups have used to build high
performance graph libraries based on sparse linear algebra~\cite{combblas,
gadepally2015graphulo, gpi2016, sundaram2015graphmat}.  A group
of graph algorithm researchers formed the GraphBLAS
forum~\cite{graphblas_web} to standardize the low level building
blocks used in these graph algorithms.  The Forum completed the
mathematical formalizations of GraphBLAS~\cite{mathgraphblas16} and
has undertaken the task to define the
binding of the C-programming language onto the mathematical definition of
the GraphBLAS -- the so-called GraphBLAS C specification.   This work
has been undertaken by a subcommittee of the GraphBLAS forum, comprised of
the authors of this paper.  The challenge in formulating
the GraphBLAS C specification was to balance conflicting 
objectives: (i) simplicity and ease of use,
(ii) enabling high-performance implementations, and (iii) adherence to
the underlying mathematics.

This paper summarizes the GraphBLAS C specification and the
motivation behind our  decisions.  We begin by summarizing the 
mathematical ideas behind the GraphBLAS and how those ideas
influenced our notation.  We then explain data structures,
algebraic objects, and objects that control the semantics of the functions
in the GraphBLAS C specification.   We then define the
core operations in the GraphBLAS C  specification and the signatures
for a subset of the functions  within the API.  We then present a betweenness centrality 
algorithm that uses the GraphBLAS C  specification.  We close with results
and concluding remarks.
