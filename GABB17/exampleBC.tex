\section{Example: Betweenness Centrality}

\label{sec:example}
Betweenness centrality (BC) is a popular metric to assess the centrality of vertices in a graph. It is based on shortest paths where the BC score
of a vertex $v$ is the normalized ratio of the number of shortest paths between any pair of vertices that go through $v$ to the total number of shortest paths in the graph. 
Equation~\ref{eqn:bc} formally defines BC where $\sigma_{st}$ denotes the number of shortest paths from $s$ to $t$, and $\sigma_{st}(v)$ is the number of such paths 
passing through vertex $v$.

\begin{equation}
	BC(v) = \sum_{s \neq v \neq t \in V} \frac{\sigma_{st}(v)}{\sigma_{st}}
\label{eqn:bc}
\end{equation}

BC is efficiently computed using Brandes' algorithm~\cite{brandes2001faster}, which runs in $O(mn)$ time or unweighted graphs and avoids the expensive explicit all-pairs shortest paths computation.
For each starting vertex $s$, Brandes' algorithm calls a subroutine that computes the BC contributions of to every other vertex. This subroutine, BC\_update, is
shown in Figure~\ref{Fig:BClisting} using the C GraphBLAS API. 

BC\_update performs two sweeps over the graph. The forward sweep performs a breadth-first search where it also keeps track of the number
of independent shortest paths that reach each vertex. This is performed by the do-while loop in lines 38-44. 
The backward sweep rolls back and tallies the BC contributions to each vertex. This is performed by the for loop in lines 61-71. 


\begin{figure*}[h]
\caption{C function using GraphBLAS primitives that computes the BC-metric updates ${\it delta}$, given Boolean $n \times n$ adjacency matrix $A$, a set of source vertices $s$, 
and the number of source vertices (i.e. the length of s) ${\it nsver}$.
}
\label{Fig:BClisting}
{\scriptsize
\lstinputlisting[language=C,numbers=left]{GabbBC3M.c}
}
\end{figure*}

In the forward sweep, variable {\tt p} keeps track of the number of independent shortest paths that reach each vertex, variable {\tt q} contains the current frontier, and {\tt sigma} stores
the final breadth-first search tree. 
GrB\_vxm call in line 40 forms the next frontier in one step by both expanding the current frontier (i.e. discovering the 1-hop neighbors of the set of vertices in the current frontier) and pruning the vertices that have 
already been discovered before. This is achieved by setting the descriptor object {\tt desc} to use the structural complement of the mask and by passing the 
variable {\tt p} as the mask parameter. The implicit cast of {\tt p} to Boolean allows GrB\_vxm to interpret {\tt p} as the set of previously discovered vertices.
The GrB\_reduce call in line 42 calculates the number of newly discovered vertices in that iteration and stores it in variable {\tt sum}. The forward sweep is over when {\tt sum} is zero.