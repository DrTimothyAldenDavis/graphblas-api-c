% This is ch1-introduction.tex of the GraphBLAS specifigation
% This is an included file. See the master file for more information.
%


\chapter{Introduction}
\index{introduction}
\label{chap:introduction}
Introduce the specification and motivate the effort.

If we have a good URL where people can learn more, put it here

\code{http://www.graphblas.org}


\section{Scope}
\label{sec:Scope}
Outline the scope of the project

\begin{quote}
We want our spec to be as simple as possible but no simplerI.
\end{quote}


\section{Glossary}
\label{sec:Glossary}
\index{glossary}

\glossaryterm{graph}
\glossarydefstart
A set of  \emph{nodes} and associated \emph{edges}.
\glossarydefend


\section{Models}
\label{sec:Models}
\index{models}
this is where we define any preliminary background formalisms.  We may not need this section.

Here are some macros we want to be sure to use.    When we want to refer to actual software or API concept,
we shousl use this font: \code{dgemm}.




\section{graphBLAS  Compliance}
\label{sec:GraphBLAS Compliance}
\index{GraphBLAS compliance}
\index{compliance}
An implementation of the Graph BLAS is complient if the following.   

\section{Normative References}
\index{normative references}
\label{sec:normative references}
\begin{itemize}
\item ISO/IEC 9899:1990, \textsl{Information Technology - Programming Languages - C}.

\item ISO/IEC 9899:1999, \textsl{Information Technology - Programming Languages - C}. 


\item ISO/IEC 14882:1998, \textsl{Information Technology - Programming Languages - C++}. 

\end{itemize}

\pagebreak
\section{Organization of this document}
\label{sec:Organization of this document}
The remainder of this document is structured as follows: 

\begin{itemize}
\item Chapter \ref{chap:Functions} ``Functions''

\item Appendix \ref{chap:Appendix A} ``Examples''

\item Appendix \ref{chap:Features History} ``Features History''
\end{itemize}


% This is the end of ch1-introduction.tex of the OpenMP specification.

