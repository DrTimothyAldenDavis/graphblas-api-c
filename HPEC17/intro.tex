%%%%%%%%%%%%%%%%%%%%%%% file intro.tex %%%%%%%%%%%%%%%%%%%%%%%%%
%
% This file contains the introduction section of the paper
%
%%%%%%%%%%%%%%%%%%%%%%%%%%%%%%%%%%%%%%%%%%%%%%%%%%%%%%%%%%%%%%%%%%%
\section{Introduction}
\label{sec:intro}
The GraphBLAS aims to standardize the mathematical concepts~\cite{mathgraphblas16} and the application programming interface (API)~\cite{graphblas_capi_17} for performing graph computations in the language of linear algebra~\cite{kepner2011graph}. The GraphBLAS C API version 1.0 has been provisionally released in May 2017~\cite{graphblasapi}. Its finalization is pending on the existence of at least two feature compliant implementations. 

During the C API development process, the members of the GraphBLAS community at large have provided valuable feedback and additional features they would like to see. Some of these ideas have made it to the version 1.0 release, some have been deemed out of scope, and some have been postponed for future releases despite being relevant for GraphBLAS. We also opted to not include many interesting ideas of our own into the first official release due to time constraints and the lack of broad discussion on their implications. This paper presents a high-level overview of those ideas that we are considering for future versions of the GraphBLAS C API.

The topics we describe in this paper are likely to become part of a future release of the C API Spec. We describe standard definitions that aim to ease the burden on the programmer in Section~\ref{sec:stddefs}, an inspector-executor interface that allow preprocessing to be performed on the inputs (matrices, vectors, and masks) of the operation with the aim of accelerating the overall operation in Section~\ref{sec:inspecExec}, especially if it is performed multiple times with the same inputs.
We describe more fine grained synchronization constructs in Section~\ref{sec:dag}, extensions to descriptor objects in Section~\ref{sec:masks}, user-defined types in Section~\ref{sec:usrTypes}, and additional mathematical operations Section~\ref{sec:kronProd}. Finally, we include a discussion about object type promotion that allows automatic generation of higher dimensional objects from objects of relatively lower dimensions in Section~\ref{sec:promotion}. We also include some experimental results on the practical benefits of the inspector-executor approach in Section~\ref{sec:inspecExec}. 

