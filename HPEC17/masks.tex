%%%%%%%%%%%%%%%%%%%%%%%%%%%%%%%%%%%%%%%%%%%%%%%%%%%%%%%%%%%%%%%%%%%
\section{API-transparent Extensions}
\label{Sec:Extensions}

A defining characteristic of the GraphBLAS C API is its use of
\emph{descriptors} to modify the behavior of a method.  In the current
definition of the API, descriptors can specify preprocessing steps for
input data (typically, whether an input matrix should be transposed or
not before the operation) as well as controlling how the result value
is written to the output object (complement or not the mask, preserve
or replace the elements of the output object that are masked).

This descriptor-based approach allows one to extend the functionality of
methods without changing their interface, since all main GraphBLAS methods
already include an optional descriptor as the last argument. We envision
some new uses for descriptors in future versions of the GraphBLAS C API.

First, we plan to provide a {\sf GrB\_STRUCTURE\_ONLY} modifier for masks.
In the present specification of GraphBLAS, masks need to be matrices
or vectors of a predefined type. And only those elements of a mask
that evaluate to a Boolean {\sf true} value are considered active.
Specifying {\sf GrB\_STRUCTURE\_ONLY} in the {\sf GrB\_MASK} field of
a descriptor would direct GraphBLAS to consider all elements present
in the mask as active, irrespective of their value.  As a side effect,
vectors and matrices of any type could be used as masks.

Another item under consideration is the use of assertions about the
properties of the output object. Those assertions could be used to
implement optimizations for certain operations. For example, specifying
{\sf GrB\_FIXED\_STRUCTURE} in the {\sf GrB\_OUTP} field of a descriptor
would assert that the output object will not change structure (pattern
of nonzero elements) during this operation. Therefore, computation
can happen in place with new values simply replacing old values. This
would accomplish similar optimization results as achieved by a split
analyze/compute execution. Moreover, a spectrum of standard properties,
from more restrictive to more permissive can be specified. For example,
{\sf GrB\_SYMMETRIC} can assert that an output is a symmetric matrix,
whereas {\sf GrB\_FIXED\_NVALS} can assert that the number of elements
in the output is constant (even though the structure may change).

In the case of assertions, it is important to specify how \emph{hard}
or \emph{soft} those assertions are. That is, do they cause a run-time
error if violated or are they just ``hints'' that the implementation
can use to improve performance but should be able to recover from,
maybe with big performance penalties, if they prove to be wrong.
