%%%%%%%%%%%%%%%%%%%%%%% file GrBDescrip.tex %%%%%%%%%%%%%%%%%%%%%%%%%
%
% This file contains the description of GraphBLAS C API
%
%%%%%%%%%%%%%%%%%%%%%%%%%%%%%%%%%%%%%%%%%%%%%%%%%%%%%%%%%%%%%%%%%%%

\renewcommand{\vector}[1]{{\bf #1}}
\renewcommand{\matrix}[1]{{\bf #1}}

%%%%%
\section{The GraphBLAS C API}
\label{sec:GrBCspec}

This section gives a brief overview of important elements 
of Version 1.0 of the GraphBLAS C API specification in order give some context 
for future extensions proposed in this paper.  More details of the API can be 
found in \cite{graphblas_capi_17}, and the full specification can be downloaded 
from the GraphBLAS Forum website \cite{graphblas_web}.  Like the original BLAS, 
this library performs operations on matrices and vectors.  However there are 
significant differences to specifically support graph operations as described 
in \cite{mathgraphblas16}. 

\begin{figure}[b]
\hrule
\footnotesize
\[
\matrix{C}\langle \matrix{M} \rangle ~ = ~ [\matrix{C} ~\odot] ~ \matrix{A}^{[T]} \oplus.\otimes \matrix{B}^{[T]}
\]
\begin{center}(a) The mathematical definition.\end{center}

\begin{verbatim}

  GrB_Info GrB_mxm(GrB_Matrix               C,
                   const GrB_Matrix         M,  
                   const GrB_BinaryOp       accum,
                   const GrB_Semiring       op,
                   const GrB_Matrix         A, 
                   const GrB_Matrix         B,
                   const GrB_Descriptor     desc);
\end{verbatim}
\begin{center}(b) The {\sf mxm} function signature.\end{center}
\caption{GraphBLAS matrix-matrix multiply definition and function signature where 
$\odot$ (corresponding to {\sf accum}) performs an optional accumulate
operation with the output matrix, and $\matrix{M}$ is an optional write mask that
controls which elements of the output matrix get modified.\label{Fig:mxmfig}}
\hrule
\end{figure}

Since nearly all operations in the API follow the same basic structure, we will
highlight important aspects of the API by examining one operation: matrix-matrix
multiply as shown in Figure~\ref{Fig:mxmfig}.  Figure~\ref{Fig:mxmfig}(a) illustrates
the mathematical notation for that operation (optional items in brackets). This is similar to what is found in
\cite{mathgraphblas16}, with some additions supported by the API -- write masks and
accumulation -- which are contained in the signature in Figure~\ref{Fig:mxmfig}(b).
In the signature for every GraphBLAS operation, the output argument always 
appears first. This is followed by the write mask and accumulate operation, if supported. 
Next, the arguments describing the input objects and the math to be performed are
listed. Finally, the last argument is the optional descriptor.  All GraphBLAS methods
return a status code ({\sf GrB\_Info}) regarding the success of the call.

\subsection{Objects}

At its foundation, the GraphBLAS C API is built on opaque data types exposed
through the API by handles.  
A few of these opaque types are collections -- matrices and vectors --
where matrices are typically very sparse and vectors can be
sparse as well.  The {\sf mxm} signature consists of one output matrix, {\sf C}, and 
two input matrices, {\sf A} and {\sf B}. The dimensions and 
domain (type) of the object are specified at construction time and remain fixed for the lifetime of 
the object.  A number of domains corresponding to the C built-in types are
supported, and the API allows for definition of user-defined types as well.

The GraphBLAS C API also has opaque types for algebraic objects including
unary and binary operators, monoids, and semirings, which differ from the traditional
linear algebra of the BLAS.  The traditional arithmetic addition and
multiplication operations can be replaced (e.g., min and plus, respectively) through
user-defined capabilities that can operate on data from multiple domains.  The {\sf op} 
argument in Figure~\ref{Fig:mxmfig} corresponds to a GraphBLAS semiring, denoted by $\oplus.\otimes$ in the 
mathematical description, and it specifies the binary operations that take the 
place of arithmetic addition and multiplication, respectively. The {\sf accum}
argument in the signature, denoted by $\odot$, is an optional binary operator that can
be used to combine the results with existing values in the output matrix. If accumulation
is not desired, {\sf GrB\_NULL} is specified.

The API supports a few optional control objects: write masks and descriptors.  A
mask is specified using GraphBLAS matrix (or a vector for 1-dimensional operations).
Since this is a write mask, it is applied when assigning results to the output matrix
{\em after} accumulation has been performed.
The descriptor is a lightweight object that can modify the semantics of the
operation by allowing the user to specify which input parameters need to be transposed
prior to the operation, whether the mask should be structurally complemented, or whether
the output should be cleared before assignment.
Note that {\sf GrB\_NULL} can be specified if default behavior is desired.

\subsection{Operations}

The API supports the basic operations that were outlined in \cite{mathgraphblas16}:
inserting and extracting data, matrix multiply, element wise operations, subgraph assignment
and extraction, apply, reduction, and transpose.  It also provides a number of
variants that are useful in a number of graph algorithms. For example, instead of
just row reduction of a matrix to a vector, the API also supports reduction of both
matrices and vectors to scalars.  Variants of both subgraph assignment and extraction
support these operaitons on a single row or column of a matrix.  The API also 
supports the ability to assign a constant
scalar value to an entire subgraph (through a variant of subgraph assignment). 

\subsection{Execution Models}

The API also supports two execution modes: blocking and nonblocking. In blocking mode,
each API method completes the operation before proceeding to the next.  However, 
because the API is meant to support high-performance computing in a parallel and
distributed environment we also provide a non-blocking mode.  In this mode, the methods
{\em may} return immediately after the input arguments have been verified but before
any computation has been carried out.  This mode gives an implementation the flexibility
to choose an execution strategy that might reduce computation time through fused
operations, lazy evaluation, asynchronous evaluation, and/or asynchronous execution.  When
using nonblocking mode, a slightly different mechanism is used for reporting errors which
is beyond the scope of this paper.
