\section{InspectorExecutor}

In its non-blocking mode of execution, GraphBLAS allows specific implementations to support
various execution approaches. In particular, it allows for non-terminating methods to be 
executed is several stages, possibly interleaved with stages from other methods. This is called \emph{split execution}. In a particular split execution,
a method can go through and \emph{analyze} stage followed by a \emph{perform} stage.
The analyze stage computes various characteristics of the object being produced, whereas the
perform stage does the actual calculations.

It may be desirable to augment GraphBLAS with additional constructs to control this particular
analyze/perform split explicitly. In particular, having the application communicate properties of
an object that do not change between multiple perform stages, and therefore greatly simplifies
or eliminates the need for analyze stages, could be valuable.

As a concrete example, the current GraphBLAS specification exposes {\sf mxm} as a single operation to the user. Under the hood, it is understood that {\sf mxm} is typically implemented in two phases: analyze and compute. The analyze phase consists of allocating memory to the output matrix. The compute phase computes the value at each nonzero of the output matrix. The implementer has the option of deciding whether to set the nonzero structure of the output matrix in the analyze phase or the compute phase.

It may be desirable in some circumstances to expose these two phases of {\sf mxm} to the user as {\sf mxm\_analyze} and {\sf mxm\_compute}. 

For typical use cases, both the size and nonzero structure of the output matrix \textbf{C} is not known \emph{a priori} before the {\sf mxm} operation is run. However, there may exist situations where the user is computing a sequence of output matrices \textbf{C} for which the output nonzeroes share the same \emph{structural zeroes} and therefore, memory allocation. In this case, it may be advantageous for the user to call {\sf mxm\_analyze} and {\sf mxm\_compute} for the first matrix-multiplication in the sequence, and {\sf mxm\_compute} for the rest.

We decided to defer such facilities from version 1.0 of the GraphBLAS specification. We believe
that additional experience with implementation and use of GraphBLAS is necessary before
we can define the proper interfaces for explicit split execution.

\todo{Cite MKL and NVIDIA's inspector executor efforts}

