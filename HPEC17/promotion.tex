\section{Rank Promotion}
\label{Sec:promotion}

\renewcommand{\vector}[1]{{\bf #1}}
\renewcommand{\matrix}[1]{{\bf #1}}

\emph{Rank promotion} is the conversion of an object of lower rank (\eg,
scalar or rank 0) to an object of a higher rank (\eg, vector or rank 1).
It is a common feature in array programming languages suchs as Fortran~90+
and MATLAB.  Typically, a scalar is converted to a matrix or vector
by replicating it in every element of the matrix or vector. A vector is
conveterd to a matrix by replicating it either along the rows or along the
columns of the matrix.  The replication factor can be stated explicitly
or implicitly calculated in order to result in a valid operation.

In GraphBLAS, scalars, whether of built-in or user-defined type, are
always of rank 0. Vectors and matrices are of rank 1 and 2, respectively.
Our proposal is to support \emph{automatic} rank promotion by allowing,
in most cases, the use of an object of lower rank in an input argument.

For example, in the GraphBLAS matrix multiply method 
\begin{quote} 
{\sf GrB\_mxm(C,Mask,accum,op,A,B,desc)}
\end{quote}
{\sf A} and {\sf B} are input matrices. One could replace either (or both)
of them by a scalar or vector.  Assume no tranposition specified by the
descriptor {\sf desc}, and let {\sf C} be an $m \times n$ matrix, {\sf A}
be a scalar and {\sf B} be an $p \times n$ matrix. The scalar {\sf A}
would be converted into an $m \times p$ matrix, with all its elements
set to the value of {\sf A}.  After that, the matrix multiplication would
proceed as specified in the standard.  We note that the same requirements
for domain compatibility would still hold.

If, instead, {\sf A} were an $m$-element vector, it would be converted into
an $m \times p$ matrix by replicating it $p$ times as a column of the matrix.
Replication as rows could be achieved by specifying transposition of
{\sf A} in the descriptor.

A tentative list of GraphBLAS methods supporting automatic rank
promotion is shown in Table~\ref{Tab:Promotion}. We should note that 
rank promotion can already be accomplished explicitly with the existing
GraphBLAS methods. However, doing it automatically, as we propose,
saves a matrix or vector instantiation just for that purpose.

\begin{table}[htb]
	\hrule
	\caption{Tentative list of automatic rank promotions in GraphBLAS.}
	\label{Tab:Promotions}
	Matrices: $\matrix{A},\matrix{B},\matrix{C},\matrix{M}$ \\
	Vectors: $\vector{a},\vector{b},\vector{u},\vector{w}$ \\
	Scalars: $a,b,u$ \\
	$\Delta$ denotes a descriptor \\
	$\mathbb{S}$ is a semiring \\
	$\odot$ is a binary operator used for accumulation \\
	\begin{center}
		\begin{tabular}{|l|l|} \hline
			Method		& Promotions \\ \hline
			${\sf GrB\_mxm}(\matrix{C},\matrix{M},\odot,\mathbb{S},\matrix{A},\matrix{B},\Delta)$	& $a \rightarrow \matrix{A}$ \\
														& $\vector{a} \rightarrow \matrix{A}$ \\
														& $b \rightarrow \matrix{B}$ \\
														& $\vector{b} \rightarrow \matrix{B}$ \\
			\hline
			${\sf GrB\_vxm}(\vector{w},\matrix{m},\odot,\mathbb{S},\vector{u},\matrix{A},\Delta)$	& $u \rightarrow \vector{u}$ \\
														& $a \rightarrow \matrix{A}$ \\
                                                                                                                & $\vector{a} \rightarrow \matrix{A}$ \\
			\hline
			${\sf GrB\_mxv}(\vector{w},\matrix{m},\odot,\mathbb{S},\matrix{A},\vector{u},\Delta)$	& $u \rightarrow \vector{u}$ \\
														& $a \rightarrow \matrix{A}$ \\
                                                                                                                & $\vector{a} \rightarrow \matrix{A}$ \\
			\hline
		\end{tabular}
	\end{center}
	\hrule
\end{table}
