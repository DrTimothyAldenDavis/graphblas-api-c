% This is ch2-functions.tex of the GraphBLAS specification.
% This is an included file. See the master file for more information.
%

%

\chapter{Functions}
\index{functions}
\label{chap:Functions}
This chapter describes the functions in the Graph BLAS standard

For my own reference, I'm leaving in a few latex fragments I'd llike to remember for when we add our own text.


\begin{boxedcode}
\#pragma\plc{ }omp\plc{ directive-name [clause[ [},\plc{] clause] ... ] new-line}
\end{boxedcode}

As an example of how a construct might look, here is the single construct from OpenMP

\subsection{\code{single} Construct}
\index{single@{\code{single}}}
\index{constructs!single@{\code{single}}}
\label{subsec:single Construct}
\summary
The \code{single} construct specifies that the associated structured block is executed by only 
one of the threads in the team (not necessarily the master thread), in the context of its 
implicit task. The other threads in the team, which do not execute the block, wait at an 
implicit barrier at the end of the \code{single} construct unless a \code{nowait} clause is specified.

\parbox{\linewidth}{%
\syntax
\ccppspecificstart}
The syntax of the single construct is as follows:

\begin{boxedcode}
\#pragma omp single \plc{[clause[ [},\plc{] clause] ... ] new-line}
   \plc{structured-block}
\end{boxedcode}

\begin{samepage}
where \plc{clause} is one of the following:

\begin{indentedcodelist}
private(\plc{list})
firstprivate(\plc{list})
copyprivate(\plc{list})
nowait
\end{indentedcodelist}
\ccppspecificend
\end{samepage}

\fortranspecificstart
The syntax of the \code{single} construct is as follows:

\begin{boxedcode}
!\$omp single \plc{[clause[ [},\plc{] clause] ... ]}
   \plc{structured-block} 
!\$omp end single \plc{[end\_clause[ [},\plc{] end\_clause] ... ]}
\end{boxedcode}

where \plc{clause} is one of the following:

\begin{indentedcodelist}
private(\plc{list})
firstprivate(\plc{list})
\end{indentedcodelist}

and \plc{end\_clause} is one of the following: 

\begin{indentedcodelist}
copyprivate(\plc{list})
nowait
\end{indentedcodelist}
\fortranspecificend

\binding
The binding thread set for a \code{single} region is the current team. A \code{single} region 
binds to the innermost enclosing \code{parallel} region. Only the threads of the team 
executing the binding \code{parallel} region participate in the execution of the structured 
block and the implied barrier of the \code{single} region if the barrier is not eliminated by a 
\code{nowait} clause.

\descr
The method of choosing a thread to execute the structured block is implementation 
defined. There is an implicit barrier at the end of the \code{single} construct unless a 
\code{nowait} clause is specified. 

\restrictions
Restrictions to the \code{single} construct are as follows: 

\begin{itemize}
\item The \code{copyprivate} clause must not be used with the \code{nowait} clause.

\item At most one \code{nowait} clause can appear on a \code{single} construct.

\cppspecificstart
\item A throw executed inside a \code{single} region must cause execution to resume within the 
same \code{single} region, and the same thread that threw the exception must catch it.
\cppspecificend
\end{itemize}


\crossreferences
\begin{itemize}
\item \code{private} and \code{firstprivate} clauses, see 
\specref{subsec:Data-Sharing Attribute Clauses}.

\item \code{copyprivate} clause, see 
\specref{subsubsec:copyprivate clause}.
\end{itemize}


% This is the end of ch2-functions.tex of the GraphBLAS specification.

