\subsection{{\sf extract}: Selecting Sub-Graphs}
\label{Sec:extract}

Extract a sub-matrix (vector) from a larger matrix (vector). 

In the standard version of {\sf extract}, GraphBLAS index arrays
specify the elements in the source vector/matrix that are copied to
the destination. \scott{this statement goes with the next two subsections only.}

If an accumulation function is specified, then that function is
used to combine current elements in the destination with those elements
extracted from the source. If no accumulation function is specified, then
the destination is overwritten with the extracted elements.  \scott{this 
statement should be moved into individual Description sections.}

%--------------------------------------------------------------

\subsubsection{{\sf extract}: Standard vector variant}

For vectors, only one index array is used to specify
elements, and for matrices two index arrays (for row and column indices)
are needed.  The size of the destination vector is the same size as
the one index array provided.  

\paragraph{\syntax}

\begin{verbatim}
        GrB_info GrB_extract(GrB_Vector             *w,
                             const GrB_Vector        mask,
                             const GrB_BinaryOp      accum,
                             const GrB_Vector        u,
                             const GrB_Index        *indices,
                             const GrB_Index         nindices,
                             const GrB_Descriptor    desc);
\end{verbatim}

\paragraph{Parameters}

\begin{itemize}[leftmargin=1in]
    \item[{\sf w}]       ({\sf INOUT}) The vector to store the extracted subgraph.

    \item[{\sf mask}]    ({\sf IN}) Output mask. The mask
    specifies which elements of {\sf w} can be modified.
    If no mask is necessary (i.e., compute all elements of result
    vector), {\sf GrB\_NULL} should be specified.

    \item[{\sf accum}]   ({\sf IN})  Operator used for accumulating entries into existing {\sf w} entries. 
			If no accumulation is desired, {\sf GrB\_NULL} should be specified.

    \item[{\sf u}]       ({\sf IN}) The vector from which to extract the subgraph.
    \item[{\sf indices}] ({\sf IN}) The set of element indices specifying location of
                              elements from {\sf u} that are extracted. Can
                              be set to {\sf GrB\_ALL} if all elements are
                              to be extracted.
    \item[{\sf nindices}] ({\sf IN}) The number of indices in {\sf indices} array.

    \item[{\sf desc}]     ({\sf IN}) Operation descriptor. If a
    \emph{default} descriptor is desired, {\sf GrB\_NULL} is to be
    used.  Valid fields and values are as follows: \\ ~\\
    \begin{tabular}{lllp{2.5in}}
    Param & Field  & Value & Description \\
    \hline
    {\sf w}    & {\sf GrB\_OUTP} & {\sf GrB\_REPLACE} & Output vector {\sf w} is cleared (all elements removed) before result is stored in it. \\
    {\sf mask} & {\sf GrB\_MASK} & {\sf GrB\_SCMP}   & Use the structural complement of {\sf mask}. \\
    \end{tabular}
\end{itemize}

\paragraph{Return Values}

\begin{itemize}[leftmargin=2.1in]
\item[{\sf GrB\_SUCCESS}]    operation completed successfully.
\item[{\sf GrB\_PANIC}]      unknown internal error.
\item[{\sf GrB\_OUTOFMEM}]   not enough memory available for operation.
\item[{\sf GrB\_NOOBJECT}]   one or more of the GraphBLAS objects has
                             not been initialized by a call to {\sf new}.
\item[{\sf GrB\_INVALID\_VALUE}]    {\sf w} or {\sf indices} pointer is {\sf NULL}.

\item[{\sf GrB\_INDEX\_OUTOFBOUNDS}]
        A value in {\sf indices} references a non-existent element in {\sf u}.
\item[{\sf GrB\_DIMENSION\_MISMATCH}] 
        The size of {\sf indices} is not equal to the size of {\sf u}, or
        the dimensions of the mask (if specified) do not match {\sf w}.
\item[{\sf GrB\_DOMAIN\_MISMATCH}]   Mismatch between $\bold{D}({\sf u})$ and
                                     $\bold{D}({\sf w})$, and/or the domains of the 
                                     {\sf accum} operation (if used), or if 
                                     $\bold{D}(\sf mask)$ is incorrect.
\end{itemize}

\paragraph{Description}

Vectors $\vector{u}$ and $\vector{mask}$ are computed from input
parameters {\sf u} and {\sf mask}, respectively, as specified by
descriptor {\sf desc}.  If either $\vector{u}$ or $\vector{mask}$
cannot be computed from the input parameters, the method returns {\sf
GrB\_DOMAIN\_MISMATCH}.

A consistency check is performed to verify that all indices in array
{\sf indices} are valid. That is, they must fall within the range of allowed
indices for vector $\vector{u}$ ($0 \leq {\sf indices}[i] < \bold{n}(\vector{u})
\forall i = 0,\ldots,{\sf nindices}-1$).  Otherwise, the method returns {\sf
GrB\_INDEX\_OUTOFBOUNDS}.

A new vector $\vector{w} = \langle \bold{D}({\sf w}),{\sf n},
\bold{L}(\vector{w}) = \{ (i,w_i), i = 0,\ldots,{\sf nindices}-1 : {\sf indices}[i]
\in \bold{i}(\vector{u}) \wedge \vector{mask}[i] = \true \} \rangle$
is created.  The value $w_i$ is set to $\vector{u}[{\sf indices}[i]]$. If
casting from $\bold{D}(\vector{u})$ to $\bold{D}(\vector{w})$ is not
allowed, the method returns {\sf GrB\_DOMAIN\_MISMATCH}.

Finally, output parameter {\sf w} is computed from vector $\vector{w}$ as
specified by descriptor {\sf desc} and accumulation function {\sf accum}.


%--------------------------------------------------------------

\subsubsection{{\sf extract}: Standard matrix variant}

For matrices, the destination matrix has
the same number of rows as the size of the row index array and the same
number of columns as the size of the column index array.

\paragraph{\syntax}

\begin{verbatim}                 
        GrB_info GrB_extract(GrB_Matrix                *C,
                             const GrB_Matrix           Mask,
                             const GrB_BinaryFunction   accum,
                             const GrB_Matrix           A,
                             const GrB_Index           *rowIDs,
                             const GrB_Index            nrows,
                             const GrB_Index           *colIDs,
                             const GrB_Index            ncols,
                             GrB_Descriptor const       desc);
\end{verbatim}

\paragraph{Parameters}

\begin{itemize}[leftmargin=1in]
    \item[{\sf C}]     ({\sf INOUT}) The matrix to store the extracted subgraph.

    \item[{\sf Mask}]  ({\sf IN}) Output mask. The mask
    specifies which elements of {\sf C} can be modified.
    If no mask is necessary (i.e., compute all elements of result
    vector), {\sf GrB\_NULL} should be specified.

    \item[{\sf accum}]  ({\sf IN})  Operator used for accumulating entries into existing {\sf C} entries. 
			If no accumulation is desired, {\sf GrB\_NULL} should be specified.

    \item[{\sf A}]      ({\sf IN})  The matrix from which to extract the subgraph.
    \item[{\sf rowIDs}] ({\sf IN})     The set of row indices specifying rows from source that
                              are extracted. Can
                              be set to {\sf GrB\_ALL} if all rows are
                              to be extracted.
    \item[{\sf nrows}]  ({\sf IN}) The number of indices in array {\sf rowIDs}.
    \item[{\sf colIDs}] ({\sf IN}) The set of column indices specifying
                              columns from source that are extracted. Can
                              be set to {\sf GrB\_ALL} if all columns are
                              to be extracted.
    \item[{\sf ncols}]  ({\sf IN}) The number of indices in array {\sf colIDs}.

    \item[{\sf desc}]   ({\sf IN}) Operation descriptor. If a
    \emph{default} descriptor is desired, {\sf GrB\_NULL} is to be
    used.  Valid fields and values are as follows: \\ ~\\
    \begin{tabular}{lllp{2.5in}}
    Param & Field  & Value & Description \\
    \hline
    {\sf C}    & {\sf GrB\_OUTP} & {\sf GrB\_REPLACE} & Output matrix {\sf C} is cleared (all elements removed) before result is stored in it. \\
    {\sf Mask} & {\sf GrB\_MASK} & {\sf GrB\_SCMP}   & Use the structural complement of {\sf Mask}. \\
    {\sf A}    & {\sf GrB\_INP0} & {\sf GrB\_TRAN}   & Apply transpose to {\sf A} before extract.) \\
    \end{tabular}
\end{itemize}

\paragraph{Return Values}

\begin{itemize}[leftmargin=2.1in]
\item[{\sf GrB\_SUCCESS}]     operation completed successfully.
\item[{\sf GrB\_PANIC}]        unknown internal error.
\item[{\sf GrB\_OUTOFMEM}]    not enough memory available for operation.
\item[{\sf GrB\_NOOBJECT}]   one or more of the GraphBLAS objects has
                             not been initialized by a call to {\sf new}.
\item[{\sf GrB\_INVALID\_VALUE}]    {\sf C}, {\sf rowIDs} or {\sf colIDs} pointer is {\sf NULL}.

\item[{\sf GrB\_INDEX\_OUTOFBOUNDS}]
        A value in {\sf rowIDs} references a non-existent row in {\sf A}, or
        the value in {\sf colIDs} references a non-existent column in {\sf A}.
\item[{\sf GrB\_DIMENSION\_MISMATCH}] 
        The size of {\sf rowIDs} (namely {\sf nrows}) is not equal to the number of rows in {\sf C}, or
        the size of {\sf colIDs} (namely {\sf ncols}) is not equal to the number of columns in {\sf C}, or
        the dimensions of the mask (if specified) do not match {\sf C}.
\item[{\sf GrB\_DOMAIN\_MISMATCH}]    Mismatch between $\bold{D}({\sf A})$ and $\bold{D}({\sf C})$, 
                                      and/or the domains of the 
                                      {\sf accum} operation (if used), or if $\bold{D}(\sf Mask)$ is incorrect.
\end{itemize}


\paragraph{Description}

Matrices $\matrix{A}$ and $\matrix{Mask}$ are computed from input
parameters {\sf A} and {\sf Mask}, respectively, as specified by
descriptor {\sf desc}.  If either $\matrix{A}$ or $\matrix{Mask}$
cannot be computed from the input parameters, the method returns {\sf
GrB\_DOMAIN\_MISMATCH}.

A consistency check is performed to verify that all indices in arrays
{\sf rows} and {\sf cols} are valid. That is, they must fall within the range of allowed
indices for matrix $\matrix{A}$ ($0 \leq {\sf rowIDs}[i] < \bold{m}(\matrix{A})
\forall i = 0,\ldots,{\sf nrows}-1$ and
$0 \leq {\sf colIDs}[j] < \bold{n}(\matrix{A}) \forall j = 0,\ldots,{\sf ncols}-1$).  Otherwise, the method returns {\sf
GrB\_INDEX\_OUTOFBOUNDS}.

A new matrix $\matrix{C} = \langle \bold{D}({\sf C}),{\sf nrows},{\sf ncols},
\bold{L}(\vector{C}) = \{ (i,j,C_{ij}), i = 0,\ldots,{\sf nrows}-1, j = 0, \ldots,{\sf ncols}-1 : {\sf rowIDs}[i]
\in \bold{i}(\matrix{A}) \wedge {\sf colIDs}[j] \in \bold{j}(\matrix{A}) \wedge \vector{Mask}[i,j] = \true \} \rangle$ is created. 
\scott{The use of $\bold{i}(\matrix{A})$ and $\bold{j}(\matrix{A})$ here is not correct.  They give the non-empty rows and columns of the matrix but not an indication if a value is stored at the intersection of this row and column.} 
The value $C_{ij}$ is set to $\matrix{A}[{\sf rowIDs}[i],{\sf colIDs}[j]]$. If
casting from $\bold{D}(\vector{A})$ to $\bold{D}(\vector{C})$ is not
allowed, the method returns {\sf GrB\_DOMAIN\_MISMATCH}.

Finally, output parameter {\sf C} is computed from matrix $\matrix{C}$ as
specified by descriptor {\sf desc} and accumulation function {\sf accum}.

%-----------------------------------------------------------------------------
\subsubsection{{\sf extract}: Column (and row) variant}

Extract from one column of a matrix into a vector.  Note that with the transpose
descriptor for the source matrix, elements of an arbitrary row of the matrix
can be extracted with this function as well.

\paragraph{\syntax}

\begin{verbatim}
        GrB_info GrB_extract(GrB_Vector             *w,
                             const GrB_Vector        mask,
                             const GrB_BinaryOp      accum,
                             const GrB_Matrix        A,
                             const GrB_Index        *rowIDs,
                             GrB_Index               nrows,
                             GrB_Index               colID,
                             const GrB_Descriptor    desc); 
\end{verbatim}

\paragraph{Parameters}

\begin{itemize}[leftmargin=1in]
    \item[{\sf w}]      ({\sf INOUT})  The vector into which to place the extracted values.

    \item[{\sf mask}]   ({\sf IN}) Output mask vector. The mask
    specifies which elements of {\sf w} can be modified.
    If no mask is necessary (i.e., compute all elements of result
    vector), {\sf GrB\_NULL} should be specified.

    \item[{\sf accum}]  ({\sf IN})  Operator used for accumulating entries into existing {\sf w} entries. 
			If no accumulation is desired, {\sf GrB\_NULL} should be specified.

    \item[{\sf A}]      ({\sf IN}) ) The matrix from which to extract the column.

    \item[{\sf rowIDs}] ({\sf IN})   An array of row indices to extract. Can
                              be set to a special array, {\sf GrB\_ALL}, if all elements
                              are to be extracted from the column.
    \item[{\sf nrows}]  ({\sf IN}) The number of indices in array {\sf rowIDs}.
    \item[{\sf colID}]  ({\sf IN}) The index of the column to extract.

    \item[{\sf desc}]   ({\sf IN}) Operation descriptor. If a
    \emph{default} descriptor is desired, {\sf GrB\_NULL} is to be
    used.  Valid fields and values are as follows: \\ ~ \\
    \begin{tabular}{lllp{2.5in}}
    Param & Field  & Value & Description \\
    \hline
    {\sf w}    & {\sf GrB\_OUTP} & {\sf GrB\_REPLACE} & Output vector {\sf w} is cleared (all elements removed) before result is stored in it. \\
    {\sf mask} & {\sf GrB\_MASK} & {\sf GrB\_SCMP} & Use the structural complement of {\sf mask}. \\
    {\sf A}    & {\sf GrB\_INP0} & {\sf GrB\_TRAN} & Apply transpose to {\sf A} before extract. \\
    \end{tabular}
\end{itemize}

\paragraph{Return Values}

\begin{itemize}[leftmargin=2.1in]
\item[{\sf GrB\_SUCCESS}]             Operation completed successfully.
\item[{\sf GrB\_PANIC}]               Unknown internal error.
\item[{\sf GrB\_OUTOFMEM}]    not enough memory available for operation.
\item[{\sf GrB\_NOOBJECT}]   one or more of the GraphBLAS objects has
                             not been initialized by a call to {\sf new}.
\item[{\sf GrB\_INVALID\_VALUE}]    {\sf w} or {\sf rowIDs} pointer is {\sf NULL}.

\item[{\sf GrB\_INDEX\_OUTOFBOUNDS}]  The indexes specify a position that outside the dimensions of src.
\item[{\sf GrB\_DIMENSION\_MISMATCH}] 
       The size of {\sf rowIDs} is greater than the size of {\sf w} , 
       or the dimensions of the mask (if specified) do not match {\sf w}.
\item[{\sf GrB\_DOMAIN\_MISMATCH}]    Mismatch between $\bold{D}(\sf A)$, 
                                      $\bold{D}({\sf w})$, and/or the domains of the 
                                      {\sf accum} operation (if used), or if $\bold{D}(\sf mask)$ is incorrect.
\end{itemize}

\paragraph{Description}

Matrix $\matrix{A}$ and vector $\vector{mask}$ are computed from input
parameters {\sf A} and {\sf mask}, respectively, as specified by
descriptor {\sf desc}.  If either $\matrix{A}$ or $\vector{mask}$
cannot be computed from the input parameters, the method returns {\sf
GrB\_DOMAIN\_MISMATCH}.

A consistency check is performed to verify that all indices in array
{\sf rowIDs} and index {\sf colID} are valid. That is, they must fall within the range of allowed
indices for matrix $\matrix{A}$ ($0 \leq {\sf rowIDs}[i] < \bold{n}(\matrix{A})
\forall i = 0,\ldots,{\sf nrows}-1$ and
$0 \leq {\sf colID} < \bold{n}(\matrix{A})$).  Otherwise, the method returns {\sf
GrB\_INDEX\_OUTOFBOUNDS}.

A new vector $\vector{w} = \langle \bold{D}({\sf w}),{\sf nrows},
\bold{L}(\vector{w}) = \{ (i,w_{i}), i = 0,\ldots,{\sf nrows}-1 : {\sf rowIDs}[i]
\in \bold{i}(\matrix{A}) \wedge {\sf colID} \in \bold{j}(\matrix{A}) \wedge \vector{mask}[i] = \true \} \rangle$
is created.  The value $w_{i}$ is set to $\matrix{A}[{\sf rowIDs}[i],{\sf colID}]$. If
casting from $\bold{D}(\matrix{A})$ to $\bold{D}(\vector{w})$ is not
allowed, the method returns {\sf GrB\_DOMAIN\_MISMATCH}.

Finally, output parameter {\sf w} is computed from vector $\vector{w}$ as
specified by descriptor {\sf desc} and accumulation function {\sf accum}.

%-----------------------------------------------------------------------------
\subsubsection{{\sf extract}: Single element vector variant}
\label{Sec:extract_single_element_vec}

\scott{(1) The descriptor is not necessary. (2) There is no masking. (3) Accumulation
with the scalar is much less useful.  I argue that this should be moved to Vector 
methods: Vector\_getElement or Vector\_getValue.}

Extract one element of a vector into a scalar. 

\paragraph{\syntax}

\begin{verbatim}
        GrB_info GrB_extract(<type>                 *val,
                             const GrB_BinaryOp      accum,
                             const GrB_Vector        u,
                             GrB_Index               index,
                             const GrB_Descriptor    desc,
                             char                   *err); 
\end{verbatim}

\paragraph{Parameters}

\begin{itemize}[leftmargin=1in]
    \item[{\sf val}]   ({\sf INOUT}) The scalar into which to assign the extracted value.
    \item[{\sf accum}] ({\sf IN}) Operator used for accumulation into {\sf val}. If no accumulation is desired,
                        {\sf GrB\_NULL} should be specified.
    \item[{\sf u}]     ({\sf IN})  The vector from which to extract the scalar.
    \item[{\sf index}] ({\sf IN}) The index of location in {\sf u} to extract

    \item[{\sf desc}]  ({\sf IN}) Operation descriptor (optional). If a
    \emph{default} descriptor is desired, {\sf GrB\_NULL} is to be
    used.  Valid fields and values are as follows: \\ ~ \\
    \begin{tabular}{lllp{2.5in}}
    Param & Field  & Value & Description \\
    \hline
    {\sf ---} & {\sf ---} & {\sf ---} & There are no valid descriptor fields for this operation. \\
    \end{tabular}
    \item[{\sf err}]   ({\sf INOUT?})  A null terminated string containing additional error
                         information on computations within the sequence 
                         terminated by this method. 

\end{itemize}

\paragraph{Return Values}

\begin{itemize}[leftmargin=2.1in]
\item[{\sf GrB\_SUCCESS}]          Operation completed successfully.
\item[{\sf GrB\_PANIC}]            Unknown internal error.
\item[{\sf GrB\_NOOBJECT}]   one or more of the GraphBLAS objects has
                             not been initialized by a call to {\sf new}.
\item[{\sf GrB\_INVALID\_VALUE}]    {\sf val} or {\sf err???} pointer is {\sf NULL}.
\item[{\sf GrB\_NOVALUE}]        No stored value at specified location.
\item[{\sf GrB\_INDEX\_OUTOFBOUNDS}]  Index {\sf index} is out of 
                                      bounds of the vector. 
\item[{\sf GrB\_DOMAIN\_MISMATCH}]    Mismatch between $\bold{D}(\sf u)$ and type of {\sf val},
                                      and/or the domains of the 
                                      {\sf accum} operation (if used).
\end{itemize}

\paragraph{Description}

If $({\sf index},u_{\sf index}) \notin \bold{L}({\sf u})$, the method returns {\sf GrB\_NOVALUE}.
Otherwise, if ${\sf accum} = {\sf GrB\_NULL}$, then ${\sf val} \leftarrow {\sf u}[{\sf index}]$.
If an accumulation function is provided, then ${\sf val} \leftarrow {\sf accum}({\sf val},{\sf u}[{\sf index}])$.
If computing the new value of {\sf val} requires casting and that is not allowed by the descriptor,
the method returns {\sf GrB\_DOMAIN\_MISMATCH}.

%--------------------------------------------------------------

\subsubsection{{\sf extract}: Single element matrix variant}
\label{Sec:extract_single_element_mat}

\scott{(1) The descriptor is not necessary. (2) There is no masking. (3) Accumulation
with the scalar is much less useful.  I argue that this should be moved to Matrix 
methods: Matrix\_getElement or Matrix\_getValue.}

Extract one element of a matrix into a scalar. 

\paragraph{\syntax}

\begin{verbatim}
        GrB_info GrB_extract(<type>                 *val,
                             const GrB_BinaryOp      accum,
                             const GrB_Matrix        A,
                             GrB_Index               rowID,
                             GrB_Index               colID,
                             const GrB_Descriptor    desc,
                             char                   *err); 

\end{verbatim}

\paragraph{Parameters}

\begin{itemize}[leftmargin=1in]
    \item[{\sf val}]   ({\sf INOUT? OUT?}) The scalar into which to assign the extracted value.
    \item[{\sf accum}] ({\sf IN}) Operator used for accumulation into {\sf val}. If no accumulation is desired,
                        {\sf GrB\_NULL} should be specified.
    \item[{\sf A}]     ({\sf IN}) The matrix from which to extract the scalar.
    \item[{\sf rowID}] ({\sf IN}) The row index of location to extract.
    \item[{\sf colID}] ({\sf IN}) The column index of location to extract.

    \item[{\sf desc}]  ({\sf IN}) Operation descriptor. If a
    \emph{default} descriptor is desired, {\sf GrB\_NULL} is to be
    used.  Valid fields and values are as follows: \\ ~ \\
    \begin{tabular}{lllp{2.5in}}
    Param & Field  & Value & Description \\
    \hline
    {\sf ---} & {\sf ---} & {\sf ---} & \scott{An argument can be made that are no valid descriptor fields for this operation.  Transpose of A is not necessary, just switch rowID and colID on call.} \\
    {\sf A}   & {\sf GrB\_INP0} & {\sf GrB\_TRAN} &  Transpose {\sf A} before extracting element \\
    \end{tabular}
    \item[{\sf err}]   ({\sf INOUT?}) A null terminated string containing additional error
                         information on computations within the sequence 
                         terminated by this method. 

\end{itemize}

\paragraph{Return Values}

\begin{itemize}[leftmargin=2.1in]
\item[{\sf GrB\_SUCCESS}]             Operation completed successfully.
\item[{\sf GrB\_PANIC}]               Unknown internal error.
\item[{\sf GrB\_NOOBJECT}]   one or more of the GraphBLAS objects has
                             not been initialized by a call to {\sf new}.
\item[{\sf GrB\_INVALID\_VALUE}]    {\sf val} or {\sf err???} pointer is {\sf NULL}.
\item[{\sf GrB\_NOVALUE}]             There is no value stored at the specified location.
\item[{\sf GrB\_INDEX\_OUTOFBOUNDS}]  Either the row or the column index,
                                      {\sf rowID} or {\sf colID}, is out of matrix bounds.
\item[{\sf GrB\_DOMAIN\_MISMATCH}]    Mismatch between the type of $(\sf val)$, 
                                      $\bold{D}({\sf A})$, and/or the domains of the 
                                      {\sf accum} operation (if used).
\end{itemize}

\paragraph{Description}

A matrix $\matrix{A}$ is computed from input parameter {\sf A} as specified by descriptor {\sf desc}.
If $\matrix{A}$ cannot be computed, the method returns {\sf GrB\_DOMAIN\_MISMATCH}.

If $({\sf rowID},{\sf colID}, A_{{\sf rowID},{\sf colID}}) \notin \bold{L}(\matrix{A})$, 
the method returns {\sf GrB\_NOVALUE}.  Otherwise, if 
${\sf accum} = {\sf GrB\_NULL}$, then ${\sf val} \leftarrow \matrix{A}[{\sf rowID},{\sf colID}]$.
If an accumulation function is provided, then 
${\sf val} \leftarrow {\sf accum}({\sf val},\matrix{A}[{\sf rowID},{\sf colID}])$.
If computing the new value of {\sf val} requires casting and that is not allowed,
the method returns {\sf GrB\_DOMAIN\_MISMATCH}.
