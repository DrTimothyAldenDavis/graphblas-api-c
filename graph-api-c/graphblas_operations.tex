\section{GraphBLAS Operations}
\label{Sec:Operations}

%
% TGM: I find this version of the table to be hopelessly confusing.  Please check out my proposed replacement
%
\comment{
\begin{table*}[h]
\hrule
\begin{center}
\caption{A Mathematical overview of the fundamental GraphBLAS operations supported.}
\label{Tab:GraphBLASOps}
\begin{tabular}{l|rrl}
{\sf Operation Name} & \multicolumn{3}{c}{Mathematical Description}  \\
\hline
{\sf mxm}          & $\matrix{C}(\neg\matrix{M})$ & $\oplus=$ & $\matrix{A}^T \oplus.\otimes \matrix{B}^T$  \\
{\sf mxv}          & $\vector{u}(\neg\vector{m})$ & $\oplus=$ & $\matrix{A}^T \oplus.\otimes \vector{v}$  \\
{\sf vxm}          & $\vector{u}(\neg\vector{m})$ & $\oplus=$ & $\vector{v} \oplus.\otimes \matrix{A}^T$  \\
{\sf eWiseMult}    & $\matrix{C}(\neg\matrix{M})$ & $\oplus=$ & $\matrix{A}^T \otimes \matrix{B}^T$  \\
{\sf eWiseAdd}     & $\matrix{C}(\neg\matrix{M})$ & $\oplus=$ & $\matrix{A}^T \oplus  \matrix{B}^T$  \\
{\sf reduce} (row) & $\vector{u}(\neg\vector{m})$ & $\oplus=$ & $\oplus_j\matrix{A}^T(:,j)$  \\
{\sf apply}        & $\matrix{C}(\neg\matrix{M})$ & $\oplus=$ & $f(\matrix{A}^T)$ \\
{\sf transpose}    & $\matrix{C}(\neg\matrix{M})$ & $\oplus=$ & $\matrix{A}^T$ \\
{\sf extract}      & $\matrix{C}(\neg\matrix{M})$ & $\oplus=$ & $\matrix{A}^T(\vector{i},\vector{j})$ \\
{\sf assign}       & $\matrix{C}(\neg\matrix{M})(\vector{i},\vector{j})$ & $\oplus=$ & $\matrix{A}^T$ \\
{\sf buildMatrix}  & $\matrix{C}(\neg\matrix{M})$ & $\oplus=$ & $\mathbb{S}^{m\times n}(\vector{i},\vector{j},\vector{v},\oplus_{dup})$ \\
{\sf buildVector}  & $\vector{u}(\neg\vector{m})$ & $\oplus=$ & $\mathbb{S}^{n}(\vector{i},\vector{v})$ \\
{\sf extractTuples}& $(\vector{i},\vector{j},\vector{v})$ & $=$ & $\matrix{A}(\neg\matrix{M})$ \\
\end{tabular}
\end{center}
\hrule
\end{table*}
}

\begin{table*}[h]
\hrule
\begin{center}
\caption{A Mathematical overview of the fundamental GraphBLAS operations supported
in this specification.  Input matrices $\matrix{A}$ and $\matrix{B}$ may be optionally
transposed.   Use of an optional mask is indicated, for example when applied to the matrix  $\matrix{C}$, 
as $\matrix{C}(\matrix{M})$. The mask or its structural compliment controls 
which values are written into the output result.}
\label{Tab:GraphBLASOps}
\begin{tabular}{l|rrl}
{\sf Operation Name} & \multicolumn{3}{c}{Mathematical Description}  \\
\hline
{\sf mxm}          & $\matrix{C}(\matrix{M})$ & $\oplus=$ & $\matrix{A} \oplus.\otimes \matrix{B}$  \\
{\sf mxv}          & $\vector{u}(\vector{m})$ & $\oplus=$ & $\matrix{A} \oplus.\otimes \vector{v}$  \\
{\sf vxm}          & $\vector{u}(\vector{m})$ & $\oplus=$ & $\vector{v} \oplus.\otimes \matrix{A}$  \\
{\sf eWiseMult}    & $\matrix{C}(\matrix{M})$ & $\oplus=$ & $\matrix{A} \otimes \matrix{B}$  \\
{\sf eWiseAdd}     & $\matrix{C}(\matrix{M})$ & $\oplus=$ & $\matrix{A} \oplus  \matrix{B}$  \\
{\sf reduce} (row) & $\vector{u}(\vector{m})$ & $\oplus=$ & $\oplus_j\matrix{A}(:,j)$  \\
{\sf apply}        & $\matrix{C}(\matrix{M})$ & $\oplus=$ & $f(\matrix{A})$ \\
{\sf transpose}    & $\matrix{C}(\matrix{M})$ & $\oplus=$ & $\matrix{A}$ \\
{\sf extract}      & $\matrix{C}(\matrix{M})$ & $\oplus=$ & $\matrix{A}(\vector{i},\vector{j})$ \\
{\sf assign}       & $\matrix{C}(\matrix{M})(\vector{i},\vector{j})$ & $\oplus=$ & $\matrix{A}$ \\
{\sf buildMatrix}  & $\matrix{C}(\matrix{M})$ & $\oplus=$ & $\mathbb{S}^{m\times n}(\vector{i},\vector{j},\vector{v},\oplus_{dup})$ \\
{\sf buildVector}  & $\vector{u}(\vector{m})$ & $\oplus=$ & $\mathbb{S}^{n}(\vector{i},\vector{v})$ \\
{\sf extractTuples}& $(\vector{i},\vector{j},\vector{v})$ & $=$ & $\matrix{A}(\matrix{M})$ \\
\end{tabular}
\end{center}
\hrule
\end{table*}




The GraphBLAS operations are defined in the GraphBLAS math spec and summarized in 
Table~\ref{Tab:GraphBLASOps}.   In addition to methods that implement these fundamental GraphBLAS 
operations, we support a number of variants that have been found to be especially useful in 
algorithm development.

A GraphBLAS operation is only valid when the domains of the GraphBLAS objects are mathematically
consistent.  The C programming language defines implicit casts between data types.  For example, floats,
doubles and ints can be freely mixed according to the rules defined for implicit casts.  It is the responsibility of
the user to assure that these casts are appropriate for the algorithm in question.  For example, a cast to int
implies truncation of a floating point type.  Depending on the operation, this truncation error could lead to erroneous
results.  Furthermore, casting a wider type onto a narrower type can lead to overflow errors.  The GraphBLAS
operations do not attempt to protect a user from these sorts of errors.

GraphBLAS operations also make assumptions about the numbers of dimensions and sizes of objects in
an operation.   An operation will test these sizes and report an error if they are inconsistent.  For example, 
when multiplying two matrices, C=A*B, the number of rows of C must equal the number of rows of A, the number of
columns of A must match the number of rows of B, and the number of columns of C must match then number
of columns of B.  This is the behavior expected given the mathematical definition of the operations.   

For most of the GraphBLAS operations involving matrices, an optional descriptor can modify the matrix
associated with an input GraphBLAS matrix object.  For example, if an input matrix is an argument to a GraphBLAS
operation and the associated descriptor indicates the transpose option, then the operation occurs as if on the 
transposed matrix.  In this case, the relationships between the sizes in each dimension shift in the mathematically 
expected way. 

The methods that implement the GraphBLAS operations test the domains and the sizes in each dimension based
on the input arguments and the descriptors.  Error conditions are reported when mismatches occur.


%A mathematical overview of the the fundamental GraphBLAS operations that are
%discussed in this section are shown in Table~\ref{Tab:GraphBLASOps}.  This
%ection also specifies variants to some of these operations where they have
%been found especially useful in algorithm development.
%

%
%  Note ... I've written this assuming the treatment of mask that Aydin and Tim
% prefer.  We'll change this text is we go the way Jose wants us to go.
%
When a GraphBLAS operation supports the use of an optional mask, that mask is
specified through a GraphBLAS vector (for one-dimensional masks) or
a GraphBLAS matrix (for two-dimensional masks).  When a mask is used, it is 
applied to the result from the operation and then that result is either assigned to the 
provided output matrix or, if a binary accumulation operation is provided, the result is 
accumulated into the corresponding elements of the provided output matrix.

Given a GraphBLAS vector $\vector{v} = \langle D,N, \{ (i,v_i) \} \rangle$, a
one-dimensional mask $\vector{m} = \langle N, \{ i : \mbox{\tt (bool)}v_i = \true \} \rangle$
is derived for use in the operation, where $\mbox{\tt (bool)}v_i$ denotes
casting the value $v_i$ to a Boolean value (\true\ or \false).
We note that, if cast is disallowed for the mask by the operation descriptor, then
$\bold{D}(\vector{v})$ must be {\sf GrB\_BOOL}.

Given a GraphBLAS matrix $\matrix{A} = \langle D, M, N, \{ (i,j,A_{ij}) \} \rangle$,
a two-dimensional mask $\matrix{M} = \langle M,N, \{ (i,j) : \mbox{\tt (bool)}A_{ij} = \true \} \rangle$
is derived for use in the operation, where $\mbox{\tt (bool)}A_{ij}$ denotes
casting the value $A_{ij}$ to a Boolean value (\true\ or \false).
%We note that, if cast is disallowed for the mask by the operation descriptor, then
%$\bold{D}(\matrix{A})$ must be {\sf GrB\_BOOL}.

In both the one- and two-dimensional cases, the mask may go through a structural
complement operation ($\S$~\ref{Sec:Masks}) as specified in the descriptor, before a final
mask is generated for use in the operation.



