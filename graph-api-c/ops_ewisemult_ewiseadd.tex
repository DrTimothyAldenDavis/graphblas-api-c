\subsection{{\sf eWiseMult}: Element-wise multiplication}

{\bf Note:} The difference between {\sf eWiseAdd} and {\sf eWiseMult} is not 
about the element-wise operation but how the index sets are treated.
{\sf eWiseAdd} returns an object whose indices are the ``union'' of the 
indices of the inputs whereas {\sf eWiseMult} returns an object whose indices 
are the ``intersection'' of the indices of the inputs. In both cases, the 
passed semiring, monoid, or operator operates on the set of values from the 
resulting index set. 

\comment{
\scott{Two points need to be revisited for ewiseX operations.  First, 'add' and 'mult'
really are a misnomer and have been the source of much confusion.  ewiseunion and
ewiseintersection really are much more descriptive.  Second, we are not in agreement
about whether or not to support the semiring interface as you need to then carefully
document which of the semirings binary operators will be used.  I recommend removal,
while providing helper functions that can 'wrap' semirings that extract the 
appropriate binary function: extract\_add(sr) or extract\_mult(sr).}

\aydin{As everyone will remember, I am all for both suggestions: rename to 
intersect/union (ewise union/intersect is an oxymoron so ewise- prefix should 
not be there) and don't support the confusing semiring version}
}

%-----------------------------------------------------------------------------

\subsubsection{{\sf eWiseMult}: Vector variant}

Perform element-wise (general) multiplication on the intersection of elements 
of two vectors, producing a third vector as result.

\paragraph{\syntax}

\begin{verbatim}
        GrB_Info GrB_eWiseMult(GrB_Vector             w,
                               const GrB_Vector       mask,
                               const GrB_BinaryOp     accum,
                               const GrB_Semiring     op,
                               const GrB_Vector       u,
                               const GrB_Vector       v,
                               const GrB_Descriptor   desc);
                            
        GrB_Info GrB_eWiseMult(GrB_Vector             w,
                               const GrB_Vector       mask,
                               const GrB_BinaryOp     accum,
                               const GrB_Monoid       op,
                               const GrB_Vector       u,
                               const GrB_Vector       v,
                               const GrB_Descriptor   desc);

        GrB_Info GrB_eWiseMult(GrB_Vector             w,
                               const GrB_Vector       mask,
                               const GrB_BinaryOp     accum,
                               const GrB_BinaryOp     op,
                               const GrB_Vector       u,
                               const GrB_Vector       v,
                               const GrB_Descriptor   desc);
\end{verbatim}

\paragraph{Parameters}

\begin{itemize}[leftmargin=1.1in]
    \item[{\sf w}]    ({\sf INOUT}) An existing GraphBLAS vector.  On input,
    the vector provides values that may be accumulated with the result of the
    element-wise operation.  On output, this vector holds the results of the
    operation.

    \item[{\sf mask}] ({\sf IN}) An optional ``write'' mask that controls which
    results from this operation are stored into the output vector {\sf w}. The 
    mask dimensions must match those of the vector {\sf w}. If the 
    {\sf GrB\_STRUCTURE} descriptor is {\em not} set for the mask, the domain of the
    {\sf mask} vector must be of type {\sf bool} or any of the predefined 
    ``built-in'' types in Table~\ref{Tab:PredefinedTypes}.  If the default
    mask is desired (\ie, a mask that is all {\sf true} with the dimensions of {\sf w}), 
    {\sf GrB\_NULL} should be specified.

    \item[{\sf accum}] ({\sf IN}) An optional binary operator used for accumulating
    entries into existing {\sf w} entries.
    %: ${\sf accum} = \langle \bDout({\sf accum}),\bDin1({\sf accum}),
    %\bDin2({\sf accum}), \odot \rangle$. 
    If assignment rather than accumulation is
    desired, {\sf GrB\_NULL} should be specified.

    \item[{\sf op}]    ({\sf IN}) The semiring, monoid, or binary operator 
    used in the element-wise ``product'' operation.  Depending on which type is
    passed, the following defines the binary operator, 
    $F_b=\langle \bDout({\sf op}),\bDin1({\sf op}),\bDin2({\sf op}),\otimes\rangle$, used:
    \begin{itemize}[leftmargin=1.1in]
    \item[BinaryOp:] $F_b = \langle \bDout({\sf op}), \bDin1({\sf op}),
    \bDin2({\sf op}),\bold{\bigodot}({\sf op})\rangle$.  
    \item[Monoid:] $F_b = \langle \bold{D}({\sf op}), \bold{D}({\sf op}),
    \bold{D}({\sf op}),\bold{\bigodot}({\sf op})\rangle$;
    the identity element is ignored. 
    \item[Semiring:] $F_b = \langle \bDout({\sf op}), \bDin1({\sf op}),
    \bDin2({\sf op}),\bold{\bigotimes}({\sf op})\rangle$; the
    additive monoid is ignored.
    \end{itemize}

    \item[{\sf u}]    ({\sf IN}) The GraphBLAS vector holding the values for
    the left-hand vector in the operation.

    \item[{\sf v}]    ({\sf IN}) The GraphBLAS vector holding the values for
    the right-hand vector in the operation.

    \item[{\sf desc}] ({\sf IN}) An optional operation descriptor. If
    a \emph{default} descriptor is desired, {\sf GrB\_NULL} should be
    specified. Non-default field/value pairs are listed as follows:  \\

    \hspace*{-2em}\begin{tabular}{lllp{2.7in}}
        Param & Field  & Value & Description \\
        \hline
        {\sf w}    & {\sf GrB\_OUTP} & {\sf GrB\_REPLACE} & Output vector {\sf w}
        is cleared (all elements removed) before the result is stored in it.\\

        {\sf mask} & {\sf GrB\_MASK} & {\sf GrB\_STRUCTURE}   & Use the structure
        of the {\sf mask} and don't examine the stored values.\\

        {\sf mask} & {\sf GrB\_MASK} & {\sf GrB\_SCMP}   & Use the structural
        complement of {\sf mask}. \\
    \end{tabular}
\end{itemize}

\paragraph{Return Values}

\begin{itemize}[leftmargin=2.1in]
    \item[{\sf GrB\_SUCCESS}]         In blocking mode, the operation completed
    successfully. In non-blocking mode, this indicates that the compatibility 
    tests on dimensions and domains for the input arguments passed successfully. 
    Either way, output vector {\sf w} is ready to be used in the next method of 
    the sequence.

    \item[{\sf GrB\_PANIC}]           Unknown internal error.

    \item[{\sf GrB\_INVALID\_OBJECT}] This is returned in any execution mode 
    whenever one of the opaque GraphBLAS objects (input or output) is in an invalid 
    state caused by a previous execution error.  Call {\sf GrB\_error()} to access 
    any error messages generated by the implementation.

    \item[{\sf GrB\_OUT\_OF\_MEMORY}] Not enough memory available for the operation.

    \item[{\sf GrB\_UNINITIALIZED\_OBJECT}] One or more of the GraphBLAS objects
    has not been initialized by a call to {\sf new} (or {\sf dup} for vector
    parameters).

    \item[{\sf GrB\_DIMENSION\_MISMATCH}] Mask or vector dimensions are incompatible.

    \item[{\sf GrB\_DOMAIN\_MISMATCH}]    The domains of the various vectors are
    incompatible with the corresponding domains of the binary operator ({\sf op}) or
    accumulation operator, or the mask's domain is not compatible with {\sf bool}
    (in the case where {\sf desc[GrB\_MASK].GrB\_STRUCTURE} is not set).
\end{itemize}

\paragraph{Description}

This variant of {\sf GrB\_eWiseMult} computes the element-wise ``product'' of
two GraphBLAS vectors: ${\sf w} = {\sf u} \otimes {\sf v}$, or, if an optional
binary accumulation operator ($\odot$) is provided, ${\sf w} = {\sf w} \odot
\left({\sf u} \otimes {\sf v}\right)$.  Logically, this operation occurs in
three steps:
\begin{enumerate}[leftmargin=0.75in]
\item[\bf Setup] The internal vectors and mask used in the computation are formed 
and their domains and dimensions are tested for compatibility.
\item[\bf Compute] The indicated computations are carried out.
\item[\bf Output] The result is written into the output vector, possibly under 
control of a mask.
\end{enumerate}

Up to four argument vectors are used in the {\sf GrB\_eWiseMult} operation:
\begin{enumerate}
	\item ${\sf w} = \langle \bold{D}({\sf w}),\bold{size}({\sf w}),\bold{L}({\sf w}) = \{(i,w_i) \} \rangle$
	\item ${\sf mask} = \langle \bold{D}({\sf mask}),\bold{size}({\sf mask}),\bold{L}({\sf mask}) = \{(i,m_i) \} \rangle$ (optional)
	\item ${\sf u} = \langle \bold{D}({\sf u}),\bold{size}({\sf u}),\bold{L}({\sf u}) = \{(i,u_i) \} \rangle$
	\item ${\sf v} = \langle \bold{D}({\sf v}),\bold{size}({\sf v}),\bold{L}({\sf v}) = \{(i,v_i) \} \rangle$
\end{enumerate}

The argument vectors, the ``product'' operator ({\sf op}), and the accumulation 
operator (if provided) are tested for domain compatibility as follows:
\begin{enumerate}
	\item If {\sf mask} is not {\sf GrB\_NULL}, and ${\sf desc[GrB\_MASK].GrB\_STRUCTURE}$
    is not set, then $\bold{D}({\sf mask})$ must be from one of the pre-defined types of 
    Table~\ref{Tab:PredefinedTypes}.

	\item $\bold{D}({\sf u})$ must be compatible with $\bDin1({\sf op})$.

	\item $\bold{D}({\sf v})$ must be compatible with $\bDin2({\sf op})$.

	\item $\bold{D}({\sf w})$ must be compatible with $\bDout({\sf op})$.

	\item If {\sf accum} is not {\sf GrB\_NULL}, then $\bold{D}({\sf w})$ must be
    compatible with $\bDin1({\sf accum})$ and $\bDout({\sf accum})$ of the accumulation operator and $\bDout({\sf op})$ of
    {\sf op} must be compatible with $\bDin2({\sf accum})$ of the accumulation operator.
\end{enumerate}
Two domains are compatible with each other if values from one domain can be cast 
to values in the other domain as per the rules of the C language.
In particular, domains from Table~\ref{Tab:PredefinedTypes} are all compatible 
with each other. A domain from a user-defined type is only compatible with itself.
If any compatibility rule above is violated, execution of {\sf GrB\_eWiseMult} ends
and the domain mismatch error listed above is returned.

From the argument vectors, the internal vectors and mask used in 
the computation are formed ($\leftarrow$ denotes copy):
\begin{enumerate}
	\item Vector $\vector{\widetilde{w}} \leftarrow {\sf w}$.

	\item One-dimensional mask, $\vector{\widetilde{m}}$, is computed from 
    argument {\sf mask} as follows:
	\begin{enumerate}
		\item If ${\sf mask} = {\sf GrB\_NULL}$, then $\vector{\widetilde{m}} = 
        \langle \bold{size}({\sf w}), \{i, \ \forall \ i : 0 \leq i < 
        \bold{size}({\sf w}) \} \rangle$.

		\item If {\sf mask} $\ne$ {\sf GrB\_NULL},  
        \begin{enumerate}
            \item If ${\sf desc[GrB\_MASK].GrB\_STRUCTURE}$ is set, then
            $\vector{\widetilde{m}} = 
            \langle \bold{size}({\sf mask}), \{i : i \in \bold{ind}({\sf mask}) \} \rangle$,
            \item Otherwise, $\vector{\widetilde{m}} = 
            \langle \bold{size}({\sf mask}), \{i : i \in \bold{ind}({\sf mask}) \wedge
            ({\sf bool}){\sf mask}(i) = \true \} \rangle$.
        \end{enumerate}

		\item	If ${\sf desc[GrB\_MASK].GrB\_SCMP}$ is set, then 
        $\vector{\widetilde{m}} \leftarrow \neg \vector{\widetilde{m}}$.
	\end{enumerate}

	\item Vector $\vector{\widetilde{u}} \leftarrow {\sf u}$.

	\item Vector $\vector{\widetilde{v}} \leftarrow {\sf v}$.
\end{enumerate}

The internal vectors and mask are checked for dimension compatibility. The following 
conditions must hold:
\begin{enumerate}
	\item $\bold{size}(\vector{\widetilde{w}}) = \bold{size}(\vector{\widetilde{m}})
    = \bold{size}(\vector{\widetilde{u}}) = \bold{size}(\vector{\widetilde{v}})$.
\end{enumerate}
If any compatibility rule above is violated, execution of {\sf GrB\_eWiseMult} ends and 
the dimension mismatch error listed above is returned.

From this point forward, in {\sf GrB\_NONBLOCKING} mode, the method can 
optionally exit with {\sf GrB\_SUCCESS} return code and defer any computation 
and/or execution error codes.

We are now ready to carry out the element-wise ``product'' and any additional 
associated operations.  We describe this in terms of two intermediate vectors:
\begin{itemize}
    \item $\vector{\widetilde{t}}$: The vector holding the element-wise ``product'' of
    $\vector{\widetilde{u}}$ and vector $\vector{\widetilde{v}}$.
    \item $\vector{\widetilde{z}}$: The vector holding the result after 
    application of the (optional) accumulation operator.
\end{itemize}

The intermediate vector $\vector{\widetilde{t}} = \langle
\bDout({\sf op}), \bold{size}(\vector{\widetilde{u}}),
\bold{L}(\vector{\widetilde{t}}) =
\{(i,t_i) : \bold{ind}(\vector{\widetilde{u}}) \cap 
\bold{ind}(\vector{\widetilde{v}})
 \neq \emptyset \} \rangle$
is created.  The value of each of its elements is computed by:
\[t_i = (\vector{\widetilde{u}}(i)
\otimes \vector{\widetilde{v}}(i)), \forall i \in 
(\bold{ind}(\vector{\widetilde{u}}) \cap 
\bold{ind}(\vector{\widetilde{v}}))\]



The intermediate vector $\vector{\widetilde{z}}$ is created as follows, using what is called a \emph{standard vector accumulate}:
\begin{itemize}
    \item If ${\sf accum} = {\sf GrB\_NULL}$, then $\vector{\widetilde{z}} = \vector{\widetilde{t}}$.

    \item If ${\sf accum}$ is a binary operator, then $\vector{\widetilde{z}}$ is defined as
        \[ \vector{\widetilde{z}} = \langle \bDout({\sf accum}), \bold{size}(\vector{\widetilde{w}}),
        %\bold{L}(\vector{\widetilde{z}}) =
        \{(i,z_{i}) \ \forall \ i \in \bold{ind}(\vector{\widetilde{w}}) \cup 
        \bold{ind}(\vector{\widetilde{t}}) \} \rangle.\]

    The values of the elements of $\vector{\widetilde{z}}$ are computed based on the 
    relationships between the sets of indices in $\vector{\widetilde{w}}$ and 
    $\vector{\widetilde{t}}$.
\[
    z_{i} = \vector{\widetilde{w}}(i) \odot \vector{\widetilde{t}}(i), \ \mbox{if}\  
    i \in  (\bold{ind}(\vector{\widetilde{t}}) \cap \bold{ind}(\vector{\widetilde{w}})),
\]
\[
    z_{i} = \vector{\widetilde{w}}(i), \ \mbox{if}\  
    i \in (\bold{ind}(\vector{\widetilde{w}}) - (\bold{ind}(\vector{\widetilde{t}})
    \cap \bold{ind}(\vector{\widetilde{w}}))),
\]
\[
    z_{i} = \vector{\widetilde{t}}(i), \ \mbox{if}\  i \in  
    (\bold{ind}(\vector{\widetilde{t}}) - (\bold{ind}(\vector{\widetilde{t}})
    \cap \bold{ind}(\vector{\widetilde{w}}))),
\]
where $\odot  = \bigodot({\sf accum})$, and the difference operator refers to set difference.
\end{itemize}




Finally, the set of output values that make up vector $\vector{\widetilde{z}}$ 
are written into the final result vector {\sf w}, using
what is called a \emph{standard vector mask and replace}. 
This is carried out under control of the mask which acts as a ``write mask''.
\begin{itemize}
\item If {\sf desc[GrB\_OUTP].GrB\_REPLACE} is set, then any values in {\sf w} 
on input to this operation are deleted and the contents of the new output vector,
{\sf w}, is defined as,
\[ 
\bold{L}({\sf w}) = \{(i,z_{i}) : i \in (\bold{ind}(\vector{\widetilde{z}}) 
\cap \bold{ind}(\vector{\widetilde{m}})) \}. 
\]

\item If {\sf desc[GrB\_OUTP].GrB\_REPLACE} is not set, the elements of 
$\vector{\widetilde{z}}$ indicated by the mask are copied into the result 
vector, {\sf w}, and elements of {\sf w} that fall outside the set indicated by 
the mask are unchanged:
\[ 
\bold{L}({\sf w}) = \{(i,w_{i}) : i \in (\bold{ind}({\sf w}) 
\cap \bold{ind}(\neg \vector{\widetilde{m}})) \} \cup \{(i,z_{i}) : i \in 
(\bold{ind}(\vector{\widetilde{z}}) \cap \bold{ind}(\vector{\widetilde{m}})) \}. 
\]
\end{itemize}

In {\sf GrB\_BLOCKING} mode, the method exits with return value 
{\sf GrB\_SUCCESS} and the new content of vector {\sf w} is as defined above
and fully computed.  
In {\sf GrB\_NONBLOCKING} mode, the method exits with return value 
{\sf GrB\_SUCCESS} and the new content of vector {\sf w} is as defined above 
but may not be fully computed; however, it can be used in the next GraphBLAS 
method call in a sequence.



%-----------------------------------------------------------------------------

\subsubsection{{\sf eWiseMult}: Matrix variant}

Perform element-wise (general) multiplication on the intersection of elements 
of two matrices, producing a third matrix as result.

\paragraph{\syntax}

\begin{verbatim}
        GrB_Info GrB_eWiseMult(GrB_Matrix             C,
                               const GrB_Matrix       Mask,
                               const GrB_BinaryOp     accum,
                               const GrB_Semiring     op,
                               const GrB_Matrix       A,
                               const GrB_Matrix       B,
                               const GrB_Descriptor   desc);

        GrB_Info GrB_eWiseMult(GrB_Matrix             C,
                               const GrB_Matrix       Mask,
                               const GrB_BinaryOp     accum,
                               const GrB_Monoid       op, 
                               const GrB_Matrix       A,
                               const GrB_Matrix       B,
                               const GrB_Descriptor   desc);

        GrB_Info GrB_eWiseMult(GrB_Matrix             C,
                               const GrB_Matrix       Mask,
                               const GrB_BinaryOp     accum,
                               const GrB_BinaryOp     op, 
                               const GrB_Matrix       A,
                               const GrB_Matrix       B,
                               const GrB_Descriptor   desc);
\end{verbatim}

\paragraph{Parameters}

\begin{itemize}[leftmargin=1.1in]
    \item[{\sf C}]    ({\sf INOUT}) An existing GraphBLAS matrix. On input,
    the matrix provides values that may be accumulated with the result of the
    element-wise operation.  On output, the matrix holds the results of the
    operation.

    \item[{\sf Mask}] ({\sf IN}) An optional ``write'' mask that controls which
    results from this operation are stored into the output matrix {\sf C}. The 
    mask dimensions must match those of the matrix {\sf C}. If the 
    {\sf GrB\_STRUCTURE} descriptor is {\em not} set for the mask, the domain of the 
    {\sf Mask} matrix must be of type {\sf bool} or any of the predefined 
    ``built-in'' types in Table~\ref{Tab:PredefinedTypes}.  If the default
    mask is desired (\ie, a mask that is all {\sf true} with the dimensions of {\sf C}), 
    {\sf GrB\_NULL} should be specified.

    \item[{\sf accum}] ({\sf IN}) An optional binary operator used for accumulating
    entries into existing {\sf C} entries.
    %: ${\sf accum} = \langle \bDout({\sf accum}),\bDin1({\sf accum}),
    %\bDin2({\sf accum}), \odot \rangle$. 
    If assignment rather than accumulation is
    desired, {\sf GrB\_NULL} should be specified.

    \item[{\sf op}]   ({\sf IN}) The semiring, monoid, or binary operator 
    used in the element-wise ``product'' operation.  Depending on which type is
    passed, the following defines the binary operator, 
    $F_b=\langle \bDout({\sf op}),\bDin1({\sf op}),\bDin2({\sf op}),\otimes\rangle$, used:
    \begin{itemize}[leftmargin=1.1in]
    \item[BinaryOp:] $F_b = \langle \bDout({\sf op}), \bDin1({\sf op}),
    \bDin2({\sf op}),\bold{\bigodot}({\sf op})\rangle$.  
    \item[Monoid:] $F_b = \langle \bold{D}({\sf op}), \bold{D}({\sf op}),
    \bold{D}({\sf op}),\bold{\bigodot}({\sf op})\rangle$;
    the identity element is ignored. 
    \item[Semiring:] $F_b = \langle \bDout({\sf op}), \bDin1({\sf op}),
    \bDin2({\sf op}),\bold{\bigotimes}({\sf op})\rangle$; the
    additive monoid is ignored.
    \end{itemize}
    
    \item[{\sf A}]    ({\sf IN}) The GraphBLAS matrix holding the values
    for the left-hand matrix in the operation.

    \item[{\sf B}]    ({\sf IN}) The GraphBLAS matrix holding the values for
    the right-hand matrix in the operation.

    \item[{\sf desc}] ({\sf IN}) An optional operation descriptor. If
    a \emph{default} descriptor is desired, {\sf GrB\_NULL} should be
    specified. Non-default field/value pairs are listed as follows:  \\

    \hspace*{-2em}\begin{tabular}{lllp{2.7in}}
        Param & Field  & Value & Description \\
        \hline
        {\sf C}    & {\sf GrB\_OUTP} & {\sf GrB\_REPLACE} & Output matrix {\sf C}
        is cleared (all elements removed) before the result is stored in it.\\

        {\sf Mask} & {\sf GrB\_MASK} & {\sf GrB\_STRUCTURE}   & Use the structure
        of the {\sf Mask} and don't examine the stored values.\\

        {\sf Mask} & {\sf GrB\_MASK} & {\sf GrB\_SCMP}   & Use the structural
        complement of {\sf Mask}. \\

        {\sf A}    & {\sf GrB\_INP0} & {\sf GrB\_TRAN}   & Use transpose of {\sf A}
        for the operation. \\

        {\sf B}    & {\sf GrB\_INP1} & {\sf GrB\_TRAN}   & Use transpose of {\sf B}
        for the operation. \\
    \end{tabular}
\end{itemize}

\paragraph{Return Values}

\begin{itemize}[leftmargin=2.1in]
    \item[{\sf GrB\_SUCCESS}]         In blocking mode, the operation completed
    successfully. In non-blocking mode, this indicates that the compatibility 
    tests on dimensions and domains for the input arguments passed successfully. 
    Either way, output matrix {\sf C} is ready to be used in the next method of
    the sequence.

    \item[{\sf GrB\_PANIC}]           Unknown internal error.

    \item[{\sf GrB\_INVALID\_OBJECT}] This is returned in any execution mode 
    whenever one of the opaque GraphBLAS objects (input or output) is in an invalid 
    state caused by a previous execution error.  Call {\sf GrB\_error()} to access 
    any error messages generated by the implementation.

    \item[{\sf GrB\_OUT\_OF\_MEMORY}] Not enough memory available for the operation.

    \item[{\sf GrB\_UNINITIALIZED\_OBJECT}] One or more of the GraphBLAS objects 
    has not been initialized by a call to {\sf new} (or {\sf Matrix\_dup} for matrix
    parameters).

    \item[{\sf GrB\_DIMENSION\_MISMATCH}] Mask and/or matrix
    dimensions are incompatible.

    \item[{\sf GrB\_DOMAIN\_MISMATCH}]    The domains of the various matrices are
    incompatible with the corresponding domains of the binary operator ({\sf op}) or
    accumulation operator, or the mask's domain is not compatible with {\sf bool}
    (in the case where {\sf desc[GrB\_MASK].GrB\_STRUCTURE} is not set).
\end{itemize}

\paragraph{Description}

This variant of {\sf GrB\_eWiseMult} computes the element-wise ``product'' of
two GraphBLAS matrices: ${\sf C} = {\sf A} \otimes {\sf B}$, or, if an optional
binary accumulation operator ($\odot$) is provided, ${\sf C} = {\sf C} \odot
\left({\sf A} \otimes {\sf B}\right)$.  Logically, this operation occurs in
three steps:
\begin{enumerate}[leftmargin=0.85in]
\item[\bf Setup] The internal matrices and mask used in the computation are formed and their 
domains and dimensions are tested for compatibility.
\item[\bf Compute] The indicated computations are carried out.
\item[\bf Output] The result is written into the output matrix, possibly under control of a mask.
\end{enumerate}

Up to four argument matrices are used in the {\sf GrB\_eWiseMult} operation:
\begin{enumerate}
	\item ${\sf C} = \langle \bold{D}({\sf C}),\bold{nrows}({\sf C}),
    \bold{ncols}({\sf C}),\bold{L}({\sf C}) = \{(i,j,C_{ij}) \} \rangle$

	\item ${\sf Mask} = \langle \bold{D}({\sf Mask}),\bold{nrows}({\sf Mask}),
    \bold{ncols}({\sf Mask}),\bold{L}({\sf Mask}) = \{(i,j,M_{ij}) \} \rangle$ (optional)

	\item ${\sf A} = \langle \bold{D}({\sf A}),\bold{nrows}({\sf A}),
    \bold{ncols}({\sf A}),\bold{L}({\sf A}) = \{(i,j,A_{ij}) \} \rangle$

	\item ${\sf B} = \langle \bold{D}({\sf B}),\bold{nrows}({\sf B}),
    \bold{ncols}({\sf B}),\bold{L}({\sf B}) = \{(i,j,B_{ij}) \} \rangle$
\end{enumerate}

The argument matrices, the ``product'' operator ({\sf op}), and the accumulation 
operator (if provided) are tested for domain compatibility as follows:
\begin{enumerate}
	\item If {\sf Mask} is not {\sf GrB\_NULL}, and ${\sf desc[GrB\_MASK].GrB\_STRUCTURE}$
    is not set, then $\bold{D}({\sf Mask})$ must be from one of the pre-defined types of 
    Table~\ref{Tab:PredefinedTypes}.

	\item $\bold{D}({\sf A})$ must be compatible with $\bDin1({\sf op})$.

	\item $\bold{D}({\sf B})$ must be compatible with $\bDin2({\sf op})$.

	\item $\bold{D}({\sf C})$ must be compatible with $\bDout({\sf op})$.

	\item If {\sf accum} is not {\sf GrB\_NULL}, then $\bold{D}({\sf C})$ must be
    compatible with $\bDin1({\sf accum})$ and $\bDout({\sf accum})$ of the accumulation operator and 
    $\bDout({\sf op})$ of {\sf op} must be compatible with $\bDin2({\sf accum})$ of the accumulation operator.
\end{enumerate}
Two domains are compatible with each other if values from one domain can be cast 
to values in the other domain as per the rules of the C language.
In particular, domains from Table~\ref{Tab:PredefinedTypes} are all compatible 
with each other. A domain from a user-defined type is only compatible with itself.
If any compatibility rule above is violated, execution of {\sf GrB\_eWiseMult} ends and 
the domain mismatch error listed above is returned.

From the argument matrices, the internal matrices and mask used in 
the computation are formed ($\leftarrow$ denotes copy):
\begin{enumerate}
	\item Matrix $\matrix{\widetilde{C}} \leftarrow {\sf C}$.

	\item Two-dimensional mask, $\matrix{\widetilde{M}}$, is computed from
    argument {\sf Mask} as follows:
	\begin{enumerate}
		\item If ${\sf Mask} = {\sf GrB\_NULL}$, then $\matrix{\widetilde{M}} = 
        \langle \bold{nrows}({\sf C}), \bold{ncols}({\sf C}), \{(i,j), 
        \forall i,j : 0 \leq i <  \bold{nrows}({\sf C}), 0 \leq j < 
        \bold{ncols}({\sf C}) \} \rangle$.

		\item If {\sf Mask} $\ne$ {\sf GrB\_NULL},
        \begin{enumerate}
            \item If ${\sf desc[GrB\_MASK].GrB\_STRUCTURE}$ is set, then 
            $\matrix{\widetilde{M}} = \langle \bold{nrows}({\sf Mask}), 
            \bold{ncols}({\sf Mask}), \{(i,j) : (i,j) \in \bold{ind}({\sf Mask}) \} \rangle$,
            \item Otherwise, $\matrix{\widetilde{M}} = \langle \bold{nrows}({\sf Mask}), 
            \bold{ncols}({\sf Mask}), \\ \{(i,j) : (i,j) \in \bold{ind}({\sf Mask}) \wedge 
            ({\sf bool}){\sf Mask}(i,j) = \true\} \rangle$.
        \end{enumerate}

		\item	If ${\sf desc[GrB\_MASK].GrB\_SCMP}$ is set, then 
        $\matrix{\widetilde{M}} \leftarrow \neg \matrix{\widetilde{M}}$.
	\end{enumerate}

	\item Matrix $\matrix{\widetilde{A}} \leftarrow
    {\sf desc[GrB\_INP0].GrB\_TRAN} \ ? \ {\sf A}^T : {\sf A}$.

	\item Matrix $\matrix{\widetilde{B}} \leftarrow
    {\sf desc[GrB\_INP1].GrB\_TRAN} \ ? \ {\sf B}^T : {\sf B}$.
\end{enumerate}

The internal matrices and masks are checked for dimension compatibility. The following
conditions must hold:
\begin{enumerate}
	\item $\bold{nrows}(\matrix{\widetilde{C}}) = \bold{nrows}(\matrix{\widetilde{M}})
	     = \bold{nrows}(\matrix{\widetilde{A}}) = \bold{nrows}(\matrix{\widetilde{C}})$.

	\item $\bold{ncols}(\matrix{\widetilde{C}}) = \bold{ncols}(\matrix{\widetilde{M}})
	     = \bold{ncols}(\matrix{\widetilde{A}}) = \bold{ncols}(\matrix{\widetilde{C}})$.
\end{enumerate}
If any compatibility rule above is violated, execution of {\sf GrB\_eWiseMult} ends and
the dimension mismatch error listed above is returned.

From this point forward, in {\sf GrB\_NONBLOCKING} mode, the method can 
optionally exit with {\sf GrB\_SUCCESS} return code and defer any computation 
and/or execution error codes.

We are now ready to carry out the element-wise ``product'' and any additional 
associated operations.  We describe this in terms of two intermediate matrices:
\begin{itemize}
    \item $\matrix{\widetilde{T}}$: The matrix holding the element-wise product of
    $\matrix{\widetilde{A}}$ and $\matrix{\widetilde{B}}$.
    \item $\matrix{\widetilde{Z}}$: The matrix holding the result after 
    application of the (optional) accumulation operator.
\end{itemize}

The intermediate matrix $\matrix{\widetilde{T}} = \langle
\bDout({\sf op}), \bold{nrows}(\matrix{\widetilde{A}}), \bold{ncols}(\matrix{\widetilde{A}}),
%\bold{L}(\matrix{\widetilde{T}}) =
\{(i,j,T_{ij}) : \bold{ind}(\matrix{\widetilde{A}}) \cap 
\bold{ind}(\matrix{\widetilde{B}}) \neq \emptyset \} \rangle$
is created.  The value of each of its elements is computed by 
\[T_{ij} = (\matrix{\widetilde{A}}(i,j)
\otimes \matrix{\widetilde{B}}(i,j)), \forall (i,j) \in 
\bold{ind}(\matrix{\widetilde{A}}) \cap 
\bold{ind}(\matrix{\widetilde{B}})\]


The intermediate matrix $\matrix{\widetilde{Z}}$ is created as follows, using what is called a \emph{standard matrix accumulate}:
\begin{itemize}
    \item If ${\sf accum} = {\sf GrB\_NULL}$, then $\matrix{\widetilde{Z}} = \matrix{\widetilde{T}}$.

    \item If ${\sf accum}$ is a binary operator, then $\matrix{\widetilde{Z}}$ is defined as
        \[ \langle \bDout({\sf accum}), \bold{nrows}(\matrix{\widetilde{C}}), \bold{ncols}(\matrix{\widetilde{C}}),
        %\bold{L}(\matrix{\widetilde{Z}}) =
        \{(i,j,Z_{ij})  \forall (i,j) \in \bold{ind}(\matrix{\widetilde{C}}) \cup 
        \bold{ind}(\matrix{\widetilde{T}}) \} \rangle.\]

    The values of the elements of $\matrix{\widetilde{Z}}$ are computed based on the
    relationships between the sets of indices in $\matrix{\widetilde{C}}$ and 
    $\matrix{\widetilde{T}}$.
\[
    Z_{ij} = \matrix{\widetilde{C}}(i,j) \odot \matrix{\widetilde{T}}(i,j), \ \mbox{if}\  
    (i,j) \in  (\bold{ind}(\matrix{\widetilde{T}}) \cap \bold{ind}(\matrix{\widetilde{C}})),
\]
\[
    Z_{ij} = \matrix{\widetilde{C}}(i,j), \ \mbox{if}\  
    (i,j) \in (\bold{ind}(\matrix{\widetilde{C}}) - (\bold{ind}(\matrix{\widetilde{T}})
    \cap \bold{ind}(\matrix{\widetilde{C}}))),
\]
\[
    Z_{ij} = \matrix{\widetilde{T}}(i,j), \ \mbox{if}\  (i,j) \in  
    (\bold{ind}(\matrix{\widetilde{T}}) - (\bold{ind}(\matrix{\widetilde{T}})
    \cap \bold{ind}(\matrix{\widetilde{C}}))),
\]
where $\odot  = \bigodot({\sf accum})$, and the difference operator refers to set difference.
\end{itemize}


%\emph{Standard Matrix Mask and Replace Options}

Finally, the set of output values that make up the $\matrix{\widetilde{Z}}$ 
matrix are written into the final result matrix, {\sf C}. 
This is carried out under control of the mask which acts as a ``write mask''.
\begin{itemize}
\item If {\sf desc[GrB\_OUTP].GrB\_REPLACE} is set, then any values in {\sf C} on
input to this operation are deleted and the contents of the new output matrix,
{\sf C}, is defined as,
\[
\bold{L}({\sf C}) = \{(i,j,Z_{ij}) : (i,j) \in (\bold{ind}(\matrix{\widetilde{Z}}) 
\cap \bold{ind}(\matrix{\widetilde{M}})) \}. 
\]

\item If {\sf desc[GrB\_OUTP].GrB\_REPLACE} is not set, the elements of 
$\matrix{\widetilde{Z}}$ indicated by the mask are copied into the result 
matrix, {\sf C}, and elements of {\sf C} that fall outside the set 
indicated by the mask are unchanged:
\[
\bold{L}({\sf C}) = \{(i,j,C_{ij}) : (i,j) \in (\bold{ind}({\sf C}) 
\cap \bold{ind}(\neg \matrix{\widetilde{M}})) \} \cup \{(i,j,Z_{ij}) : (i,j) \in 
(\bold{ind}(\matrix{\widetilde{Z}}) \cap \bold{ind}(\matrix{\widetilde{M}})) \}.
\]
\end{itemize}

In {\sf GrB\_BLOCKING} mode, the method exits with return value 
{\sf GrB\_SUCCESS} and the new content of matrix {\sf C} is as defined above
and fully computed.
In {\sf GrB\_NONBLOCKING} mode, the method exits with return value 
{\sf GrB\_SUCCESS} and the new content of matrix {\sf C} is as defined above
but may not be fully computed; however, it can be used in the next GraphBLAS 
method call in a sequence.



%=============================================================================


\subsection{{\sf eWiseAdd}: Element-wise addition}

{\bf Note:} The difference between {\sf eWiseAdd} and {\sf eWiseMult} is not 
about the element-wise operation but how the index sets are treated.
{\sf eWiseAdd} returns an object whose indices are the ``union'' of the indices 
of the inputs whereas  
{\sf eWiseMult} returns an object whose indices are the ``intersection'' of the 
indices of the inputs. In both cases, the passed semiring, monoid, or operator 
operates on the set of values from the resulting index set. 

%-----------------------------------------------------------------------------

\subsubsection{{\sf eWiseAdd}: Vector variant}

Perform element-wise (general) addition on the elements of two vectors,
producing a third vector as result.

\paragraph{\syntax}

\begin{verbatim}
        GrB_Info GrB_eWiseAdd(GrB_Vector             w,
                              const GrB_Vector       mask,
                              const GrB_BinaryOp     accum,
                              const GrB_Semiring     op,
                              const GrB_Vector       u,
                              const GrB_Vector       v,
                              const GrB_Descriptor   desc);
                            
        GrB_Info GrB_eWiseAdd(GrB_Vector             w,
                              const GrB_Vector       mask,
                              const GrB_BinaryOp     accum,
                              const GrB_Monoid       op,
                              const GrB_Vector       u,
                              const GrB_Vector       v,
                              const GrB_Descriptor   desc);

        GrB_Info GrB_eWiseAdd(GrB_Vector             w,
                              const GrB_Vector       mask,
                              const GrB_BinaryOp     accum,
                              const GrB_BinaryOp     op,
                              const GrB_Vector       u,
                              const GrB_Vector       v,
                              const GrB_Descriptor   desc);
\end{verbatim}

\paragraph{Parameters}

\begin{itemize}[leftmargin=1.1in]
    \item[{\sf w}]    ({\sf INOUT}) An existing GraphBLAS vector.  On input,
    the vector provides values that may be accumulated with the result of the
    element-wise operation.  On output, this vector holds the results of the
    operation.

    \item[{\sf mask}] ({\sf IN}) An optional ``write'' mask that controls which
    results from this operation are stored into the output vector {\sf w}. The 
    mask dimensions must match those of the vector {\sf w}. If the 
    {\sf GrB\_STRUCTURE} descriptor is {\em not} set for the mask, the domain of the
    {\sf mask} vector must be of type {\sf bool} or any of the predefined 
    ``built-in'' types in Table~\ref{Tab:PredefinedTypes}.  If the default
    mask is desired (\ie, a mask that is all {\sf true} with the dimensions of {\sf w}), 
    {\sf GrB\_NULL} should be specified.

    \item[{\sf accum}] ({\sf IN}) An optional binary operator used for accumulating
    entries into existing {\sf w} entries.
    %: ${\sf accum} = \langle \bDout({\sf accum}),\bDin1({\sf accum}),
    %\bDin2({\sf accum}), \odot \rangle$. 
    If assignment rather than accumulation is
    desired, {\sf GrB\_NULL} should be specified.

    \item[{\sf op}]    ({\sf IN}) The semiring, monoid, or binary operator 
    used in the element-wise ``sum'' operation.  Depending on which type is
    passed, the following defines the binary operator, 
    $F_b=\langle \bDout({\sf op}),\bDin1({\sf op}),\bDin2({\sf op}),\oplus\rangle$, used:
    \begin{itemize}[leftmargin=1.1in]
    \item[BinaryOp:] $F_b = \langle \bDout({\sf op}), \bDin1({\sf op}),
    \bDin2({\sf op}),\bold{\bigodot}({\sf op})\rangle$.  
    \item[Monoid:] $F_b = \langle \bold{D}({\sf op}), \bold{D}({\sf op}),
    \bold{D}({\sf op}),\bold{\bigodot}({\sf op})\rangle$;
    the identity element is ignored. 
    \item[Semiring:] $F_b = \langle \bDout({\sf op}), \bDin1({\sf op}),
    \bDin2({\sf op}),\bold{\bigoplus}({\sf op})\rangle$; the
    multiplicative binary op and additive identity are ignored.
    \end{itemize}
    
    \item[{\sf u}]    ({\sf IN}) The GraphBLAS vector holding the values for
    the left-hand vector in the operation.

    \item[{\sf v}]    ({\sf IN}) The GraphBLAS vector holding the values for
    the right-hand vector in the operation.

    \item[{\sf desc}] ({\sf IN}) An optional operation descriptor. If
    a \emph{default} descriptor is desired, {\sf GrB\_NULL} should be
    specified. Non-default field/value pairs are listed as follows:  \\

    \hspace*{-2em}\begin{tabular}{lllp{2.7in}}
        Param & Field  & Value & Description \\
        \hline
        {\sf w}    & {\sf GrB\_OUTP} & {\sf GrB\_REPLACE} & Output vector {\sf w}
        is cleared (all elements removed) before the result is stored in it.\\

        {\sf mask} & {\sf GrB\_MASK} & {\sf GrB\_STRUCTURE}   & Use the structure
        of the {\sf mask} and don't examine the stored values.\\

        {\sf mask} & {\sf GrB\_MASK} & {\sf GrB\_SCMP}   & Use the structural
        complement of {\sf mask}. \\
    \end{tabular}
\end{itemize}

\paragraph{Return Values}

\begin{itemize}[leftmargin=2.1in]
    \item[{\sf GrB\_SUCCESS}]         In blocking mode, the operation completed
    successfully. In non-blocking mode, this indicates that the compatibility 
    tests on dimensions and domains for the input arguments passed successfully. 
    Either way, output vector {\sf w} is ready to be used in the next method of 
    the sequence.

    \item[{\sf GrB\_PANIC}]           Unknown internal error.

    \item[{\sf GrB\_INVALID\_OBJECT}] This is returned in any execution mode 
    whenever one of the opaque GraphBLAS objects (input or output) is in an invalid 
    state caused by a previous execution error.  Call {\sf GrB\_error()} to access 
    any error messages generated by the implementation.

    \item[{\sf GrB\_OUT\_OF\_MEMORY}] Not enough memory available for the operation.

    \item[{\sf GrB\_UNINITIALIZED\_OBJECT}] One or more of the GraphBLAS objects
    has not been initialized by a call to {\sf new} (or {\sf dup} for vector
    parameters).

    \item[{\sf GrB\_DIMENSION\_MISMATCH}] Mask or vector dimensions are incompatible.

    \item[{\sf GrB\_DOMAIN\_MISMATCH}]    The domains of the various vectors are
    incompatible with the corresponding domains of the binary operator ({\sf op}) or
    accumulation operator, or the mask's domain is not compatible with {\sf bool}
    (in the case where {\sf desc[GrB\_MASK].GrB\_STRUCTURE} is not set).
\end{itemize}

\paragraph{Description}

This variant of {\sf GrB\_eWiseAdd} computes the element-wise ``sum'' of
two GraphBLAS vectors: ${\sf w} = {\sf u} \oplus {\sf v}$, or, if an optional
binary accumulation operator ($\odot$) is provided, ${\sf w} = {\sf w} \odot
\left({\sf u} \oplus {\sf v}\right)$.  Logically, this operation occurs in
three steps:
\begin{enumerate}[leftmargin=0.75in]
\item[\bf Setup] The internal vectors and mask used in the computation are formed 
and their domains and dimensions are tested for compatibility.
\item[\bf Compute] The indicated computations are carried out.
\item[\bf Output] The result is written into the output vector, possibly under 
control of a mask.
\end{enumerate}

Up to four argument vectors are used in the {\sf GrB\_eWiseAdd} operation:
\begin{enumerate}
	\item ${\sf w} = \langle \bold{D}({\sf w}),\bold{size}({\sf w}),\bold{L}({\sf w}) = \{(i,w_i) \} \rangle$
	\item ${\sf mask} = \langle \bold{D}({\sf mask}),\bold{size}({\sf mask}),\bold{L}({\sf mask}) = \{(i,m_i) \} \rangle$ (optional)
	\item ${\sf u} = \langle \bold{D}({\sf u}),\bold{size}({\sf u}),\bold{L}({\sf u}) = \{(i,u_i) \} \rangle$
	\item ${\sf v} = \langle \bold{D}({\sf v}),\bold{size}({\sf v}),\bold{L}({\sf v}) = \{(i,v_i) \} \rangle$
\end{enumerate}

The argument vectors, the ``sum'' operator ({\sf op}), and the accumulation 
operator (if provided) are tested for domain compatibility as follows:
\begin{enumerate}
	\item If {\sf mask} is not {\sf GrB\_NULL}, and ${\sf desc[GrB\_MASK].GrB\_STRUCTURE}$
    is not set, then $\bold{D}({\sf mask})$ must be from one of the pre-defined types of 
    Table~\ref{Tab:PredefinedTypes}.

	\item $\bold{D}({\sf u})$ must be compatible with $\bDin1({\sf op})$.

	\item $\bold{D}({\sf v})$ must be compatible with $\bDin2({\sf op})$.

	\item $\bold{D}({\sf w})$ must be compatible with $\bDout({\sf op})$.

	\item $\bold{D}({\sf u})$ and $\bold{D}({\sf v})$ must be compatible with $\bDout({\sf op})$.

	\item If {\sf accum} is not {\sf GrB\_NULL}, then $\bold{D}({\sf w})$ must be
    compatible with $\bDin1({\sf accum})$ and $\bDout({\sf accum})$ of the accumulation operator and $\bDout({\sf op})$ of
    {\sf op} must be compatible with $\bDin2({\sf accum})$ of the accumulation operator.
\end{enumerate}
Two domains are compatible with each other if values from one domain can be cast 
to values in the other domain as per the rules of the C language.
In particular, domains from Table~\ref{Tab:PredefinedTypes} are all compatible 
with each other. A domain from a user-defined type is only compatible with itself.
If any compatibility rule above is violated, execution of {\sf GrB\_eWiseMult} ends
and the domain mismatch error listed above is returned.

From the argument vectors, the internal vectors and mask used in 
the computation are formed ($\leftarrow$ denotes copy):
\begin{enumerate}
	\item Vector $\vector{\widetilde{w}} \leftarrow {\sf w}$.

	\item One-dimensional mask, $\vector{\widetilde{m}}$, is computed from 
    argument {\sf mask} as follows:
	\begin{enumerate}
		\item If ${\sf mask} = {\sf GrB\_NULL}$, then $\vector{\widetilde{m}} = 
        \langle \bold{size}({\sf w}), \{i, \ \forall \ i : 0 \leq i < 
        \bold{size}({\sf w}) \} \rangle$.

		\item If {\sf mask} $\ne$ {\sf GrB\_NULL},  
        \begin{enumerate}
            \item If ${\sf desc[GrB\_MASK].GrB\_STRUCTURE}$ is set, then
            $\vector{\widetilde{m}} = 
            \langle \bold{size}({\sf mask}), \{i : i \in \bold{ind}({\sf mask}) \} \rangle$,
            \item Otherwise, $\vector{\widetilde{m}} = 
            \langle \bold{size}({\sf mask}), \{i : i \in \bold{ind}({\sf mask}) \wedge
            ({\sf bool}){\sf mask}(i) = \true \} \rangle$.
        \end{enumerate}

		\item	If ${\sf desc[GrB\_MASK].GrB\_SCMP}$ is set, then 
        $\vector{\widetilde{m}} \leftarrow \neg \vector{\widetilde{m}}$.
	\end{enumerate}

	\item Vector $\vector{\widetilde{u}} \leftarrow {\sf u}$.

	\item Vector $\vector{\widetilde{v}} \leftarrow {\sf v}$.
\end{enumerate}

The internal vectors and mask are checked for dimension compatibility. 
The following conditions must hold:
\begin{enumerate}
	\item $\bold{size}(\vector{\widetilde{w}}) = \bold{size}(\vector{\widetilde{m}})
    = \bold{size}(\vector{\widetilde{u}}) = \bold{size}(\vector{\widetilde{v}})$.
\end{enumerate}
If any compatibility rule above is violated, execution of {\sf GrB\_eWiseMult} ends and 
the dimension mismatch error listed above is returned.

From this point forward, in {\sf GrB\_NONBLOCKING} mode, the method can 
optionally exit with {\sf GrB\_SUCCESS} return code and defer any computation 
and/or execution error codes.

We are now ready to carry out the element-wise ``sum'' and any additional 
associated operations.  We describe this in terms of two intermediate vectors:
\begin{itemize}
    \item $\vector{\widetilde{t}}$: The vector holding the element-wise ``sum'' of
    $\vector{\widetilde{u}}$ and vector $\vector{\widetilde{v}}$.
    \item $\vector{\widetilde{z}}$: The vector holding the result after 
    application of the (optional) accumulation operator.
\end{itemize}

The intermediate vector $\vector{\widetilde{t}} = \langle
\bDout({\sf op}), \bold{size}(\vector{\widetilde{u}}),
\bold{L}(\vector{\widetilde{t}}) =
\{(i,t_i) : \bold{ind}(\vector{\widetilde{u}}) \cap 
\bold{ind}(\vector{\widetilde{v}})
 \neq \emptyset \} \rangle$
is created.  The value of each of its elements is computed by:
\[t_i = (\vector{\widetilde{u}}(i) \oplus \vector{\widetilde{v}}(i)), \forall i \in (\bold{ind}(\vector{\widetilde{u}}) \cap \bold{ind}(\vector{\widetilde{v}}))\]
\[t_i = \vector{\widetilde{u}}(i), \forall i \in (\bold{ind}(\vector{\widetilde{u}}) - (\bold{ind}(\vector{\widetilde{v}}) \cap \bold{ind}(\vector{\widetilde{u}})))\]
\[t_i = \vector{\widetilde{v}}(i), \forall i \in (\bold{ind}(\vector{\widetilde{v}}) - (\bold{ind}(\vector{\widetilde{v}}) \cap \bold{ind}(\vector{\widetilde{u}})))\]
where the difference operator in the previous expressions refers to set difference.


The intermediate vector $\vector{\widetilde{z}}$ is created as follows, using what is called a \emph{standard vector accumulate}:
\begin{itemize}
    \item If ${\sf accum} = {\sf GrB\_NULL}$, then $\vector{\widetilde{z}} = \vector{\widetilde{t}}$.

    \item If ${\sf accum}$ is a binary operator, then $\vector{\widetilde{z}}$ is defined as
        \[ \vector{\widetilde{z}} = \langle \bDout({\sf accum}), \bold{size}(\vector{\widetilde{w}}),
        %\bold{L}(\vector{\widetilde{z}}) =
        \{(i,z_{i}) \ \forall \ i \in \bold{ind}(\vector{\widetilde{w}}) \cup 
        \bold{ind}(\vector{\widetilde{t}}) \} \rangle.\]

    The values of the elements of $\vector{\widetilde{z}}$ are computed based on the 
    relationships between the sets of indices in $\vector{\widetilde{w}}$ and 
    $\vector{\widetilde{t}}$.
\[
    z_{i} = \vector{\widetilde{w}}(i) \odot \vector{\widetilde{t}}(i), \ \mbox{if}\  
    i \in  (\bold{ind}(\vector{\widetilde{t}}) \cap \bold{ind}(\vector{\widetilde{w}})),
\]
\[
    z_{i} = \vector{\widetilde{w}}(i), \ \mbox{if}\  
    i \in (\bold{ind}(\vector{\widetilde{w}}) - (\bold{ind}(\vector{\widetilde{t}})
    \cap \bold{ind}(\vector{\widetilde{w}}))),
\]
\[
    z_{i} = \vector{\widetilde{t}}(i), \ \mbox{if}\  i \in  
    (\bold{ind}(\vector{\widetilde{t}}) - (\bold{ind}(\vector{\widetilde{t}})
    \cap \bold{ind}(\vector{\widetilde{w}}))),
\]
where $\odot  = \bigodot({\sf accum})$, and the difference operator refers to set difference.
\end{itemize}




Finally, the set of output values that make up vector $\vector{\widetilde{z}}$ 
are written into the final result vector {\sf w}, using
what is called a \emph{standard vector mask and replace}. 
This is carried out under control of the mask which acts as a ``write mask''.
\begin{itemize}
\item If {\sf desc[GrB\_OUTP].GrB\_REPLACE} is set, then any values in {\sf w} 
on input to this operation are deleted and the contents of the new output vector,
{\sf w}, is defined as,
\[ 
\bold{L}({\sf w}) = \{(i,z_{i}) : i \in (\bold{ind}(\vector{\widetilde{z}}) 
\cap \bold{ind}(\vector{\widetilde{m}})) \}. 
\]

\item If {\sf desc[GrB\_OUTP].GrB\_REPLACE} is not set, the elements of 
$\vector{\widetilde{z}}$ indicated by the mask are copied into the result 
vector, {\sf w}, and elements of {\sf w} that fall outside the set indicated by 
the mask are unchanged:
\[ 
\bold{L}({\sf w}) = \{(i,w_{i}) : i \in (\bold{ind}({\sf w}) 
\cap \bold{ind}(\neg \vector{\widetilde{m}})) \} \cup \{(i,z_{i}) : i \in 
(\bold{ind}(\vector{\widetilde{z}}) \cap \bold{ind}(\vector{\widetilde{m}})) \}. 
\]
\end{itemize}

In {\sf GrB\_BLOCKING} mode, the method exits with return value 
{\sf GrB\_SUCCESS} and the new content of vector {\sf w} is as defined above
and fully computed.  
In {\sf GrB\_NONBLOCKING} mode, the method exits with return value 
{\sf GrB\_SUCCESS} and the new content of vector {\sf w} is as defined above 
but may not be fully computed; however, it can be used in the next GraphBLAS 
method call in a sequence.



%-----------------------------------------------------------------------------

\subsubsection{{\sf eWiseAdd}: Matrix variant}

Perform element-wise (general) addition on the elements of two matrices,
producing a third matrix as result.

\paragraph{\syntax}

\begin{verbatim}
        GrB_Info GrB_eWiseAdd(GrB_Matrix             C,
                              const GrB_Matrix       Mask,
                              const GrB_BinaryOp     accum,
                              const GrB_Semiring     op, 
                              const GrB_Matrix       A,
                              const GrB_Matrix       B,
                              const GrB_Descriptor   desc);

        GrB_Info GrB_eWiseAdd(GrB_Matrix             C,
                              const GrB_Matrix       Mask,
                              const GrB_BinaryOp     accum,
                              const GrB_Monoid       op, 
                              const GrB_Matrix       A,
                              const GrB_Matrix       B,
                              const GrB_Descriptor   desc);

        GrB_Info GrB_eWiseAdd(GrB_Matrix             C,
                              const GrB_Matrix       Mask,
                              const GrB_BinaryOp     accum,
                              const GrB_BinaryOp     op, 
                              const GrB_Matrix       A,
                              const GrB_Matrix       B,
                              const GrB_Descriptor   desc);
\end{verbatim}

\paragraph{Parameters}

\begin{itemize}[leftmargin=1.1in]
    \item[{\sf C}]    ({\sf INOUT}) An existing GraphBLAS matrix. On input,
    the matrix provides values that may be accumulated with the result of the
    element-wise operation.  On output, the matrix holds the results of the
    operation.

    \item[{\sf Mask}] ({\sf IN}) An optional ``write'' mask that controls which
    results from this operation are stored into the output matrix {\sf C}. The 
    mask dimensions must match those of the matrix {\sf C}. If the 
    {\sf GrB\_STRUCTURE} descriptor is {\em not} set for the mask, the domain of the 
    {\sf Mask} matrix must be of type {\sf bool} or any of the predefined 
    ``built-in'' types in Table~\ref{Tab:PredefinedTypes}.  If the default
    mask is desired (\ie, a mask that is all {\sf true} with the dimensions of {\sf C}), 
    {\sf GrB\_NULL} should be specified.

    \item[{\sf accum}] ({\sf IN}) An optional binary operator used for accumulating
    entries into existing {\sf C} entries.
    %: ${\sf accum} = \langle \bDout({\sf accum}),\bDin1({\sf accum}),
    %\bDin2({\sf accum}), \odot \rangle$. 
    If assignment rather than accumulation is
    desired, {\sf GrB\_NULL} should be specified.

    \item[{\sf op}]   ({\sf IN}) The semiring, monoid, or binary operator 
    used in the element-wise ``sum'' operation.  Depending on which type is
    passed, the following defines the binary operator, 
    $F_b=\langle \bDout({\sf op}),\bDin1({\sf op}),\bDin2({\sf op}),\oplus\rangle$, used:
    \begin{itemize}[leftmargin=1.1in]
    \item[BinaryOp:] $F_b = \langle \bDout({\sf op}), \bDin1({\sf op}),
    \bDin2({\sf op}),\bold{\bigodot}({\sf op})\rangle$.  
    \item[Monoid:] $F_b = \langle \bold{D}({\sf op}), \bold{D}({\sf op}),
    \bold{D}({\sf op}),\bold{\bigodot}({\sf op})\rangle$;
    the identity element is ignored. 
    \item[Semiring:] $F_b = \langle \bDout({\sf op}), \bDin1({\sf op}),
    \bDin2({\sf op}),\bold{\bigoplus}({\sf op})\rangle$; the
    multiplicative binary op and additive identity are ignored.
    \end{itemize}
    
    \item[{\sf A}]    ({\sf IN}) The GraphBLAS matrix holding the values
    for the left-hand matrix in the operation.

    \item[{\sf B}]    ({\sf IN}) The GraphBLAS matrix holding the values for
    the right-hand matrix in the operation.

    \item[{\sf desc}] ({\sf IN}) An optional operation descriptor. If
    a \emph{default} descriptor is desired, {\sf GrB\_NULL} should be
    specified. Non-default field/value pairs are listed as follows:  \\

    \hspace*{-2em}\begin{tabular}{lllp{2.7in}}
        Param & Field  & Value & Description \\
        \hline
        {\sf C}    & {\sf GrB\_OUTP} & {\sf GrB\_REPLACE} & Output matrix {\sf C}
        is cleared (all elements removed) before the result is stored in it.\\

        {\sf Mask} & {\sf GrB\_MASK} & {\sf GrB\_STRUCTURE}   & Use the structure
        of the {\sf Mask} and don't examine the stored values.\\

        {\sf Mask} & {\sf GrB\_MASK} & {\sf GrB\_SCMP}   & Use the structural
        complement of {\sf Mask}. \\

        {\sf A}    & {\sf GrB\_INP0} & {\sf GrB\_TRAN}   & Use transpose of {\sf A}
        for the operation. \\

        {\sf B}    & {\sf GrB\_INP1} & {\sf GrB\_TRAN}   & Use transpose of {\sf B}
        for the operation. \\
    \end{tabular}
\end{itemize}

\paragraph{Return Values}

\begin{itemize}[leftmargin=2.1in]
    \item[{\sf GrB\_SUCCESS}]         In blocking mode, the operation completed
    successfully. In non-blocking mode, this indicates that the compatibility 
    tests on dimensions and domains for the input arguments passed successfully. 
    Either way, output matrix {\sf C} is ready to be used in the next method of
    the sequence.

    \item[{\sf GrB\_PANIC}]           Unknown internal error.

    \item[{\sf GrB\_INVALID\_OBJECT}] This is returned in any execution mode 
    whenever one of the opaque GraphBLAS objects (input or output) is in an invalid 
    state caused by a previous execution error.  Call {\sf GrB\_error()} to access 
    any error messages generated by the implementation.

    \item[{\sf GrB\_OUT\_OF\_MEMORY}] Not enough memory available for the operation.

    \item[{\sf GrB\_UNINITIALIZED\_OBJECT}] One or more of the GraphBLAS objects 
    has not been initialized by a call to {\sf new} (or {\sf Matrix\_dup} for matrix
    parameters).

    \item[{\sf GrB\_DIMENSION\_MISMATCH}] Mask and/or matrix
    dimensions are incompatible.

    \item[{\sf GrB\_DOMAIN\_MISMATCH}]    The domains of the various matrices are
    incompatible with the corresponding domains of the binary operator ({\sf op}) or
    accumulation operator, or the mask's domain is not compatible with {\sf bool}
    (in the case where {\sf desc[GrB\_MASK].GrB\_STRUCTURE} is not set).
\end{itemize}

\paragraph{Description}

This variant of {\sf GrB\_eWiseAdd } computes the element-wise ``sum'' of
two GraphBLAS matrices: ${\sf C} = {\sf A} \oplus {\sf B}$, or, if an optional
binary accumulation operator ($\odot$) is provided, ${\sf C} = {\sf C} \odot
\left({\sf A} \oplus {\sf B}\right)$.  Logically, this operation occurs in
three steps:
\begin{enumerate}[leftmargin=0.85in]
\item[\bf Setup] The internal matrices and mask used in the computation are formed 
and their domains and dimensions are tested for compatibility.
\item[\bf Compute] The indicated computations are carried out.
\item[\bf Output] The result is written into the output matrix, possibly under 
control of a mask.
\end{enumerate}

Up to four argument matrices are used in the {\sf GrB\_eWiseMult} operation:
\begin{enumerate}
	\item ${\sf C} = \langle \bold{D}({\sf C}),\bold{nrows}({\sf C}),\bold{ncols}({\sf C}),\bold{L}({\sf C}) = \{(i,j,C_{ij}) \} \rangle$
	\item ${\sf Mask} = \langle \bold{D}({\sf Mask}),\bold{nrows}({\sf Mask}),\bold{ncols}({\sf Mask}),\bold{L}({\sf Mask}) = \{(i,j,M_{ij}) \} \rangle$ (optional)
	\item ${\sf A} = \langle \bold{D}({\sf A}),\bold{nrows}({\sf A}), \bold{ncols}({\sf A}),\bold{L}({\sf A}) = \{(i,j,A_{ij}) \} \rangle$
	\item ${\sf B} = \langle \bold{D}({\sf B}),\bold{nrows}({\sf B}), \bold{ncols}({\sf B}),\bold{L}({\sf B}) = \{(i,j,B_{ij}) \} \rangle$
\end{enumerate}

The argument matrices, the ``sum'' operator ({\sf op}), and the accumulation 
operator (if provided) are tested for domain compatibility as follows:
\begin{enumerate}
	\item If {\sf Mask} is not {\sf GrB\_NULL}, and ${\sf desc[GrB\_MASK].GrB\_STRUCTURE}$
    is not set, then $\bold{D}({\sf Mask})$ must be from one of the pre-defined types of 
    Table~\ref{Tab:PredefinedTypes}.

	\item $\bold{D}({\sf A})$ must be compatible with $\bDin1({\sf op})$.

	\item $\bold{D}({\sf B})$ must be compatible with $\bDin2({\sf op})$.

	\item $\bold{D}({\sf C})$ must be compatible with $\bDout({\sf op})$.

	\item $\bold{D}({\sf A})$ and $\bold{D}({\sf B})$ must be compatible with $\bDout({\sf op})$.

	\item If {\sf accum} is not {\sf GrB\_NULL}, then $\bold{D}({\sf C})$ must be
    compatible with $\bDin1({\sf accum})$ and $\bDout({\sf accum})$ of the accumulation operator and 
    $\bDout({\sf op})$ of {\sf op} must be compatible with $\bDin2({\sf accum})$ of the accumulation operator.
\end{enumerate}
Two domains are compatible with each other if values from one domain can be cast 
to values in the other domain as per the rules of the C language.
In particular, domains from Table~\ref{Tab:PredefinedTypes} are all compatible 
with each other. A domain from a user-defined type is only compatible with itself.
If any compatibility rule above is violated, execution of {\sf GrB\_eWiseMult} ends and 
the domain mismatch error listed above is returned.

From the argument matrices, the internal matrices and mask used in 
the computation are formed ($\leftarrow$ denotes copy):
\begin{enumerate}
	\item Matrix $\matrix{\widetilde{C}} \leftarrow {\sf C}$.

	\item Two-dimensional mask, $\matrix{\widetilde{M}}$, is computed from
    argument {\sf Mask} as follows:
	\begin{enumerate}
		\item If ${\sf Mask} = {\sf GrB\_NULL}$, then $\matrix{\widetilde{M}} = 
        \langle \bold{nrows}({\sf C}), \bold{ncols}({\sf C}), \{(i,j), 
        \forall i,j : 0 \leq i <  \bold{nrows}({\sf C}), 0 \leq j < 
        \bold{ncols}({\sf C}) \} \rangle$.

		\item If {\sf Mask} $\ne$ {\sf GrB\_NULL},
        \begin{enumerate}
            \item If ${\sf desc[GrB\_MASK].GrB\_STRUCTURE}$ is set, then 
            $\matrix{\widetilde{M}} = \langle \bold{nrows}({\sf Mask}), 
            \bold{ncols}({\sf Mask}), \{(i,j) : (i,j) \in \bold{ind}({\sf Mask}) \} \rangle$,
            \item Otherwise, $\matrix{\widetilde{M}} = \langle \bold{nrows}({\sf Mask}), 
            \bold{ncols}({\sf Mask}), \\ \{(i,j) : (i,j) \in \bold{ind}({\sf Mask}) \wedge 
            ({\sf bool}){\sf Mask}(i,j) = \true\} \rangle$.
        \end{enumerate}

		\item	If ${\sf desc[GrB\_MASK].GrB\_SCMP}$ is set, then 
        $\matrix{\widetilde{M}} \leftarrow \neg \matrix{\widetilde{M}}$.
	\end{enumerate}

	\item Matrix $\matrix{\widetilde{A}} \leftarrow
    {\sf desc[GrB\_INP0].GrB\_TRAN} \ ? \ {\sf A}^T : {\sf A}$.

	\item Matrix $\matrix{\widetilde{B}} \leftarrow
    {\sf desc[GrB\_INP1].GrB\_TRAN} \ ? \ {\sf B}^T : {\sf B}$.
\end{enumerate}

The internal matrices and masks are checked for dimension compatibility. The following
conditions must hold:
\begin{enumerate}
	\item $\bold{nrows}(\matrix{\widetilde{C}}) = \bold{nrows}(\matrix{\widetilde{M}})
	     = \bold{nrows}(\matrix{\widetilde{A}}) = \bold{nrows}(\matrix{\widetilde{C}})$.

	\item $\bold{ncols}(\matrix{\widetilde{C}}) = \bold{ncols}(\matrix{\widetilde{M}})
	     = \bold{ncols}(\matrix{\widetilde{A}}) = \bold{ncols}(\matrix{\widetilde{C}})$.
\end{enumerate}
If any compatibility rule above is violated, execution of {\sf GrB\_eWiseMult} ends and
the dimension mismatch error listed above is returned.

From this point forward, in {\sf GrB\_NONBLOCKING} mode, the method can 
optionally exit with {\sf GrB\_SUCCESS} return code and defer any computation 
and/or execution error codes.

We are now ready to carry out the element-wise ``sum'' and any additional 
associated operations.  We describe this in terms of two intermediate matrices:
\begin{itemize}
    \item $\matrix{\widetilde{T}}$: The matrix holding the element-wise sum of
    $\matrix{\widetilde{A}}$ and $\matrix{\widetilde{B}}$.
    \item $\matrix{\widetilde{Z}}$: The matrix holding the result after 
    application of the (optional) accumulation operator.
\end{itemize}

The intermediate matrix $\matrix{\widetilde{T}} = \langle
\bDout({\sf op}), \bold{nrows}(\matrix{\widetilde{A}}), \bold{ncols}(\matrix{\widetilde{A}}),
%\bold{L}(\matrix{\widetilde{T}}) =
\{(i,j,T_{ij}) : \bold{ind}(\matrix{\widetilde{A}}) \cap 
\bold{ind}(\matrix{\widetilde{B}}) \neq \emptyset \} \rangle$
is created.  The value of each of its elements is computed by 
\[T_{ij} = (\matrix{\widetilde{A}}(i,j) \oplus \matrix{\widetilde{B}}(i,j)), \forall (i,j) \in \bold{ind}(\matrix{\widetilde{A}}) \cap \bold{ind}(\matrix{\widetilde{B}})\]
\[T_{ij} = \matrix{\widetilde{A}}(i,j), \forall (i,j) \in (\bold{ind}(\matrix{\widetilde{A}}) - (\bold{ind}(\matrix{\widetilde{B}}) \cap \bold{ind}(\matrix{\widetilde{A}})))\]
\[T_{ij} = \matrix{\widetilde{B}}(i.j), \forall (i,j) \in (\bold{ind}(\matrix{\widetilde{B}}) - (\bold{ind}(\matrix{\widetilde{B}}) \cap \bold{ind}(\matrix{\widetilde{A}})))\]
where the difference operator in the previous expressions refers to set difference.


The intermediate matrix $\matrix{\widetilde{Z}}$ is created as follows, using what is called a \emph{standard matrix accumulate}:
\begin{itemize}
    \item If ${\sf accum} = {\sf GrB\_NULL}$, then $\matrix{\widetilde{Z}} = \matrix{\widetilde{T}}$.

    \item If ${\sf accum}$ is a binary operator, then $\matrix{\widetilde{Z}}$ is defined as
        \[ \langle \bDout({\sf accum}), \bold{nrows}(\matrix{\widetilde{C}}), \bold{ncols}(\matrix{\widetilde{C}}),
        %\bold{L}(\matrix{\widetilde{Z}}) =
        \{(i,j,Z_{ij})  \forall (i,j) \in \bold{ind}(\matrix{\widetilde{C}}) \cup 
        \bold{ind}(\matrix{\widetilde{T}}) \} \rangle.\]

    The values of the elements of $\matrix{\widetilde{Z}}$ are computed based on the
    relationships between the sets of indices in $\matrix{\widetilde{C}}$ and 
    $\matrix{\widetilde{T}}$.
\[
    Z_{ij} = \matrix{\widetilde{C}}(i,j) \odot \matrix{\widetilde{T}}(i,j), \ \mbox{if}\  
    (i,j) \in  (\bold{ind}(\matrix{\widetilde{T}}) \cap \bold{ind}(\matrix{\widetilde{C}})),
\]
\[
    Z_{ij} = \matrix{\widetilde{C}}(i,j), \ \mbox{if}\  
    (i,j) \in (\bold{ind}(\matrix{\widetilde{C}}) - (\bold{ind}(\matrix{\widetilde{T}})
    \cap \bold{ind}(\matrix{\widetilde{C}}))),
\]
\[
    Z_{ij} = \matrix{\widetilde{T}}(i,j), \ \mbox{if}\  (i,j) \in  
    (\bold{ind}(\matrix{\widetilde{T}}) - (\bold{ind}(\matrix{\widetilde{T}})
    \cap \bold{ind}(\matrix{\widetilde{C}}))),
\]
where $\odot  = \bigodot({\sf accum})$, and the difference operator refers to set difference.
\end{itemize}


%\emph{Standard Matrix Mask and Replace Options}

Finally, the set of output values that make up the $\matrix{\widetilde{Z}}$ 
matrix are written into the final result matrix, {\sf C}. 
This is carried out under control of the mask which acts as a ``write mask''.
\begin{itemize}
\item If {\sf desc[GrB\_OUTP].GrB\_REPLACE} is set, then any values in {\sf C} on
input to this operation are deleted and the contents of the new output matrix,
{\sf C}, is defined as,
\[
\bold{L}({\sf C}) = \{(i,j,Z_{ij}) : (i,j) \in (\bold{ind}(\matrix{\widetilde{Z}}) 
\cap \bold{ind}(\matrix{\widetilde{M}})) \}. 
\]

\item If {\sf desc[GrB\_OUTP].GrB\_REPLACE} is not set, the elements of 
$\matrix{\widetilde{Z}}$ indicated by the mask are copied into the result 
matrix, {\sf C}, and elements of {\sf C} that fall outside the set 
indicated by the mask are unchanged:
\[
\bold{L}({\sf C}) = \{(i,j,C_{ij}) : (i,j) \in (\bold{ind}({\sf C}) 
\cap \bold{ind}(\neg \matrix{\widetilde{M}})) \} \cup \{(i,j,Z_{ij}) : (i,j) \in 
(\bold{ind}(\matrix{\widetilde{Z}}) \cap \bold{ind}(\matrix{\widetilde{M}})) \}.
\]
\end{itemize}

In {\sf GrB\_BLOCKING} mode, the method exits with return value 
{\sf GrB\_SUCCESS} and the new content of matrix {\sf C} is as defined above
and fully computed.
In {\sf GrB\_NONBLOCKING} mode, the method exits with return value 
{\sf GrB\_SUCCESS} and the new content of matrix {\sf C} is as defined above
but may not be fully computed; however, it can be used in the next GraphBLAS 
method call in a sequence.

