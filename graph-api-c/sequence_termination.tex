\section{Sequence Termination}

\subsection{{\sf wait}: Wait for pending operations to complete}

Waits for a collection of pending operations to complete. Two variants are supported, one that
waits on all pending operations and one that waits on pending operations with a particular output object.

%-----------------------------------------------------------------------------
\subsubsection{{\sf wait}: Waits until all pending operations complete variant}
\label{Sec:GrB_wait}

When running in non-blocking mode, this function guarantees that all pending GraphBLAS operations are fully executed.  Note that this can be called in blocking mode without an error, but there should be no pending GraphBLAS operations to complete.

\paragraph{\syntax}

\begin{verbatim}
        GrB_Info GrB_wait();
\end{verbatim}

\paragraph{Parameters}

\paragraph{Return values}

\begin{itemize}[leftmargin=2.3in]
	\item[{\sf GrB\_SUCCESS}]	operation completed successfully.
	\item[{\sf GrB\_INDEX\_OUT\_OF\_BOUNDS}]	an index out-of-bounds execution error happened during completion of pending operations.
	\item[{\sf GrB\_OUT\_OF\_MEMORY}]		and out-of-memory execution error happened during completion of pending operations.
	\item[{\sf GrB\_PANIC}]		unknown internal error.
\end{itemize}

\paragraph{Description}

Upon successful return, all previously called GraphBLAS methods have fully
completed their execution, and any (transparent or opaque) data structures
produced or manipulated by those methods can be safely touched.  If an
error occured in any pending GraphBLAS operations, {\sf GrB\_error()}
can be used to retrieve implementation defined error information about
the problem encountered.

%-----------------------------------------------------------------------------
\subsubsection{{\sf waitObj}: Waits until pending operations on a specific object complete variant}
\label{Sec:GrB_waitOne}

When running in non-blocking mode, this function guarantees that all
pending GraphBLAS operations that have a specific GraphBLAS object as
output are fully executed.  Note that this can be called in blocking mode
without an error, but there should be no pending GraphBLAS operations
to complete.

\paragraph{\syntax}

\begin{verbatim}
        GrB_Info GrB_waitObj(GrB_Object *obj);
\end{verbatim}

\paragraph{Parameters}

\begin{itemize}[leftmargin=1.1in]
        \item[{\sf obj}] ({\sf IN}) An existing GraphBLAS object.
        The object must have been created by an explicit call to a
        GraphBLAS constructor.  Can be any of the opaque GraphBLAS
        objects such as matrix, vector, descriptor, semiring, monoid,
        binary op, unary op, or type. On successful return of {\sf
        GrB\_waitObj}, all GraphBLAS operations that produce {\sf obj}
        as output have fully completed.
\end{itemize}

\paragraph{Return values}
\begin{itemize}[leftmargin=2.3in]
	\item[{\sf GrB\_SUCCESS}]			operation completed successfully.
	\item[{\sf GrB\_INDEX\_OUT\_OF\_BOUNDS}]	an index out-of-bounds execution error happened during completion of pending operations.
	\item[{\sf GrB\_OUT\_OF\_MEMORY}]		and out-of-memory execution error happened during completion of pending operations.
	\item[{\sf GrB\_UNINITIALIZED\_OBJECT}]		object has not been initialized by a call to the respective {\sf *\_new} method.
	\item[{\sf GrB\_PANIC}]				unknown internal error.
\end{itemize}

\paragraph{Description}

Upon successful return, all previously called GraphBLAS methods that
have {\sf obj} as an {\sf OUT} or {\sf INOUT} parameter have fully
completed their execution, and any (transparent or opaque) data structures
produced or manipulated by those methods can be safely touched.  If an
error occured in any of those GraphBLAS operations, {\sf GrB\_error()}
can be used to retrieve implementation defined error information about
the problem encountered.

In non-blocking mode, a call to {\sf GrB\_waitObj(obj)} does not necessarily
end the current GraphBLAS sequence. If there are other pending methods
in the sequence, producing other objects, there is no guarantee that
those methods have completed. Those methods can still produce errors
and/or consume execution time.

%-----------------------------------------------------------------------------
\subsection{{\sf error}: Get an error message regarding internal errors}

\begin{verbatim}
        const char *GrB_error();
\end{verbatim}

\paragraph{Parameters}

\paragraph{Return value}
\begin{itemize}[leftmargin=0.3in]
	\item[] A pointer to a null-terminated string (owned by the library).
\end{itemize}

\paragraph{Description}

\scott{Copied from Basic Concepts}

After a call to any GraphBLAS method, the program can retrieve additional
error information (beyond the error code returned by the method) though a
call to the function {\sf GrB\_error()}. 
The function returns a pointer to a null terminated string and the contents of that string
are implementation dependent. In particular, a null string (not a {\sf NULL} pointer) is always a valid error string.
The pointer is valid until the next call to any GraphBLAS method by the same thread.
{\sf GrB\_error()} is a thread-safe function, in the sense that multiple threads can
call it simultaneously and each will get its own error string back, referring to the
last GraphBLAS method it called.
