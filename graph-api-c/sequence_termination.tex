\section{Sequence Termination}

\scott{Can NULL be legally passed for err? Does the implementation manipulate the char pointer? Who allocates and who owns the error string upon return}

\subsection{{\sf wait}: Waits until pending operations complete}
\label{Sec:wait}

When running in non-blocking mode, this function guarantees that all pending GraphBLAS operations are fully executed.  Note that this can be called in blocking mode without an error, but there should be no pending GraphBLAS operations to complete.

\paragraph{\syntax}

\begin{verbatim}
        GrB_Info GrB_wait(char* err)
\end{verbatim}

\paragraph{Parameters}
\begin{itemize}[leftmargin=1.1in]
\item[{\sf err}] ({\sf INOUT})  A null-terminated string containing additional error
information.
\end{itemize}

\paragraph{Return values}
\begin{itemize}[leftmargin=2.1in]
\item[{\sf GrB\_SUCCESS}]	operation completed successfully.
\item[{\sf GrB\_PANIC}]		unknown internal error; more information about the error may be found in the {\sf err} string.
\end{itemize}

\paragraph{Description}

Upon successful return, all previously called GraphBLAS methods have fully completed their execution, and 
any (transparent or opaque) data structures produced or manipulated by those methods can be safely touched.  If an error occured in any pending GraphBLAS operations, {\sf GrB\_PANIC} is returned and the {\sf err} string will contain implementation defined error information about the problem encountered.

%-----------------------------------------------------------------------------
\comment{
\subsubsection{Matrix variant}

\paragraph{\syntax}

\begin{verbatim}
        GrB_Info GrB_wait(GrB_Matrix A, char *err)
\end{verbatim}

\paragraph{Parameters}

\begin{itemize}[leftmargin=1.1in]
        \item[{\sf A}]   ({\sf IN})     An existing GraphBLAS matrix.
        \item[{\sf err}] ({\sf INOUT})  A null terminated string containing additional error
information.
\end{itemize}

\paragraph{Return values}
\begin{itemize}[leftmargin=2.1in]
\item[{\sf GrB\_SUCCESS}]	operation completed successfully.
\item[{\sf GrB\_PANIC}]		unknown internal error; more information about the error may be found in the {\sf err} string.
\item[{\sf GrB\_UNINITIALIZED\_OBJECT}]  matrix has not been initialized by a
                            call to {\sf new}.
\end{itemize}

\paragraph{Description}

Upon return, all previously called GraphBLAS methods that used {\sf A} either as 
input or output have fully completed their execution.
Any (transparent or opaque) data structures produced or manipulated by those 
methods can be safely touched.

%-----------------------------------------------------------------------------

\subsubsection{Vector variant}

\paragraph{\syntax}

\begin{verbatim}
        GrB_Info GrB_wait(GrB_Vector v, char *err)
\end{verbatim}

\paragraph{Parameters}

\begin{itemize}[leftmargin=1.1in]
	    \item[{\sf v}]   ({\sf IN})     An existing GraphBLAS vector.
        \item[{\sf err}] ({\sf INOUT})  A null terminated string containing additional error
information.
\end{itemize}

\paragraph{Return values}
\begin{itemize}[leftmargin=2.1in]
\item[{\sf GrB\_SUCCESS}]	operation completed successfully.
\item[{\sf GrB\_PANIC}]		unknown internal error; more information about the error may be found in the {\sf err} string.
\item[{\sf GrB\_UNINITIALIZED\_OBJECT}]  vector has not been initialized by a
                            call to {\sf new}.
\end{itemize}

\paragraph{Description}

Upon return, all previously called GraphBLAS methods that used {\sf v} either as
input or output have fully completed their execution.  Any (transparent or opaque)
data structures produced or manipulated by those methods can be safely touched.
}
