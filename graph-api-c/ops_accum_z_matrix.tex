
The intermediate matrix $\matrix{\widetilde{Z}}$ is created as follows, using what is called a \emph{standard matrix accumulate}:
\begin{itemize}
    \item If ${\sf accum} = {\sf GrB\_NULL}$, then $\matrix{\widetilde{Z}} = \matrix{\widetilde{T}}$.

    \item If ${\sf accum}$ is a binary operator, then $\matrix{\widetilde{Z}}$ is defined as
        \[ \langle \bDout({\sf accum}), \bold{nrows}(\matrix{\widetilde{C}}), \bold{ncols}(\matrix{\widetilde{C}}),
        %\bold{L}(\matrix{\widetilde{Z}}) =
        \{(i,j,Z_{ij})  \forall (i,j) \in \bold{ind}(\matrix{\widetilde{C}}) \cup 
        \bold{ind}(\matrix{\widetilde{T}}) \} \rangle.\]

    The values of the elements of $\matrix{\widetilde{Z}}$ are computed based on the
    relationships between the sets of indices in $\matrix{\widetilde{C}}$ and 
    $\matrix{\widetilde{T}}$.
\[
    Z_{ij} = \matrix{\widetilde{C}}(i,j) \odot \matrix{\widetilde{T}}(i,j), \ \mbox{if}\  
    (i,j) \in  (\bold{ind}(\matrix{\widetilde{T}}) \cap \bold{ind}(\matrix{\widetilde{C}})),
\]
\[
    Z_{ij} = \matrix{\widetilde{C}}(i,j), \ \mbox{if}\  
    (i,j) \in (\bold{ind}(\matrix{\widetilde{C}}) - (\bold{ind}(\matrix{\widetilde{T}})
    \cap \bold{ind}(\matrix{\widetilde{C}}))),
\]
\[
    Z_{ij} = \matrix{\widetilde{T}}(i,j), \ \mbox{if}\  (i,j) \in  
    (\bold{ind}(\matrix{\widetilde{T}}) - (\bold{ind}(\matrix{\widetilde{T}})
    \cap \bold{ind}(\matrix{\widetilde{C}}))),
\]
where $\odot  = \bigodot({\sf accum})$, and the difference operator refers to set difference.
\end{itemize}
