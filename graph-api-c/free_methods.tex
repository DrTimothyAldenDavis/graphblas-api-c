\subsection{{\sf free} method}

Destroys a previously created GraphBLAS object and releases any resources associated with the object.

\paragraph{\syntax}

\begin{verbatim}
        GrB_Info GrB_free(GrB_Object *obj);
\end{verbatim}


\paragraph{Parameters}

\begin{itemize}[leftmargin=1.1in]
	\item[{\sf obj}] ({\sf INOUT}) An existing GraphBLAS object to be destroyed. 
		The object must have been created by an explicit call to a GraphBLAS constructor.
    Can be any of the opaque GraphBLAS objects such as matrix, vector, descriptor, semiring, monoid, binary op, 
		unary op, or type. On successful completion of {\sf GrB\_free}, {\sf obj} behaves as an uninitialized object.
\end{itemize}

\paragraph{Return Values}

\begin{itemize}[leftmargin=2.1in]
\item[{\sf GrB\_SUCCESS}]        operation completed successfully
\item[{\sf GrB\_PANIC}]          unknown internal error.  If this return
value is encountered when in nonblocking mode, the error responsible for
the panic condition could be from any method involved in the computation
of the input object.  The {\sf GrB\_error()} method should be called
for additional information.
%\item[{\sf GrB\_UNINITIALIZED\_OBJECT}]       object has not been initialized by a call 
%                                 to the respective {\sf *\_new} method.
\end{itemize}

\paragraph{Description}

GraphBLAS objects consume memory and other resources managed by the
GraphBLAS runtime system. A call to {\sf GrB\_free} frees those resources
so they are available for use by other GraphBLAS objects.

The parameter passed into {\sf GrB\_free} is a handle referencing a
GraphBLAS opaque object of a data type from table~\ref{Tab:ObjTypes}.
The object must have been created by an explicit call to a GraphBLAS
constructor.  The behavior of a program that calls {\sf GrB\_free}
on a pre-defined object is implementation defined.

After the {\sf GrB\_free} method returns, the object referenced
by the input handle is destroyed and the handle has the value {\sf
GrB\_INVALID\_HANDLE}.  The handle can be used in subsequent GraphBLAS
methods but only after the handle has been reinitialized with a call
the the appropriate {\sf \_new} or {\sf \_dup} method.

Note that unlike other GraphBLAS methods, calling {\sf GrB\_free} with
an object with an invalid handle is legal.  The system may attempt to
free resources that might be associated with that object, if possible,
and return normally.

When using {\sf GrB\_free} it is possible to create a dangling reference
to an object.  This would occur when a handle is assigned to a second
variable of the same opaque type.  This creates two handles that reference
the same object. If {\sf GrB\_free} is called with one of the variables,
the object is destroyed and the handle associated with the other variable
no longer references a valid object.  This is not an error condition
that the implementation of the GraphBLAS API can be expected to catch,
hence programmers must take care to prevent this situation from occurring.
