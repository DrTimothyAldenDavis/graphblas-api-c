\pagebreak
\nolinenumbers
\section{Example: breadth first search with GraphBLAS}
{\scriptsize
\lstinputlisting[language=C,numbers=left]{BFS5M.c}
}

\pagebreak
\nolinenumbers
\section{Example: betweenness centrality with GraphBLAS}
{\scriptsize
\lstinputlisting[language=C,numbers=left]{BC1M.c}
}

\comment{
\pagebreak
\linenumbers
\section{Algebraless option}

\subsection{Vector-matrix multiply ({\sf vxm})}

Multiplies a vector by a matrix. The result is a vector.

\paragraph{C99 Syntax}

\begin{verbatim}
#include "GraphBLAS.h"
GrB_info GrB_vxm(GrB_Vector *u, const GrB_operation f, const GrB_operation g,
         const GrB_Vector v, const GrB_Matrix A
         [, const GrB_Vector m[, const GrB_Descriptor d]])
\end{verbatim}

\paragraph{Input Parameters}

\begin{itemize}
	\item[{\sf f}] ({\sf ARG0}) Additive operation used in the vector-matrix
	multiply.

	\item[{\sf g}] ({\sf ARG1}) Multiplicative oepration useind in the vector-matrix multiply.

	\item[{\sf v}] ({\sf ARG2}) Vector to be multiplied.

	\item[{\sf A}] ({\sf ARG3}) Matrix to be multiplied.

	\item[{\sf m}] ({\sf MASK}) Operation mask (optional). The mask
	specifies which elements of the result vector are to be computed.
	If no mask is necessary (i.e., compute all elements of result
	vector), {\sf GrB\_NULL} can be used or the mask can be omitted.

	\item[{\sf d}] Operation descriptor (optional). The descriptor
	is used to specify details of the operation, such as transpose
	the matrix or not, invert the mask or not (see below). If a
	\emph{default} descriptor is desired, {\sf GrB\_NULL} can be
	used or the descriptor can be omitted.
\end{itemize}

\paragraph{Output Parameter}

\begin{itemize}
	\item[{\sf u}] ({\sf OUTP}) Address of result vector.
\end{itemize}

\paragraph{Return Value}

\begin{tabular}{rl} 
{\sf GrB\_SUCCESS} 	& operation completed successfully \\
{\sf GrB\_PANIC}	& unknown internal error \\
{\sf GrB\_OUTOFMEM}	& not enough memory available for operation \\
{\sf GrB\_MISMATCH}	& mismatch among vectors, matrix and/or algebra
\end{tabular}

\paragraph{Description}

Let $\otimes: D_1 \times D_2 \rightarrow D_3$ and $\oplus: D_3 \times
D_3 \rightarrow D_3$ be the operations defined by arguments {\sf f}
and {\sf g}, respectively.

Vectors $\vector{v}, \vector{m}$ and matrix $\matrix{A}$ are computed
from input parameters {\sf v}, {\sf m} and {\sf A}, respectively,
as specified by descriptor {\sf d}.  $\bold{D}(\vector{v}) = D_1$ and
$\bold{D}(\matrix{A}) = D_2$.  $\bold{D}(\vector{m}) = {\sf GrB\_BOOL}$.
If {\sf m} is {\sf GrB\_NULL} then $\vector{m}$ is a Boolean vector of
size $\bold{n}(\vector{A})$ and with all elements set to {\sf true}.

If either $\vector{v}, \vector{m}$ or $\matrix{A}$ cannot be computed
from the input parameters as described above, the method returns {\sf
GrB\_MISMATCH}.

A consistency check is performed to verify that $\bold{n}(\vector{v})
= \bold{m}(\matrix{A})$ and $\bold{n}(\vector{m}) =
\bold{n}(\matrix{A})$. If a consistency check fails, the operation is
aborted and the method returns {\sf GrB\_MISMATCH}.

A new vector $\vector{u} = \langle D_3, \bold{n}(\matrix{A}),
\bold{L}(\vector{u}) = \{(i,u_i) : \vector{m}(i) = {\sf true} \} \rangle$
is created.  The value of each of its elements is computed by $u_i =
\bigoplus_{j \in \vector{i}(\vector{v}) \cap \vector{i}(\matrix{A}(:,i))}
(\vector{v}(j) \otimes \matrix{A}(j,i))$.  If $\vector{i}(\vector{v})
\cap \vector{i}(\matrix{A}(:,i)) = \emptyset$ then the pair $(i,u_i)$
is not included in $\bold{L}(\vector{u})$.

Finally, output parameter {\sf u} is computed from vector $\vector{u}$
as specified by descriptor {\sf d}.  A consistency check is performed
to verify that $\bold{n}({\sf u}) = \bold{n}(\vector{u})$. If the
consistency check fails, the operation is aborted and the method return
{\sf GrB\_MISMATCH}.

\jose{The domains are now defined by the oeprations. Which means we need
different versions of common operations such as {\sf GrB\_PLUS}:
{\sf GrB\_PLUSI64} (for {\tt int64\_t}), {\sf GrB\_PLUSU64} (for {\tt uint64\_t}), {\sf GrB\_PLUSF64} (for {\tt double}),
{\sf GrB\_PLUSI32} (for {\tt int32\_t}), {\sf GrB\_PLUSU32} (for {\tt uint32\_t}), {\sf GrB\_PLUSF32} (for {\tt float}),
}

\pagebreak
\nolinenumbers
\section{Example: BFS with Algebraless-GraphBLAS}
{\scriptsize
\lstinputlisting[language=C,numbers=left]{BFS5AL.c}
}

\pagebreak
\nolinenumbers
\section{Example: BC with Algebraless-GraphBLAS}
{\scriptsize
\lstinputlisting[language=C,numbers=left]{BC1AL.c}
}

\pagebreak
\linenumbers
\section{Monoid option}

A GraphBLAS \emph{monoid} $M = \langle D_1,D_2,D_3,\oplus,0 \rangle$ is
defined by three domains $D_1$, $D_2$, $D_3$, an operation $\oplus:
D_1 \times D_2 \rightarrow D_3$, and an identity $0 \in D_1 \cap D_2$.
It is required that $D_1 \subseteq D_3$ and $D_2 \subseteq D_3$. For a given GraphBLAS monoid $M=\langle
D_1, D_2, D_3,\oplus,0 \rangle$ we define $\bold{D}_1(M) =
D_1$, $\bold{D}_2(M) = D_2$, $\bold{D}_3(M) = D_3$, $\bold{\bigoplus}(M)
= \oplus$ and $\bold{0}(S) = 0$.

\subsection{Element-wise multiplication ({\sf ewisemult})}

\subsubsection{Vector variant}

Perform element-wise (general) multiplication on the elements of two vectors,
producing a third vector as result.

\paragraph{C99 Syntax}

\begin{verbatim}
#include "GraphBLAS.h"
GrB_info GrB_ewisemult(GrB_Vector* w, const GrB_Monoid s, const GrB_Vector u,
         const GrB_Vector v[, const GrB_Vector m[, const GrB_Descriptor d]])
\end{verbatim}

\paragraph{Input Parameters}

\begin{itemize}
	\item[{\sf s}] ({\sf ARG0}) Monoid used in the vector-wise multiplication.

	\item[{\sf u}] ({\sf ARG1}) Left vector to be multiplied.

	\item[{\sf v}] ({\sf ARG2}) Right vector to be multiplied.

	\item[{\sf m}] ({\sf MASK}) Operation mask (optional). The mask
	specifies which elements of the result vector are to be computed.
	If no mask is necessary (i.e., compute all elements of result
	vector), {\sf GrB\_NULL} can be used or the mask can be omitted.

	\item[{\sf d}] Operation descriptor (optional). The descriptor
	is used to specify details of the operation, such as 
	invert the mask or not (see below). If a
	\emph{default} descriptor is desired, {\sf GrB\_NULL} can be
	used or the descriptor can be omitted.
\end{itemize}

\paragraph{Output Parameter}

\begin{itemize}
	\item[{\sf w}] ({\sf OUTP}) Address of result vector.
\end{itemize}

\paragraph{Return Value}

\begin{tabular}{rl} 
{\sf GrB\_SUCCESS} 	& operation completed successfully \\
{\sf GrB\_PANIC}	& unknown internal error \\
{\sf GrB\_OUTOFMEM}	& not enough memory available for operation \\
{\sf GrB\_MISMATCH}	& mismatch among vectors and/or algebra
\end{tabular}

\paragraph{Description}

Vectors $\vector{v}, \vector{m}$ and $\vector{u}$ are computed from
input parameters {\sf v}, {\sf m} and {\sf u}, respectively, as specified
by descriptor {\sf d}. (See below for the properties of a descriptor. In
the simplest form, these are just copies, but additional preprocessing,
including casting, can be specified.)  $\bold{D}(\vector{u}) =
\bold{D}_1({\sf s})$ and $\bold{D}(\vector{v}) = \bold{D}_2({\sf s})$.
$\bold{D}(\vector{m}) = {\sf GrB\_BOOL}$.  If {\sf m} is {\sf GrB\_NULL} or omitted,
then $\vector{m}$ is a Boolean vector of size $\bold{n}(\vector{u})$
and with all elements set to {\sf true}.

If either $\vector{v}, \vector{m}$ or $\vector{u}$ cannot be computed
from the input parameters as described above, the method returns {\sf
GrB\_MISMATCH}.

A consistency check is performed to verify that $\bold{n}(\vector{v})
= \bold{n}(\vector{u}) = \bold{n}(\vector{m})$. If a consistency
check fails, the operation is aborted and the method returns {\sf
GrB\_MISMATCH}.

A new vector $\vector{w} = \langle \bold{D}_3({\sf s}),
\bold{n}(\vector{u}), \bold{L}(\vector{w}) = \{(i,w_i)  \forall i \in
\vector{i}(\vector{v}) \cap \vector{i}(\vector{u}) : \vector{m}(i)
= {\sf true} \} \rangle$ is created.  The value of each of its
elements is computed by $w_i = \vector{u}(i) \otimes \vector{v}(i)$,
where $\otimes$ is the multiplicative operation of algebra {\sf s}.
If $\vector{i}(\vector{v}) \cap \vector{i}(\vector{u}) = \emptyset$
then $\bold{L}(\vector{w}) = \emptyset$.

Finally, output parameter {\sf w} is computed from vector $\vector{w}$
as specified by descriptor {\sf d}. (Again, in the simplest case this
is just a copy, but additional postprocessing, including casting and
accumulation of result values, can be specified.)  A consistency check is
performed to verify that $\bold{n}({\sf w}) = \bold{n}(\vector{w})$. If
the consistency check fails, the operation is aborted and the method
return {\sf GrB\_MISMATCH}.

\subsection{Perform a reduction across the elements of an object ({\sf reduce})}

Computes the reduction of the values of the elements of a vector or matrix.

\subsubsection{Vector variant}

\paragraph{C99 Syntax}

\begin{verbatim}
#include "GraphBLAS.h"
GrB_info GrB_reduce(scalar *t, const GrB_Monoid s, const GrB_Vector v)
\end{verbatim}

\paragraph{Input Parameters}

\begin{itemize}
	\item[{\sf v}] Vector to be reduced.
	\item[{\sf s}] Monoid defining the reduction.
\end{itemize}

\paragraph{Output Parameters}

\begin{itemize}
	\item[{\sf t}] Value of the reduction. It must
	be a pointer to one of the types in 
	the left column of Table~\ref{Tab:PredefinedTypes} or
	{\tt void*}.
\end{itemize}

\paragraph{Return Value}

\begin{tabular}{rl}
{\sf GrB\_SUCCESS}	& operation completed successfully \\
{\sf GrB\_PANIC}	& unknown internal error \\
{\sf GrB\_NOVECTOR}	& vector does not exist \\
{\sf GrB\_MISMATCH}	& mismatch between vector domain and scalar type \\
\end{tabular}

\pagebreak
\nolinenumbers
\section{Example: BC with Monoid option}
{\scriptsize
\lstinputlisting[language=C,numbers=left]{BC1M.c}
}
}

\pagebreak