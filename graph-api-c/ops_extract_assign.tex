\subsubsection{{\sf extract}: Selecting Sub-Graphs}

Extract a sub-matrix from a larger matrix. 

%-----------------------------------------------------------------------------
\paragraph{Standard Matrix and Vector Versions}

Describe behaviour with or without accum.  Can you create structural zero's through assignment (no accumulate)

\subparagraph{C99 Syntax}

\begin{verbatim}
GrB_info GrB_extract(GrB_Vector *dst, const GrB_Function accum, const GrB_Vector src,
                     const GrB_IndexArray i[, const GrB_Vector m[, const GrB_Descriptor d]])
GrB_info GrB_extract(GrB_Matrix *dst, const GrB_Function s, const GrB_Matrix src,
                     const GrB_IndexArray i, const GrB_IndexArray j
                     [, const GrB_Vector m[, const GrB_Descriptor d]])
\end{verbatim}

\subparagraph{Input Parameters}

\begin{itemize}
	\item[{\sf dst}]   ({\sf ARG0}) The matrix or vector to assign the extracted subgraph.
	\item[{\sf accum}] ({\sf ARG1}) Function used for accumulation into dst.  {\sf GrB\_NULL}
                       can be used if no accumulation into dst is used.
	\item[{\sf src}]   ({\sf ARG2}) The matrix or vector from which to extract the subgraph.
	\item[{\sf i}]     ({\sf ARG3}) The set of row indices specifying locations from src that
                       are assigned to dst.
	\item[{\sf j}]     ({\sf ARG4}) (Matrix version only) The set of column indices specifying
                       locations from src that are assigned to dst.

	\item[{\sf m}]     ({\sf MASK}) Operation mask (optional). The mask
	specifies which elements of the result vector can be assigned.
	If no mask is necessary (i.e., compute all elements of result
	vector), {\sf GrB\_NULL} can be used or the mask can be omitted.

	\item[{\sf d}] Operation descriptor (optional). The descriptor
    is used to specify details of the operation. Valid options are transpose
    of src (ARG2), and invert (structural complement) of src (ARG2). If
    a \emph{default} descriptor is desired,	{\sf GrB\_NULL} can be
    used or the descriptor can be omitted.
\end{itemize}

\subparagraph{Return Value}

\scott{Are invalid/unused descriptors an error?}

\begin{tabular}{rl} 
{\sf GrB\_SUCCESS} 	& operation completed successfully \\
{\sf GrB\_PANIC}	& unknown internal error \\
{\sf GrB\_OUTOFMEM}	& not enough memory available for operation \\
{\sf GrB\_DIMENSION\_MISMATCH} & 
       The size of i is greater than the number of rows in src, or
       the size of j is greater than the number of columns in src (matrix version) \\
{\sf GrB\_INDEX\_OUTOFBOUNDS} &
       A value in i references a nonexistent row in dst, or
	   the value in j references a nonexistent column in dst (matrix version).\\
{\sf GrB\_MISMATCH} & 
	   mismatch among vectors, matrix and/or semiring
\end{tabular}


%-----------------------------------------------------------------------------
\paragraph{Flat Variant}
\subparagraph{C99 Syntax}
\subparagraph{Input Parameters}
\subparagraph{Return Value}

%-----------------------------------------------------------------------------
\paragraph{Indexed Variant}
\subparagraph{C99 Syntax}
\subparagraph{Input Parameters}
\subparagraph{Return Value}

%-----------------------------------------------------------------------------
%-----------------------------------------------------------------------------
\subsubsection{{\sf assign}: Modifying Sub-Graphs}

Assign a matrix to a set of indices (sub-matrix) of a larger matrix

\scott{the variants beyond the standard version need to be discussed perhaps in the large group; not currently part of any prior document.}

%-----------------------------------------------------------------------------
\paragraph{Standard Matrix and Vector Versions}

\subparagraph{C99 Syntax}

\begin{verbatim}
GrB_info GrB_assign(GrB_Vector *dst, const GrB_Function accum, const GrB_Vector src,
                    const GrB_IndexArray i[, const GrB_Vector m[, const GrB_Descriptor d]])
GrB_info GrB_assign(GrB_Matrix *dst, const GrB_Function s, const GrB_Matrix src,
                    const GrB_IndexArray i, const GrB_IndexArray j
                    [, const GrB_Vector m[, const GrB_Descriptor d]])
\end{verbatim}

\subparagraph{Input Parameters}

\begin{itemize}
	\item[{\sf dst}]   ({\sf ARG0}) The matrix or vector into which to assign the subgraph.
	\item[{\sf accum}] ({\sf ARG1}) Function used for accumulation into dst.  {\sf GrB\_NULL}
                       can be used if no accumulation into dst is desired.
	\item[{\sf src}]   ({\sf ARG2}) The matrix or vector containing the subgraph.
	\item[{\sf i}]     ({\sf ARG3}) The set of row indices specifying locations in dst that
                       are assigned from src.
	\item[{\sf j}]     ({\sf ARG4}) (Matrix version only) The set of column indices specifying
                       locations in dst that are assigned from src.

	\item[{\sf m}]     ({\sf MASK}) Operation mask (optional). The mask
	specifies which elements of the result vector can be assigned.
	If no mask is necessary (i.e., compute all elements of result
	vector), {\sf GrB\_NULL} can be used or the mask can be omitted.

	\item[{\sf d}] Operation descriptor (optional). The descriptor
    is used to specify details of the operation. Valid options are transpose
    of src (ARG2), and invert (structural complement) of src (ARG2). If
    a \emph{default} descriptor is desired,	{\sf GrB\_NULL} can be
    used or the descriptor can be omitted.
\end{itemize}

\subparagraph{Return Value}

Are invalid/unused descriptors an error?

%-----------------------------------------------------------------------------
\paragraph{Flat variant}

Set a subset of elements of a vector to a specified constant value.

\begin{verbatim}
#include "GraphBLAS.h"
GrB_info GrB_assign(GrB_Vector *v,scalar s[, const GrB_Vector m])
\end{verbatim}

\subparagraph{Input Parameters}

\begin{itemize}
	\item[{\sf v}] Vector to be assigned.
	\item[{\sf s}] Scalar value for the elements.
	\item[{\sf m}] (Optional) mask for assignment. \aydin{Maybe say in the document that GrB\_Vector's domain could only be GrB\_Index for this function} \jose{Any domain that can be cast to {\sf GrB\_BOOL} will do.}
\end{itemize}

\subparagraph{Return Value}

\begin{tabular}{rl}
{\sf GrB\_SUCCESS}	& operation completed successfully \\
{\sf GrB\_PANIC}	& unknown internal error \\
{\sf GrB\_NOVECTOR}	& vector does not exist \\
{\sf GrB\_MISMATCH}	& mismatch between vector domain and scalar type \\
\end{tabular}

%-----------------------------------------------------------------------------
\paragraph{Indexed variant}

Set some of the elements of a vector to a given value.
\scott{Set one element of...}

\subparagraph{C99 Syntax}

\begin{verbatim}
#include "GraphBLAS.h"
GrB_info GrB_assign(GrB_Vector *v,scalar s,GrB_index i)
\end{verbatim}

\subparagraph{Input Parameters}

\begin{itemize}
	\item[{\sf v}] Vector to be assigned.
	\item[{\sf s}] Scalar value for the elements.
	\item[{\sf i}] Index of element to be assigned
\end{itemize}

\subparagraph{Return Value}

\begin{tabular}{rl}
{\sf GrB\_SUCCESS}	& operation completed successfully \\
{\sf GrB\_PANIC}	& unknown internal error \\
{\sf GrB\_NOVECTOR}	& vector does not exist \\
{\sf GrB\_MISMATCH}	& mismatch between vector domain and scalar type \\
\end{tabular}
