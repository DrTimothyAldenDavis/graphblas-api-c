
The intermediate vector $\vector{\widetilde{z}}$ is created as follows, using what is called a \emph{standard vector accumulate}:
\begin{itemize}
    \item If ${\sf accum} = {\sf GrB\_NULL}$, then $\vector{\widetilde{z}} = \vector{\widetilde{t}}$.

    \item If ${\sf accum}$ is a binary operator, then $\vector{\widetilde{z}}$ is defined as
        \[ \vector{\widetilde{z}} = \langle \bDout({\sf accum}), \bold{size}(\vector{\widetilde{w}}),
        %\bold{L}(\vector{\widetilde{z}}) =
        \{(i,z_{i}) \ \forall \ i \in \bold{ind}(\vector{\widetilde{w}}) \cup 
        \bold{ind}(\vector{\widetilde{t}}) \} \rangle.\]

    The values of the elements of $\vector{\widetilde{z}}$ are computed based on the 
    relationships between the sets of indices in $\vector{\widetilde{w}}$ and 
    $\vector{\widetilde{t}}$.
\[
    z_{i} = \vector{\widetilde{w}}(i) \odot \vector{\widetilde{t}}(i), \ \mbox{if}\  
    i \in  (\bold{ind}(\vector{\widetilde{t}}) \cap \bold{ind}(\vector{\widetilde{w}})),
\]
\[
    z_{i} = \vector{\widetilde{w}}(i), \ \mbox{if}\  
    i \in (\bold{ind}(\vector{\widetilde{w}}) - (\bold{ind}(\vector{\widetilde{t}})
    \cap \bold{ind}(\vector{\widetilde{w}}))),
\]
\[
    z_{i} = \vector{\widetilde{t}}(i), \ \mbox{if}\  i \in  
    (\bold{ind}(\vector{\widetilde{t}}) - (\bold{ind}(\vector{\widetilde{t}})
    \cap \bold{ind}(\vector{\widetilde{w}}))),
\]
where $\odot  = \bigodot({\sf accum})$, and the difference operator refers to set difference.
\end{itemize}

