\vfill
\pagebreak

%-----------------------------------------------------------------------------
\scott{Content related to Scalar for earlier parts of the spec}
%-----------------------------------------------------------------------------

\subsection*{2.4 Opaque Objects}

\scott{Scalars need to be added in various places with vector and matrix and to Table 2.1.}


%-----------------------------------------------------------------------------
\section*{3.x Scalars (section before Vectors)}
\label{Sec:Scalars}

\scott{Jose: I need help with this section}

A scalar $\scalar{s} = \langle D, \{ s_0 \} \rangle$ is defined by a domain $D$ and 
a set of zero or one elements, where $s_0 \in D$. We define $\mathbf{size}(\scalar{s})$ 
to be constant one, and
$\mathbf{L}(\scalar{s}) = \{ s_0 \}$. The set $\mathbf{L}(\scalar{s})$ is
called the \emph{contents} of scalar $\scalar{s}$. We also define $\mathbf{D}(\scalar{d})
= D$. For a scalar $\scalar{s}$, $\scalar{s}(0)$ is a reference to $s_0$
if $\mathbf{L}(\scalar{s})$ is not empty, and is undefined otherwise.

\scott{Do we need a val(s) which gives ues a reference to the value stored in 
s or should we somehow use a zero index?}

%-----------------------------------------------------------------------------
%-----------------------------------------------------------------------------
\subsection{Scalar Methods}


%-----------------------------------------------------------------------------
\subsubsection{{\sf Scalar\_new}: Create a new scalar}

Creates a new empty scalar with specified domain.

\paragraph{\syntax}

\begin{verbatim}
        GrB_Info GrB_Scalar_new(GrB_Scalar *s,
                                GrB_Type    d);
\end{verbatim}

\paragraph{Parameters}

\begin{itemize}[leftmargin=1.1in]
    \item[{\sf s}] ({\sf INOUT}) On successful return, contains a handle
                                 to the newly created GraphBLAS scalar.
    \item[{\sf d}] ({\sf IN})    The type corresponding to the domain of the 
                                 scalar being created.  Can be one of the 
                                 predefined GraphBLAS types in 
                                 Table~\ref{Tab:PredefinedTypes}, or an existing 
                                 user-defined GraphBLAS type.
\end{itemize}

\paragraph{Return Values}

\begin{itemize}[leftmargin=2.1in]
    \item[{\sf GrB\_SUCCESS}]         In blocking mode, the operation completed
    successfully. In non-blocking mode, this indicates that the API checks 
    for the input arguments passed successfully. Either way, output scalar 
    {\sf s} is ready to be used in the next method of the sequence.

    \item[{\sf GrB\_PANIC}]           Unknown internal error.
    
    \item[{\sf GrB\_INVALID\_OBJECT}] This is returned in any execution mode 
    whenever one of the opaque GraphBLAS objects (input or output) is in an invalid 
    state caused by a previous execution error.  Call {\sf GrB\_error()} to access 
    any error messages generated by the implementation.

    \item[{\sf GrB\_OUT\_OF\_MEMORY}] Not enough memory available for operation.
    
    \item[{\sf GrB\_UNINITIALIZED\_OBJECT}]  The {\sf GrB\_Type} object has not 
    been initialized by a call to {\sf GrB\_Type\_new} (needed for user-defined types).
    
    \item[{\sf GrB\_NULL\_POINTER}]  The {\sf s} pointer is {\sf NULL}.
\end{itemize}

\paragraph{Description}

Creates a new scalar $\scalar{s}$ of domain $\mathbf{D}({\sf d})$ and empty 
$\mathbf{L}(\scalar{s})$ (size is one). The method returns a handle to the new scalar in {\sf s}.

It is not an error to call this method more than once on the same variable;  
however, the handle to the previously created object will be overwritten. 

%-----------------------------------------------------------------------------
\subsubsection{{\sf Scalar\_dup}: Create a copy of a GraphBLAS scalar}

Creates a new scalar with the same domain and contents as another scalar.

\paragraph{\syntax}

\begin{verbatim}
        GrB_Info GrB_Scalar_dup(GrB_Scalar       *t,
                                const GrB_Scalar  s);
\end{verbatim}

\paragraph{Parameters}

\begin{itemize}[leftmargin=1.1in]
    \item[{\sf t}]  ({\sf INOUT}) On successful return, contains a handle
                                  to the newly created GraphBLAS scalar.
    \item[{\sf s}]  ({\sf IN})    The GraphBLAS scalar to be duplicated.
\end{itemize}

\paragraph{Return Values}

\begin{itemize}[leftmargin=2.1in]
    \item[{\sf GrB\_SUCCESS}]         In blocking mode, the operation completed
    successfully. In non-blocking mode, this indicates that the API checks 
    for the input arguments passed successfully. Either way, output scalar 
    {\sf t} is ready to be used in the next method of the sequence.

    \item[{\sf GrB\_PANIC}]           Unknown internal error.
    
    \item[{\sf GrB\_INVALID\_OBJECT}] This is returned in any execution mode 
    whenever one of the opaque GraphBLAS objects (input or output) is in an invalid 
    state caused by a previous execution error.  Call {\sf GrB\_error()} to access 
    any error messages generated by the implementation.

    \item[{\sf GrB\_OUT\_OF\_MEMORY}] Not enough memory available for operation.
    
    \item[{\sf GrB\_UNINITIALIZED\_OBJECT}]  The GraphBLAS scalar, {\sf s}, has 
    not been initialized by a call to {\sf Scalar\_new} or {\sf Scalar\_dup}.
    
    \item[{\sf GrB\_NULL\_POINTER}]  The {\sf t} pointer is {\sf NULL}.
\end{itemize}

\paragraph{Description}

Creates a new scalar $\scalar{t}$ of domain $\mathbf{D}({\sf s})$ and contents 
$\mathbf{L}({\sf s})$ (size is one). The method returns a handle to the new scalar in {\sf t}.

It is not an error to call this method more than once with the same output variable;  
however, the handle to the previously created object will be overwritten. 


%-----------------------------------------------------------------------------
\subsubsection{{\sf Scalar\_clear}: Clear/remove a stored value from a scalar}

Removes the stored value from a scalar.

\paragraph{\syntax}

\begin{verbatim}
        GrB_Info GrB_Scalar_clear(GrB_Scalar s);
\end{verbatim}

\paragraph{Parameters}

\begin{itemize}[leftmargin=1.1in]
    \item[{\sf s}] ({\sf INOUT}) An existing GraphBLAS scalar to clear.
\end{itemize}

\paragraph{Return Values}

\begin{itemize}[leftmargin=2.1in]
    \item[{\sf GrB\_SUCCESS}]         In blocking mode, the operation completed
    successfully. In non-blocking mode, this indicates that the API checks 
    for the input arguments passed successfully. Either way, output scalar 
    {\sf s} is ready to be used in the next method of the sequence.

    \item[{\sf GrB\_PANIC}]           Unknown internal error.
    
    \item[{\sf GrB\_INVALID\_OBJECT}] This is returned in any execution mode 
    whenever one of the opaque GraphBLAS objects (input or output) is in an invalid 
    state caused by a previous execution error.  Call {\sf GrB\_error()} to access 
    any error messages generated by the implementation.

    \item[{\sf GrB\_OUT\_OF\_MEMORY}] Not enough memory available for operation.
    
    \item[{\sf GrB\_UNINITIALIZED\_OBJECT}]  The GraphBLAS scalar, {\sf s}, has 
    not been initialized by a call to {\sf Scalar\_new} or {\sf Scalar\_dup}.
    
\end{itemize}

\paragraph{Description}

Removes the stored value from an existing scalar. After the call to
{\sf GrB\_Scalar\_clear(v)}, 
$\mathbf{L}(\scalar{v}) = \emptyset$. The size of the scalar does not change. 


%-----------------------------------------------------------------------------
\subsubsection{{\sf Scalar\_nvals}: Number of stored elements in a scalar}
\label{Sec:Scalar_nvals}

Retrieve the number of stored elements in a scalar (either zero or one).

\paragraph{\syntax}

\begin{verbatim}
        GrB_Info GrB_Scalar_nvals(GrB_Index        *nvals,
                                  const GrB_Scalar  s);
\end{verbatim}

\paragraph{Parameters}

\begin{itemize}[leftmargin=1.1in]
    \item[{\sf nvals}] ({\sf OUT}) On successful return, this is set to the number of 
                                   stored elements in the scalar (zero or one).
    \item[{\sf s}]     ({\sf IN})  An existing GraphBLAS scalar being queried.
\end{itemize}


\paragraph{Return Values}

\begin{itemize}[leftmargin=2.1in]
    \item[{\sf GrB\_SUCCESS}]  In blocking or non-blocking mode, the operation 
    completed successfully and the value of {\sf nvals} has been set. 

    \item[{\sf GrB\_PANIC}]    Unknown internal error.
    
    \item[{\sf GrB\_INVALID\_OBJECT}] This is returned in any execution mode 
    whenever one of the opaque GraphBLAS objects (input or output) is in an invalid 
    state caused by a previous execution error.  Call {\sf GrB\_error()} to access 
    any error messages generated by the implementation.

    \item[{\sf GrB\_OUT\_OF\_MEMORY}] Not enough memory available for operation.
    
    \item[{\sf GrB\_UNINITIALIZED\_OBJECT}]  The GraphBLAS scalar, {\sf s}, has 
    not been initialized by a call to {\sf Scalar\_new} or {\sf Scalar\_dup}.
    
    \item[{\sf GrB\_NULL\_POINTER}]  The {\sf nvals} pointer is {\sf NULL}.
\end{itemize}

\paragraph{Description}


Return $\mathbf{nvals}({\sf s})$ in {\sf nvals}. This is the number of stored 
elements in scalar {\sf s}, which is the size of $\mathbf{L}(\scalar{v})$, and
can only be either zero or one (see Section~\ref{Sec:Scalars}).

%-----------------------------------------------------------------------------
\subsubsection{{\sf Scalar\_setElement}: Set the single element in a scalar}

Set the single element of a scalar to a given value.

\paragraph{\syntax}

\begin{verbatim}
        GrB_Info GrB_Scalar_setElement(GrB_Scalar   s,
                                       <type>       val);
\end{verbatim}

\paragraph{Parameters}

\begin{itemize}[leftmargin=1.1in]
    \item[{\sf s}]   ({\sf INOUT}) An existing GraphBLAS scalar for which the 
    element is to be assigned.

    \item[{\sf val}]   ({\sf IN}) Scalar value to assign.  The type must
    be compatible with the domain of {\sf s}.
\end{itemize}

\paragraph{Return Values}

\begin{itemize}[leftmargin=2.1in]
    \item[{\sf GrB\_SUCCESS}]         In blocking mode, the operation completed
    successfully. In non-blocking mode, this indicates that the compatibility 
    tests on index/dimensions and domains for the input arguments passed successfully. 
    Either way, the output scalar {\sf s} is ready to be used in the next method of 
    the sequence.

    \item[{\sf GrB\_PANIC}]   Unknown internal error.
    
    \item[{\sf GrB\_INVALID\_OBJECT}] This is returned in any execution mode 
    whenever one of the opaque GraphBLAS objects (input or output) is in an invalid 
    state caused by a previous execution error.  Call {\sf GrB\_error()} to access 
    any error messages generated by the implementation.

    \item[{\sf GrB\_OUT\_OF\_MEMORY}]  Not enough memory available for operation.
    
    \item[{\sf GrB\_UNINITIALIZED\_OBJECT}]  The GraphBLAS scalar, {\sf s}, has 
    not been initialized by a call to {\sf Scalar\_new} or {\sf Scalar\_dup}.
    
    \item[{\sf GrB\_DOMAIN\_MISMATCH}]     The domains of {\sf s} and {\sf val}
    are incompatible.
\end{itemize}

\paragraph{Description}

First, {\sf val} and output GraphBLAS scalar are tested for domain compatibility as follows:
$\mathbf{D}({\sf val})$ must be compatible with $\mathbf{D}({\sf s})$. Two domains 
are compatible with each other if values from one domain can be cast to values 
in the other domain as per the rules of the C language. In particular, domains 
from Table~\ref{Tab:PredefinedTypes} are all compatible with each other. A domain 
from a user-defined type is only compatible with itself. If any compatibility 
rule above is violated, execution of {\sf GrB\_Scalar\_setElement} ends and 
the domain mismatch error listed above is returned.

We are now ready to carry out the assignment {\sf val}; that is:
\[
    {\sf s}(0) = {\sf val}
\]
If {\sf s} already had a stored value, it will be overwritten; otherwise,
the new value is stored in {\sf s}.

In {\sf GrB\_BLOCKING} mode, the method exits with return value 
{\sf GrB\_SUCCESS} and the new contents of {\sf s} is as defined above
and fully computed.  
In {\sf GrB\_NONBLOCKING} mode, the method exits with return value 
{\sf GrB\_SUCCESS} and the new content of scalar {\sf s} is as defined above 
but may not be fully computed; however, it can be used in the next GraphBLAS 
method call in a sequence.


%-----------------------------------------------------------------------------

\subsubsection{{\sf Scalar\_extractElement}: Extract a single element from a scalar.}
\label{Sec:Scalar_extractElement}

Assign a non-opaque scalar with the value of the element stored in a GraphBLAS scalar. 

\paragraph{\syntax}

\begin{verbatim}
        GrB_Info GrB_Scalar_extractElement(<type>           *val,
                                           const GrB_Scalar  s); 
\end{verbatim}

\paragraph{Parameters}

\begin{itemize}[leftmargin=1in]
    \item[{\sf val}]   ({\sf INOUT}) Pointer to a non-opaque scalar of type that is 
    compatible with the domain of scalar {\sf s}. On successful return, {\sf val} 
    holds the result of the operation, and any previous value in {\sf val} is 
    overwritten.

    \item[{\sf s}]     ({\sf IN}) The GraphBLAS scalar from which an element
    is extracted.
\end{itemize}

\paragraph{Return Values}

\begin{itemize}[leftmargin=2.1in]
    \item[{\sf GrB\_SUCCESS}]  In blocking or non-blocking mode, the operation 
    completed successfully. This indicates that the compatibility tests on 
    dimensions and domains for the input arguments passed successfully, and
    the output scalar, {\sf val}, has been computed and is ready to be used in 
    the next method of the sequence.

    \item[{\sf GrB\_PANIC}]   Unknown internal error.
    
    \item[{\sf GrB\_INVALID\_OBJECT}] This is returned in any execution mode 
    whenever one of the opaque GraphBLAS objects (input or output) is in an invalid 
    state caused by a previous execution error.  Call {\sf GrB\_error()} to access 
    any error messages generated by the implementation.

    \item[{\sf GrB\_OUT\_OF\_MEMORY}]  Not enough memory available for operation.
    
    \item[{\sf GrB\_UNINITIALIZED\_OBJECT}]  The GraphBLAS scalar, {\sf s}, has 
    not been initialized by a call to {\sf Scalar\_new} or {\sf Scalar\_dup}.
    
    \item[{\sf GrB\_NULL\_POINTER}]    {\sf val} pointer is {\sf NULL}.
    
    \item[{\sf GrB\_DOMAIN\_MISMATCH}]     The domains of the scalar or scalar
    are incompatible.

    \item[{\sf GrB\_NO\_VALUE}]  There is no stored value in the scalar.
\end{itemize}

\paragraph{Description}

First, {\sf val} and input GraphBLAS scalar are tested for domain compatibility as follows:
$\mathbf{D}({\sf val})$ must be compatible with $\mathbf{D}({\sf s})$. Two domains 
are compatible with each other if values from one domain can be cast to values 
in the other domain as per the rules of the C language. In particular, domains 
from Table~\ref{Tab:PredefinedTypes} are all compatible with each other. A domain 
from a user-defined type is only compatible with itself. If any compatibility 
rule above is violated, execution of {\sf GrB\_Scalar\_extractElement} ends and 
the domain mismatch error listed above is returned.

Then, if no value is currently stored in the GraphBLAS scalar, execution of 
{\sf GrB\_Scalar\_extractElement} ends and the "no value" error listed above is returned.

Finally the extract into the output argument, {\sf val} can be performed;  
that is:
\[
    {\sf val} = {\sf s}(0)
\]

In both {\sf GrB\_BLOCKING} mode {\sf GrB\_NONBLOCKING} mode
if the method exits with return value {\sf GrB\_SUCCESS}, the  new 
contents of {\sf val} are as defined above.  

