\subsection{{\sf error} method}
\label{Sec:GrB_error}

Fetch an error message regarding errors encountered during computing associated with an object.

\paragraph{\syntax}

\begin{verbatim}
        GrB_Info GrB_error(const char          **error,
                           const GrB_Object      obj);
\end{verbatim}

\paragraph{Parameters}

\begin{itemize}[leftmargin=1.1in]
	\item[{\sf error}] ({\sf OUT}) A pointer to a null-terminated
		string. The contents of the string are implementation
		defined.

        \item[{\sf obj}] ({\sf IN}) An existing GraphBLAS object.
        The object must have been created by an explicit call to a
        GraphBLAS constructor.  Can be any of the opaque GraphBLAS
        objects such as matrix, vector, descriptor, semiring, monoid,
        binary op, unary op, or type. On successful return of {\sf
        GrB\_wait}, all GraphBLAS operations that produce {\sf obj}
        as output have fully completed.
\end{itemize}


\paragraph{Return value}
\begin{itemize}[leftmargin=2.3in]
	\item[{\sf GrB\_SUCCESS}]			operation completed successfully.
	\item[{\sf GrB\_UNINITIALIZED\_OBJECT}]		object has not been initialized by a call to the respective {\sf *\_new} method.
	\item[{\sf GrB\_PANIC}]				unknown internal error.
\end{itemize}

\paragraph{Description}

\scott{Copied from Basic Concepts}

At any point during execution, the program can retrieve additional
error information on an object through a
call to the function {\sf GrB\_error()}. 
The function returns a pointer to a null terminated string and the contents of that string
are implementation dependent. In particular, a null string (not a {\sf NULL} pointer) is always a valid error string.


