\subsection{{\sf error}: Retrieve an error string}
\label{Sec:GrB_error}

Retrieve an error-message about any errors encountered during the processing associated with an object.

\paragraph{\syntax}

\begin{verbatim}
        GrB_Info GrB_error(const char          **error,
                           const GrB_Object      obj);
\end{verbatim}

\paragraph{Parameters}

\begin{itemize}[leftmargin=1.1in]
	\item[{\sf error}] ({\sf OUT}) A pointer to a null-terminated
		string. The contents of the string are implementation
		defined.

        \item[{\sf obj}] ({\sf IN}) An existing GraphBLAS object.
        The object must have been created by an explicit call to a
        GraphBLAS constructor.  Can be any of the opaque GraphBLAS
        objects such as matrix, vector, descriptor, semiring, monoid,
        binary op, unary op, or type.
\end{itemize}


\paragraph{Return value}
\begin{itemize}[leftmargin=2.3in]
	\item[{\sf GrB\_SUCCESS}]                operation completed successfully.
	\item[{\sf GrB\_UNINITIALIZED\_OBJECT}]  object has not been initialized by a call to the respective {\sf *\_new}, or other constructor, method.
	\item[{\sf GrB\_PANIC}]                  unknown internal error.
\end{itemize}

\paragraph{Description}

This method retrieves a message related to any errors that were encountered 
during the last GraphBLAS method that had the opaque GraphBLAS object, 
{\sf obj}, as an {\sf OUT} or {\sf INOUT} parameter.   The function returns a 
pointer to a null-terminated string and the contents of that string are 
implementation-dependent.  In particular, a null 
string (not a {\sf NULL} pointer) is always a valid error string.  The string 
that is returned is owned by {\sf obj} and will be valid until the next time 
{\sf obj} is used as an {\sf OUT} or {\sf INOUT} parameter or the object is freed
by a call to {\sf GrB\_free(obj)}.
{\color{red}
This is a thread-safe function.  It can be safely called by multiple threads for the 
same object in a race-free program.
% This is a thread-safe function. , in the sense that multiple threads can call it
% simultaneously and each will get its own error string back, referring to the
% object passed as {\sf obj}.} \scott{Is this really true?} \jose{Yes, I think it is.}
