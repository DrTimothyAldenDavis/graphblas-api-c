\subsection{{\sf apply}: Apply a unary function to the elements of a matrix}


\paragraph{C99 Syntax}

\begin{verbatim}
        // Vector version
        GrB_info GrB_apply(GrB_Vector                *dst,
                           const GrB_Vector           mask,
                           const GrB_BinaryFunction   accum,
                           const GrB_UnaryFunction    op,
                           const GrB_Vector           src
                        [, const GrB_Descriptor       desc]);

        // Matrix version
        GrB_info GrB_apply(GrB_Matrix                *dst,
                           const GrB_Matrix           Mask,
                           const GrB_BinaryFunction   accum,
                           const GrB_UnaryFunction    op,
                           const GrB_Matrix           src
                        [, const GrB_Descriptor       desc]);
\end{verbatim}

\paragraph{Parameters}

\begin{itemize}[leftmargin=1.1in]
    \item[{\sf dst}]   ({\sf OUTP}) The matrix/vector to assign the result.

    \item[{\sf mask}] ({\sf MASK}) Output mask. The mask
    specifies which elements of {\sf dst} can be modified.
    If no mask is necessary (i.e., compute all elements of result
    vector), {\sf GrB\_NULL} should be specified.

    \item[{\sf accum}]  Function used for accumulating entries into existing {\sf dst} entries. 
			If no accumulation is desired, {\sf GrB\_NULL} should be specified.

    \item[{\sf op}]    The unary function (operation) to apply to each stored element in src.
    \item[{\sf src}]   ({\sf ARG0}) The matrix to apply the function to.
    \item[{\sf desc}]   Operation descriptor (optional). If a
    \emph{default} descriptor is desired, {\sf GrB\_NULL} can be
    used or the descriptor can be omitted.  Valid fields and values are as follows: \\
    \begin{tabular}{lll}
    Field  & Value & Description \\
    \hline
    {\sf ARG0} & {\sf GrB\_CAST} & Allow casting from $\bold{D}({\sf src})$ to $\bold{D}({\sf dst})$ \\
    {\sf ARG0} & {\sf GrB\_TRAN} & Transpose {\sf src} \\
    {\sf MASK} & {\sf GrB\_SCMP} & Use the structural complement of {\sf mask}. \\
    {\sf OUTP}& {\sf GrB\_ACC}  & Use the {\sf accum} function to combine with existing values in {\sf dst}.\\
    \end{tabular}
\end{itemize}

\paragraph{Return Values}

\scott{Are invalid descriptors an error or ignored?}

\begin{itemize}[leftmargin=2.1in]
\item[{\sf GrB\_SUCCESS}]     operation completed successfully.
\item[{\sf GrB\_PANIC}]        unknown internal error.
\item[{\sf GrB\_DIMENSION\_MISMATCH}]            
        If the size/shape of dst is not the same as either mask or
        of the transpose of src.
\item[{\sf GrB\_DOMAIN\_MISMATCH}]  
        domain mismatch among matrices, unary function, and/or
        accum function \scott{elaborate}
\end{itemize}

\paragraph{Description}

The dst matrix must have already been created with the proper dimensions
prior to calling this function.  The unary function operators on stored values in src and the results are assigned to corresponding location in dst.
Src can be optionally transposed first.  The results of the unary function can be optionally accumulated with existing values in dst using accum.  Assignment into dst can be optionally masked.

%=========================================================================

\subsection{{\sf reduce}: Perform a reduction across the elements of an object}

Computes the reduction of the values of the elements of a vector or matrix.  There are a number of different variants.

%-----------------------------------------------------------------------------
\subsubsection{{\sf reduce}: Matrix to vector variant}

This performs a row-wise reduction of a matrix to produce a vector.  If column-wise reduction
is desired, the input matrix should be transposed which can be specified using the descriptor.

\paragraph{C99 Syntax}

\begin{verbatim}
        // reduce rows or columns to a  vector
        GrB_info GrB_reduce(GrB_Vector               *dst,
                            const GrB_Vector          mask,
                            const GrB_BinaryFunction  accum,
                            const GrB_Monoid          md,  
                            const GrB_Matrix          A
                         [, const GrB_Descriptor      desc]);
\end{verbatim}

\paragraph{Parameters}

\begin{itemize}[leftmargin=1.1in]
    \item[{\sf dst}]   ({\sf OUTP}) Vector/Matrix to be assigned.

    \item[{\sf mask}] ({\sf MASK}) Output mask. The mask specifies which elements
    of {\sf dst} can be modified. If no mask is necessary (i.e., compute all
    elements of {\sf dst}), {\sf GrB\_NULL} should be specified.

    \item[{\sf accum}] Function used for accumulation into {\sf dst}.  If no accumulation
                        is desired, {\sf GrB\_NULL} should be specified.

    \item[{\sf md}]    The monoid with the function to reduce two src elements and the identity to fill for empty rows/columns. 
    \item[{\sf A}]     ({\sf ARG0}) The matrix on which to operate.

    \item[{\sf desc}]   Operation descriptor (optional). If a
    \emph{default} descriptor is desired, {\sf GrB\_NULL} can be
    used or the descriptor can be omitted.  Valid fields and values are as follows: \\
    \begin{tabular}{lll}
    Field  & Value & Description \\
    \hline
    {\sf MASK} & {\sf GrB\_NOCAST} & Prohibit casting from $\bold{D}({\sf mask})$ to {\sf bool} domain. \\
    {\sf MASK} & {\sf GrB\_SCMP} & Use the structural complement of {\sf mask}. \\
    {\sf ARG0} & {\sf GrB\_NOCAST} & Prohibit casting from $\bold{D}(\matrix{A})$ to $\bold{D}({\sf dst})$ \\
    {\sf ARG0} & {\sf GrB\_TRAN} & Transpose {\sf A} matrix first to perform a 
    column-wise reduction. \\
    \end{tabular}

\end{itemize}

\paragraph{Return Values}
\aydin{Aydin to fill}

\paragraph{Description}

Matrix \matrix{A} is computed from
input parameter {\sf A} as specified
by descriptor {\sf desc}. Specifically, it computes the vector of row ``sums'' if GrB\_TRAN is not set, and
the vector of column ``sums'' if GrB\_TRAN  is set. It then stores this result vector into {\sf dst}. 

When called on a $m$-by-$n$ matrix \matrix{A}, this function returns a vector ${\sf dst}$ of length $m$. Each entry
${\sf dst}(i)$ is the ``sum'' of all entries in the $i$th row of {\sf A} with respect to the monoid {\sf md}. 

Given that the binary operation of the monoid {\sf md} is denoted by $\oplus$ and its identity element is $\overline{0}$ in this discussion, 
this amounts to $$ {\sf dst}(i) = \overline{0} \oplus \matrix{A}(i,0) \oplus \matrix{A}(i,1) \oplus  \ldots \oplus  \matrix{A}(i,n) $$

This function requires a GrB\_Monoid because for cases where $\matrix{A}(i,j) = 0$ for all $j=1 \ldots n$, we need to instantiate {\sf dst}(i) with some value (which
is $\overline{0}$ in this case).

\aydin{Can we say if that row is empty, then we return a "sparse" vector? } 
\scott{I believe so, which is why I don't think that reduction to a vector requires a monoid.  However we need to decide if this spec supports dense vectors and matrices need to make sure the structural zero values are incorporated properly and consistently everywhere.}



%-----------------------------------------------------------------------------
\subsubsection{{\sf reduce}: Scalar variants}

\scott{TODO: do we need to support an input mask for this variant?}

\scott{TODO: Need to have conversation about whether op can/should also take a semiring}

\scott{TODO: Should we support accum in this variant?}


\paragraph{C99 Syntax}

\begin{verbatim}
        // semiring versions, to be removed?
        //GrB_info GrB_reduce(<type>                   *dst,
        //                    const GrB_BinaryFunction  accum,
        //                    const GrB_Semiring        op,
        //                    const GrB_Vector/Matrix   src
        //                 [, const GrB_Descriptor      desc]);
                         
        // vector/monoid version
        GrB_info GrB_reduce(<type>                   *dst,
                            const GrB_BinaryFunction  accum,
                            const GrB_Monoid          op,
                            const GrB_Vector          src
                         [, const GrB_Descriptor      desc]);
        // matrix/monoid version
        GrB_info GrB_reduce(<type>                   *dst,
                            const GrB_BinaryFunction  accum,
                            const GrB_Monoid          op,
                            const GrB_Matrix          src
                         [, const GrB_Descriptor      desc]);
\end{verbatim}


\paragraph{Parameters}

\begin{itemize}[leftmargin=1.1in]
    \item[{\sf dst}]    ({\sf OUTP}) Scalar to store final reduced value into.  The type must be
                        a pointer to one of the types in the left column of 
                        Table~\ref{Tab:PredefinedTypes} or {\tt void*}. It must be
                        compatible with the {\sf src} or {\sf accum} (if used) domains.

    \item[{\sf accum}]  Function used for accumulation into {\sf dst}.  If no accumulation
                        is desired, {\sf GrB\_NULL} should be specified.

    \item[{\sf op}]     ({\sf ARG0}) Semiring(?)/monoid defining the reduction.
    \item[{\sf src}]    Vector/matrix to be reduced.

    \item[{\sf desc}]   Operation descriptor (optional). If a
    \emph{default} descriptor is desired, {\sf GrB\_NULL} can be
    used or the descriptor can be omitted.  Valid fields and values are as follows: \\
    \begin{tabular}{lll}
    Field  & Value & Description \\
    \hline
    {\sf ARG0} & {\sf GrB\_NOCAST} & Prohibit casting from $\bold{D}({\sf src})$ to 
    $\bold{D}({\sf dst})$.  \\
    \end{tabular}\\
    \scott{what about casting to op, or from op to accum, or from accum to dst?}
\end{itemize}

\paragraph{Return Values}

\begin{itemize}[leftmargin=2.1in]
\item[{\sf GrB\_SUCCESS}]             operation completed successfully.
\item[{\sf GrB\_PANIC}]               unknown internal error.
\item[{\sf GrB\_NOVECTOR/NOMATRIX}]   {\sf src} vector/matrix does not exist
\item[{\sf GrB\_DOMAIN\_MISMATCH}]  
        domain mismatch among matrices, monoid operation, and/or
        accum function \scott{elaborate}
\end{itemize}

\paragraph{Description}

Let $0 = \bold{0}({\sf op})$, whether ${\sf s}$ is a semiring or monoid.
Let $\oplus = \bigoplus({\sf op})$ if op is a semiring and let 
$\oplus = \bigodot({\sf op})$ if op is a monoid.

We must have $\bold{D}_3({\sf s}) = \bold{D}_1({\sf s})$.
Otherwise, the method returns {\sf GrB\_MISMATCH}.

Vector $\vector{v}$ is computed from input parameter ${\sf v}$ as
specified by descriptor {\sf d}. $\bold{D}(\vector{v}) = \bold{D}_2({\sf s})$
and $\bold{n}(\vector{v}) = \bold{n}({\sf v})$. If $\vector{v}$ cannot be computed
from the input parameters, the method returns {\sf GrB\_MISMATCH}.

A scalar variable $t$ such that $\bold{D}(t) = \bold{D_1}({\sf s})$ is
created and initialized $t \leftarrow \bold{0}({\sf s})$. 
We then compute the recurrence $t \leftarrow t \oplus v_i, \forall i \in \vector{i}(\vector{v})$.

Finally, output parameter {\sf t} is computed from scalar $t$.

%=========================================================================

\subsection{{\sf transpose}: Transpose rows and columns of a matrix}

This version materializes a new matrix that is the transpose of the source matrix.

\scott{ If we adopt the concept of a vector having only one index and conventional orientation of column, then there is no materialized transpose for vector. For vector transpose only makes sense in the context of other operations.}


\scott{TODO: Need to discuss if we will support dst and src referencing the same matrix
with the meaning that in-place transpose is to be performed. With immutable dimensions
and non-square matrices this should be disallowed.}

\paragraph{C99 Syntax}

\begin{verbatim}
        GrB_info GrB_transpose(GrB_Matrix               *dst,
                               const GrB_Matrix          Mask,
                               const GrB_BinaryFunction  accum,
                               const GrB_Matrix          src
                            [, const GrB_Descriptor      desc]);
\end{verbatim}

\paragraph{Parameters}

\begin{itemize}[leftmargin=1.1in]
    \item[{\sf dst}]   ({\sf OUTP}) The matrix to assign the result.

    \item[{\sf Mask}]  ({\sf MASK}) Output mask. The mask specifies which elements
    of {\sf dst} can be modified. If no mask is necessary (i.e., compute all
    elements of {\sf dst}), {\sf GrB\_NULL} should be specified.

    \item[{\sf accum}] Function used for accumulation into {\sf dst}.  If no accumulation
                        is desired, {\sf GrB\_NULL} should be specified.

    \item[{\sf src}]   ({\sf ARG0}) The matrix to transpose.

    \item[{\sf desc}]  Operation descriptor (optional). The descriptor
    is used to specify details of the operation. Valid options are 
    invert (structural complement) of mask ({\sf ARG3}). If
    a \emph{default} descriptor is desired,    {\sf GrB\_NULL} can be
    used or the descriptor can be omitted.

    \item[{\sf desc}]   Operation descriptor (optional). If a
    \emph{default} descriptor is desired, {\sf GrB\_NULL} can be
    used or the descriptor can be omitted.  Valid fields and values are as follows: \\
    \begin{tabular}{lll}
    Field  & Value & Description \\
    \hline
    {\sf MASK} & {\sf GrB\_NOCAST} & Prohibit casting from $\bold{D}({\sf Mask})$ to {\sf bool} domain. \\
    {\sf MASK} & {\sf GrB\_SCMP}   & Use the structural complement of {\sf Mask}. \\
    {\sf ARG0} & {\sf GrB\_NOCAST} & Prohibit casting from $\bold{D}({\sf src})$ to $\bold{D}({\sf dst})$ or to the \\ & & accum input domain (if specified). \\
    \end{tabular}
\end{itemize}

\paragraph{Return Values}

\scott{Are invalid descriptors an error or ignored?}

\begin{itemize}[leftmargin=2.1in]
\item[{\sf GrB\_SUCCESS}]     operation completed successfully.
\item[{\sf GrB\_PANIC}]        unknown internal error.
\item[{\sf GrB\_DIMENSION\_MISMATCH}]      
        If the size/shape of dst is not the same as either mask or
        of the transpose of src.
\item[{\sf GrB\_DOMAIN\_MISMATCH}]  
        domain mismatch among matrices and/or
        accum function \scott{elaborate}
\end{itemize}

\paragraph{Description}

The dst matrix must have already been created with the proper dimensions
prior to calling this function.  Stored values are inserted into dst to
create a transpose of the src matrix where 
$dst(j,i) = src(i,j) \forall i,j$ where $src(i,j) \neq 0$.
If accum is specified then the function is used to combine with existing values in dst using {\sf eWiseAdd} semantics:
$dst(j,i) \oplus = src(i,j) \forall i,j$ where $src(i,j) \neq 0$.
When the mask is specified the destination location will only be assigned if the corresponding location in the mask has a stored value.
