\section{Basic concepts}

\subsection{Domains}

GraphBLAS defines two kinds of collections: matrices and vectors.
For any given collection, the elements of the collection belong to
a \emph{domain}, which is the set of valid values for the element.
In GraphBLAS, domains correspond to the valid values for types from
the host language (in our case, the C programming language).  For any
variable or object $V$ in GraphBLAS we denote as $\bold{D}(V)$ the
domain of $V$. That is, the set of possible values that elements of
$V$ can take.  The predefined types, and corresponding domains, of
GraphBLAS are shown in Table~\ref{Tab:PredefinedTypes}.  The Boolean
type is defined in {\tt stdbool.h}, the integral types are defined in
{\tt stdint.h}, and the floating-point types are native to the language.
GraphBLAS also supports user defined types. In that case, the domain is
the set of valid values for a variable of that type.

\begin{table}
\hrule
\begin{center}
\caption{Predefined types and corresponding domains for GraphBLAS in C. 
	    \aydin{There will be a way to introduce new GraphBLAS identifiers,
        similar to MPI\_Type\_Commit, these are just predefined stuff}
		\scott{An example would be nice, 
		especially where there is not a 1-to-1 correspondence to a built-in
		type, e.g. \{0,1\}.}}
\label{Tab:PredefinedTypes}
\begin{tabular}{l|l|l}
type	& domain & GraphBLAS identifier \\ \hline
{\tt bool}	& $\{ {\tt false}, {\tt true} \}$	& {\sf GrB\_BOOL} \\
{\tt int8\_t}	& $\mathbb{Z} \cap [-2^{7},2^{7})$ 	& {\sf GrB\_INT8} \\
{\tt uint8\_t}	& $\mathbb{Z} \cap [0,2{^8})$ 		& {\sf GrB\_UINT8} \\
{\tt int16\_t}	& $\mathbb{Z} \cap [-2^{15},2^{15})$ 	& {\sf GrB\_INT16} \\
{\tt uint16\_t}	& $\mathbb{Z} \cap [0,2^{16})$ 		& {\sf GrB\_UINT16} \\
{\tt int32\_t}	& $\mathbb{Z} \cap [-2^{31},2^{31})$ 	& {\sf GrB\_INT32} \\
{\tt uint32\_t}	& $\mathbb{Z} \cap [0,2^{32})$ 		& {\sf GrB\_UINT32} \\
{\tt int64\_t}	& $\mathbb{Z} \cap [-2^{63},2^{63})$ 	& {\sf GrB\_INT64} \\
{\tt uint64\_t}	& $\mathbb{Z} \cap [0,2^{64})$ 		& {\sf GrB\_UINT64} \\
{\tt float}	& IEEE 754 {\sf binary32} 		& {\sf GrB\_FLOAT} \\
{\tt double}	& IEEE 754 {\sf binary64} 		& {\sf GrB\_DOUBLE} \\
\end{tabular}
\end{center}
\hrule
\end{table}

\subsection{Operations}

In GraphBLAS, a \emph{binary operation} is a function that maps two input
values to one output value. A \emph{unary operation} is a function that 
maps one input value to one output value. The value of the output is uniquely
determined by the value of the input(s).
Binary functions are defined over two input domains and produce an output from
a (possibly different) third domain. Unary functions are specified
over one input domain and produce an output from a (possibly different)
second domain.
The predefined operations of GraphBLAS are listed in
Table~\ref{Tab:PredefinedOperations}.

\begin{table}
\hrule
\begin{center}
\caption{Predefined operations for GraphBLAS in C. (Just a sample.)
\scott{We need to specify the complete set of predefined operations. 
Notably missing are min and max.}}
\label{Tab:PredefinedOperations}
\begin{tabular}{l|l|l|l}
		& GraphBLAS		&									& \\
kind		& identifier 		& domains								& description \\ \hline
		& {\sf GrB\_NOP}	& 									& no operation \\
unary		& {\sf GrB\_LNOT}	& ${\tt bool} \rightarrow {\tt bool}$      				& logical inverse \\
binary		& {\sf GrB\_LAND}	& ${\tt bool} \times {\tt bool} \rightarrow {\tt bool}$      		& logical AND \\
binary		& {\sf GrB\_LOR}	& ${\tt bool} \times {\tt bool} \rightarrow {\tt bool}$      		& logical OR \\
binary		& {\sf GrB\_LXOR}	& ${\tt bool} \times {\tt bool} \rightarrow {\tt bool}$      		& logical XOR \\
binary		& {\sf GrB\_PLUSI64}	& ${\tt int64\_t} \times {\tt int64\_t} \rightarrow {\tt int64\_t}$ 	& signed integer addition \\
binary		& {\sf GrB\_PLUSU64}	& ${\tt uint64\_t} \times {\tt uint64\_t} \rightarrow {\tt uint64\_t}$ 	& unsigned integer addition \\
binary		& {\sf GrB\_PLUSF64}	& ${\tt double} \times {\tt double} \rightarrow {\tt double}$ 		& floating-point addition \\
binary		& {\sf GrB\_MINUSI64}	& ${\tt int64\_t} \times {\tt int64\_t} \rightarrow {\tt int64\_t}$ 	& signed integer subtraction \\
binary		& {\sf GrB\_MINUSU64}	& ${\tt uint64\_t} \times {\tt uint64\_t} \rightarrow {\tt uint64\_t}$ 	& unsigned integer subtraction \\
binary		& {\sf GrB\_MINUSF64}	& ${\tt double} \times {\tt double} \rightarrow {\tt double}$ 		& floating-point subtraction \\
binary		& {\sf GrB\_TIMESI64}	& ${\tt int64\_t} \times {\tt int64\_t} \rightarrow {\tt int64\_t}$ 	& signed integer multiplication \\
binary		& {\sf GrB\_TIMESU64}	& ${\tt uint64\_t} \times {\tt uint64\_t} \rightarrow {\tt uint64\_t}$ 	& unsigned integer multiplication \\
binary		& {\sf GrB\_TIMESF64}	& ${\tt double} \times {\tt double} \rightarrow {\tt double}$ 		& floating-point multiplication \\
binary		& {\sf GrB\_DIVI64}	& ${\tt int64\_t} \times {\tt int64\_t} \rightarrow {\tt int64\_t}$ 	& signed integer division \\
binary		& {\sf GrB\_DIVU64}	& ${\tt uint64\_t} \times {\tt uint64\_t} \rightarrow {\tt uint64\_t}$ 	& unsigned integer division \\
binary		& {\sf GrB\_DIVF64}	& ${\tt double} \times {\tt double} \rightarrow {\tt double}$ 		& floating-point division \\
binary		& {\sf GrB\_EQI64}	& ${\tt int64\_t} \times {\tt int64\_t} \rightarrow {\tt bool}$		& signed integer  equal \\
binary		& {\sf GrB\_EQU64}	& ${\tt uint64\_t} \times {\tt uint64\_t} \rightarrow {\tt bool}$ 	& unsigned integer  equal \\
binary		& {\sf GrB\_EQF64}	& ${\tt double} \times {\tt double} \rightarrow {\tt bool}$ 		& floating-point  equal \\
binary		& {\sf GrB\_NEI64}	& ${\tt int64\_t} \times {\tt int64\_t} \rightarrow {\tt bool}$		& signed integer not equal \\
binary		& {\sf GrB\_NEU64}	& ${\tt uint64\_t} \times {\tt uint64\_t} \rightarrow {\tt bool}$ 	& unsigned integer not equal \\
binary		& {\sf GrB\_NEF64}	& ${\tt double} \times {\tt double} \rightarrow {\tt bool}$ 		& floating-point not equal \\
binary		& {\sf GrB\_GTI64}	& ${\tt int64\_t} \times {\tt int64\_t} \rightarrow {\tt bool}$		& signed integer greater than  \\
binary		& {\sf GrB\_GTU64}	& ${\tt uint64\_t} \times {\tt uint64\_t} \rightarrow {\tt bool}$ 	& unsigned integer greater than  \\
binary		& {\sf GrB\_GTF64}	& ${\tt double} \times {\tt double} \rightarrow {\tt bool}$ 		& floating-point greater than  \\
binary		& {\sf GrB\_LTI64}	& ${\tt int64\_t} \times {\tt int64\_t} \rightarrow {\tt bool}$		& signed integer less than  \\
binary		& {\sf GrB\_LTU64}	& ${\tt uint64\_t} \times {\tt uint64\_t} \rightarrow {\tt bool}$ 	& unsigned integer less than  \\
binary		& {\sf GrB\_LTF64}	& ${\tt double} \times {\tt double} \rightarrow {\tt bool}$ 		& floating-point less than  \\
binary		& {\sf GrB\_GEI64}	& ${\tt int64\_t} \times {\tt int64\_t} \rightarrow {\tt bool}$		& signed integer greater than or equal \\
binary		& {\sf GrB\_GEU64}	& ${\tt uint64\_t} \times {\tt uint64\_t} \rightarrow {\tt bool}$ 	& unsigned integer greater than or equal \\
binary		& {\sf GrB\_GEF64}	& ${\tt double} \times {\tt double} \rightarrow {\tt bool}$ 		& floating-point greater than or equal \\
binary		& {\sf GrB\_LEI64}	& ${\tt int64\_t} \times {\tt int64\_t} \rightarrow {\tt bool}$		& signed integer less than or equal \\
binary		& {\sf GrB\_LEU64}	& ${\tt uint64\_t} \times {\tt uint64\_t} \rightarrow {\tt bool}$ 	& unsigned integer less than or equal \\
binary		& {\sf GrB\_LEF64}	& ${\tt double} \times {\tt double} \rightarrow {\tt bool}$ 		& floating-point less than or equal \\
\end{tabular}
\end{center}
\hrule
\end{table}

\subsection{Index, Index Arrays and Scalar Arrays}

\scott{TBD: based on recent email conversation with Aydin}

In order to interface with third-party software packages, operations like buildMatrix and extractTuples needs to specify the how data should be laid out (i.e. non-opaque structures).  To this end we define, explicitly the types for indices and the arrays used by these operations.

The index type is a typedef used to give a GraphBLAS name to a concrete type. We define it as follows:

\begin{verbatim}
    typedef uint64_t GrB_Index;
\end {verbatim}

The index array, is a pointer to a set of GrB\_Index values that are stored in a contiguous block of memory:

\begin{verbatim}
    typedef uint64_t* GrB_IndexArray;
\end{verbatim}

\scott{Do we really want to do it this way or should we replace all references to GrB\_IndexArray with GrB\_Index*}

Placeholder for discussion of scalar arrays: getting stored values into and out of matrices and vectors.  \scott{It probably also needs to be a contiguous array of arbitrary type (void* ?)}
