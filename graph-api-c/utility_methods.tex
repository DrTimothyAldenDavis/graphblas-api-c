\section{Utility Methods}

\scott{TODO: there are many useful functions possible.  Need to get input.}

%-----------------------------------------------------------------------------
\subsection{{\sf init}: Initialize a GraphBLAS context}

Creates and initializes a GraphBLAS C API context.  The argument
to {\sf GrB\_init} defines the mode for the context.  The two
available modes are:

\begin{itemize}
\item {\sf GrB\_Blocking}: Methods in a sequence return after
computations in the method have completed and output arguments
are available to subsequent statements in an application.  When
executing in {\sf GrB\_Blocking} mode, the methods execute 
in program order.

\item {\sf GrB\_NonBlocking}: Methods in a sequence return after
arguments in the method have been tested for consistency with the
method but potentially before computations complete or output 
arguments are available to subsequent statements in an application.
When executing in {\sf GrB\_NonBlocking} mode, the methods 
in a sequence may execute in any order that preserves the 
mathematically result defined by the squence.

\end{itemize}

{\sf GrB\_init()} may be called with {\sf GrB\_NULL} to select 
the default mode.  This mode is implementation defined.

\paragraph{C99 Syntax}

\begin{verbatim}
        GrB_info GrB_init(GrB_Mode m);
\end{verbatim}


\paragraph{Parameters}

\begin{itemize}[leftmargin=1.1in]
	\item[{\sf m}] Mode for the GraphBLAS context.
\end{itemize}

\paragraph{Return Values}

\begin{itemize}[leftmargin=2.1in]
\item[{\sf GrB\_SUCCESS}]        operation completed successfully
\item[{\sf GrB\_PANIC}]          unknown internal error
\item[{\sf GrB\_NOMODE}]         mode does not exist
\end{itemize}


%-----------------------------------------------------------------------------
\subsection{{\sf finalize}: finalize a GraphBLAS context}

Terminates and frees any internal resources created to 
support the GraphBLAS C API context.
An application may not create a new context after 
{\sf GrB\_finalize} has been called.

\paragraph{C99 Syntax}

\begin{verbatim}
        GrB_info GrB_finalize();
\end{verbatim}

\paragraph{Return Values}

\begin{itemize}[leftmargin=2.1in]
\item[{\sf GrB\_SUCCESS}]        operation completed successfully
\item[{\sf GrB\_PANIC}]          unknown internal error
\end{itemize}

%-----------------------------------------------------------------------------
\subsection{{\sf free}: Destroy object}

Destroys a previously created GraphBLAS object.

\paragraph{C99 Syntax}

\begin{verbatim}
        GrB_info GrB_free(GrB_Object o);
\end{verbatim}

\ajy{polymorphic?  is there where \_Generic is specified?}
\jose{This is \emph{one} of the polymorphic functions in the C API.
I changed the introduction to say that polymorphism must be supported by an
extensio of C99. {\tt \_Generic} is just one way of accomplishing that.}

\paragraph{Parameters}

\begin{itemize}[leftmargin=1.1in]
	\item[{\sf o}] GraphBLAS object to be destroyed. Can be a matrix, vector or descriptor.
\end{itemize}

\paragraph{Return Values}

\begin{itemize}[leftmargin=2.1in]
\item[{\sf GrB\_SUCCESS}]        operation completed successfully
\item[{\sf GrB\_PANIC}]          unknown internal error
\item[{\sf GrB\_NOOBJECT}]       object does not exist
\end{itemize}
