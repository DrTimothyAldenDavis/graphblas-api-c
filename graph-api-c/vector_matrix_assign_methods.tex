\subsection{Vector Methods}

%-----------------------------------------------------------------------------
\subsubsection{{\sf Vector\_new}: Create new vector}

Creates a new vector with specified domain and size.

\paragraph{\syntax}

\begin{verbatim}
        GrB_Info GrB_Vector_new(GrB_Vector *v,
                                GrB_Type    d,
                                GrB_Index   nsize);
\end{verbatim}

\paragraph{Parameters}

\begin{itemize}[leftmargin=1.1in]
    \item[{\sf v}] ({\sf INOUT}) On successful return, contains a handle
                                 to the newly created GraphBLAS vector.
    \item[{\sf d}] ({\sf IN})    The type corresponding to the domain of the 
                                 vector being created.  Can be one of the 
                                 predefined GraphBLAS types in 
                                 Table~\ref{Tab:PredefinedTypes}, or an existing 
                                 user-defined GraphBLAS type.
    \item[{\sf nsize}] ({\sf IN}) The size of the vector being created.
\end{itemize}

\paragraph{Return Values}

\begin{itemize}[leftmargin=2.1in]
    \item[{\sf GrB\_SUCCESS}]         In blocking mode, operation completed
    successfully. In non-blocking mode, this indicates that the API checks 
    for the input arguments passed successfully. Either way, output vector 
    {\sf v} is ready to be used in the next method of the sequence.

    \item[{\sf GrB\_PANIC}]           Unknown internal error.
    
    \item[{\sf GrB\_INVALID\_OBJECT}] This is returned in any execution mode 
    whenever one of the opaque GraphBLAS objects (input or output) is in an invalid 
    state caused by a previous execution error.  Call {GrB\_error()} to access 
    any error messages generated by the implementation.

    \item[{\sf GrB\_OUT\_OF\_MEMORY}] Not enough memory available for operation.
    
    \item[{\sf GrB\_UNINITIALIZED\_OBJECT}]  The {\sf GrB\_Type} object has not 
    been initialized by a call to {\sf new} (needed for user-defined types).
    
    \item[{\sf GrB\_NULL\_POINTER}]  The {\sf v} pointer is {\sf NULL}.
    
    \item[{\sf GrB\_INVALID\_VALUE}] {\sf nsize} is zero
\end{itemize}

\paragraph{Description}

Creates a new vector $\vector{v}$ of domain $\bold{D}({\sf d})$, size {\sf nsize}, 
and empty $\bold{L}(\vector{v})$. It returns a handle to it in {\sf v}.

It is not an error to call this method more than once on the same variable;  
however, the handle to the previously created object will be overwritten. 

%\scott{In the context of non-blocking can this operation be deferred?}
%\aydin{why not?}

%-----------------------------------------------------------------------------
\subsubsection{{\sf Vector\_dup}: Create a copy of a GraphBLAS vector}

Creates a new vector with the same domain, size, and contents as another vector.

\paragraph{\syntax}

\begin{verbatim}
        GrB_Info GrB_Vector_dup(GrB_Vector       *w,
                                const GrB_Vector  u);
\end{verbatim}

\paragraph{Parameters}

\begin{itemize}[leftmargin=1.1in]
    \item[{\sf w}]  ({\sf INOUT}) On successful return, contains a handle
                                  to the newly created GraphBLAS vector.
    \item[{\sf u}]  ({\sf IN})    The GraphBLAS vector to be duplicated.
\end{itemize}

\paragraph{Return Values}

\begin{itemize}[leftmargin=2.1in]
    \item[{\sf GrB\_SUCCESS}]         In blocking mode, operation completed
    successfully. In non-blocking mode, this indicates that the API checks 
    for the input arguments passed successfully. Either way, output vector 
    {\sf w} is ready to be used in the next method of the sequence.

    \item[{\sf GrB\_PANIC}]           Unknown internal error.
    
    \item[{\sf GrB\_INVALID\_OBJECT}] This is returned in any execution mode 
    whenever one of the opaque GraphBLAS objects (input or output) is in an invalid 
    state caused by a previous execution error.  Call {GrB\_error()} to access 
    any error messages generated by the implementation.

    \item[{\sf GrB\_OUT\_OF\_MEMORY}] Not enough memory available for operation.
    
    \item[{\sf GrB\_UNINITIALIZED\_OBJECT}]  The GraphBLAS vector, {\sf u}, has 
    not been initialized by a call to {\sf Vector\_new} or {\sf Vector\_dup}.
    
    \item[{\sf GrB\_NULL\_POINTER}]  The {\sf w} pointer is {\sf NULL}.
\end{itemize}

\paragraph{Description}

Creates a new vector $\vector{w}$ of domain $\bold{D}({\sf u})$, size 
$\bold{size}({\sf u})$, and contents $\bold{L}({\sf u})$. It returns a 
handle to it in {\sf w}.

It is not an error to call this method more than once on the same variable;  
however, the handle to the previously created object will be overwritten. 

%-----------------------------------------------------------------------------
\subsubsection{{\sf Vector\_clear}: Clear a vector}

Removes all the elements (tuples) from a vector.

\paragraph{\syntax}

\begin{verbatim}
        GrB_Info GrB_Vector_clear(GrB_Vector v);
\end{verbatim}

\paragraph{Parameters}

\begin{itemize}[leftmargin=1.1in]
    \item[{\sf v}] ({\sf INOUT}) An existing GraphBLAS vector to clear.
\end{itemize}

\paragraph{Return Values}

\begin{itemize}[leftmargin=2.1in]
    \item[{\sf GrB\_SUCCESS}]         In blocking mode, operation completed
    successfully. In non-blocking mode, this indicates that the API checks 
    for the input arguments passed successfully. Either way, output vector 
    {\sf v} is ready to be used in the next method of the sequence.

    \item[{\sf GrB\_PANIC}]           Unknown internal error.
    
    \item[{\sf GrB\_INVALID\_OBJECT}] This is returned in any execution mode 
    whenever one of the opaque GraphBLAS objects (input or output) is in an invalid 
    state caused by a previous execution error.  Call {GrB\_error()} to access 
    any error messages generated by the implementation.

    \item[{\sf GrB\_OUT\_OF\_MEMORY}] Not enough memory available for operation.
    
    \item[{\sf GrB\_UNINITIALIZED\_OBJECT}]  The GraphBLAS vector, {\sf *v}, has 
    not been initialized by a call to {\sf Vector\_new} or {\sf Vector\_dup}.
    
\end{itemize}

\paragraph{Description}

Removes {\em contents} of an existing vector. After the call {\sf GrB\_Vector\_clear(\&v)},
$\bold{L}(\vector{v}) = \emptyset$. The size of the vector does not change. 

%-----------------------------------------------------------------------------
\subsubsection{{\sf Vector\_size}: Size of a vector}

Retrieve the size of a vector.

\paragraph{\syntax}

\begin{verbatim}
        GrB_Info GrB_Vector_size(GrB_Index        *nsize,
                                 const GrB_Vector  v);
\end{verbatim}

\paragraph{Parameters}

\begin{itemize}[leftmargin=1.1in]
    \item[{\sf nsize}] ({\sf OUT}) On successful return, is set to the size 
                                   of the vector.
    \item[{\sf v}]     ({\sf IN})  An existing GraphBLAS vector being queried.
\end{itemize}

\paragraph{Return Values}

\begin{itemize}[leftmargin=2.1in]
    \item[{\sf GrB\_SUCCESS}]   In blocking or non-blocking mode, the operation 
    completed successfully and the value of {\sf nsize} has been set.

    \item[{\sf GrB\_PANIC}]     Unknown internal error.
    
    \item[{\sf GrB\_INVALID\_OBJECT}] This is returned in any execution mode 
    whenever one of the opaque GraphBLAS objects (input or output) is in an invalid 
    state caused by a previous execution error.  Call {GrB\_error()} to access 
    any error messages generated by the implementation.

    \item[{\sf GrB\_UNINITIALIZED\_OBJECT}]  The GraphBLAS vector, {\sf v}, has 
    not been initialized by a call to {\sf Vector\_new} or {\sf Vector\_dup}.
    
    \item[{\sf GrB\_NULL\_POINTER}]  {\sf nsize} pointer is {\sf NULL}.
\end{itemize}

\paragraph{Description}

Return $\bold{size}({\sf v})$ in {\sf nsize}.

%-----------------------------------------------------------------------------
\subsubsection{{\sf Vector\_nvals}: Number of stored elements in a vector}
\label{Sec:Vector_nvals}

Retrieve the number of stored elements (tuples) in a vector.

\paragraph{\syntax}

\begin{verbatim}
        GrB_Info GrB_Vector_nvals(GrB_Index        *nvals,
                                  const GrB_Vector  v);
\end{verbatim}

\paragraph{Parameters}

\begin{itemize}[leftmargin=1.1in]
    \item[{\sf nvals}] ({\sf OUT}) On successful return, is set to the number of 
                                   stored elements (tuples) in the vector.
    \item[{\sf v}]     ({\sf IN})  An existing GraphBLAS vector being queried.
\end{itemize}


\paragraph{Return Values}

\begin{itemize}[leftmargin=2.1in]
    \item[{\sf GrB\_SUCCESS}]  In blocking or non-blocking mode, the operation 
    completed successfully and the value of {\sf nvals} has been set. 

    \item[{\sf GrB\_PANIC}]    Unknown internal error.
    
    \item[{\sf GrB\_INVALID\_OBJECT}] This is returned in any execution mode 
    whenever one of the opaque GraphBLAS objects (input or output) is in an invalid 
    state caused by a previous execution error.  Call {GrB\_error()} to access 
    any error messages generated by the implementation.

    \item[{\sf GrB\_OUT\_OF\_MEMORY}] Not enough memory available for operation.
    
    \item[{\sf GrB\_UNINITIALIZED\_OBJECT}]  The GraphBLAS vector, {\sf v}, has 
    not been initialized by a call to {\sf Vector\_new} or {\sf Vector\_dup}.
    
    \item[{\sf GrB\_NULL\_POINTER}]  The {\sf nvals} pointer is {\sf NULL}.
\end{itemize}

\paragraph{Description}


Return $\bold{nvals}({\sf v})$ in {\sf nvals}. This is the number of stored 
elements in vector {\sf v} (the size of $\bold{L}(\vector{v})$ in 
Section~\ref{Sec:Vectors}).

%-----------------------------------------------------------------------------

\subsubsection{{\sf Vector\_build}: Store elements from tuples into a vector}
\label{Sec:Vector_build}

\paragraph{\syntax}

\begin{verbatim}
        GrB_Info GrB_Vector_build(GrB_Vector            w,
                                  const GrB_Index       *indices,
                                  const <type>          *values,
                                  GrB_Index              nvals,
                                  const GrB_BinaryOp     dup);
\end{verbatim}

\paragraph{Parameters}

\begin{itemize}[leftmargin=1.1in]
    \item[{\sf w}]       ({\sf INOUT}) An existing Vector object to store the result.
    \item[{\sf indices}] ({\sf IN}) Pointer to an array of indices. 
    \item[{\sf values}]  ({\sf IN}) Pointer to an array of scalars of a type that
                                     is compatible with the domain of vector {\sf w}.
    \item[{\sf nvals}]   ({\sf IN}) The number of entries contained in each array (the same for \arg{indices} and \arg{values}).
    \item[{\sf dup}]     ({\sf IN}) An associative and commutative binary operator to apply when duplicate values for
	    the same location are present in the input arrays. All three domains of {\sf dup} must be the same; hence
	$dup=\langle D_{dup},D_{dup},D_{dup},\oplus \rangle$.
\end{itemize}

\paragraph{Return Values}

\begin{itemize}[leftmargin=2.1in]
    \item[{\sf GrB\_SUCCESS}]         In blocking mode, operation completed
    successfully. In non-blocking mode, this indicates that the API checks 
    for the input arguments passed successfully. Either way, output vector 
    {\sf w} is ready to be used in the next method of the sequence.

    \item[{\sf GrB\_PANIC}]           Unknown internal error.
    
    \item[{\sf GrB\_INVALID\_OBJECT}] This is returned in any execution mode 
    whenever one of the opaque GraphBLAS objects (input or output) is in an invalid 
    state caused by a previous execution error.  Call {GrB\_error()} to access 
    any error messages generated by the implementation.

    \item[{\sf GrB\_OUT\_OF\_MEMORY}] Not enough memory available for operation.
    
    \item[{\sf GrB\_UNINITIALIZED\_OBJECT}]  Either GraphBLAS object, {\sf w} or
    {\sf dup}, has not been initialized by a call to its respective {\sf new} method (or
    {\sf by GrB\_Vector\_dup} for {\sf w}).
    
    \item[{\sf GrB\_NULL\_POINTER}]  {\sf indices} or {\sf values} 
    pointer is {\sf NULL}.

    \item[{\sf GrB\_INDEX\_OUT\_OF\_BOUNDS}] A value in {\sf indices} is outside 
    the allowed range for {\sf w}.
    
	\item[{\sf GrB\_DOMAIN\_MISMATCH}]    Either the domains of {\sf values} and {\sf w}
		are incompatible with each other, or the domains of the GraphBLAS binary operator {\sf dup} are not all the same.
	
	\item[{\sf GrB\_OUTPUT\_NOT\_EMPTY}]    Output vector {\sf w} already contains valid tuples (elements).
	In other words, {\sf GrB\_Vector\_nvals(C)} returns a positive value.
\end{itemize}

\paragraph{Description}

An internal vector  $\vector{\widetilde{w}} = \langle D_{dup},\bold{size}({\sf w}),\emptyset \rangle$ is created, which only differs from ${\sf w}$ in its domain.

Each tuple $\{ {\sf indices[k]}, {\sf values[k]}\}$, where $1\leq k \leq {\sf nvals}$, is a contribution to the output in the form of 

$$\vector{\widetilde{w}}({\sf indices[k]}) = (D_{dup})\, {\sf values[k]}.$$

If multiple values for the same location are present in the input arrays, the 
{\sf dup} binary operand is used to reduce them before assignment into $\vector{\widetilde{w}}$. 

More generally,

\[
\vector{\widetilde{w}}_{i}
= \bigoplus_{k:\, {\sf indices[k]} = i}  (D_{dup})\, {\sf values[k]}
,\] 

where $\oplus$ is the {\sf dup} binary operator. Finally, the resulting $\vector{\widetilde{w}}$ is copied into ${\sf w}$ via typecasting its values. 

If $\oplus$ is not associative or not commutative, the result is undefined.  The nonopaque input arrays {\sf indices} and {\sf values} should be of the same length {\sf nvals}.
The nonopaque input arrays {\sf indices} and {\sf values} are available to be modified by the user on return from this method.

It is an error to call this function on an output object with existing elements. In other words, 
{\sf GrB\_Vector\_nvals(C)} should evaluate to zero prior to calling this function.

After {\sf GrB\_Vector\_build} returns, it is safe for a programmer to 
modify or delete the arrays {\sf indices} or {\sf values}.


%-----------------------------------------------------------------------------
\subsubsection{{\sf Vector\_setElement}: Set a single element in a vector}

Set one element of a vector to a given value.

\paragraph{\syntax}

\begin{verbatim}
        GrB_Info GrB_Vector_setElement(GrB_Vector   w,
                                       <type>       val,
                                       GrB_Index    index);
\end{verbatim}

\paragraph{Parameters}

\begin{itemize}[leftmargin=1.1in]
    \item[{\sf w}]   ({\sf INOUT}) An existing GraphBLAS vector for which an 
    element is to be assigned.

    \item[{\sf val}]   ({\sf IN}) Value to assign.  The type must
    be compatible with the domain of {\sf w}.

    \item[{\sf index}] ({\sf IN}) The location of the element to be assigned.
\end{itemize}

\paragraph{Return Values}

\begin{itemize}[leftmargin=2.1in]
    \item[{\sf GrB\_SUCCESS}]         In blocking mode, the operation completed
    successfully. In non-blocking mode, this indicates that the compatibility 
    tests on index/dimensions and domains for the input arguments passed successfully. 
    Either way, output vector {\sf w} is ready to be used in the next method of 
    the sequence.

    \item[{\sf GrB\_PANIC}]   Unknown internal error.
    
    \item[{\sf GrB\_INVALID\_OBJECT}] This is returned in any execution mode 
    whenever one of the opaque GraphBLAS objects (input or output) is in an invalid 
    state caused by a previous execution error.  Call {GrB\_error()} to access 
    any error messages generated by the implementation.

    \item[{\sf GrB\_OUT\_OF\_MEMORY}]  Not enough memory available for operation.
    
    \item[{\sf GrB\_UNINITIALIZED\_OBJECT}]  The GraphBLAS vector, {\sf w}, has 
    not been initialized by a call to {\sf Vector\_new} or {\sf Vector\_dup}.
    
    \item[{\sf GrB\_INVALID\_INDEX}]  {\sf index} specifies a location 
    that is outside the dimensions of {\sf w}.

    \item[{\sf GrB\_DOMAIN\_MISMATCH}]     The domains of the vector or scalar
    are incompatible.
\end{itemize}

\paragraph{Description}

First, the scalar and output vector are tested for domain compatibility as follows:
$\bold{D}({\sf val})$ must be compatible with $\bold{D}({\sf w})$. Two domains 
are compatible with each other if values from one domain can be cast to values 
in the other domain as per the rules of the C language. In particular, domains 
from Table~\ref{Tab:PredefinedTypes} are all compatible with each other. A domain 
from a user-defined type is only compatible with itself. If any compatibility 
rule above is violated, execution of {\sf GrB\_Vector\_setElement} ends and 
the domain mismatch error listed above is returned.

Then, the {\sf index} parameter is checked for a valid value where the following
condition must hold:
\[
	0\ \leq\ {\sf index}\ <\ \bold{size}({\sf w})
\]
If this condition is violated, execution of {\sf GrB\_Vector\_extractElement} 
ends and the invalid index error listed above is returned.

We are now ready to carry out the assignment {\sf val}; that is:
\[
    {\sf w}({\sf index}) = {\sf val}
\]
If a value existed at this location in {\sf w}, it will be overwritten; otherwise,
and new value is stored in {\sf w}.

In {\sf GrB\_BLOCKING} mode, the method exits with return value 
{\sf GrB\_SUCCESS} and the new contents of {\sf w} is as defined above
and fully computed.  
In {\sf GrB\_NONBLOCKING} mode, the method exits with return value 
{\sf GrB\_SUCCESS} and the new content of vector {\sf w} is as defined above 
but may not be fully computed; however, it can be used in the next GraphBLAS 
method call in a sequence.


%-----------------------------------------------------------------------------

\subsubsection{{\sf Vector\_extractElement}: Extract a single element from a vector.}
\label{Sec:extract_single_element_vec}

Extract one element of a vector into a scalar. 

\paragraph{\syntax}

\begin{verbatim}
        GrB_Info GrB_Vector_extractElement(<type>           *val,
                                           const GrB_Vector  u,
                                           GrB_Index         index); 
\end{verbatim}

\paragraph{Parameters}

\begin{itemize}[leftmargin=1in]
    \item[{\sf val}]   ({\sf INOUT}) Pointer to a scalar of type that is compatible with the domain of vector w. 
    On successful return, this scalar holds the result of the operation. Any previous value is 
    overwritten.

    \item[{\sf u}]     ({\sf IN}) The GraphBLAS vector from which an element
    is extracted.
    
    \item[{\sf index}] ({\sf IN}) The location in {\sf u} to extract.
\end{itemize}

\paragraph{Return Values}

\begin{itemize}[leftmargin=2.1in]
    \item[{\sf GrB\_SUCCESS}]  In blocking or non-blocking mode, the operation 
    completed successfully. This indicates that the compatibility tests on 
    dimensions and domains for the input arguments passed successfully, and
    the output scalar, {\sf val}, has been computed and is ready to be used in 
    the next method of the sequence.

    \item[{\sf GrB\_PANIC}]   Unknown internal error.
    
    \item[{\sf GrB\_INVALID\_OBJECT}] This is returned in any execution mode 
    whenever one of the opaque GraphBLAS objects (input or output) is in an invalid 
    state caused by a previous execution error.  Call {GrB\_error()} to access 
    any error messages generated by the implementation.

    \item[{\sf GrB\_OUT\_OF\_MEMORY}]  Not enough memory available for operation.
   % \scott{Is this error possible?}
    %\aydin{I think it might be possible. We don't know the internal "extract" algorithm so it might not be an "in-place" algorithm. Better safe than sorry}
    
    \item[{\sf GrB\_UNINITIALIZED\_OBJECT}]  The GraphBLAS vector, {\sf u}, has 
    not been initialized by a call to {\sf Vector\_new} or {\sf Vector\_dup}.
    
    \item[{\sf GrB\_NULL\_POINTER}]    {\sf val} pointer is {\sf NULL}.

    \item[{\sf GrB\_NO\_VALUE}]  There is no stored value at specified location.
    
    \item[{\sf GrB\_INVALID\_INDEX}]  {\sf index} specifies a location 
    that is outside the dimensions of {\sf w}.

    \item[{\sf GrB\_DOMAIN\_MISMATCH}]     The domains of the vector or scalar
    are incompatible.
\end{itemize}

\paragraph{Description}

First, the scalar and input vector are tested for domain compatibility as follows:
$\bold{D}({\sf val})$ must be compatible with $\bold{D}({\sf u})$. Two domains 
are compatible with each other if values from one domain can be cast to values 
in the other domain as per the rules of the C language. In particular, domains 
from Table~\ref{Tab:PredefinedTypes} are all compatible with each other. A domain 
from a user-defined type is only compatible with itself. If any compatibility 
rule above is violated, execution of {\sf GrB\_Vector\_extractElement} ends and 
the domain mismatch error listed above is returned.

Then, the {\sf index} parameter is checked for a valid value where the following
condition must hold:
\[
	0\ \leq\ {\sf index}\ <\ \bold{size}({\sf u})
\]
If this condition is violated, execution of {\sf GrB\_Vector\_extractElement} 
ends and the invalid index error listed above is returned.

We are now ready to carry out the extract into the output argument, {\sf val};  
that is:
\[
    {\sf val} = {\sf u}({\sf index})
\]
where the following condition must be true:
\[
    {\sf index} \in \bold{ind}({\sf u})
\]
If this condition is violated, execution of {\sf GrB\_Vector\_extractElement} 
ends and the "no value" error listed above is returned.

In both {\sf GrB\_BLOCKING} mode {\sf GrB\_NONBLOCKING} mode
if the method exits with return value {\sf GrB\_SUCCESS}, the  new 
contents of {\sf val} are as defined above.  In other words, the method
does not return until any operations required to fully compute 
the GraphBLAS vector {\sf u} have completed. 

In {\sf GrB\_NONBLOCKING} mode, if the return value is 
not {\sf GrB\_SUCCESS}, an error in a method occurring earlier in the sequence
may have occurred that prevents completion of the GraphBLAS vector {\sf u}.
The GrB\_error() method should be called for additional information 
about these errors.


%-----------------------------------------------------------------------------

\subsubsection{{\sf Vector\_extractTuples}: Extract tuples from a vector}
\label{Sec:Vector_extractTuples}

Extract the contents of a GraphBLAS vector into non-opaque data structures.

\paragraph{\syntax}

\begin{verbatim}
        GrB_Info GrB_Vector_extractTuples(GrB_Index            *indices,
                                          <type>               *values, 
                                          const GrB_Vector      v);

\end{verbatim}

\begin{itemize}[leftmargin=1.1in]
    \item[{\sf indices}] ({\sf OUT}) Pointer to an array of indices that is sufficient to
                        hold all of the stored values' indices (no checking is performed).
    \item[{\sf values}] ({\sf OUT}) Pointer to an array of scalars of a type that is sufficient to
                        hold all of the stored values (no checking is performed) whose
                        type is compatible with $\bold{D}(\vector{v})$.
    \item[{\sf v}]      ({\sf IN})  An existing GraphBLAS vector.
\end{itemize}

\paragraph{Return Values}

\begin{itemize}[leftmargin=2.1in]
    \item[{\sf GrB\_SUCCESS}]  In blocking or non-blocking mode, the operation 
    completed successfully. This indicates that the compatibility tests on 
    the input argument passed successfully, and the output arrays, {\sf indices}
    and {\sf values}, have been computed.

    \item[{\sf GrB\_PANIC}]   Unknown internal error.
    
    \item[{\sf GrB\_INVALID\_OBJECT}] This is returned in any execution mode 
    whenever one of the opaque GraphBLAS objects (input or output) is in an invalid 
    state caused by a previous execution error.  Call {GrB\_error()} to access 
    any error messages generated by the implementation.

    \item[{\sf GrB\_OUT\_OF\_MEMORY}]  Not enough memory available for operation.
    %\scott{Is this error possible?}
    %\aydin{I think here it is really possible. We don't know the internal "extracttuples" algorithm so it might not be an "in-place" algorithm. Better safe than sorry}
    
    \item[{\sf GrB\_UNINITIALIZED\_OBJECT}]  The GraphBLAS vector, {\sf v}, has 
    not been initialized by a call to {\sf Vector\_new} or {\sf Vector\_dup}.
    
    \item[{\sf GrB\_NULL\_POINTER}] {\sf indices} or {\sf values} pointer is {\sf NULL}.
     
    \item[{\sf GrB\_DOMAIN\_MISMATCH}] The domains of the {\sf v} vector or 
    {\sf values} array are incompatible with one another.
\end{itemize}


\paragraph{Description}


This method will extract all the tuples from the GraphBLAS vector {\sf v}.  
The values associated with those tuples are placed in the
array {\sf values} and the indices are placed in the array {\sf indices}. 
Both {\sf indices} and {\sf values} pre-allocated by the user to have enough
space to hold at least {\sf GrB\_Vector\_nvals(v)} elements before calling
this function. 

Upon return of this function, entries of {\sf indices} are unique, but they do not have to be sorted.
Each tuple $(i,v_i)$ in {\sf v} is unzipped and copied into a distinct $k$th location in output vectors:
$$ \{{\sf indices[k]}, {\sf values[k]}\} \leftarrow (i,v_i),$$
where $0 \leq k < {\sf GrB\_Vector\_nvals(v)}$. No gaps in
output vectors are allowed; i.e.\, if {\sf indices[k]} and {\sf values[k]} exist upon return, so does
{\sf indices[j]} and {\sf values[j]} for all $j$ such that $0 \leq j < k$.


In both {\sf GrB\_BLOCKING} mode {\sf GrB\_NONBLOCKING} mode
if the method exits with return value {\sf GrB\_SUCCESS}, the  new 
contents of the arrays {\sf indices} and {\sf values} are as defined above.  In other words, the method
does not return until any operations required to fully compute 
the GraphBLAS vector {\sf v} have completed. 

In {\sf GrB\_NONBLOCKING} mode, if the return value is 
not {\sf GrB\_SUCCESS}, an error in a method occurring earlier in the sequence
may have occurred that prevents completion of the GraphBLAS vector {\sf v}.
The GrB\_error() method should be called for additional information 
about these errors.

%-----------------------------------------------------------------------------
\subsubsection{{\sf Vector\_assign}: Standard vector variant}

Assign values (and implied zeros) from one GraphBLAS vector to a subset of a 
vector as specified by a set of indices. The size of the input vector is the
same size as the index array provided.

\paragraph{\syntax}

\begin{verbatim}
        GrB_Info GrB_Vector_assign(GrB_Vector              w,
                                   const GrB_Vector        u,
                                   const GrB_Index        *indices,
                                   const GrB_Index         nindices);
\end{verbatim}

\paragraph{Parameters}

\begin{itemize}[leftmargin=1in]
    \item[{\sf w}] ({\sf INOUT}) An existing GraphBLAS vector, that will
    be modified by the assign.  On output, this vector holds the results
    of the operation.

    \item[{\sf u}] ({\sf IN}) The GraphBLAS vector whose contents are
    assigned to a subset of {\sf w}.

    \item[{\sf indices}] ({\sf IN}) Pointer to the ordered set (array)
    of indices corresponding to the locations in {\sf w} that are to be
    assigned.  If all elements of {\sf w} are to be assigned in order from
    $0$ to ${\sf nindices} - 1$, then {\sf GrB\_ALL} should be specified.
    Regardless of execution mode and return value, this array may be
    manipulated by the caller after this operation returns without
    affecting any deferred computations for this operation.  If this
    array contains duplicate values, it implies in assignment of more
    than one value to the same location which leads to undefined results.

    \item[{\sf nindices}] ({\sf IN}) The number of values in {\sf
    indices} array.  Must be equal to $\bold{size}({\sf u})$.
\end{itemize}

\paragraph{Return Values}

\begin{itemize}[leftmargin=2.1in]
    \item[{\sf GrB\_SUCCESS}]         In blocking mode, the operation completed
    successfully. In non-blocking mode, this indicates that the compatibility 
    tests on dimensions and domains for the input arguments passed successfully. 
    Either way, output vector {\sf w} is ready to be used in the next method of 
    the sequence.

    \item[{\sf GrB\_PANIC}]            Unknown internal error.
    
    \item[{\sf GrB\_INVALID\_OBJECT}] This is returned in any execution mode 
    whenever one of the opaque GraphBLAS objects (input or output) is in an invalid 
    state caused by a previous execution error.  Call {\sf GrB\_error()} to access 
    any error messages generated by the implementation.

    \item[{\sf GrB\_OUT\_OF\_MEMORY}]  Not enough memory available for operation.
    
    \item[{\sf GrB\_UNINITIALIZED\_OBJECT}] One or more of the GraphBLAS objects
    has not been initialized by a call to {\sf new} (or {\sf dup} for vector
    parameters).

    \item[{\sf GrB\_INDEX\_OUT\_OF\_BOUNDS}]  A value in {\sf indices} is greater
    than or equal to $\bold{size}({\sf w})$.  In non-blocking mode, this can be
    reported as an execution error.
    
    \item[{\sf GrB\_DIMENSION\_MISMATCH}] ${\sf nindices} \neq \bold{size}({\sf u})$. 
    
    \item[{\sf GrB\_DOMAIN\_MISMATCH}]    The domains of the various vectors are
	incompatible with each other.

    \item[{\sf GrB\_NULL\_POINTER}] Argument {\sf indices} is a {\sf NULL} pointer.
\end{itemize}

\paragraph{Description}

This variant of {\sf GrB\_assign} computes the result of assigning values of elements from 
a source GraphBLAS vector to elements of a destination GraphBLAS vector. 
More explicitly:
\[
\begin{aligned}
	{\sf w}({\sf indices}[i]) = &\ {\sf u}(i),
    \forall i : 0 \leq i < {\sf nindices}.
\end{aligned}
\]  
If a particular element ${\sf u}(i)$ does not exist, the corresponding
element of {\sf w} is also removed. 
Elements of {\sf w} outside the range $({\sf indices}[i])$ are not affected.

Two argument vectors are used in the {\sf GrB\_assign} operation:
\begin{enumerate}
	\item ${\sf w} = \langle \bold{D}({\sf w}),\bold{size}({\sf w}),
    \bold{L}({\sf w}) = \{(i,w_i) \} \rangle$
    
	\item ${\sf u} = \langle \bold{D}({\sf u}),\bold{size}({\sf u}),
    \bold{L}({\sf u}) = \{(i,u_i) \} \rangle$
\end{enumerate}

The argument vectors 
are tested for domain compatibility as follows:
\begin{enumerate}
	\item $\bold{D}({\sf w})$ must be compatible with $\bold{D}({\sf u})$.
\end{enumerate}
Two domains are compatible with each other if values from one domain can be cast 
to values in the other domain as per the rules of the C language.
In particular, domains from Table~\ref{Tab:PredefinedTypes} are all compatible 
with each other. A domain from a user-defined type is only compatible with itself.
If any compatibility rule above is violated, execution of {\sf GrB\_assign} ends
and the domain mismatch error listed above is returned.

From the arguments, the internal index array used in 
the computation is formed ($\leftarrow$ denotes copy):
\begin{enumerate}
    \item The internal index array, $\grbarray{\widetilde{I}}$, is computed from 
    argument {\sf indices} as follows:
	\begin{enumerate}
		\item	If ${\sf indices} = {\sf GrB\_ALL}$, then 
        $\grbarray{\widetilde{I}}[i] = i, \forall i : 0 \leq i < {\sf nindices}$.

		\item	Otherwise, $\grbarray{\widetilde{I}}[i] = {\sf indices}[i], 
        \forall i : 0 \leq i < {\sf nindices}$.
    \end{enumerate}
\end{enumerate}

The vectors are checked for dimension compatibility. 
The following conditions must hold:
\begin{enumerate}
    \item ${\sf nindices} = \bold{size}(\vector{\widetilde{u}})$.
\end{enumerate}
If any compatibility rule above is violated, execution of {\sf GrB\_assign} ends and 
the dimension mismatch error listed above is returned.

From this point forward, in {\sf GrB\_NONBLOCKING} mode, the method can 
optionally exit with {\sf GrB\_SUCCESS} return code and defer any computation 
and/or execution error codes.

We are now ready to carry out the assign.
At this point, if any value of $\grbarray{\widetilde{I}}[i]$ is outside the valid 
range of indices for vector $\vector{\widetilde{w}}$, computation ends and the 
method returns the index-out-of-bounds error listed above. In 
{\sf GrB\_NONBLOCKING} mode, the error can be deferred until a 
sequence-terminating {\sf GrB\_wait()} is called.  Regardless, the result 
vector, {\sf w}, is invalid from this point forward in the 
sequence.

Executing the assign consists of updating the contents of vector {\sf w}
according to the following expression:
\[
	\bold{L}({\sf w}) = (\bold{L}({\sf w}) - \{(\grbarray{\widetilde{I}}[i],w) \forall i : \grbarray{\widetilde{I}}[i] \in \bold{ind}({\sf w})\})
	\cup \{(\grbarray{\widetilde{I}}[i],u_i) \forall i \in \bold{ind}({\sf u})\},
\]
where the difference operator in the previous expressions refers to set difference.
That is, first existing elements of {\sf w} with index in $\grbarray{\widetilde{I}}$ are removed,
and then the elements of {\sf u} are added in the right place.

In {\sf GrB\_BLOCKING} mode, the method exits with return value 
{\sf GrB\_SUCCESS} and the new content of vector {\sf w} is as defined above
and fully computed.  
In {\sf GrB\_NONBLOCKING} mode, the method exits with return value 
{\sf GrB\_SUCCESS} and the new content of vector {\sf w} is as defined above 
but may not be fully computed; however, it can be used in the next GraphBLAS 
method call in a sequence.

%-----------------------------------------------------------------------------

\subsubsection{{\sf Vector\_assign}: Constant vector variant}

Assign the same value to a specified subset of a GraphBLAS vector.  With the use of {\sf GrB\_ALL}, the 
entire destination vector can be filled with the constant.

\paragraph{\syntax}

\begin{verbatim}
        GrB_Info GrB_Vector_assign(GrB_Vector              w,
                                   <type>                  val,
                                   const GrB_Index        *indices,
                                   const GrB_Index         nindices);
\end{verbatim}

\paragraph{Parameters}

\begin{itemize}[leftmargin=1.1in]
    \item[{\sf w}] ({\sf INOUT}) An existing GraphBLAS vector, that will
    be modified by the assign.  On output, this vector holds the results
    of the operation.

    \item[{\sf val}] ({\sf IN}) Scalar value to assign to (a subset of)
    {\sf w}.

    \item[{\sf indices}]  ({\sf IN}) Pointer to the ordered set (array)
    of indices corresponding to the locations in {\sf w} that are to
    be assigned.  If all elements of {\sf w} are to be assigned in
    order from $0$ to ${\sf nindices} - 1$, then {\sf GrB\_ALL} should
    be specified.  Regardless of execution mode and return value, this
    array may be manipulated by the caller after this operation returns
    without affecting any deferred computations for this operation.
    In this variant, the specific order of the values in the array has no
    effect on the result.  Unlike other variants, if there are duplicated
    values in this array the result is still defined.

    \item[{\sf nindices}] ({\sf IN}) The number of values in {\sf
    indices} array.  Must be in the range: $[0, \bold{size}({\sf w}))$.
    If {\sf nindices} is zero, the operation becomes a NO-OP.
\end{itemize}

\paragraph{Return Values}

\begin{itemize}[leftmargin=2.1in]
    \item[{\sf GrB\_SUCCESS}]         In blocking mode, the operation completed
    successfully. In non-blocking mode, this indicates that the compatibility 
    tests on dimensions and domains for the input arguments passed successfully. 
    Either way, output vector {\sf w} is ready to be used in the next method of 
    the sequence.

    \item[{\sf GrB\_PANIC}]            Unknown internal error.
    
    \item[{\sf GrB\_INVALID\_OBJECT}] This is returned in any execution mode 
    whenever one of the opaque GraphBLAS objects (input or output) is in an invalid 
    state caused by a previous execution error.  Call {\sf GrB\_error()} to access 
    any error messages generated by the implementation.

    \item[{\sf GrB\_OUT\_OF\_MEMORY}]  Not enough memory available for operation.
    
    \item[{\sf GrB\_UNINITIALIZED\_OBJECT}] One or more of the GraphBLAS objects
    has not been initialized by a call to {\sf new} (or {\sf dup} for vector
    parameters).

    \item[{\sf GrB\_INDEX\_OUT\_OF\_BOUNDS}]  A value in {\sf indices} is greater
    than or equal to $\bold{size}({\sf w})$.  In non-blocking mode, this can be
    reported as an execution error.
    
    \item[{\sf GrB\_DIMENSION\_MISMATCH}] {\sf nindices} is not less than $\bold{size}({\sf w})$. 

    \item[{\sf GrB\_DOMAIN\_MISMATCH}]    The domains of the vector and scalar are
	incompatible with each other.

    \item[{\sf GrB\_NULL\_POINTER}] Argument {\sf indices} is a {\sf NULL} pointer.
\end{itemize}


\paragraph{Description}

This variant of {\sf GrB\_assign} computes the result of assigning a constant
scalar value to locations in a destination GraphBLAS vector: 
\[
\begin{aligned}
	{\sf w}({\sf indices}[i]) = &\ {\sf val}, \ 
    \forall \  i : 0 \leq i < {\sf nindices}.
\end{aligned}
\]  

One argument vectors are used in the {\sf GrB\_assign} operation:
\begin{enumerate}
	\item ${\sf w} = \langle \bold{D}({\sf w}),\bold{size}({\sf w}),
    \bold{L}({\sf w}) = \{(i,w_i) \} \rangle$
\end{enumerate}

The argument scalar and vectors
are tested for domain compatibility as follows:
\begin{enumerate}
	\item $\bold{D}({\sf w})$ must be compatible with $\bold{D}({\sf val})$.
\end{enumerate}
Two domains are compatible with each other if values from one domain can be cast 
to values in the other domain as per the rules of the C language.
In particular, domains from Table~\ref{Tab:PredefinedTypes} are all compatible 
with each other. A domain from a user-defined type is only compatible with itself.
If any compatibility rule above is violated, execution of {\sf GrB\_assign} ends
and the domain mismatch error listed above is returned.

From the arguments, the internal index array used in 
the computation is formed ($\leftarrow$ denotes copy):
\begin{enumerate}
    \item The internal index array, $\grbarray{\widetilde{I}}$, is computed from 
    argument {\sf indices} as follows:
	\begin{enumerate}
		\item	If ${\sf indices} = {\sf GrB\_ALL}$, then 
        $\grbarray{\widetilde{I}}[i] = i, \ \forall \ i : 0 \leq i < {\sf nindices}$.

		\item	Otherwise, $\grbarray{\widetilde{I}}[i] = {\sf indices}[i], 
        \ \forall \ i : 0 \leq i < {\sf nindices}$.
    \end{enumerate}
\end{enumerate}

If any compatibility rule above is violated, execution of {\sf GrB\_assign} ends and 
the dimension mismatch error listed above is returned.

From this point forward, in {\sf GrB\_NONBLOCKING} mode, the method can 
optionally exit with {\sf GrB\_SUCCESS} return code and defer any computation 
and/or execution error codes.

We are now ready to carry out the assign.
At this point, if any value of $\grbarray{\widetilde{I}}[i]$ is outside the valid 
range of indices for vector $\vector{\widetilde{w}}$, computation ends and the 
method returns the index-out-of-bounds error listed above. In 
{\sf GrB\_NONBLOCKING} mode, the error can be deferred until a 
sequence-terminating {\sf GrB\_wait()} is called.  Regardless, the result 
vector, {\sf w}, is invalid from this point forward in the 
sequence.

Executing the assign consists of updating the contents of vector {\sf w}
according to the following expression:
\[
	\bold{L}({\sf w}) = (\bold{L}({\sf w}) - \{(\grbarray{\widetilde{I}}[i],w) \forall i : \grbarray{\widetilde{I}}[i] \in \bold{ind}({\sf w})\})
	\cup \{(\grbarray{\widetilde{I}}[i],{\sf val}) \forall i \in \bold{ind}({\sf u})\},
\]
where the difference operator in the previous expressions refers to set difference.
That is, first existing elements of {\sf w} with index in $\grbarray{\widetilde{I}}$ are removed,
and then elements with a value of {\sf val} are added in the right place.

In {\sf GrB\_NONBLOCKING} mode, the method exits with return value 
{\sf GrB\_SUCCESS} and the new content of vector {\sf w} is as defined above 
but may not be fully computed; however, it can be used in the next GraphBLAS 
method call in a sequence.


%==============================================================================
\subsection{Matrix Methods}

%-----------------------------------------------------------------------------
\subsubsection{{\sf Matrix\_new}: Create new matrix}

Creates a new matrix with specified domain and dimensions.

\paragraph{\syntax}

\begin{verbatim}
        GrB_Info GrB_Matrix_new(GrB_Matrix *A,
                                GrB_Type    d,
                                GrB_Index   nrows,
                                GrB_Index   ncols);
\end{verbatim}

\paragraph{Parameters}

\begin{itemize}[leftmargin=1.1in]
    \item[{\sf A}] ({\sf INOUT}) On successful return, contains a handle to 
                                 the newly created GraphBLAS matrix.
    \item[{\sf d}] ({\sf IN})    The type corresponding to the domain of the matrix 
                                 being created. Can be one of the predefined
                                 GraphBLAS types in Table~\ref{Tab:PredefinedTypes}, 
                                 or an existing user-defined GraphBLAS type.
    \item[{\sf nrows}] ({\sf IN}) The number of rows of the matrix being created.
    \item[{\sf ncols}] ({\sf IN}) The number of columns of the matrix being created.
\end{itemize}


\paragraph{Return Values}

\begin{itemize}[leftmargin=2.1in]
    \item[{\sf GrB\_SUCCESS}]         In blocking mode, operation completed
    successfully. In non-blocking mode, this indicates that the API checks 
    for the input arguments passed successfully. Either way, output matrix 
    {\sf A} is ready to be used in the next method of the sequence.

    \item[{\sf GrB\_PANIC}]           Unknown internal error.
    
    \item[{\sf GrB\_INVALID\_OBJECT}] This is returned in any execution mode 
    whenever one of the opaque GraphBLAS objects (input or output) is in an invalid 
    state caused by a previous execution error.  Call {GrB\_error()} to access 
    any error messages generated by the implementation.

    \item[{\sf GrB\_OUT\_OF\_MEMORY}] Not enough memory available for operation.
    
    \item[{\sf GrB\_UNINITIALIZED\_OBJECT}]  The {\sf GrB\_Type} object has not 
    been initialized by a call to {\sf new} (needed for user-defined types).
    
    \item[{\sf GrB\_NULL\_POINTER}]  The {\sf A} pointer is {\sf NULL}.
    
    \item[{\sf GrB\_INVALID\_VALUE}] {\sf nrows} or {\sf ncols} is zero.
\end{itemize}

\paragraph{Description}

Creates a new matrix $\matrix{A}$ of domain $\bold{D}({\sf d})$, size 
{\sf nrows $\times$ ncols}, and empty $\bold{L}(\matrix{A})$. It returns a
handle to it in {\sf A}.

It is not an error to call this method more than once on the same variable;  
however, the handle to the previously created object will be overwritten. 

%-----------------------------------------------------------------------------
\subsubsection{{\sf Matrix\_dup}: Create a copy of a GraphBLAS matrix}

Creates a new matrix with the same domain, dimensions, and contents as 
another matrix.

\paragraph{\syntax}

\begin{verbatim}
        GrB_Info GrB_Matrix_dup(GrB_Matrix       *C,
                                const GrB_Matrix  A);
\end{verbatim}

\paragraph{Parameters}

\begin{itemize}[leftmargin=1.1in]
    \item[{\sf C}] ({\sf INOUT}) On successful return, contains a handle to 
                                 the newly created GraphBLAS matrix.
    \item[{\sf A}] ({\sf IN})    The GraphBLAS matrix to be duplicated.
\end{itemize}


\paragraph{Return Values}

\begin{itemize}[leftmargin=2.1in]
    \item[{\sf GrB\_SUCCESS}]         In blocking mode, operation completed
    successfully. In non-blocking mode, this indicates that the API checks 
    for the input arguments passed successfully. Either way, output matrix 
    {\sf C} is ready to be used in the next method of the sequence.

    \item[{\sf GrB\_PANIC}]           Unknown internal error.
    
    \item[{\sf GrB\_INVALID\_OBJECT}] This is returned in any execution mode 
    whenever one of the opaque GraphBLAS objects (input or output) is in an invalid 
    state caused by a previous execution error.  Call {GrB\_error()} to access 
    any error messages generated by the implementation.

    \item[{\sf GrB\_OUT\_OF\_MEMORY}] Not enough memory available for operation.
    
    \item[{\sf GrB\_UNINITIALIZED\_OBJECT}]  The GraphBLAS matrix, {\sf A}, has 
    not been initialized by a call to {\sf Matrix\_new} or {\sf Matrix\_dup}.
    
    \item[{\sf GrB\_NULL\_POINTER}]   The {\sf C} pointer is {\sf NULL}.
\end{itemize}

\paragraph{Description}

Creates a new matrix $\matrix{C}$ of domain $\bold{D}({\sf A})$, size 
$\bold{nrows}({\sf A}) \times \bold{ncols}({\sf A})$, and contents 
$\bold{L}({\sf A})$. It returns a handle to it in {\sf C}.

It is not an error to call this method more than once on the same variable;  
however, the handle to the previously created object will be overwritten. 

%-----------------------------------------------------------------------------
\subsubsection{{\sf Matrix\_clear}: Clear a matrix}

Removes all elements from a matrix.

\paragraph{\syntax}

\begin{verbatim}
        GrB_Info GrB_Matrix_clear(GrB_Matrix A);
\end{verbatim}

\paragraph{Parameters}

\begin{itemize}[leftmargin=1.1in]
    \item[{\sf A}] ({\sf IN}) An exising GraphBLAS matrix to clear.
\end{itemize}

\paragraph{Return Values}

\begin{itemize}[leftmargin=2.1in]
    \item[{\sf GrB\_SUCCESS}]         In blocking mode, operation completed
    successfully. In non-blocking mode, this indicates that the API checks 
    for the input arguments passed successfully. Either way, output matrix 
    {\sf A} is ready to be used in the next method of the sequence.

    \item[{\sf GrB\_PANIC}]           Unknown internal error.
    
    \item[{\sf GrB\_INVALID\_OBJECT}] This is returned in any execution mode 
    whenever one of the opaque GraphBLAS objects (input or output) is in an invalid 
    state caused by a previous execution error.  Call {GrB\_error()} to access 
    any error messages generated by the implementation.

    \item[{\sf GrB\_OUT\_OF\_MEMORY}] Not enough memory available for operation.
    
    \item[{\sf GrB\_UNINITIALIZED\_OBJECT}]  The GraphBLAS matrix, {\sf *A}, has 
    not been initialized by a call to {\sf Matrix\_new} or {\sf Matrix\_dup}.
    
\end{itemize}

\paragraph{Description}

Removes all elements (tuples) from an existing matrix.

%-----------------------------------------------------------------------------
\subsubsection{{\sf Matrix\_nrows}: Number of rows in a matrix}

Retrieve the number of rows in a matrix.

\paragraph{\syntax}

\begin{verbatim}
        GrB_Info GrB_Matrix_nrows(GrB_Index        *nrows,
                                  const GrB_Matrix  A);
\end{verbatim}

\paragraph{Parameters}

\begin{itemize}[leftmargin=1.1in]
    \item[{\sf nrows}] ({\sf OUT}) On successful return, contains the number of rows in the matrix.
    \item[{\sf A}] ({\sf IN}) An existing GraphBLAS matrix being queried.
\end{itemize}


\paragraph{Return Values}

\begin{itemize}[leftmargin=2.1in]
    \item[{\sf GrB\_SUCCESS}]   In blocking or non-blocking mode, the operation 
    completed successfully and the value of {\sf nrows} has been set.

    \item[{\sf GrB\_PANIC}]     Unknown internal error.
    
    \item[{\sf GrB\_INVALID\_OBJECT}] This is returned in any execution mode 
    whenever one of the opaque GraphBLAS objects (input or output) is in an invalid 
    state caused by a previous execution error.  Call {GrB\_error()} to access 
    any error messages generated by the implementation.

    \item[{\sf GrB\_UNINITIALIZED\_OBJECT}]  The GraphBLAS matrix, {\sf A}, has 
    not been initialized by a call to {\sf Matrix\_new} or {\sf Matrix\_dup}.
    
    \item[{\sf GrB\_NULL\_POINTER}]  {\sf nrows} pointer is {\sf NULL}.
\end{itemize}

\paragraph{Description}

Return $\bold{nrows}({\sf A})$ in {\sf nrows} (the number of rows).

%-----------------------------------------------------------------------------
\subsubsection{{\sf Matrix\_ncols}: Number of columns in a matrix}

Retrieve the number of columns in a matrix.

\paragraph{\syntax}

\begin{verbatim}
        GrB_Info GrB_Matrix_ncols(GrB_Index        *ncols,
                                  const GrB_Matrix  A);
\end{verbatim}

\paragraph{Parameters}

\begin{itemize}[leftmargin=1.1in]
    \item[{\sf ncols}] ({\sf OUT}) On successful return, contains the number of columns in the matrix.
    \item[{\sf A}] ({\sf IN}) An existing GraphBLAS matrix being queried.
\end{itemize}

\paragraph{Return Values}

\begin{itemize}[leftmargin=2.1in]
    \item[{\sf GrB\_SUCCESS}]   In blocking or non-blocking mode, the operation 
    completed successfully and the value of {\sf ncols} has been set.

    \item[{\sf GrB\_PANIC}]     Unknown internal error.
    
    \item[{\sf GrB\_INVALID\_OBJECT}] This is returned in any execution mode 
    whenever one of the opaque GraphBLAS objects (input or output) is in an invalid 
    state caused by a previous execution error.  Call {GrB\_error()} to access 
    any error messages generated by the implementation.

    \item[{\sf GrB\_UNINITIALIZED\_OBJECT}]  The GraphBLAS matrix, {\sf A}, has 
    not been initialized by a call to {\sf Matrix\_new} or {\sf Matrix\_dup}.
    
    \item[{\sf GrB\_NULL\_POINTER}]  {\sf ncols} pointer is {\sf NULL}.
\end{itemize}

\paragraph{Description}

Return $\bold{ncols}({\sf A})$ in {\sf ncols} (the number of columns).

%-----------------------------------------------------------------------------
\subsubsection{{\sf Matrix\_nvals}: Number of stored elements in a matrix}
\label{Sec:Matrix_nvals}

Retrieve the number of stored elements (tuples) in a matrix.

\paragraph{\syntax}

\begin{verbatim}
        GrB_Info GrB_Matrix_nvals(GrB_Index        *nvals,
                                  const GrB_Matrix  A);
\end{verbatim}

\paragraph{Parameters}

\begin{itemize}[leftmargin=1.1in]
    \item[{\sf nvals}] ({\sf OUT}) On successful return, contains the number of 
    stored elements (tuples) in the matrix.
    \item[{\sf A}] ({\sf IN}) An existing GraphBLAS matrix being queried.
\end{itemize}

\paragraph{Return Values}

\begin{itemize}[leftmargin=2.1in]
    \item[{\sf GrB\_SUCCESS}]  In blocking or non-blocking mode, the operation 
    completed successfully and the value of {\sf nvals} has been set. 

    \item[{\sf GrB\_PANIC}]    Unknown internal error.
    
    \item[{\sf GrB\_INVALID\_OBJECT}] This is returned in any execution mode 
    whenever one of the opaque GraphBLAS objects (input or output) is in an invalid 
    state caused by a previous execution error.  Call {GrB\_error()} to access 
    any error messages generated by the implementation.

    \item[{\sf GrB\_OUT\_OF\_MEMORY}] Not enough memory available for operation.
    
    \item[{\sf GrB\_UNINITIALIZED\_OBJECT}]  The GraphBLAS matrix, {\sf A}, has 
    not been initialized by a call to {\sf Matrix\_new} or {\sf Matrix\_dup}.
    
    \item[{\sf GrB\_NULL\_POINTER}]  The {\sf nvals} pointer is {\sf NULL}.
\end{itemize}

\paragraph{Description}

Return in {\sf nvals} the number of tuples (the size of $\bold{L}(\matrix{A})$
in Section~\ref{Sec:Matrices}) stored in matrix {\sf A}.

%-----------------------------------------------------------------------------

\subsubsection{{\sf Matrix\_build}: Store elements from tuples into a matrix}
\label{Sec:Matrix_build}

\paragraph{\syntax}

% AYDIN: Avoid page break due to preceding table
\begin{Verbatim}[samepage=true]    
        GrB_Info GrB_Matrix_build(GrB_Matrix             C,
                                  const GrB_Index       *row_indices,
                                  const GrB_Index       *col_indices, 
                                  const <type>          *values,
                                  GrB_Index              nvals,
                                  const GrB_BinaryOp     dup);
\end{Verbatim}

\paragraph{Parameters}

\begin{itemize}[leftmargin=1.1in]
    \item[{\sf C}]      ({\sf INOUT}) An existing Matrix object to store the result.
    \item[{\sf row\_indices}] ({\sf IN}) Pointer to an array of row indices. 
    \item[{\sf col\_indices}] ({\sf IN}) Pointer to an array of column indices. 
    \item[{\sf values}] ({\sf IN}) Pointer to an array of scalars of a type that
                                   is compatible with the domain of matrix, {\sf C}.
    \item[{\sf nvals}]  ({\sf IN}) The number of values contained in each array.
    \item[{\sf dup}]    ({\sf IN}) An associative and commutative binary function to apply when duplicate values 
                        for the same location are present in the input arrays.  
			All three domains of {\sf dup} must be the same; hence
			$dup=\langle D_{dup},D_{dup},D_{dup},\oplus \rangle$.
                        %\scott{Is {\sf GrB\_NULL} allowed?} \aydin{No, it doesn't make sense}
\end{itemize}

\paragraph{Return Values}

\begin{itemize}[leftmargin=2.1in]
    \item[{\sf GrB\_SUCCESS}]         In blocking mode, operation completed
    successfully. In non-blocking mode, this indicates that the API checks 
    for the input arguments passed successfully. Either way, output matrix 
    {\sf C} is ready to be used in the next method of the sequence.

    \item[{\sf GrB\_PANIC}]           Unknown internal error.
    
    \item[{\sf GrB\_INVALID\_OBJECT}] This is returned in any execution mode 
    whenever one of the opaque GraphBLAS objects (input or output) is in an invalid 
    state caused by a previous execution error.  Call {GrB\_error()} to access 
    any error messages generated by the implementation.

    \item[{\sf GrB\_OUT\_OF\_MEMORY}] Not enough memory available for operation.
    
    \item[{\sf GrB\_UNINITIALIZED\_OBJECT}]  Either GraphBLAS object, {\sf C} or
    {\sf dup}, has not been initialized by a call to its respective {\sf new} method (or
    {\sf by GrB\_Matrix\_dup} for {\sf C})
    
    \item[{\sf GrB\_NULL\_POINTER}]  {\sf row\_indices}, 
    {\sf col\_indices} or {\sf values} pointer is {\sf NULL}.

    \item[{\sf GrB\_INDEX\_OUT\_OF\_BOUNDS}] A value in {\sf indices} is outside 
    the allowed range for {\sf C}.
    
	\item[{\sf GrB\_DOMAIN\_MISMATCH}]    Either the domains of {\sf values} and {\sf C}
	are incompatible with each other, or the domains of the GraphBLAS binary operator {\sf dup} are not all the same.
	
	\item[{\sf GrB\_OUTPUT\_NOT\_EMPTY}]    Output matrix {\sf C} already contains valid tuples (elements).
	In other words, {\sf GrB\_Matrix\_nvals(C)} returns a positive value.
\end{itemize}

\paragraph{Description}

 An internal matrix $\matrix{\widetilde{C}} = \langle D_{dup},
     \bold{nrows}({\sf C}),
     \bold{ncols}({\sf C}),\emptyset \rangle$ is created, which only differs from ${\sf C}$ in its domain.

Each tuple $\{ {\sf row\_indices[k]}, {\sf col\_indices[k]}, {\sf values[k]}\}$, where $1\leq k \leq {\sf nvals}$, is a contribution to the output in the form of 

$$\matrix{\widetilde{C}}({\sf row\_indices[k]}, {\sf col\_indices[k]}) =  (D_{dup})\, {\sf values[k]}.$$

If multiple values for the same location are present in the input arrays, the 
{\sf dup} binary operand is used to reduce them before assignment into $\matrix{\widetilde{C}}$.

More generally,

\[
 \matrix{\widetilde{C}}_{ij}
 = \bigoplus_{k:\, {\sf row\_indices[k]} = i\, \land\, {\sf col\_indices[k]} = j}   (D_{dup})\,{\sf values[k]}

,\] 

where $\oplus$ is the {\sf dup} binary operand. Finally, the resulting $\matrix{\widetilde{C}}$ is copied into ${\sf C}$ via typecasting its values.
 
The nonopaque input arrays {\sf row\_indices}, {\sf col\_indices}, and {\sf values} should be of the same length {\sf nvals}. 

The nonopaque input arrays {\sf row\_indices}, {\sf col\_indices}, and {\sf values} are available to be modified by the user on return from this method.

It is an error to call this function on an output object with existing elements. In other words, 
{\sf GrB\_Matrix\_nvals(C)} should evaluate to zero prior to calling this function.

After {\sf GrB\_Matrix\_build} returns, it is safe for a programmer to 
modify or delete the arrays {\sf row\_indices}, {\sf col\_indices}, or {\sf values}.



%-----------------------------------------------------------------------------
\subsubsection{{\sf Matrix\_setElement}: Set a single element in matrix}

Set one element of a matrix to a given value.

\paragraph{\syntax}

\begin{verbatim}
        GrB_Info GrB_Matrix_setElement(GrB_Matrix   C,
                                       <type>       val,
                                       GrB_Index    row_index,
                                       GrB_Index    col_index); 
\end{verbatim}

\paragraph{Parameters}

\begin{itemize}[leftmargin=1.1in]
    \item[{\sf C}]   ({\sf INOUT}) An existing GraphBLAS matrix for which an 
    element is to be assigned.

    \item[{\sf val}]   ({\sf IN})  Value to assign.  The type must
    be compatible with the domain of {\sf C}.
    
    \item[{\sf row\_index}] ({\sf IN}) Row index of element to be assigned
    \item[{\sf col\_index}] ({\sf IN}) Column index of element to be assigned
\end{itemize}

\paragraph{Return Values}

\begin{itemize}[leftmargin=2.1in]
    \item[{\sf GrB\_SUCCESS}]         In blocking mode, the operation completed
    successfully. In non-blocking mode, this indicates that the compatibility 
    tests on index/dimensions and domains for the input arguments passed successfully. 
    Either way, output matrix {\sf C} is ready to be used in the next method of 
    the sequence.

    \item[{\sf GrB\_PANIC}]   Unknown internal error.
    
    \item[{\sf GrB\_INVALID\_OBJECT}] This is returned in any execution mode 
    whenever one of the opaque GraphBLAS objects (input or output) is in an invalid 
    state caused by a previous execution error.  Call {GrB\_error()} to access 
    any error messages generated by the implementation.

    \item[{\sf GrB\_OUT\_OF\_MEMORY}]  Not enough memory available for operation.
    
    \item[{\sf GrB\_UNINITIALIZED\_OBJECT}]  The GraphBLAS matrix, {\sf C}, has 
    not been initialized by a call to {\sf Matrix\_new} or {\sf Matrix\_dup}.
    
    \item[{\sf GrB\_INVALID\_INDEX}]  The ({\sf row\_index, col\_index}) tuple
    specifies a position that outside the dimensions of {\sf C}.
    
    \item[{\sf GrB\_DOMAIN\_MISMATCH}]     The domains of the matrix or scalar
    are incompatible.
\end{itemize}

\paragraph{Description}

First, the scalar and output matrix are tested for domain compatibility as follows:  
$\bold{D}({\sf val})$ must be compatible with $\bold{D}({\sf C})$. Two domains 
are compatible with each other if values from one domain can be cast to values 
in the other domain as per the rules of the C language.  In particular, domains 
from Table~\ref{Tab:PredefinedTypes} are all compatible with each other. A domain 
from a user-defined type is only compatible with itself.  If any compatibility 
rule above is violated, execution of {\sf GrB\_Matrix\_extractElement} ends and
the domain mismatch error listed above is returned.

Then, both index parameters are checked for valid values where following
conditions must hold:
\[
\begin{aligned}
    0\ \leq\ {\sf row\_index} & \ <\ \bold{nrows}({\sf C}), \\
    0\ \leq\ {\sf col\_index} & \ <\ \bold{ncols}({\sf C})
\end{aligned}
\]
If either of these conditions is violated, execution of 
{\sf GrB\_Matrix\_extractElement} ends and the invalid 
index error listed above is returned. 

We are now ready to carry out the assignment of {\sf val}; that is,
\[
{\sf C}({\sf row\_index},{\sf col\_index}) = {\sf val} 
\]
If a value existed at this location in {\sf C}, it will be overwritten; otherwise,
and new value is stored in {\sf C}.

In {\sf GrB\_BLOCKING} mode, the method exits with return value 
{\sf GrB\_SUCCESS} and the new contents of {\sf C} is as defined above
and fully computed.  
In {\sf GrB\_NONBLOCKING} mode, the method exits with return value 
{\sf GrB\_SUCCESS} and the new content of vector {\sf C} is as defined above 
but may not be fully computed; however, it can be used in the next GraphBLAS 
method call in a sequence.


%-----------------------------------------------------------------------------

\subsubsection{{\sf Matrix\_extractElement}: Extract a single element from a matrix}
\label{Sec:extract_single_element_mat}

%\scott{Is OUTOFMEMORY error possible (perhaps only in non-blocking)? }
%\scott{still need to deal with notation for type of first parameter.}

Extract one element of a matrix into a scalar. 

\paragraph{\syntax}

\begin{verbatim}
        GrB_Info GrB_Matrix_extractElement(<type>           *val,
                                           const GrB_Matrix  A,
                                           GrB_Index         row_index,
                                           GrB_Index         col_index); 

\end{verbatim}

\paragraph{Parameters}

\begin{itemize}[leftmargin=1in]
    \item[{\sf val}]   ({\sf OUT}) Pointer to a scalar. On successful return, 
    this scalar holds the result of the operation.  Any previous value is overwritten.

    \item[{\sf A}]     ({\sf IN}) The GraphBLAS matrix from which an element is
    extracted.
    
    \item[{\sf row\_index}] ({\sf IN}) The row index of location in {\sf A} 
    to extract.

    \item[{\sf col\_index}] ({\sf IN}) The column index of location in {\sf A} 
    to extract.
\end{itemize}

\paragraph{Return Values}

\begin{itemize}[leftmargin=2.1in]
    \item[{\sf GrB\_SUCCESS}]  In blocking or non-blocking mode, the operation 
    completed successfully. This indicates that the compatibility tests on 
    dimensions and domains for the input arguments passed successfully, and
    the output scalar, {\sf val}, has been computed and is ready to be used in 
    the next method of the sequence.

    \item[{\sf GrB\_PANIC}]   Unknown internal error.
    
    \item[{\sf GrB\_INVALID\_OBJECT}] This is returned in any execution mode 
    whenever one of the opaque GraphBLAS objects (input or output) is in an invalid 
    state caused by a previous execution error.  Call {GrB\_error()} to access 
    any error messages generated by the implementation.

    \item[{\sf GrB\_OUT\_OF\_MEMORY}]  Not enough memory available for operation.
   % \scott{Is this error possible?}
    
    \item[{\sf GrB\_UNINITIALIZED\_OBJECT}]  The GraphBLAS matrix, {\sf A}, has 
    not been initialized by a call to {\sf Matrix\_new} or {\sf Matrix\_dup}.
    
    \item[{\sf GrB\_NULL\_POINTER}]    {\sf val} pointer is {\sf NULL}.

    \item[{\sf GrB\_NO\_VALUE}]  There is no stored value at specified location.
    
    \item[{\sf GrB\_INVALID\_INDEX}]  {\sf row\_index} or {\sf col\_index} is 
    outside the allowable range (i.e., not less than $\bold{nrows}({\sf A})$ or
    $\bold{ncols}({\sf A})$, respectively).

    \item[{\sf GrB\_DOMAIN\_MISMATCH}]     The domains of the matrix and scalar
    are incompatible.
\end{itemize}

\paragraph{Description}

First, the scalar and input matrix are tested for domain compatibility as follows:  
$\bold{D}({\sf val})$ must be compatible with $\bold{D}({\sf A})$. Two domains 
are compatible with each other if values from one domain can be cast to values 
in the other domain as per the rules of the C language.  In particular, domains 
from Table~\ref{Tab:PredefinedTypes} are all compatible with each other. A domain 
from a user-defined type is only compatible with itself.  If any compatibility 
rule above is violated, execution of {\sf GrB\_Matrix\_extractElement} ends and
the domain mismatch error listed above is returned.

Then, both index parameters are checked for valid values where following
conditions must hold:
\[
\begin{aligned}
    0\ \leq\ {\sf row\_index} & \ <\ \bold{nrows}({\sf A}), \\
    0\ \leq\ {\sf col\_index} & \ <\ \bold{ncols}({\sf A})
\end{aligned}
\]
If either of these conditions is violated, execution of 
{\sf GrB\_Matrix\_extractElement} ends and the invalid 
index error listed above is returned. 

We are now ready to carry out the extract into the output argument, {\sf val}; 
that is,
\[
{\sf val} = {\sf A}({\sf row\_index},{\sf col\_index})
\]
where the following condition must be true:
\[
    ({\sf row\_index},{\sf col\_index}) \ \in \ \bold{ind}({\sf A})
\]
If this condition is violated, execution of {\sf GrB\_Matrix\_extractElement} 
ends and the "no value" error listed above is returned.


In both {\sf GrB\_BLOCKING} mode {\sf GrB\_NONBLOCKING} mode
if the method exits with return value {\sf GrB\_SUCCESS}, the  new 
contents of  {\sf val} are as defined above.  In other words, the method
does not return until any operations required to fully compute 
the GraphBLAS matrix {\sf A} have completed. 

In {\sf GrB\_NONBLOCKING} mode, if the return value is 
other than  {\sf GrB\_SUCCESS}, an error in a method occurring earlier in the sequence
may have occurred that prevents completion of the GraphBLAS matrix {\sf A}.
The GrB\_error() method should be called for additional information 
about such errors.


%-----------------------------------------------------------------------------

\subsubsection{{\sf Matrix\_extractTuples}: Extract tuples from a matrix}
\label{Sec:Matrix_extractTuples}

Extract the contents of a GraphBLAS matrix into non-opaque data structures.

\paragraph{\syntax}

\begin{verbatim}
        GrB_Info GrB_Matrix_extractTuples(GrB_Index            *row_indices,
                                          GrB_Index            *col_indices,
                                          <type>               *values, 
                                          const GrB_Matrix      A);
\end{verbatim}

\paragraph{Parameters}

\begin{itemize}[leftmargin=1.1in]
    \item[{\sf row\_indices}] ({\sf OUT}) Pointer to an array of row indices that is sufficient to
                        hold all of the row indices (no checking is performed).
    \item[{\sf col\_indices}] ({\sf OUT}) Pointer to an array of column indices that is sufficient to
                        hold all of the column indices (no checking is performed). 
    \item[{\sf values}] ({\sf OUT}) Pointer to an array of scalars of a type that is sufficient to
                        hold all of the stored values (no checking is performed) whose
                        type is compatible with $\bold{D}(\matrix{A})$.
    \item[{\sf A}]      ({\sf IN}) An existing GraphBLAS matrix.
\end{itemize}

\paragraph{Return Values}

\begin{itemize}[leftmargin=2.1in]
    \item[{\sf GrB\_SUCCESS}]  In blocking or non-blocking mode, the operation 
    completed successfully. This indicates that the compatibility tests on 
    the input argument passed successfully, and the output arrays, {\sf indices}
    and {\sf values}, have been computed.

    \item[{\sf GrB\_PANIC}]   Unknown internal error.
    
    \item[{\sf GrB\_INVALID\_OBJECT}] This is returned in any execution mode 
    whenever one of the opaque GraphBLAS objects (input or output) is in an invalid 
    state caused by a previous execution error.  Call {GrB\_error()} to access 
    any error messages generated by the implementation.

    \item[{\sf GrB\_OUT\_OF\_MEMORY}]  Not enough memory available for operation.
	 %   \scott{Is this error possible?} \jose{out-of-memory is always possible -- we don't restrict implementation}
    
    \item[{\sf GrB\_UNINITIALIZED\_OBJECT}]  The GraphBLAS matrix, {\sf A}, has 
    not been initialized by a call to {\sf Matrix\_new} or {\sf Matrix\_dup}.
    
    \item[{\sf GrB\_NULL\_POINTER}]  {\sf row\_indices}, {\sf col\_indices} or 
    {\sf values} pointer is {\sf NULL}.
    
    \item[\sf GrB\_DOMAIN\_MISMATCH] The domains of the {\sf A} matrix and 
    {\sf values} array are incompatible with one another.
\end{itemize}

\paragraph{Description}


This method will extract all the tuples from the GraphBLAS matrix {\sf A}.  
The values associated with those tuples are placed in the
array {\sf values}, the column indices are placed in the array {\sf col\_indices}, 
and the row indices are placed in the array {\sf row\_indices}. 
These output arrays are pre-allocated by the user before calling
this function such that each output array has enough
space to hold at least {\sf GrB\_Matrix\_nvals(v)} elements. 

Upon return of this function, a pair of $\{{\sf row\_indices[k], col\_indices[k]}\}$ are unique for every valid $k$, 
but they do not have to be sorted in any particular order.
Each tuple $(i,j,A_{ij})$ in {\sf A} is unzipped and copied into a distinct $k$th location in output vectors:  
$$\{{\sf row\_indices[k]}, {\sf col\_indices[k]}, {\sf values[k]}\} \leftarrow (i,j,A_{ij}),$$ where $0 \leq k < {\sf GrB\_Matrix\_nvals(v)}$. 
No gaps in output vectors are allowed; i.e.\, if {\sf row\_indices[k]},  {\sf col\_indices[k]}  and {\sf values[k]} exist upon return, 
so does {\sf row\_indices[j]}, {\sf col\_indices[j]} and {\sf values[j]} for all $j$ such that $0 \leq j < k$.


In both {\sf GrB\_BLOCKING} mode {\sf GrB\_NONBLOCKING} mode
if the method exits with return value {\sf GrB\_SUCCESS}, the  new 
contents of the arrays {\sf row\_indices}, {\sf col\_indices} and {\sf values} are as defined above.  In other words, the method
does not return until any operations required to fully compute 
the GraphBLAS vector {\sf A} have completed. 

In {\sf GrB\_NONBLOCKING} mode, if the return value is 
not {\sf GrB\_SUCCESS}, an error in a method occurring earlier in the sequence
may have occurred that prevents completion of the GraphBLAS vector {\sf A}.
The GrB\_error() method should be called for additional information 
about these errors.

%-----------------------------------------------------------------------------
\subsubsection{{\sf Matrix\_assign}: Standard matrix variant}

Assign values from one GraphBLAS matrix to a subset of another GraphBLAS
matrix as specified by two sets of indices. The dimensions of the input matrix are
the same size as the row and column index arrays provided.

\paragraph{\syntax}

\begin{verbatim}
        GrB_Info GrB_Matrix_assign(GrB_Matrix              C,
                                   const GrB_Matrix        A,
                                   const GrB_Index        *row_indices,
                                   const GrB_Index         nrows,
                                   const GrB_Index        *col_indices,
                                   const GrB_Index         ncols,
                                   const GrB_Descriptor    desc);
\end{verbatim}

\paragraph{Parameters}

\begin{itemize}[leftmargin=1in]
    \item[{\sf C}]    ({\sf INOUT}) An existing GraphBLAS Matriix,
    that will be modified by the assign On output, this matrix holds
    the results of the operation.

    \item[{\sf A}]       ({\sf IN}) The GraphBLAS matrix whose contents
    are assigned to a subset of {\sf C}.

    \item[{\sf row\_indices}]  ({\sf IN}) Pointer to the ordered set
    (array) of indices corresponding to the rows of {\sf C} that are
    assigned.  If all rows of {\sf C} are to be assigned in order from
    $0$ to ${\sf nrows} - 1$, then {\sf GrB\_ALL} can be specified.
    Regardless of execution mode and return value, this array may be
    manipulated by the caller after this operation returns without
    affecting any deferred computations for this operation.  If this
    array contains duplicate values, it implies assignment of more than
    one value to the same location which leads to undefined results.

    \item[{\sf nrows}] ({\sf IN}) The number of values in {\sf
    row\_indices} array. Must be equal to $\bold{nrows}({\sf A})$ if {\sf
    A} is not tranposed, or equal to $\bold{ncols}({\sf A})$ if {\sf A}
    is transposed.

    \item[{\sf col\_indices}]  ({\sf IN}) Pointer to the ordered set
    (array) of indices corresponding to the columns of {\sf C} that
    are assigned.  If all columns of {\sf C} are to be assigned in order
    from $0$ to ${\sf ncols} - 1$, then {\sf GrB\_ALL} can be specified.
    Regardless of execution mode and return value, this array may be
    manipulated by the caller after this operation returns without
    affecting any deferred computations for this operation.  If this
    array contains duplicate values, it implies assignment of more than
    one value to the same location which leads to undefined results.

    \item[{\sf ncols}] ({\sf IN}) The number of values in {\sf
    col\_indices} array. Must be equal to $\bold{ncols}({\sf A})$ if {\sf
    A} is not tranposed, or equal to $\bold{nrows}({\sf A})$ if {\sf A}
    is transposed.

    \item[{\sf desc}] ({\sf IN}) An optional operation descriptor. If
    a \emph{default} descriptor is desired, {\sf GrB\_NULL} should be
    specified. Non-default field/value pairs are listed as follows:  \\

    \begin{tabular}{lllp{2.5in}}
        Param & Field  & Value & Description \\
        \hline
        {\sf A}    & {\sf GrB\_INP0} & {\sf GrB\_TRAN}   & Apply transpose to 
        {\sf A} before assigning elements.
    \end{tabular}
\end{itemize}

\paragraph{Return Values}

\begin{itemize}[leftmargin=2.1in]
    \item[{\sf GrB\_SUCCESS}] In blocking mode, the operation
    completed successfully. In non-blocking mode, this indicates that
    the compatibility tests on dimensions and domains for the input
    arguments passed successfully.  Either way, output matrix {\sf C}
    is ready to be used in the next method of the sequence.

    \item[{\sf GrB\_PANIC}] Unknown internal error.

    \item[{\sf GrB\_INVALID\_OBJECT}] This is returned in any execution
    mode whenever one of the opaque GraphBLAS objects (input or output)
    is in an invalid state caused by a previous execution error.
    Call {\sf GrB\_error()} to access any error messages generated by
    the implementation.

    \item[{\sf GrB\_OUT\_OF\_MEMORY}]  Not enough memory available
    for operation.

    \item[{\sf GrB\_UNINITIALIZED\_OBJECT}] One or more of the GraphBLAS
    objects has not been initialized by a call to {\sf new} (or {\sf dup}
    for matrix parameters).

    \item[{\sf GrB\_INDEX\_OUT\_OF\_BOUNDS}]  A value in {\sf
    row\_indices} is greater than or equal to $\bold{nrows}({\sf
    C})$, or a value in {\sf col\_indices} is greater than or equal to
    $\bold{ncols}({\sf C})$.  In non-blocking mode, this can be reported
    as an execution error.

    \item[{\sf GrB\_DIMENSION\_MISMATCH}] ${\sf nrows} \neq
    \bold{nrows}({\sf A})$ ($\bold{ncols}({\sf A})$ if {\sf A}
    is transposed), or ${\sf ncols} \neq \bold{ncols}({\sf A})$
    ($\bold{nrows}({\sf A})$ if {\sf A} is transposed).

    \item[{\sf GrB\_DOMAIN\_MISMATCH}]     The domains of the various
    matrices are incompatible with each other.

    \item[{\sf GrB\_NULL\_POINTER}] Either argument {\sf row\_indices}
    is a {\sf NULL} pointer, argument {\sf col\_indices} is a {\sf NULL}
    pointer, or both.
\end{itemize}

\paragraph{Description}

This variant of {\sf GrB\_assign} computes the result of assigning values of
elements form a source GraphBLAS matrix to elements of a destination GraphBLAS matrix.
More explicitly:
\[
\begin{aligned}
	{\sf C}({\sf row\_indices}[i],{\sf col\_indices}[j]) = &\ {\sf A}(i,j), \ \forall \ i,j \ : \ 0 \leq i < {\sf nrows},\ 0 \leq j < {\sf ncols}, \ \ \mbox{if {\sf A} is not transposed},
\\
	{\sf C}({\sf row\_indices}[i],{\sf col\_indices}[j]) = &\ {\sf A}(j,i), \ \forall \ i,j \ : \ 0 \leq i < {\sf ncols},\ 0 \leq j < {\sf nrows}, \ \ \mbox{if {\sf A} is transposed}.
\end{aligned}
\]  
If a particular element ${\sf A}(i,j)$ does not exist, the corresponding
element of {\sf C} is also removed. 
Elements of {\sf C} outside the ranges of $({\sf row\_indices}[i])$ and $({\sf col\_indics}[j])$ are not affected.

Logically, this operation occurs in three steps:
\begin{enumerate}[leftmargin=0.75in]
\item[Setup] The internal matrices used in the computation are formed 
and their domains and dimensions are tested for compatibility.
\item[Compute] The indicated computations are carried out.
\item[Output] The result is written into the output matrix.
\end{enumerate}

Two argument matrices are used in the {\sf GrB\_assign} operation:
\begin{enumerate}
	\item ${\sf C} = \langle \bold{D}({\sf C}),\bold{nrows}({\sf C}),
    \bold{ncols}({\sf C}),\bold{L}({\sf C}) = \{(i,j,C_{ij}) \} \rangle$
    
	\item ${\sf A} = \langle \bold{D}({\sf A}),\bold{nrows}({\sf A}), 
    \bold{ncols}({\sf A}),\bold{L}({\sf A}) = \{(i,j,A_{ij}) \} \rangle$
\end{enumerate}

The argument matrices 
are tested for domain compatibility as follows:
\begin{enumerate}
	\item $\bold{D}({\sf C})$ must be compatible with $\bold{D}({\sf A})$.
\end{enumerate}
Two domains are compatible with each other if values from one domain can be cast 
to values in the other domain as per the rules of the C language.
In particular, domains from Table~\ref{Tab:PredefinedTypes} are all compatible 
with each other. A domain from a user-defined type is only compatible with itself.
If any compatibility rule above is violated, execution of {\sf GrB\_assign} ends
and the domain mismatch error listed above is returned.

From the arguments, the internal matrices and index arrays used in 
the computation are formed ($\leftarrow$ denotes copy):
\begin{enumerate}
	\item Matrix $\matrix{\widetilde{C}} \leftarrow {\sf C}$.

	\item Matrix $\matrix{\widetilde{A}} \leftarrow 
    {\sf desc[GrB\_INP0].GrB\_TRAN} \ ? \ {\sf A}^T : {\sf A}$.

    \item The internal row index array, $\grbarray{\widetilde{I}}$, is computed from 
    argument {\sf row\_indices} as follows:
	\begin{enumerate}
		\item	If ${\sf row\_indices} = {\sf GrB\_ALL}$, then 
        $\grbarray{\widetilde{I}}[i] = i, \forall i : 0 \leq i < {\sf nrows}$.

		\item	Otherwise, $\grbarray{\widetilde{I}}[i] = {\sf row\_indices}[i], 
        \forall i : 0 \leq i < {\sf nrows}$.
    \end{enumerate}
    
    \item The internal column index array, $\grbarray{\widetilde{J}}$, is computed from 
    argument {\sf col\_indices} as follows:
	\begin{enumerate}
		\item	If ${\sf col\_indices} = {\sf GrB\_ALL}$, then 
        $\grbarray{\widetilde{J}}[j] = j, \forall j : 0 \leq j < {\sf ncols}$.

		\item	Otherwise, $\grbarray{\widetilde{J}}[j] = {\sf col\_indices}[j], 
        \forall j : 0 \leq j < {\sf ncols}$.
    \end{enumerate}
\end{enumerate}

The internal matrices and mask are checked for shape compatibility. The following 
conditions must hold:
\begin{enumerate}
    \item $\bold{nrows}(\matrix{\widetilde{A}}) = {\sf nrows}$.

    \item $\bold{ncols}(\matrix{\widetilde{A}}) = {\sf ncols}$.
\end{enumerate}
If any compatibility rule above is violated, execution of {\sf GrB\_assign} ends and 
the dimension mismatch error listed above is returned.

From this point forward, in {\sf GrB\_NONBLOCKING} mode, the method can 
optionally exit with {\sf GrB\_SUCCESS} return code and defer any computation 
and/or execution error codes.

We are now ready to carry out the assign and any additional 
associated operations.  We describe this in terms of an intermediate matrix:
\begin{itemize}
	\item $\matrix{\widetilde{T}}$: The matrix holding the contents from
    $\matrix{\widetilde{A}}$ in their destination locations relative to
    $\matrix{\widetilde{C}}$.
\end{itemize}

The intermediate matrix, $\matrix{\widetilde{T}}$, is created as follows:
\[ 
\begin{aligned}
\matrix{\widetilde{T}} = \langle & \bold{D}({\sf A}), 
                           \bold{nrows}(\matrix{\widetilde{C}}), 
                           \bold{ncols}(\matrix{\widetilde{C}}), \\
 & \{ (\grbarray{\widetilde{I}}[i],\grbarray{\widetilde{J}}[j],\matrix{\widetilde{A}}(i,j)) 
\ \forall \ (i,j), \ 0 \leq i < {\sf nrows}, \ 0 \leq j < {\sf ncols} :
(i,j) \in \bold{ind}(\matrix{\widetilde{A}}) \} \rangle. 
\end{aligned}
\]
At this point, if any value in the $\grbarray{\widetilde{I}}$ array is not in
the range $[0,\ \bold{nrows}(\matrix{\widetilde{C}}) )$ or any value in the 
$\grbarray{\widetilde{J}}$ array is not in the range 
$[0,\ \bold{ncols}(\matrix{\widetilde{C}}))$, the execution of {\sf GrB\_assign} 
ends and the index out-of-bounds error listed above is generated.  In 
{\sf GrB\_NONBLOCKING} mode, the error can be deferred until a 
sequence-terminating {\sf GrB\_wait()} is called.  Regardless, the result 
matrix {\sf C} is invalid from this point forward in the sequence.

Finally, the assign operation is completed by updating
the contents of matrix {\sf C} according to the following expression:
\[
	\bold{L}({\sf C}) = (\bold{L}({\matrix{\widetilde{C}}}) - \{(\grbarray{\widetilde{I}}[i],\grbarray{\widetilde{J}}[j],C) \forall i,j : (\grbarray{\widetilde{I}}[i],\grbarray{\widetilde{J}}[j]) \in \bold{ind}({\sf C})\})
	\cup \{(i,j,\matrix{\widetilde{T}}(i,j)) \forall (i,j) \in \bold{ind}({\widetilde{T}})\},
\]
where the difference operator in the previous expressions refers to set difference.
That is, first existing elements of {\sf C} with row indices in $\grbarray{\widetilde{I}}$ and column indices in $\grbarray{\widetilde{J}}$ are removed,
and then the elements of $\widetilde{T}$ are added in the right place.

In {\sf GrB\_BLOCKING} mode, the method exits with return value 
{\sf GrB\_SUCCESS} and the new content of matrix {\sf C} is as defined above
and fully computed.  
In {\sf GrB\_NONBLOCKING} mode, the method exits with return value 
{\sf GrB\_SUCCESS} and the new content of matrix {\sf C} is as defined above 
but may not be fully computed; however, it can be used in the next GraphBLAS 
method call in a sequence.

%-----------------------------------------------------------------------------
\subsubsection{{\sf Matrix\_assign}: Column variant}

Assign the contents a vector to a subset of elements in one column of a matrix. 
Note that since the output cannot be transposed, a different variant of
{\sf assign} is provided to assign to a row of a matrix.

\paragraph{\syntax}

\begin{verbatim}
        GrB_Info GrB_Matrix_assign(GrB_Matrix              C,
                                   const GrB_Vector        mask,
                                   const GrB_BinaryOp      accum,
                                   const GrB_Vector        u,
                                   const GrB_Index        *row_indices,
                                   const GrB_Index         nrows,
                                   GrB_Index               col_index,
                                   const GrB_Descriptor    desc); 
\end{verbatim}

\paragraph{Parameters}

\begin{itemize}[leftmargin=1.1in]
    \item[{\sf C}]    ({\sf INOUT}) An existing GraphBLAS Matrix.  On input,
    the matrix provides values that may be accumulated with the result of the
    assign operation.  On output, this matrix holds the results of the
    operation.

    \item[{\sf mask}] ({\sf IN}) An optional ``write'' mask that controls which
    results from this operation are stored into the specified column of the output matrix {\sf C}. The 
    mask dimensions must match those of a single column of the matrix {\sf C} and the domain of the 
    {\sf Mask} matrix must be of type {\sf bool} or any of the predefined 
    ``built-in'' types in Table~\ref{Tab:PredefinedTypes}.  If the default
    vector is desired (\ie, with correct dimensions and filled with {\sf true}), 
    {\sf GrB\_NULL} should be specified.

    \item[{\sf accum}]    ({\sf IN}) An optional operator used for accumulating
    entries into existing {\sf C} entries: ${\sf accum} = \langle D_x,
    D_y, D_z,\odot \rangle$. If assignment rather than accumulation is
    desired, {\sf GrB\_NULL} should be specified.

    \item[{\sf u}]       ({\sf IN}) The GraphBLAS vector whose contents are 
    assigned to (a subset of) a column of {\sf C}.

    \item[{\sf row\_indices}]  ({\sf IN}) Pointer to the ordered set (array) of 
    indices corresponding to the locations in the specified column of {\sf C} 
    that are to be assigned.  If all elements of the column in {\sf C} are to be 
    assigned in order from index $0$ to ${\sf nrows} - 1$, then {\sf GrB\_ALL} should be 
    specified.  Regardless of execution mode and return value, this array may be
    manipulated by the caller after this operation returns without affecting any 
    deferred computations for this operation.
    If this array contains duplicate values, it implies in assignment of more 
    than one value to the same location which leads to undefined results.
    
    \item[{\sf nrows}] ({\sf IN}) The number of values in {\sf row\_indices} array.
    Must be equal to $\bold{size}({\sf u})$.
    
    \item[{\sf col\_index}]  ({\sf IN}) The index of the column in {\sf C} to 
    assign. Must be in the range $[0, \bold{ncols}({\sf C}))$.

    \item[{\sf desc}] ({\sf IN}) An optional operation descriptor. If
    a \emph{default} descriptor is desired, {\sf GrB\_NULL} should be
    specified. Non-default field/value pairs are listed as follows:  \\

    \begin{tabular}{lllp{2.5in}}
        Param & Field  & Value & Description \\
        \hline
        {\sf C}    & {\sf GrB\_OUTP} & {\sf GrB\_REPLACE} &  Output column in 
        {\sf C} is cleared (all elements removed) before result is stored in it.\\
    
        {\sf mask} & {\sf GrB\_MASK} & {\sf GrB\_SCMP}   & Use the structural 
        complement of {\sf mask}. \\
    \end{tabular}

\end{itemize}

\paragraph{Return Values}

\begin{itemize}[leftmargin=2.1in]
    \item[{\sf GrB\_SUCCESS}]         In blocking mode, the operation completed
    successfully. In non-blocking mode, this indicates that the compatibility 
    tests on dimensions and domains for the input arguments passed successfully. 
    Either way, output matrix {\sf C} is ready to be used in the next method of 
    the sequence.

    \item[{\sf GrB\_PANIC}]            Unknown internal error.
    
    \item[{\sf GrB\_INVALID\_OBJECT}] This is returned in any execution mode 
    whenever one of the opaque GraphBLAS objects (input or output) is in an invalid 
    state caused by a previous execution error.  Call {\sf GrB\_error()} to access 
    any error messages generated by the implementation.

    \item[{\sf GrB\_OUT\_OF\_MEMORY}]  Not enough memory available for operation.
    
    \item[{\sf GrB\_UNINITIALIZED\_OBJECT}] One or more of the GraphBLAS objects
    has not been initialized by a call to {\sf new} (or {\sf dup} for vector or
    matrix parameters).

    \item[{\sf GrB\_INVALID\_INDEX}]    {\sf col\_index} is outside the allowable 
    range (i.e., greater than $\bold{ncols}({\sf C})$).

    \item[{\sf GrB\_INDEX\_OUT\_OF\_BOUNDS}]  A value in {\sf row\_indices} 
    is greater than or equal to $\bold{nrows}({\sf C})$.  In 
    non-blocking mode, this can be reported as an execution error.
    
    \item[{\sf GrB\_DIMENSION\_MISMATCH}] {\sf mask} size and number of rows
    in {\sf C} are not the same, or ${\sf nrows} \neq \bold{size}({\sf u})$.

    \item[{\sf GrB\_DOMAIN\_MISMATCH}]     The domains of the matrix and vector
    are incompatible with each other or the corresponding domains of the 
    accumulation operator, or the mask's domain is not compatible with {\sf bool}.

    \item[{\sf GrB\_NULL\_POINTER}] Argument {\sf row\_indices} is a {\sf NULL} pointer.
\end{itemize}

\paragraph{Description}

This variant of {\sf GrB\_assign} computes the result of assigning a subset of
locations in a column of a GraphBLAS matrix (in a specific order) from the 
contents of a GraphBLAS vector: \\
${\sf C}(:,{\sf col\_index}) = {\sf u}$; or, if an 
optional binary accumulation operator ($\odot$) is provided, 
${\sf C}(:,{\sf col\_index}) = 
{\sf C}(:,{\sf col\_index}) \odot {\sf u}$. Taking order of {\sf row\_indices} 
into account, it is more explicitly written as:
\[
\begin{aligned}
    {\sf C}({\sf row\_indices}[i],{\sf col\_index}) = &\ {\sf u}(i) 
    \ \forall \ i \ : \ 0 \leq i < {\sf nrows}, \mbox{~or~}
    \\
    {\sf C}({\sf row\_indices}[i],{\sf col\_index}) = &\ 
    {\sf C}({\sf row\_indices}[i],{\sf col\_index}) \odot {\sf u}(i) 
    \ \forall \ i \ : \ 0 \leq i < {\sf nrows}
\end{aligned}
\]  
Logically, this operation occurs in three steps:
\begin{enumerate}[leftmargin=0.75in]
\item[\bf Setup] The internal matrices, vectors and mask used in the computation are formed 
and their domains and dimensions are tested for compatibility.
\item[\bf Compute] The indicated computations are carried out.
\item[\bf Output] The result is written into the output matrix, possibly under 
control of a mask.
\end{enumerate}

Up to three argument vectors and matrices are used in this {\sf GrB\_assign} 
operation:
\begin{enumerate}
	\item ${\sf C} = \langle \bold{D}({\sf C}),\bold{nrows}({\sf C}),
    \bold{ncols}({\sf C}),\bold{L}({\sf C}) = \{(i,j,C_{ij}) \} \rangle$
    
	\item ${\sf mask} = \langle \bold{D}({\sf mask}),\bold{size}({\sf mask}),
    \bold{L}({\sf mask}) = \{(i,m_{i}) \} \rangle$ (optional)

	\item ${\sf u} = \langle \bold{D}({\sf u}),\bold{size}({\sf u}),
    \bold{L}({\sf u}) = \{(i,u_i) \} \rangle$
\end{enumerate}

The argument vectors, matrix, and the accumulation 
operator (if provided) are tested for domain compatibility as follows:
\begin{enumerate}
	\item The domain of {\sf mask} (if not {\sf GrB\_NULL}) must be from one of 
    the pre-defined types of Table~\ref{Tab:PredefinedTypes}.

	\item If {\sf accum} is {\sf GrB\_NULL}, then $\bold{D}({\sf C})$ must be 
    compatible with $\bold{D}({\sf u})$.

	\item If {\sf accum} is not {\sf GrB\_NULL}, then $\bold{D}({\sf C})$ must be
    compatible with $D_x$ and $D_z$ of the accumulation operator and 
    $\bold{D}({\sf u})$ must be compatible with $D_y$ of the accumulation operator.
\end{enumerate}
Two domains are compatible with each other if values from one domain can be cast 
to values in the other domain as per the rules of the C language.
In particular, domains from Table~\ref{Tab:PredefinedTypes} are all compatible 
with each other. A domain from a user-defined type is only compatible with itself.
If any compatibility rule above is violated, execution of {\sf GrB\_assign} ends
and the domain mismatch error listed above is returned.

The {\sf col\_index} parameter is checked for a valid value.  The following
condition must hold:
\begin{enumerate}
	\item $0\ \leq\ {\sf col\_index} \ <\ \bold{ncols}({\sf C})$
\end{enumerate}
If the rule above is violated, execution of {\sf GrB\_assign} ends 
and the invalid index error listed above is returned.

From the arguments, the internal vectors, mask, and index array used in 
the computation are formed ($\leftarrow$ denotes copy):
\begin{enumerate}
	\item The vector, $\vector{\widetilde{c}}$, is extracted from a column of {\sf C}
    as follows:
    \[
        \vector{\widetilde{c}} = \langle  \bold{D}({\sf C}), \bold{nrows}({\sf C}), 
        \{ (i, C_{ij}) \ \forall \ i : 0 \leq i < \bold{nrows}({\sf C}),
        j = {\sf col\_index}, (i, j) \in \bold{ind}({\sf C})\} \rangle
    \]

	\item One-dimensional mask, $\vector{\widetilde{m}}$, is computed from 
    argument {\sf mask} as follows:
	\begin{enumerate}
		\item	If ${\sf mask} = {\sf GrB\_NULL}$, then $\vector{\widetilde{m}} = 
        \langle \bold{nrows}({\sf C}), \{i,\ \forall \ i : 0 \leq i < 
        \bold{nrows}({\sf C}) \} \rangle$.

		\item	Otherwise, $\vector{\widetilde{m}} = 
        \langle \bold{size}({\sf mask}), \{i : i \in \bold{ind}({\sf mask}) \wedge
        ({\sf bool}){\sf mask}(i) = \true \} \rangle$.

		\item	If ${\sf desc[GrB\_MASK].GrB\_SCMP}$ is \true, then 
        $\vector{\widetilde{m}} \leftarrow \neg \vector{\widetilde{m}}$.
	\end{enumerate}

	\item Vector $\vector{\widetilde{u}} \leftarrow {\sf u}$.
    
    \item The internal row index array, $\grbarray{\widetilde{I}}$, is computed from 
    argument {\sf row\_indices} as follows:
	\begin{enumerate}
		\item	If ${\sf row\_indices} = {\sf GrB\_ALL}$, then 
        $\grbarray{\widetilde{I}}[i] = i, \ \forall \ i : 0 \leq i < {\sf nrows}$.

		\item	Otherwise, $\grbarray{\widetilde{I}}[i] = {\sf row\_indices}[i], 
        \forall i : 0 \leq i < {\sf nrows}$.
    \end{enumerate}
\end{enumerate}

The internal vectors, matrices, and masks are checked for dimension compatibility. 
The following conditions must hold:
\begin{enumerate}
	\item $\bold{size}(\vector{\widetilde{c}}) = \bold{size}(\vector{\widetilde{m}})$
    \item ${\sf nrows} = \bold{size}(\vector{\widetilde{u}})$.
\end{enumerate}
If any compatibility rule above is violated, execution of {\sf GrB\_assign} ends and 
the dimension mismatch error listed above is returned.

From this point forward, in {\sf GrB\_NONBLOCKING} mode, the method can 
optionally exit with {\sf GrB\_SUCCESS} return code and defer any computation 
and/or execution error codes.

We are now ready to carry out the assign and any additional 
associated operations.  We describe this in terms of two intermediate vectors:
\begin{itemize}
    \item $\vector{\widetilde{t}}$: The vector holding the elements from
    $\vector{\widetilde{u}}$ in their destination locations relative to 
    $\vector{\widetilde{c}}$.
    
    \item $\vector{\widetilde{z}}$: The vector holding the result after 
    application of the (optional) accumulation.
\end{itemize}

The intermediate vector, $\vector{\widetilde{t}}$, is created as follows:
\[
\vector{\widetilde{t}} = \langle
\bold{D}({\sf u}), \bold{size}(\vector{\widetilde{c}}),
%\bold{L}(\vector{\widetilde{t}}) =
\{(\grbarray{\widetilde{I}}[i],\vector{\widetilde{u}}(i))\ \forall \ i,
 0 \leq i < {\sf nrows} : i \in \bold{ind}(\vector{\widetilde{u}}) \} \rangle. 
\]
At this point, if any value of $\grbarray{\widetilde{I}}[i]$ is outside the valid 
range of indices for vector $\vector{\widetilde{c}}$, computation ends and the 
method returns the index out-of-bounds error listed above. In 
{\sf GrB\_NONBLOCKING} mode, the error can be deferred until a 
sequence-terminating {\sf GrB\_wait()} is called.  Regardless, the result 
matrix, {\sf C}, is invalid from this point forward in the 
sequence.

The intermediate vector $\vector{\widetilde{z}}$ is created as follows:
\begin{itemize}
    \item If ${\sf accum} = {\sf GrB\_NULL}$, then 
    $\vector{\widetilde{z}} = \vector{\widetilde{t}}$.

    \item If ${\sf accum} = \langle D_x, D_y, D_z, \odot \rangle$, then vector 
    $\vector{\widetilde{z}}$ is defined as 
        \[ 
        \vector{\widetilde{z}} = 
        \langle D_z, \bold{size}(\vector{\widetilde{c}}), 
        %\bold{L}(\vector{\widetilde{z}}) = 
        \{(i,z_{i}),   
        \ \forall \ i \in \bold{ind}(\vector{\widetilde{c}}) \cup 
        \bold{ind}(\vector{\widetilde{t}}) \} \rangle.\]
    The values of the elements of $\vector{\widetilde{z}}$ are computed based on the relationships between the sets of indices in $\vector{\widetilde{w}}$ and $\vector{\widetilde{t}}$.
\[
z_{i} = \vector{\widetilde{c}}(i) \odot \vector{\widetilde{t}}(i), \ \mbox{if}\  
i \in  (\bold{ind}(\vector{\widetilde{t}}) \cap \bold{ind}(\vector{\widetilde{c}})),
\]
\[
z_{i} = \vector{\widetilde{c}}(i), \ \mbox{if}\  i \in  (\bold{ind}(
\vector{\widetilde{c}}) - (\bold{ind}(\vector{\widetilde{t}}) \cap 
\bold{ind}(\vector{\widetilde{c}}))),
\]
\[
z_{i} = \vector{\widetilde{t}}(i), \ \mbox{if}\  i \in  (\bold{ind}(
\vector{\widetilde{t}}) - (\bold{ind}(\vector{\widetilde{t}}) \cap 
\bold{ind}(\vector{\widetilde{c}}))).
\]
where the difference operator in the previous expressions refers to set difference.
\end{itemize}

Finally, the set of output values that make up the $\vector{\widetilde{z}}$ 
vector are written into the column of the final result matrix, 
${\sf C}(:,{\sf col\_index})$.  This is carried out under control of the mask 
which acts as a ``write mask''.
\begin{itemize}
    \item If {\sf desc[GrB\_OUTP].GrB\_REPLACE} is set then any values in 
    ${\sf C}(:,{\sf col\_index})$ on input to {\sf GrB\_assign()} are deleted 
    and the new content of the column is given by:
    \[ 
		\bold{L}({\sf C}) = \{ (i,j,C_{ij}) : j \neq {\sf col\_index} \} \cup \{(i,{\sf col\_index},z_{i}) : i \in 
    (\bold{ind}(\vector{\widetilde{z}}) \cap \bold{ind}(\vector{\widetilde{m}})) \}. 
    \]

    \item If {\sf desc[GrB\_OUTP].GrB\_REPLACE} is not set, the elements of 
    $\vector{\widetilde{z}}$ indicated by the mask are copied into the column 
    of the final result matrix, ${\sf C}(:,{\sf col\_index})$, and elements of 
    this column that fall outside the set indicated by the mask are unchanged:
		\begin{eqnarray} 
			\bold{L}({\sf C}) & = & \{ (i,j,C_{ij}) : j \neq {\sf col\_index} \} \cup \nonumber \\
			& & \{(i,{\sf col\_index},\vector{\widetilde{c}}(i)) : i \in (\bold{ind}(\vector{\widetilde{c}}) 
		\cap \bold{ind}(\neg \vector{\widetilde{m}})) \} \cup \nonumber \\
			& & \{(i,{\sf col\_index},z_{i}) : i \in 
    (\bold{ind}(\vector{\widetilde{z}}) \cap \bold{ind}(\vector{\widetilde{m}})) \}. \nonumber
		\end{eqnarray}
\end{itemize}

In {\sf GrB\_BLOCKING} mode, the method exits with return value 
{\sf GrB\_SUCCESS} and the new content of vector {\sf w} is as defined above
and fully computed.  
In {\sf GrB\_NONBLOCKING} mode, the method exits with return value 
{\sf GrB\_SUCCESS} and the new content of vector {\sf w} is as defined above 
but may not be fully computed; however, it can be used in the next GraphBLAS 
method call in a sequence.


%-----------------------------------------------------------------------------
\subsubsection{{\sf Matrix\_assign}: Row variant}

Assign the contents a vector to a subset of elements in one row of a matrix. 
Note that since the output cannot be transposed, a different variant of
{\sf assign} is provided to assign to a column of a matrix.

\paragraph{\syntax}

\begin{verbatim}
        GrB_Info GrB_Matrix_assign(GrB_Matrix              C,
                                   const GrB_Vector        mask,
                                   const GrB_BinaryOp      accum,
                                   const GrB_Vector        u,
                                   GrB_Index               row_index,
                                   const GrB_Index        *col_indices,
                                   const GrB_Index         ncols,
                                   const GrB_Descriptor    desc); 
\end{verbatim}

\paragraph{Parameters}

\begin{itemize}[leftmargin=1.1in]
    \item[{\sf C}]    ({\sf INOUT}) An existing GraphBLAS Matrix.  On input,
    the matrix provides values that may be accumulated with the result of the
    assign operation.  On output, this matrix holds the results of the
    operation.

    \item[{\sf mask}] ({\sf IN}) An optional ``write'' mask that controls which
    results from this operation are stored into the specified row of the output matrix {\sf C}. The 
    mask dimensions must match those of a single row of the matrix {\sf C} and the domain of the 
    {\sf Mask} matrix must be of type {\sf bool} or any of the predefined 
    ``built-in'' types in Table~\ref{Tab:PredefinedTypes}.  If the default
    vector is desired (\ie, with correct dimensions and filled with {\sf true}), 
    {\sf GrB\_NULL} should be specified.
    
    \item[{\sf accum}]    ({\sf IN}) An optional operator used for accumulating
    entries into existing {\sf C} entries: ${\sf accum} = \langle D_x,
    D_y, D_z,\odot \rangle$. If assignment rather than accumulation is
    desired, {\sf GrB\_NULL} should be specified.

    \item[{\sf u}]       ({\sf IN}) The GraphBLAS vector whose contents are 
    assigned to (a subset of) a row of {\sf C}.

    \item[{\sf row\_index}]  ({\sf IN}) The index of the row in {\sf C} to 
    assign. Must be in the range $[0, \bold{nrows}({\sf C}))$.

    \item[{\sf col\_indices}]  ({\sf IN}) Pointer to the ordered set (array) of 
    indices corresponding to the locations in the specified row of {\sf C} 
    that are to be assigned.  If all elements of the row in {\sf C} are to be 
    assigned in order from index $0$ to ${\sf ncols} - 1$, then {\sf GrB\_ALL} should be 
    specified.  Regardless of execution mode and return value, this array may be
    manipulated by the caller after this operation returns without affecting any 
    deferred computations for this operation.
    If this array contains duplicate values, it implies in assignment of more 
    than one value to the same location which leads to undefined results.
    
    \item[{\sf ncols}] ({\sf IN}) The number of values in {\sf col\_indices} array.
    Must be equal to $\bold{size}({\sf u})$.

    \item[{\sf desc}] ({\sf IN}) An optional operation descriptor. If
    a \emph{default} descriptor is desired, {\sf GrB\_NULL} should be
    specified. Non-default field/value pairs are listed as follows:  \\

    \begin{tabular}{lllp{2.5in}}
        Param & Field  & Value & Description \\
        \hline
        {\sf C}    & {\sf GrB\_OUTP} & {\sf GrB\_REPLACE} &   Output row in 
        {\sf C} is cleared (all elements removed) before result is stored in it. \\
        
        {\sf mask} & {\sf GrB\_MASK} & {\sf GrB\_SCMP} & Use the structural 
        complement of {\sf mask}. \\
    \end{tabular}

\end{itemize}

\paragraph{Return Values}

\begin{itemize}[leftmargin=2.1in]
    \item[{\sf GrB\_SUCCESS}]         In blocking mode, the operation completed
    successfully. In non-blocking mode, this indicates that the compatibility 
    tests on dimensions and domains for the input arguments passed successfully. 
    Either way, output matrix {\sf C} is ready to be used in the next method of 
    the sequence.

    \item[{\sf GrB\_PANIC}]            Unknown internal error.

    \item[{\sf GrB\_INVALID\_OBJECT}] This is returned in any execution mode 
    whenever one of the opaque GraphBLAS objects (input or output) is in an invalid 
    state caused by a previous execution error.  Call {\sf GrB\_error()} to access 
    any error messages generated by the implementation.

    \item[{\sf GrB\_OUT\_OF\_MEMORY}]  Not enough memory available for operation.

    \item[{\sf GrB\_UNINITIALIZED\_OBJECT}] One or more of the GraphBLAS objects
    has not been initialized by a call to {\sf new} (or {\sf dup} for vector or
    matrix parameters).

    \item[{\sf GrB\_INVALID\_INDEX}]    {\sf row\_index} is outside the allowable 
    range (i.e., greater than $\bold{nrows}({\sf C})$).

    \item[{\sf GrB\_INDEX\_OUT\_OF\_BOUNDS}]  A value in {\sf col\_indices} 
    is greater than or equal to $\bold{ncols}({\sf C})$.  In 
    non-blocking mode, this can be reported as an execution error.

    \item[{\sf GrB\_DIMENSION\_MISMATCH}] {\sf mask} size and number of columns
    in {\sf C} are not the same, or ${\sf ncols} \neq \bold{size}({\sf u})$.

    \item[{\sf GrB\_DOMAIN\_MISMATCH}]     The domains of the matrix and vector
    are incompatible with each other or the corresponding domains of the 
    accumulation operator, or the mask's domain is not compatible with {\sf bool}.

    \item[{\sf GrB\_NULL\_POINTER}] Argument {\sf col\_indices} is a {\sf NULL} pointer.
\end{itemize}

\paragraph{Description}

This variant of {\sf GrB\_assign} computes the result of assigning a subset of
locations in a row of a GraphBLAS matrix (in a specific order) from the 
contents of a GraphBLAS vector: 
${\sf C}({\sf row\_index},:) = {\sf u}$; or, if an 
optional binary accumulation operator ($\odot$) is provided, 
${\sf C}({\sf row\_index},:) = 
{\sf C}({\sf row\_index},:) \odot {\sf u}$. Taking order into account it is
more explicitly written as:
\[
\begin{aligned}
    {\sf C}({\sf row\_index},{\sf col\_indices}[j]) = &\ {\sf u}(j),
    \ \forall \ j \ : \ 0 \leq j < {\sf ncols}, \mbox{~or~}
    \\
    {\sf C}({\sf row\_index},{\sf col\_indices}[j]) = &\ {\sf C}({\sf row\_index},{\sf col\_indices}[j]) \odot {\sf u}(j), 
    \ \forall \ j \ : \ 0 \leq j < {\sf ncols}
\end{aligned}
\]  
Logically, this operation occurs in three steps:
\begin{enumerate}[leftmargin=0.75in]
\item[\bf Setup] The internal matrices, vectors and mask used in the computation are formed 
and their domains and dimensions are tested for compatibility.
\item[\bf Compute] The indicated computations are carried out.
\item[\bf Output] The result is written into the output matrix, possibly under 
control of a mask.
\end{enumerate}

Up to three argument vectors and matrices are used in this {\sf GrB\_assign} 
operation:
\begin{enumerate}
	\item ${\sf C} = \langle \bold{D}({\sf C}),\bold{nrows}({\sf C}),
    \bold{ncols}({\sf C}),\bold{L}({\sf C}) = \{(i,j,C_{ij}) \} \rangle$
    
	\item ${\sf mask} = \langle \bold{D}({\sf mask}),\bold{size}({\sf mask}),
    \bold{L}({\sf mask}) = \{(i,m_{i}) \} \rangle$ (optional)

	\item ${\sf u} = \langle \bold{D}({\sf u}),\bold{size}({\sf u}),
    \bold{L}({\sf u}) = \{(i,u_i) \} \rangle$
\end{enumerate}

The argument vectors, matrix, and the accumulation 
operator (if provided) are tested for domain compatibility as follows:
\begin{enumerate}
	\item The domain of {\sf mask} (if not {\sf GrB\_NULL}) must be from one of 
    the pre-defined types of Table~\ref{Tab:PredefinedTypes}.

	\item If {\sf accum} is {\sf GrB\_NULL}, then $\bold{D}({\sf C})$ must be 
    compatible with $\bold{D}({\sf u})$.

	\item If {\sf accum} is not {\sf GrB\_NULL}, then $\bold{D}({\sf C})$ must be
    compatible with $D_x$ and $D_z$ of the accumulation operator and 
    $\bold{D}({\sf u})$ must be compatible with $D_y$ of the accumulation operator.
\end{enumerate}
Two domains are compatible with each other if values from one domain can be cast 
to values in the other domain as per the rules of the C language.
In particular, domains from Table~\ref{Tab:PredefinedTypes} are all compatible 
with each other. A domain from a user-defined type is only compatible with itself.
If any compatibility rule above is violated, execution of {\sf GrB\_assign} ends
and the domain mismatch error listed above is returned.

The {\sf row\_index} parameter is checked for a valid value.  The following
condition must hold:
\begin{enumerate}
	\item $0\ \leq\ {\sf row\_index} \ <\ \bold{nrows}({\sf C})$
\end{enumerate}
If the rule above is violated, execution of {\sf GrB\_assign} ends 
and the invalid index error listed above is returned.

From the arguments, the internal vectors, mask, and index array used in 
the computation are formed ($\leftarrow$ denotes copy):
\begin{enumerate}
	\item The vector, $\vector{\widetilde{c}}$, is extracted from a row of {\sf C}
    as follows:
    \[
        \vector{\widetilde{c}} = \langle  \bold{D}({\sf C}), \bold{ncols}({\sf C}), 
        \{ (j, C_{ij}) \ \forall \ j : 0 \leq j < \bold{ncols}({\sf C}),
        i = {\sf row\_index}, (i, j) \in \bold{ind}({\sf C})\} \rangle
    \]

	\item One-dimensional mask, $\vector{\widetilde{m}}$, is computed from 
    argument {\sf mask} as follows:
	\begin{enumerate}
		\item	If ${\sf mask} = {\sf GrB\_NULL}$, then $\vector{\widetilde{m}} = 
        \langle \bold{ncols}({\sf C}), \{i, \forall i : 0 \leq i < 
        \bold{ncols}({\sf C}) \} \rangle$.

		\item	Otherwise, $\vector{\widetilde{m}} = 
        \langle \bold{size}({\sf mask}), \{i : i \in \bold{ind}({\sf mask}) \wedge
        ({\sf bool}){\sf mask}(i) = \true \} \rangle$.

		\item	If ${\sf desc[GrB\_MASK].GrB\_SCMP}$ is \true, then 
        $\vector{\widetilde{m}} \leftarrow \neg \vector{\widetilde{m}}$.
	\end{enumerate}

	\item Vector $\vector{\widetilde{u}} \leftarrow {\sf u}$.
    
    \item The internal column index array, $\grbarray{\widetilde{J}}$, is computed from 
    argument {\sf col\_indices} as follows:
	\begin{enumerate}
		\item	If ${\sf col\_indices} = {\sf GrB\_ALL}$, then 
        $\grbarray{\widetilde{J}}[j] = j, \forall j : 0 \leq j < {\sf ncols}$.

		\item	Otherwise, $\grbarray{\widetilde{J}}[j] = {\sf col\_indices}[j], 
        \forall j : 0 \leq j < {\sf ncols}$.
    \end{enumerate}
\end{enumerate}

The internal vectors, matrices, and masks are checked for dimension compatibility. 
The following conditions must hold:
\begin{enumerate}
	\item $\bold{size}(\vector{\widetilde{c}}) = \bold{size}(\vector{\widetilde{m}})$
    \item ${\sf ncols} = \bold{size}(\vector{\widetilde{u}})$.
\end{enumerate}
If any compatibility rule above is violated, execution of {\sf GrB\_assign} ends and 
the dimension mismatch error listed above is returned.

From this point forward, in {\sf GrB\_NONBLOCKING} mode, the method can 
optionally exit with {\sf GrB\_SUCCESS} return code and defer any computation 
and/or execution error codes.

We are now ready to carry out the assign and any additional 
associated operations.  We describe this in terms of two intermediate vectors:
\begin{itemize}
    \item $\vector{\widetilde{t}}$: The vector holding the elements from
    $\vector{\widetilde{u}}$ in their destination locations relative to 
    $\vector{\widetilde{c}}$.
    
    \item $\vector{\widetilde{z}}$: The vector holding the result after 
    application of the (optional) accumulation.
\end{itemize}

The intermediate vector, $\vector{\widetilde{t}}$, is created as follows:
\[
\vector{\widetilde{t}} = \langle
\bold{D}({\sf u}), \bold{size}(\vector{\widetilde{c}}),
%\bold{L}(\vector{\widetilde{t}}) =
\{(\grbarray{\widetilde{J}}[j],\vector{\widetilde{u}}(j)) \ \forall \ j, \ 
0 \leq j < {\sf ncols} : j \in \bold{ind}(\vector{\widetilde{u}}) \} \rangle. 
\]
At this point, if any value of $\grbarray{\widetilde{J}}[j]$ is outside the valid 
range of indices for vector $\vector{\widetilde{c}}$, computation ends and the 
method returns the index out-of-bounds error listed above. In 
{\sf GrB\_NONBLOCKING} mode, the error can be deferred until a 
sequence-terminating {\sf GrB\_wait()} is called.  Regardless, the result 
matrix, {\sf C}, is invalid from this point forward in the 
sequence.

The intermediate vector $\vector{\widetilde{z}}$ is created as follows:
\begin{itemize}
    \item If ${\sf accum} = {\sf GrB\_NULL}$, then 
    $\vector{\widetilde{z}} = \vector{\widetilde{t}}$.

    \item If ${\sf accum} = \langle D_x, D_y, D_z, \odot \rangle$, then vector 
    $\vector{\widetilde{z}}$ is defined as 
        \[ 
        \vector{\widetilde{z}} =
        \langle D_z, \bold{size}(\vector{\widetilde{c}}), 
        %\bold{L}(\vector{\widetilde{z}}) = 
        \{(j,z_{j}) \ \forall \ j \in \bold{ind}(\vector{\widetilde{c}}) \cup 
        \bold{ind}(\vector{\widetilde{t}}) \} \rangle.
        \]
    The values of the elements of $\vector{\widetilde{z}}$ are computed based on the relationships between the sets of indices in $\vector{\widetilde{w}}$ and $\vector{\widetilde{t}}$.
\[
z_{j} = \vector{\widetilde{c}}(j) \odot \vector{\widetilde{t}}(j), \ \mbox{if}\  
j \in  (\bold{ind}(\vector{\widetilde{t}}) \cap \bold{ind}(\vector{\widetilde{c}})),
\]
\[
z_{j} = \vector{\widetilde{c}}(j), \ \mbox{if}\  j \in  (\bold{ind}(
\vector{\widetilde{c}}) - (\bold{ind}(\vector{\widetilde{t}}) \cap 
\bold{ind}(\vector{\widetilde{c}}))),
\]
\[
z_{j} = \vector{\widetilde{t}}(j), \ \mbox{if}\  j \in  (\bold{ind}(
\vector{\widetilde{t}}) - (\bold{ind}(\vector{\widetilde{t}}) \cap 
\bold{ind}(\vector{\widetilde{c}}))).
\]
where the difference operator in the previous expressions refers to set difference.
\end{itemize}

Finally, the set of output values that make up the $\vector{\widetilde{z}}$ 
vector are written into the column of the final result matrix, 
${\sf C}({\sf row\_index},:)$.  This is carried out under control of the mask 
which acts as a ``write mask''.
\begin{itemize}
    \item If {\sf desc[GrB\_OUTP].GrB\_REPLACE} is set then any values in 
    ${\sf C}({\sf row\_index},:)$ on input to {\sf GrB\_assign()} are deleted 
    and the new content of the column is given by:
    \[ 
		\bold{L}({\sf C}) = \{ (i,j,C_{ij}) : i \neq {\sf row\_index} \} \cup \{({\sf row\_index},j,z_{j}) : j \in 
    (\bold{ind}(\vector{\widetilde{z}}) \cap \bold{ind}(\vector{\widetilde{m}})) \}. 
    \]

    \item If {\sf desc[GrB\_OUTP].GrB\_REPLACE} is not set, the elements of 
    $\vector{\widetilde{z}}$ indicated by the mask are copied into the column 
    of the final result matrix, ${\sf C}({\sf row\_index},:)$, and elements of 
    this column that fall outside the set indicated by the mask are unchanged:
	\begin{eqnarray} 
		\bold{L}({\sf C}) & = & \{ (i,j,C_{ij}) : i \neq {\sf row\_index} \} \cup \nonumber \\
		& & \{({\sf row\_index},j,\vector{\widetilde{c}}(j)) : j \in (\bold{ind}(\vector{\widetilde{c}}) \cap \bold{ind}(\neg \vector{\widetilde{m}})) \} \cup \nonumber \\ 
		& & \{({\sf row\_index},j,z_{j})                     : j \in (\bold{ind}(\vector{\widetilde{z}}) \cap \bold{ind}(\vector{\widetilde{m}})) \}. \nonumber
	\end{eqnarray}
\end{itemize}

In {\sf GrB\_BLOCKING} mode, the method exits with return value 
{\sf GrB\_SUCCESS} and the new content of vector {\sf w} is as defined above
and fully computed.  
In {\sf GrB\_NONBLOCKING} mode, the method exits with return value 
{\sf GrB\_SUCCESS} and the new content of vector {\sf w} is as defined above 
but may not be fully computed; however, it can be used in the next GraphBLAS 
method call in a sequence.


%-----------------------------------------------------------------------------
\subsubsection{{\sf Matrix\_assign}: Constant matrix variant}

Assign the same value to a specified subgraph.  With the use of {\sf GrB\_ALL}, 
the entire destination matrix can be filled with the constant.

\paragraph{\syntax}

\begin{verbatim}
        GrB_Info GrB_Matrix_assign(GrB_Matrix              C,
                                   const GrB_Matrix        Mask,
                                   const GrB_BinaryOp      accum,
                                   <type>                  val,
                                   const GrB_Index        *row_indices,
                                   const GrB_Index         nrows,
                                   const GrB_Index        *col_indices,
                                   const GrB_Index         ncols,
                                   const GrB_Descriptor    desc);
\end{verbatim}

\paragraph{Parameters}

\begin{itemize}[leftmargin=1.1in]
    \item[{\sf C}]    ({\sf INOUT}) An existing GraphBLAS Matrix.  On input,
    the matrix provides values that may be accumulated with the result of the
    assign operation.  On output, this matrix holds the results of the
    operation.

    \item[{\sf Mask}] ({\sf IN}) An optional ``write'' mask that controls which
    results from this operation are stored into the output matrix {\sf C}. The 
    mask dimensions must match those of the matrix {\sf C} and the domain of the 
    {\sf Mask} matrix must be of type {\sf bool} or any of the predefined 
    ``built-in'' types in Table~\ref{Tab:PredefinedTypes}.  If the default
    matrix is desired (\ie, with correct dimensions and filled with {\sf true}), 
    {\sf GrB\_NULL} should be specified.

    \item[{\sf accum}]    ({\sf IN}) An optional operator used for accumulating
    entries into existing {\sf C} entries: ${\sf accum} = \langle D_x,
    D_y, D_z,\odot \rangle$. If assignment rather than accumulation is
    desired, {\sf GrB\_NULL} should be specified.

    \item[{\sf val}]    ({\sf IN}) Scalar value to assign to (a subset of) {\sf C}.
    
    \item[{\sf row\_indices}]  ({\sf IN}) Pointer to the ordered set (array) of 
    indices corresponding to the rows of {\sf C} that are assigned.  If all rows
    of {\sf C} are to be assigned in order from $0$ to ${\sf nrows} - 1$, then 
    {\sf GrB\_ALL} can be specified.  Regardless of execution mode and return 
    value, this array may be manipulated by the caller after this operation 
    returns without affecting any deferred computations for this operation.  
    Unlike other variants, if there are duplicated values in this array the 
    result is still defined.

    \item[{\sf nrows}] ({\sf IN}) The number of values in {\sf row\_indices} 
	array. Must be in the range: $[0, \bold{nrows}({\sf C})]$.  If
    {\sf nrows} is zero, the operation becomes a NO-OP.

    \item[{\sf col\_indices}]  ({\sf IN}) Pointer to the ordered set (array) of 
    indices corresponding to the columns of {\sf C} that are assigned.  If all columns
    of {\sf C} are to be assigned in order from $0$ to ${\sf ncols} - 1$, then 
    {\sf GrB\_ALL} can be specified.  Regardless of execution mode and return 
    value, this array may be manipulated by the caller after this operation 
    returns without affecting any deferred computations for this operation.
    Unlike other variants, if there are duplicated values in this array the 
    result is still defined.

    \item[{\sf ncols}] ({\sf IN}) The number of values in {\sf col\_indices} 
	array. Must be in the range: $[0, \bold{ncols}({\sf C})]$.  If
    {\sf ncols} is zero, the operation becomes a NO-OP.

    \item[{\sf desc}] ({\sf IN}) An optional operation descriptor. If
    a \emph{default} descriptor is desired, {\sf GrB\_NULL} should be
    specified. Non-default field/value pairs are listed as follows:  \\

    \begin{tabular}{lllp{2.5in}}
    Param & Field  & Value & Description \\
    \hline
        {\sf C}    & {\sf GrB\_OUTP} & {\sf GrB\_REPLACE} & Output matrix {\sf C} 
        is cleared (all elements removed) before result is stored in it.\\
    
        {\sf mask} & {\sf GrB\_MASK} & {\sf GrB\_SCMP} & Use the structural 
        complement of {\sf Mask}. \\
    \end{tabular}

\end{itemize}

\paragraph{Return Values}

\begin{itemize}[leftmargin=2.1in]
    \item[{\sf GrB\_SUCCESS}]         In blocking mode, the operation completed
    successfully. In non-blocking mode, this indicates that the compatibility 
    tests on dimensions and domains for the input arguments passed successfully. 
    Either way, output matrix {\sf C} is ready to be used in the next method of 
    the sequence.

    \item[{\sf GrB\_PANIC}]            Unknown internal error.
    
    \item[{\sf GrB\_INVALID\_OBJECT}] This is returned in any execution mode 
    whenever one of the opaque GraphBLAS objects (input or output) is in an invalid 
    state caused by a previous execution error.  Call {\sf GrB\_error()} to access 
    any error messages generated by the implementation.

    \item[{\sf GrB\_OUT\_OF\_MEMORY}]  Not enough memory available for operation.

    \item[{\sf GrB\_UNINITIALIZED\_OBJECT}] One or more of the GraphBLAS objects
    has not been initialized by a call to {\sf new} (or {\sf dup} for vector
    parameters).

    \item[{\sf GrB\_INDEX\_OUT\_OF\_BOUNDS}]  A value in {\sf row\_indices} is greater
    than or equal to $\bold{nrows}({\sf C})$, or a value in {\sf col\_indices} is greater
    than or equal to $\bold{ncols}({\sf C})$.  In non-blocking mode, this can be
    reported as an execution error.

    \item[{\sf GrB\_DIMENSION\_MISMATCH}]  {\sf Mask} and {\sf C} dimensions are
    incompatible, {\sf nrows} is not less than $\bold{nrows}({\sf C})$, or
    {\sf ncols} is not less than $\bold{ncols}({\sf C})$. 

    \item[{\sf GrB\_DOMAIN\_MISMATCH}]     The domains of the matrix and scalar are
	incompatible with each other or the corresponding domains of the 
    accumulation operator, or the mask's domain is not compatible with {\sf bool}.

    \item[{\sf GrB\_NULL\_POINTER}] Either argument {\sf row\_indices} is a {\sf NULL} pointer,
	    argument {\sf col\_indices} is a {\sf NULL} pointer, or both.
\end{itemize}


\paragraph{Description}


This variant of {\sf GrB\_assign} computes the result of assigning a constant
scalar value to locations in a destination GraphBLAS matrix: 
${\sf C}({\sf row\_indices, col\_indices}) = {\sf val}$; or, if an optional 
binary accumulation operator ($\odot$) is provided, 
${\sf C}({\sf row\_indices, col\_indices}) = 
{\sf w}({\sf row\_indices, col\_indices}) \odot {\sf val}$.  
More explicitly:
\[
\begin{aligned}
	{\sf C}({\sf row\_indices}[i], {\sf col\_indices}[j]) =\ & {\sf val} \mbox{~~or~~} \\
    {\sf C}({\sf row\_indices}[i], {\sf col\_indices}[j]) =\ & 
              {\sf C}({\sf row\_indices}[i], {\sf col\_indices}[j]) \odot {\sf val} \\
    & \forall \ (i,j) \ : \ 0 \leq i < {\sf nrows},  0 \leq j < {\sf ncols}
\end{aligned}
\]  
Logically, this operation occurs in three steps:
\begin{enumerate}[leftmargin=0.75in]
\item[Setup] The internal vectors and mask used in the computation are formed 
and their domains and dimensions are tested for compatibility.
\item[Compute] The indicated computations are carried out.
\item[Output] The result is written into the output matrix, possibly under 
control of a mask.
\end{enumerate}

Up to two argument matrices are used in the {\sf GrB\_assign} operation:
\begin{enumerate}
	\item ${\sf C} = \langle \bold{D}({\sf C}),\bold{nrows}({\sf C}),
    \bold{ncols}({\sf C}),\bold{L}({\sf C}) = \{(i,j,C_{ij}) \} \rangle$
    
	\item ${\sf Mask} = \langle \bold{D}({\sf Mask}),\bold{nrows}({\sf Mask}),
    \bold{ncols}({\sf Mask}),\bold{L}({\sf Mask}) = \{(i,j,M_{ij}) \} \rangle$ (optional)
\end{enumerate}

The argument scalar, matrices, and the accumulation 
operator (if provided) are tested for domain compatibility as follows:
\begin{enumerate}
	\item The domain of {\sf Mask} (if not {\sf GrB\_NULL}) must be from one of 
    the pre-defined types of Table~\ref{Tab:PredefinedTypes}.

	\item If {\sf accum} is {\sf GrB\_NULL}, then $\bold{D}({\sf C})$ must be 
    compatible with $\bold{D}({\sf val})$.

	\item If {\sf accum} is not {\sf GrB\_NULL}, then $\bold{D}({\sf C})$ must be
    compatible with $D_x$ and $D_z$ of the accumulation operator and 
    $\bold{D}({\sf val})$ must be compatible with $D_y$ of the accumulation operator.
\end{enumerate}
Two domains are compatible with each other if values from one domain can be cast 
to values in the other domain as per the rules of the C language.
In particular, domains from Table~\ref{Tab:PredefinedTypes} are all compatible 
with each other. A domain from a user-defined type is only compatible with itself.
If any compatibility rule above is violated, execution of {\sf GrB\_assign} ends
and the domain mismatch error listed above is returned.

From the arguments, the internal matrices, index arrays, and mask used in 
the computation are formed ($\leftarrow$ denotes copy):
\begin{enumerate}
	\item Matrix $\matrix{\widetilde{C}} \leftarrow {\sf C}$.

	\item Two-dimensional mask $\matrix{\widetilde{M}}$ is computed from 
    argument {\sf Mask} as follows:
	\begin{enumerate}

		\item	If ${\sf Mask} = {\sf GrB\_NULL}$, then $\matrix{\widetilde{M}} = 
        \langle \bold{nrows}({\sf C}), \bold{ncols}({\sf C}), \{(i,j), 
        \forall i,j : 0 \leq i <  \bold{nrows}({\sf C}), 0 \leq j < 
        \bold{ncols}({\sf C}) \} \rangle$.

		\item	Otherwise, $\matrix{\widetilde{M}} = \langle 
        \bold{nrows}({\sf Mask}), \bold{ncols}({\sf Mask}), \{(i,j) : 
        (i,j) \in \bold{ind}({\sf Mask}) \wedge 
        ({\sf bool}){\sf Mask}(i,j) = \true\} \rangle$.

		\item	If ${\sf desc[GrB\_MASK].GrB\_SCMP}$ is set, then 
        $\matrix{\widetilde{M}} \leftarrow \neg \matrix{\widetilde{M}}$.
	\end{enumerate}

    \item The internal row index array, $\grbarray{\widetilde{I}}$, is computed from 
    argument {\sf row\_indices} as follows:
	\begin{enumerate}
		\item	If ${\sf row\_indices} = {\sf GrB\_ALL}$, then 
        $\grbarray{\widetilde{I}}[i] = i, \forall i : 0 \leq i < {\sf nrows}$.

		\item	Otherwise, $\grbarray{\widetilde{I}}[i] = {\sf row\_indices}[i], 
        \forall i : 0 \leq i < {\sf nrows}$.
    \end{enumerate}
    
    \item The internal column index array, $\grbarray{\widetilde{J}}$, is computed from 
    argument {\sf col\_indices} as follows:
	\begin{enumerate}
		\item	If ${\sf col\_indices} = {\sf GrB\_ALL}$, then 
        $\grbarray{\widetilde{J}}[j] = j, \forall j : 0 \leq j < {\sf ncols}$.

		\item	Otherwise, $\grbarray{\widetilde{J}}[j] = {\sf col\_indices}[j], 
        \forall j : 0 \leq j < {\sf ncols}$.
    \end{enumerate}
\end{enumerate}

The internal matrix and mask are checked for dimension compatibility. 
The following conditions must hold:
\begin{enumerate}
    \item $\bold{nrows}(\matrix{\widetilde{C}}) = \bold{nrows}(\matrix{\widetilde{M}})$.

    \item $\bold{ncols}(\matrix{\widetilde{C}}) = \bold{ncols}(\matrix{\widetilde{M}})$.

    \item $0 \leq {\sf nrows} \leq \bold{nrows}(\matrix{\widetilde{C}})$.

    \item $0 \leq {\sf ncols} \leq \bold{ncols}(\matrix{\widetilde{C}})$.
\end{enumerate}
If any compatibility rule above is violated, execution of {\sf GrB\_assign} ends and 
the dimension mismatch error listed above is returned.

From this point forward, in {\sf GrB\_NONBLOCKING} mode, the method can 
optionally exit with {\sf GrB\_SUCCESS} return code and defer any computation 
and/or execution error codes.

We are now ready to carry out the assign and any additional 
associated operations.  We describe this in terms of two intermediate vectors:
\begin{itemize}
	\item $\matrix{\widetilde{T}}$: The matrix holding the copies of the scalar 
    {\sf val} in their destination locations relative to 
    $\matrix{\widetilde{C}}$.
    
	\item $\matrix{\widetilde{Z}}$: The matrix holding the result after 
    application of the (optional) accumulation operator.
\end{itemize}

The intermediate matrix, $\matrix{\widetilde{T}}$, is created as follows:
\[
\begin{aligned}
\matrix{\widetilde{T}} = \langle & \bold{D}({\sf val}),
                           \bold{nrows}(\matrix{\widetilde{C}}), 
                           \bold{ncols}(\matrix{\widetilde{C}}), \\
 & \{ (\grbarray{\widetilde{I}}[i],\grbarray{\widetilde{J}}[j], {\sf val}) 
 \ \forall \ (i,j), \ 0 \leq i < {\sf nrows}, \ 0 \leq j < {\sf ncols} \} \rangle. 
\end{aligned}
\]
If either $\grbarray{\widetilde{I}}$ or $\grbarray{\widetilde{J}}$ is empty, this 
operation results in an empty matrix, $\matrix{\widetilde{T}}$.  Otherwise, if 
any value in the $\grbarray{\widetilde{I}}$ array is not in
the range $[0,\ \bold{nrows}(\matrix{\widetilde{C}}) )$ or any value in the 
$\grbarray{\widetilde{J}}$ array is not in the range 
$[0,\ \bold{ncols}(\matrix{\widetilde{C}}))$, the execution of {\sf GrB\_assign} 
ends and the index out-of-bounds error listed above is generated. In 
{\sf GrB\_NONBLOCKING} mode, the error can be deferred until a 
sequence-terminating {\sf GrB\_wait()} is called.  Regardless, the result 
matrix {\sf C} is invalid from this point forward in the sequence.

The intermediate matrix $\matrix{\widetilde{Z}}$ is created as follows:
\begin{itemize}
    \item If ${\sf accum} = {\sf GrB\_NULL}$, then 
    $\matrix{\widetilde{Z}} = \matrix{\widetilde{T}}$.

    \item If ${\sf accum} = \langle D_x, D_y, D_z, \odot \rangle$, then matrix 
    $\matrix{\widetilde{Z}}$ is defined as 
        \[ 
        \matrix{\widetilde{Z}} = 
        \langle D_z, \bold{nrows}(\matrix{\widetilde{C}}),
        \bold{ncols}(\matrix{\widetilde{C}}), 
        %\bold{L}(\matrix{\widetilde{Z}}) =
	    \{(i,j,Z_{ij}) \ \forall \ (i,j) \in \bold{ind}(\matrix{\widetilde{C}}) \cup 
        \bold{ind}(\matrix{\widetilde{T}}) \} \rangle.\]
    The values of the elements of $\matrix{\widetilde{Z}}$ are computed based on 
    the relationships between the sets of indices in $\matrix{\widetilde{C}}$ and 
    $\matrix{\widetilde{T}}$.
\[
	Z_{ij} = \matrix{\widetilde{C}}(i,j) \odot \matrix{\widetilde{T}}(i,j), 
    \mbox{~if~} (i,j) \in  (\bold{ind}(\matrix{\widetilde{T}}) \cap 
    \bold{ind}(\matrix{\widetilde{C}})),
\]
\[
	Z_{ij} = \matrix{\widetilde{C}}(i,j), \mbox{~if~} (i,j) \in  
    (\bold{ind}(\matrix{\widetilde{C}}) - (\bold{ind}(\matrix{\widetilde{T}}) \cap 
    \bold{ind}(\matrix{\widetilde{C}}))),
\]
\[
	Z_{ij} = \matrix{\widetilde{T}}(i,j), \mbox{~if~} (i,j) \in  
    (\bold{ind}(\matrix{\widetilde{T}}) - (\bold{ind}(\matrix{\widetilde{T}}) \cap 
    \bold{ind}(\matrix{\widetilde{C}}))).
\]
where the difference operator in the previous expressions refers to set difference.
\end{itemize}

Finally, the set of output values that make up the $\matrix{\widetilde{Z}}$ 
matrix are written into the final result matrix, ${\sf C}$. 
This is carried out under control of the mask which acts as a ``write mask''.
\begin{itemize}
\item If {\sf desc[GrB\_OUTP].GrB\_REPLACE} is set then any values in ${\sf C}$ 
on input to {\sf GrB\_assign()} are deleted and the new output matrix ${\sf C}$ is,
		\[ \bold{L}({\sf C}) = \{(i,j,Z_{ij}) : (i,j) \in (\bold{ind}(\matrix{\widetilde{Z}}) 
\cap \bold{ind}(\matrix{\widetilde{M}})) \}. \]

\item If {\sf desc[GrB\_OUTP].GrB\_REPLACE} is not set, the elements of 
$\matrix{\widetilde{Z}}$ indicated by 
the mask are copied into the result matrix, ${\sf C}$, and elements of 
${\sf C}$ that fall outside the set indicated by the mask are unchanged:
		\[ \bold{L}({\sf C}) = \{(i,j,C_{ij}) : (i,j) \in (\bold{ind}(\matrix{\sf C}) 
		\cap \bold{ind}(\neg \matrix{\widetilde{M}})) \} \cup \{(i,j,Z_{ij}) : (i,j) \in 
(\bold{ind}(\matrix{\widetilde{Z}}) \cap \bold{ind}(\matrix{\widetilde{M}})) \}. \]
\end{itemize}

In {\sf GrB\_BLOCKING} mode, the method exits with return value 
{\sf GrB\_SUCCESS} and the new content of matrix {\sf C} is as defined above
and fully computed.  
In {\sf GrB\_NONBLOCKING} mode, the method exits with return value 
{\sf GrB\_SUCCESS} and the new content of matrix {\sf C} is as defined above 
but may not be fully computed; however, it can be used in the next GraphBLAS 
method call in a sequence.

