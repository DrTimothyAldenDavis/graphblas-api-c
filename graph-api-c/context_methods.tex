\section{Context Methods}

The methods in this section set up and tear down the GraphBLAS
context within which all GraphBLAS methods must occur.  The initialization
of this context also includes the specification of which execution mode is
to be used.

%-----------------------------------------------------------------------------
\subsection{{\sf init}: Initialize a GraphBLAS context}

Creates and initializes a GraphBLAS C API context.

\paragraph{\syntax}

\begin{verbatim}
        GrB_Info GrB_init(GrB_Mode mode);
\end{verbatim}


\paragraph{Parameters}

\begin{itemize}[leftmargin=1.1in]
	\item[{\sf mode}] Mode for the GraphBLAS context, {\sf GrB\_BLOCKING} or {\sf GrB\_NONBLOCKING}.
\end{itemize}

\paragraph{Return Values}

\scott{Need an error for when init() is called a second time or after finalize. Or if finalize is called without init.  PANIC?}

\begin{itemize}[leftmargin=2.1in]
\item[{\sf GrB\_SUCCESS}]           operation completed successfully.
\item[{\sf GrB\_PANIC}]             unknown internal error.  
\item[{\sf GrB\_INVALID\_VALUE}]    invalid mode specified, or method called multiple times.
\end{itemize}

\paragraph{Description}

Creates and initializes a GraphBLAS C API context.  The argument
to {\sf GrB\_init} defines the mode for the context.  The two
available modes are:

\begin{itemize}
\item {\sf GrB\_BLOCKING}: Methods in a sequence return after computations
in the method have completed and output arguments are available to
subsequent statements in an application.  When executing in {\sf
GrB\_BLOCKING} mode, the methods execute in program order.

\item {\sf GrB\_NONBLOCKING}: Methods in a sequence return after arguments
in the method have been tested for dimension and domain compatibility
within the method but potentially before computations complete.  Output
arguments are available to subsequent GraphBLAS methods in an application.
When executing in {\sf GrB\_NONBLOCKING} mode, the methods in a sequence
may execute in any order that preserves the mathematically result defined
by the sequence.
\end{itemize}

An application can only create one context per execution instance.

%-----------------------------------------------------------------------------
\subsection{{\sf finalize}: Finalize a GraphBLAS context}

Terminates and frees any internal resources created to support the
GraphBLAS C API context.

\paragraph{\syntax}

\begin{verbatim}
        GrB_Info GrB_finalize();
\end{verbatim}

\paragraph{Return Values}

\begin{itemize}[leftmargin=2.1in]
\item[{\sf GrB\_SUCCESS}]        operation completed successfully.
\item[{\sf GrB\_PANIC}]          unknown internal error.
\end{itemize}

\paragraph{Description}

Terminates and frees any internal resources created to support the
GraphBLAS C API context.  An application may not create a new context
after {\sf GrB\_finalize} has been called.
