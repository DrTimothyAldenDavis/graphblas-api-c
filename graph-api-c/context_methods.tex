\section{Context Methods}

The methods in this section set up and tear down the GraphBLAS
context within which all GraphBLAS methods must be executed.  The initialization
of this context also includes the specification of which execution mode is
to be used.

%-----------------------------------------------------------------------------
\subsection{{\sf init}: Initialize a GraphBLAS context}

Creates and initializes a GraphBLAS C API context.

\paragraph{\syntax}

\begin{verbatim}
        GrB_Info GrB_init(GrB_Mode mode);
\end{verbatim}


\paragraph{Parameters}

\begin{itemize}[leftmargin=1.1in]
	\item[{\sf mode}] Mode for the GraphBLAS context. Must be either {\sf GrB\_BLOCKING}
    or {\sf GrB\_NONBLOCKING}.
\end{itemize}

\paragraph{Return Values}

\begin{itemize}[leftmargin=2.1in]
\item[{\sf GrB\_SUCCESS}]           operation completed successfully.
\item[{\sf GrB\_PANIC}]             unknown internal error.
\item[{\sf GrB\_INVALID\_VALUE}]    invalid mode specified, or method called multiple times.
\end{itemize}

\paragraph{Description}

The {\sf init} method creates and initializes a GraphBLAS C API context.  The argument
to {\sf GrB\_init} defines the mode for the context.  The two
available modes are:

\begin{itemize}
\item {\sf GrB\_BLOCKING}: In this mode, each method in a sequence returns after
its computations have completed and output arguments are available to
subsequent statements in an application.  When executing in {\sf
GrB\_BLOCKING} mode, the methods execute in program order.

\item {\sf GrB\_NONBLOCKING}: In this mode, methods in a sequence may return after arguments
in the method have been tested for dimension and domain compatibility
within the method but potentially before their computations complete.  Output
arguments are available to subsequent GraphBLAS methods in an application.
When executing in {\sf GrB\_NONBLOCKING} mode, the methods in a sequence
may execute in any order that preserves the mathematical result defined
by the sequence.
\end{itemize}

An application can only create one context per execution instance.  An application
may only call {\sf GrB\_Init} once.  Calling {\sf GrB\_Init} more
than once results in undefined behavior.

%-----------------------------------------------------------------------------
\subsection{{\sf finalize}: Finalize a GraphBLAS context}
\label{Sec:GrB_finalize}

Terminates and frees any internal resources created to support the
GraphBLAS C API context.

\paragraph{\syntax}

\begin{verbatim}
        GrB_Info GrB_finalize();
\end{verbatim}

\paragraph{Return Values}

\begin{itemize}[leftmargin=2.1in]
\item[{\sf GrB\_SUCCESS}]        operation completed successfully.
\item[{\sf GrB\_PANIC}]          unknown internal error.
\end{itemize}

\paragraph{Description}

The {\sf finalize} method terminates and frees any internal resources created to support the
GraphBLAS C API context.  {\sf GrB\_finalize} may only be called after
a context has been initialized by calling {\sf GrB\_init}, or else undefined
behavior occurs.  After {\sf GrB\_finalize}
has been called to finalize a GraphBLAS context, calls to any GraphBLAS methods,
including {\sf GrB\_finalize}, will result in undefined behavior.


%-----------------------------------------------------------------------------
\subsection{{\sf getVersion}: Get the version number of the standard.}

Query the library for the version number of the standard that this library
implements.

\paragraph{\syntax}

\begin{verbatim}
        GrB_Info GrB_getVersion(unsigned int *version,
                                unsigned int *subversion);
\end{verbatim}


\paragraph{Parameters}

\begin{itemize}[leftmargin=1.1in]
	\item[{\sf version}] ({\sf OUT})  On successful return will hold the value
    of the major version number.
	\item[{\sf version}] ({\sf OUT})  On successful return will hold the value
    of the subversion number.
\end{itemize}


\paragraph{Return Values}

\begin{itemize}[leftmargin=2.1in]
\item[{\sf GrB\_SUCCESS}]        operation completed successfully.
\item[{\sf GrB\_PANIC}]          unknown internal error.
\end{itemize}

\paragraph{Description}

The {\sf getVersion} method is used to query the major and minor version number of the
GraphBLAS C API specification that the library implements at runtime.  To
support compile time queries the following two macros shall also be defined by
the library.
\begin{verbatim}
        #define GRB_VERSION     2
        #define GRB_SUBVERSION  0
\end{verbatim}
