\chapter{Possible Extensions}
\label{Chp:Extensions}

This chapter covers material that is currently under discussion for inclusion in 
future revisions (post-1.0) of the GraphBLAS. The material falls into the
following major categories:

\begin{itemize}
\item Standard definitions.
\item Explicitly split execution.
\end{itemize}

\section{Standard Definitions}

The current GraphBLAS specification could be augmented with a list of 
pre-defined objects, such as descriptors, semirings and monoids, that are commonly used
in building graph algorithms. These standard pre-defined objects would
simplify coding and ensure more consistency across algorithms. 
We emphasize
that individual application writers are always free to create their own set of
pre-defined objects.

Additionally, GraphBLAS could include pre-defined macros that support
shorter calling syntax for methods, by hiding default arguments. 
Again, individual application writers are always free to create their
own set of pre-defined objects.

We propose that such pre-defined objects and macros would
be included in a compilation unit through a construct like:

\begin{verbatim}
#include "GrB_stddef.h"
\end{verbatim}

\section{Explicitly Split Execution}

In its non-blocking mode of execution, GraphBLAS allows specific implementations to support
various execution approaches. In particular, it allows for non-terminating methods to be 
executed is several stages, possibly interleaved with stages from other methods. This is called \emph{split execution}. In a particular split execution,
a method can go through and \emph{analyze} stage followed by a \emph{perform} stage.
The analyze stage computes various characteristics of of the object being produced, whereas the
perform stage does the actual calculations.

It may be desirable to augment GraphBLAS with addidional constructs to control this particular
analyze/perform split explicitly. In particular, having the application communicate properties of
an object that do not change between multiple perform stages, and therefore greatly simplifies
or eliminates the need for analyze stages, could be valuable.

We decided to defer such facilities from version 1.0 of the GraphBLAS specification. We believe
that additional experience with implementation and use of GraphBLAS is necessary before
we can define the proper interfaces for explicit split execution.

\section{Dependency DAGs and {\sf wait} on Specific Objects}

\scott{NEED TO FILL}
