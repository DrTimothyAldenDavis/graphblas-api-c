\subsection{Algebra Methods}


\subsubsection{Create new function ({\sf Function\_new})}

Creates a new function with specified domain, operations and elements.

\paragraph{C99 Syntax}

\begin{verbatim}
#include "GraphBLAS.h"
GrB_info GrB_Function_new(GrB_Function *s,GrB_type t1, GrB_type t2, GrB_type t3,
                          GrB_BinaryFunction a)
\end{verbatim}

GrB\_operation is a typedef to a standard C \emph{function pointer}.


\subsubsection{Create new monoid ({\sf Monoid\_new})}

Creates a new monoid with specified domain, operations and elements.

\paragraph{C99 Syntax}

\begin{verbatim}
#include "GraphBLAS.h"
GrB_info GrB_Monoid_new(GrB_Monoid *s,GrB_type t1, GrB_type t2, GrB_type t3,
                        GrB_BinaryFunction a, <type> z)
\end{verbatim}

\paragraph{Input Parameters}

\begin{itemize}
	\item[{\sf t1}] The type defining the first domain of the monoid being created. Should be one of the predefined
	GraphBLAS types in Table~\ref{Tab:PredefinedTypes}, or a user created type.
	\item[{\sf t2}] The type defining the second domain of the monoid being created. Should be one of the predefined
	GraphBLAS types in Table~\ref{Tab:PredefinedTypes}, or a user created type.
	\item[{\sf t3}] The type defining the third domain of the monoid being created. Should be one of the predefined
	GraphBLAS types in Table~\ref{Tab:PredefinedTypes}, or a user created type.
	\item[{\sf a}] The additive operation of the monoid.
	\item[{\sf z}] The $0$ element of the monoid. Must be of type corresponding to {\sf t3} as per Table~\ref{Tab:PredefinedTypes}.
\end{itemize}

\paragraph{Output Parameter}

\begin{itemize}
	\item[{\sf s}] Identifier of the newly created monoid.
\end{itemize}

\paragraph{Return Value}

\begin{tabular}{rl} 
{\sf GrB\_SUCCESS} 	& operation completed successfully \\
{\sf GrB\_PANIC}	& unknown internal error \\
{\sf GrB\_OUTOFMEM}	& not enough memory available for operation \\
\end{tabular}

\paragraph{Description}

Creates a new monoid $S = \langle \bold{D}({\sf t1}), \bold{D}({\sf t2}), \bold{D}({\sf t3}), {\sf a}, {\sf z} \rangle$ and
returns its identifier in {\sf s}.


\subsubsection{Create new semiring ({\sf Semiring\_new})}

Creates a new semiring with specified domain, operations and elements.

\paragraph{C99 Syntax}

\begin{verbatim}
#include "GraphBLAS.h"
GrB_info GrB_Semiring_new(GrB_Semiring *s,GrB_type t1, GrB_type t2, GrB_type t3,
                          GrB_BinaryFunction a, GrB_BinaryFunction m,
                          <type> z[, <type> o]))
\end{verbatim}

\scott{This signature is a little confusing partially because we have not been explicit with the domain/type specifications earlier. 
If the GrB\_type is the name of an enum for domains (not types) then I would suggest calling it GrB\_domain.  Not sure how {\sf ti} 
can be used as both a type and a parameter and may be related to the domain/type issue.}

\paragraph{Input Parameters}

\begin{itemize}
	\item[{\sf t1}] The type defining the first domain of the semiring being created. Should be one of the predefined
	GraphBLAS types in Table~\ref{Tab:PredefinedTypes}, or a user created type.
	\item[{\sf t2}] The type defining the second domain of the semiring being created. Should be one of the predefined
	GraphBLAS types in Table~\ref{Tab:PredefinedTypes}, or a user created type.
	\item[{\sf t3}] The type defining the third domain of the semiring being created. Should be one of the predefined
	GraphBLAS types in Table~\ref{Tab:PredefinedTypes}, or a user created type.
	\item[{\sf a}] The additive operation of the semiring.
	\item[{\sf m}] The multiplicative operation of the semiring.
	\item[{\sf z}] The $0$ element of the semiring. Must be of type corresponding to {\sf t3} as per Table~\ref{Tab:PredefinedTypes}.
	\item[{\sf o}] The $1$ element of the semiring. Must be of type corresponding to {\sf t3} as per Table~\ref{Tab:PredefinedTypes}.
\end{itemize}

\paragraph{Output Parameter}

\begin{itemize}
	\item[{\sf s}] Identifier of the newly created semiring.
\end{itemize}

\paragraph{Return Value}

\begin{tabular}{rl} 
{\sf GrB\_SUCCESS} 	& operation completed successfully \\
{\sf GrB\_PANIC}	& unknown internal error \\
{\sf GrB\_OUTOFMEM}	& not enough memory available for operation \\
\end{tabular}

\paragraph{Description}

Creates a new semiring $S = \langle \bold{D}({\sf t1}), \bold{D}({\sf t2}), \bold{D}({\sf t3}), {\sf a}, {\sf m}, {\sf z}, {\sf o} \rangle$ and
returns its identifier in {\sf s}.

