\subsection{Algebra Methods}
\label{Sec:AlgebraMethods}

%-----------------------------------------------------------------------------

\subsubsection{{\sf Type\_new}: Create a new GraphBLAS (user-defined) type}
\label{Sec:TypeNew}

Creates a new user-defined GraphBLAS type. This type can then be used to create new
operators, monoids, semirings, vectors and matrices.

\paragraph{\syntax}

\begin{verbatim}
        GrB_Info GrB_Type_new(GrB_Type  *utype,
                              size_t     sizeof(ctype));
\end{verbatim}

\paragraph{Parameters}

\begin{itemize}[leftmargin=1.1in]
    \item[{\sf utype}] ({\sf INOUT}) On successful return, contains a handle 
                                     to the newly created user-defined GraphBLAS 
                                     type object.
	\item[{\sf ctype}] ({\sf IN})    A C type that defines the new GraphBLAS 
                                     user-defined type.
\end{itemize}

\paragraph{Return Values}

\begin{itemize}[leftmargin=2.1in]
\item[{\sf GrB\_SUCCESS}]           operation completed successfully.
\item[{\sf GrB\_PANIC}]             unknown internal error.
\item[{\sf GrB\_OUT\_OF\_MEMORY}]          not enough memory available for operation.
\item[{\sf GrB\_NULL\_POINTER}]    {\sf utype} pointer is {\sf NULL}.
\end{itemize}

\paragraph{Description}

Given a C type {\sf ctype}, this method returns in {\sf utype} a handle to
a new GraphBLAS type equivalent to that C type.  Variables of this {\sf ctype} 
must be a struct, union, or fixed-size array. In particular, given two variables, 
{\tt src} and {\tt dst}, of type {\sf ctype}, the following operation must be a 
valid way to copy the contents of {\tt src} to {\tt dst}:

\begin{center}
{\tt memcpy(\&dst, \&src, sizeof({\sf ctype}))}
\end{center}

A new user-defined type {\sf utype} should be destroyed with a call to 
{\sf GrB\_free(utype)} when no longer needed.

It is not an error to call this method more than once on the same variable;  
however, the handle to the previously created object will be overwritten. 

%-----------------------------------------------------------------------------
\subsubsection{{\sf UnaryOp\_new}: Create a new GraphBLAS unary operator}

Initializes a new GraphBLAS unary operator with a specified user-defined 
function and its types (domains).

\paragraph{\syntax}

\begin{verbatim}
        GrB_Info GrB_UnaryOp_new(GrB_UnaryOp *unary_op,
                                 void       (*unary_func)(void*, const void*),
                                 GrB_Type     d_out,
                                 GrB_Type     d_in);
\end{verbatim}

\paragraph{Parameters}

\begin{itemize}[leftmargin=1.1in]
    \item[{\sf unary\_op}] ({\sf INOUT}) On successful return, contains a
                           handle to the newly created GraphBLAS unary operator object.
    \item[{\sf unary\_func}] ({\sf IN})  a pointer to a user-defined function that takes 
                           one input parameter of {\sf d\_in}'s type
			   and returns a value of {\sf d\_out}'s type, both passed as {\sf void} pointers.
                           Specifically the signature of the function is expected to 
                           be of the form:
          \begin{verbatim}
          void func(void *out, const void *in);
          \end{verbatim}
    \item[{\sf d\_out}] ({\sf IN})  The {\sf GrB\_Type} of the return value of the unary 
                           operator being created.  Should be one of the predefined 
                           GraphBLAS types in Table~\ref{Tab:PredefinedTypes}, or a 
                           user-defined GraphBLAS type.
    \item[{\sf d\_in}] ({\sf IN})  The {\sf GrB\_Type} of the input 
                           argument of the unary operator being created.  Should be 
                           one of the predefined GraphBLAS types in 
                           Table~\ref{Tab:PredefinedTypes}, or a user-defined GraphBLAS type.
\end{itemize}

\paragraph{Return Values}

\begin{itemize}[leftmargin=2.1in]
\item[{\sf GrB\_SUCCESS}]           operation completed successfully.
\item[{\sf GrB\_PANIC}]             unknown internal error.
\item[{\sf GrB\_OUT\_OF\_MEMORY}]          not enough memory available for operation.
\item[{\sf GrB\_UNINITIALIZED\_OBJECT}]          any {\sf GrB\_Type} parameter (for
                                    user-defined types) has not been
                                    initialized by a call to {\sf GrB\_Type\_new}.
\item[{\sf GrB\_NULL\_POINTER}]    {\sf unary\_op} or {\sf unary\_func}
                                    pointers are {\sf NULL}.

\jose{Domain mismatch not possible.}
%\item[{\sf GrB\_DOMAIN\_MISMATCH}]  the types in the function pointer signature
%                                    are not compatible with the {\sf GrB\_Type}
%                                    parameters specified when user-defined types
%                                    are specified.
\end{itemize}

\paragraph{Description}

\newenvironment{code}{\tt}{}

Creates a new GraphBLAS unary operator $f_u = \langle \mathbf{D}({\sf d\_out}), 
\mathbf{D}({\sf d\_in}), {\sf unary\_func} \rangle$ and returns a handle to it 
in {\sf unary\_op}.

The implementation of {\sf unary\_func} must be such that it works
even if the {\sf d\_out} and {\sf d\_in} arguments are aliased.
In other words, for all invocations of the function:
\begin{quote}
\begin{verbatim}
unary_func(out,in);
\end{verbatim}
\end{quote}
the value of {\sf out} must be the same as if the following code
was executed:

\begin{quote}
\begin{code}
    $\mathbf{D}({\sf d\_in})$ tmp = malloc(sizeof($\mathbf{D}({\sf d\_in}$))); \\
    memcpy(tmp,in,sizeof($\mathbf{D}({\sf d\_in}$))); \\
    unary\_func(out,tmp); \\
    free(tmp);
\end{code}
\end{quote}

It is not an error to call this method more than once on the same variable;  
however, the handle to the previously created object will be overwritten. 

%-----------------------------------------------------------------------------

\subsubsection{{\sf BinaryOp\_new}: Create a new GraphBLAS binary operator}

Initializes a new GraphBLAS binary operator with a specified user-defined 
function and its types (domains).

\paragraph{\syntax}

\begin{verbatim}
        GrB_Info GrB_BinaryOp_new(GrB_BinaryOp *binary_op,
                                  void        (*binary_func)(void*,
                                                             const void*,
                                                             const void*),
                                  GrB_Type      d_out,
                                  GrB_Type      d_in1,
                                  GrB_Type      d_in2);
\end{verbatim}

\paragraph{Parameters}

\begin{itemize}[leftmargin=1.1in]
    \item[{\sf binary\_op}] ({\sf INOUT}) On successful return, contains a 
          handle to the newly created GraphBLAS binary operator object.
    \item[{\sf binary\_func}] ({\sf IN}) A pointer to a user-defined function that 
          takes two input parameters of types {\sf d\_in1} and {\sf d\_in2} and returns a value of
		type {\sf d\_out}, all passed as {\sf void} pointers.
          Specifically the signature of the function is expected to 
          be of the form:
      \begin{verbatim}
      void func(void *out, const void *in1, const void *in2);
      \end{verbatim}
    \item[{\sf d\_out}]  ({\sf IN}) The {\sf GrB\_Type} of the return
          value of the binary operator being created. Should be one of the
          predefined GraphBLAS types in Table~\ref{Tab:PredefinedTypes}, or a 
          user-defined GraphBLAS type.
    \item[{\sf d\_in1}]  ({\sf IN}) The {\sf GrB\_Type} of the left hand 
          argument of the binary operator being created. Should be one of the
          predefined GraphBLAS types in Table~\ref{Tab:PredefinedTypes}, or a
          user-defined GraphBLAS type.
    \item[{\sf d\_in2}]  ({\sf IN}) The {\sf GrB\_Type} of the right hand 
          argument of the binary operator being created. Should be one of the
          predefined GraphBLAS types in Table~\ref{Tab:PredefinedTypes}, or a 
          user-defined GraphBLAS type.
\end{itemize}

\paragraph{Return Values}

\begin{itemize}[leftmargin=2.1in]
\item[{\sf GrB\_SUCCESS}]           operation completed successfully.
\item[{\sf GrB\_PANIC}]             unknown internal error.
\item[{\sf GrB\_OUT\_OF\_MEMORY}]          not enough memory available for operation.
\item[{\sf GrB\_UNINITIALIZED\_OBJECT}]          the {\sf GrB\_Type} (for user-defined types)
                                    has not been initialized by a call to {\sf GrB\_Type\_new}.
\item[{\sf GrB\_NULL\_POINTER}]    {\sf binary\_op} or {\sf binary\_func} pointer is {\sf NULL}.

\jose{Domain mistmatch not possible.}
%\item[{\sf GrB\_DOMAIN\_MISMATCH}]  the types in the function pointer signature are not   
%                                    compatible with the {\sf GrB\_Type} parameters specified.
\end{itemize}

\paragraph{Description}

Creates a new GraphBLAS binary operator $f_b = \langle \mathbf{D}({\sf d\_out}), 
\mathbf{D}({\sf d\_in1}), \mathbf{D}({\sf d\_in2}), {\sf binary\_func} \rangle$ and
returns a handle to it in {\sf binary\_op}.

The implementation of {\sf binary\_func} must be such that it works
even if any of the {\sf d\_out}, {\sf d\_in1}, and {\sf d\_in2} arguments are aliased to each other.
In other words, for all invocations of the function:
\begin{quote}
\begin{verbatim}
binary_func(out,in1,in2);
\end{verbatim}
\end{quote}
the value of {\sf out} must be the same as if the following code
was executed:

\begin{quote}
\begin{code}
    $\mathbf{D}({\sf d\_in1})$ tmp1 = malloc(sizeof($\mathbf{D}({\sf d\_in1}$))); \\
    $\mathbf{D}({\sf d\_in2})$ tmp2 = malloc(sizeof($\mathbf{D}({\sf d\_in2}$))); \\
    memcpy(tmp1,in1,sizeof($\mathbf{D}({\sf d\_in1}$))); \\
    memcpy(tmp2,in2,sizeof($\mathbf{D}({\sf d\_in2}$))); \\
    binary\_func(out,tmp1,tmp2); \\
    free(tmp2); \\
    free(tmp1);
\end{code}
\end{quote}

It is not an error to call this method more than once on the same variable;  
however, the handle to the previously created object will be overwritten. 

%-----------------------------------------------------------------------------

\subsubsection{{\sf Monoid\_new}: Create new GraphBLAS monoid}

Creates a new monoid with specified binary operator and identity value.

\paragraph{\syntax}

\begin{verbatim}
        GrB_Info GrB_Monoid_new(GrB_Monoid    *monoid,
                                GrB_BinaryOp   binary_op,
                                <type>         identity);
\end{verbatim}

\paragraph{Parameters}

\begin{itemize}[leftmargin=1.1in]
    \item[{\sf monoid}] ({\sf INOUT}) On successful return, contains a
                         handle to the newly created GraphBLAS monoid object.
    \item[{\sf binary\_op}] ({\sf IN}) An existing GraphBLAS associative binary 
                         operator whose input and output types are the same.
    \item[{\sf identity}]  ({\sf IN}) The value of the identity element of the 
                         monoid. Must be the same type as the type used by the
                         {\sf binary\_op} operator.
\end{itemize}

\paragraph{Return Values}

\begin{itemize}[leftmargin=2.1in]
\item[{\sf GrB\_SUCCESS}]           operation completed successfully.
\item[{\sf GrB\_PANIC}]             unknown internal error.
\item[{\sf GrB\_OUT\_OF\_MEMORY}]   not enough memory available for operation.
\item[{\sf GrB\_UNINITIALIZED\_OBJECT}]  the {\sf GrB\_BinaryOp} has not been
                                    initialized by a call to {\sf GrB\_BinaryOp\_new}.
\item[{\sf GrB\_NULL\_POINTER}]     {\sf monoid} pointer is {\sf NULL}.
\item[{\sf GrB\_DOMAIN\_MISMATCH}]  all three argument types of the binary operator and
                                    the type of the identity value are not the same.
\end{itemize}

\paragraph{Description}

Creates a new monoid $M = \langle \mathbf{D}({\sf binary\_op}), {\sf binary\_op}, 
{\sf identity} \rangle$ and returns a handle to it in {\sf monoid}.

If {\sf binary\_op} is not associative, the results of GraphBLAS operations that
require associativity of this monoid will be undefined.

It is not an error to call this method more than once on the same variable;  
however, the handle to the previously created object will be overwritten. 

%-----------------------------------------------------------------------------
\subsubsection{{\sf Semiring\_new}: Create new GraphBLAS semiring}

Creates a new semiring with specified domain, operators, and elements.

\paragraph{\syntax}

\begin{verbatim}
        GrB_Info GrB_Semiring_new(GrB_Semiring  *semiring,
                                  GrB_Monoid     add_op,
                                  GrB_BinaryOp   mul_op);
\end{verbatim}

\paragraph{Parameters}

\begin{itemize}[leftmargin=1.1in]
    \item[{\sf semiring}] ({\sf INOUT}) On successful return, contains a 
    handle to the newly created GraphBLAS semiring.
    \item[{\sf add\_op}]  ({\sf IN}) An existing GraphBLAS commutative monoid that 
    specifies the addition operator and its identity.
    \item[{\sf mul\_op}]  ({\sf IN}) An existing GraphBLAS binary operator that 
    specifies the semiring's multiplication operator. In addition, {\sf mul\_op}'s
    output domain, $\bDout({\sf mul\_op})$, must be the same as the {\sf add\_op}'s
    domain $\mathbf{D}(\mbox{\sf add\_op})$.
\end{itemize}


\paragraph{Return Values}

\begin{itemize}[leftmargin=2.1in]
\item[{\sf GrB\_SUCCESS}]           operation completed successfully.
\item[{\sf GrB\_PANIC}]             unknown internal error.
\item[{\sf GrB\_OUT\_OF\_MEMORY}]   not enough memory available for this method to complete.
\item[{\sf GrB\_UNINITIALIZED\_OBJECT}]   the {\sf add\_op} object has not been
                                    initialized with a call to {\sf GrB\_Monoid\_new}
                                    or the {\sf mul\_op} object has not been
                                    not been initialized by a call to 
                                    {\sf GrB\_BinaryOp\_new}.
\item[{\sf GrB\_NULL\_POINTER}]    {\sf semiring} pointer is {\sf NULL}.
\item[{\sf GrB\_DOMAIN\_MISMATCH}]  the output domain of {\sf mul\_op} does not
                                    match the domain of the {\sf add\_op} monoid.
\end{itemize}

\paragraph{Description}

Creates a new semiring $S = \langle \bDout({\sf mul\_op}), 
\bDin1({\sf mul\_op}), \bDin2({\sf mul\_op}), {\sf add\_op}, 
{\sf mul\_op}, \mathbf{0}({\sf add\_op})\rangle$ and returns a handle to it in 
{\sf semiring}.  Note that $\bDout({\sf mul\_op})$ must be the same as 
$\mathbf{D}({\sf add\_op})$.

If {\sf add\_op} is not commutative, then GraphBLAS operations using this semiring
will be undefined.

It is not an error to call this method more than once on the same variable;  
however, the handle to the previously created object will be overwritten. 
