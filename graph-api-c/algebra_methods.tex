\subsection{Algebra Methods}
\label{Sec:AlgebraMethods}


%-----------------------------------------------------------------------------

\subsubsection{{\sf Type\_new}: Create a new GraphBLAS (user-defined) type}

Creates a new user-defined GraphBLAS type. This type can then be used to create new
operators, monoids, semirings, vectors and matrices.

\paragraph{\syntax}

\begin{verbatim}
        GrB_info GrB_Type_new(GrB_Type	 *utype,
                              <in_type>	  ctype);
\end{verbatim}

\paragraph{Parameters}

\begin{itemize}[leftmargin=1.1in]
    \item[{\sf utype}] ({\sf INOUT}) On successful return, contains the identifier 
                                     of the newly created user-defined GraphBLAS type.
	\item[{\sf ctype}] ({\sf IN})    A C type that defines the new GraphBLAS 
                                     user-defined type ({\sf in\_type}).
\end{itemize}

\paragraph{Return Values}

\begin{itemize}[leftmargin=2.1in]
\item[{\sf GrB\_SUCCESS}]           operation completed successfully.
\item[{\sf GrB\_PANIC}]             unknown internal error.
\item[{\sf GrB\_OUTOFMEM}]          not enough memory available for operation.
\item[{\sf GrB\_INVALID\_VALUE}]    {\sf utype} pointer is {\sf NULL}.
\item[{\sf GrB\_INVALID\_VALUE}]    {\sf utype} object is already initialized.
\end{itemize}

\paragraph{Description}
Given a C type, {\sf ctype}, this method returns in {\sf utype} a new GraphBLAS type equivalent to that C type.
Variables of this {\sf ctype} must be a struct, union, or fixed-size array. In particular, given two variables,
{\tt src} and {\tt dst}, of type {\sf ctype}, the following operation must be a valid way to copy the contents of
{\tt src} to {\tt dst}:

\begin{center}
{\tt memcpy(\&dst, \&src, sizof({\sf ctype}))}
\end{center}

A new user-defined type {\sf utype} should be destroyed with a call to 
{\sf GrB\_free(utype)} when no longer needed.

%-----------------------------------------------------------------------------
\subsubsection{{\sf UnaryOp\_new}: Create a new GraphBLAS unary operator}

Initializes a new GraphBLAS unary operator with a specified user-defined 
function and its types (domains).

\paragraph{\syntax}

\begin{verbatim}
        GrB_info GrB_UnaryOp_new(GrB_UnaryOp *unary_op,
                                 GrB_Type     d1,
                                 GrB_Type     d2,
                                 void        *unary_func);
\end{verbatim}

\paragraph{Parameters}

\begin{itemize}[leftmargin=1.1in]
    \item[{\sf unary\_op}] ({\sf INOUT}) On successful return, contains the 
                           identifier of the newly created GraphBLAS UnaryOp.
    \item[{\sf d1}] ({\sf IN})  The {\sf GrB\_Type} of the input 
                           argument of the unary operator being created.  Should be 
                           one of the predefined GraphBLAS types in 
                           Table~\ref{Tab:PredefinedTypes}, or a user-defined GraphBLAS type.
    \item[{\sf d2}] ({\sf IN})  The {\sf GrB\_Type} of the return value of the unary 
                           operator being created.  Should be one of the predefined 
                           GraphBLAS types in Table~\ref{Tab:PredefinedTypes}, or a 
                           user-defined GraphBLAS type.
    \item[{\sf unary\_func}] ({\sf IN})  a pointer to a user-defined function that takes 
                           one input parameter of a type consistent with {\sf d1}'s type
                           and returns a value of type consistent with {\sf d2}'s type.  \scott{DEFINE "CONSISTENT"}
\end{itemize}


\paragraph{Return Values}

\begin{itemize}[leftmargin=2.1in]
\item[{\sf GrB\_SUCCESS}]           operation completed successfully.
\item[{\sf GrB\_PANIC}]             unknown internal error.
\item[{\sf GrB\_OUTOFMEM}]          not enough memory available for operation.
\item[{\sf GrB\_NOOBJECT}]          any {\sf GrB\_Type} parameter (for
                                    user-defined types) has not been
                                    initialized by a call to {\sf new}.
\item[{\sf GrB\_INVALID\_VALUE}]    {\sf unary\_op} or {\sf unary\_func}
                                    pointers are {\sf NULL}.
\item[{\sf GrB\_INVALID\_VALUE}]    {\sf unary\_op} object is already initialized.
\item[{\sf GrB\_DOMAIN\_MISMATCH}]  the types in the function pointer signature
                                    are not consistent with the {\sf GrB\_Type}
                                    parameters specified when user-defined types
                                    are specified. \scott{HOW IS THIS ENFORCED?}
\end{itemize}

\paragraph{Description}

Creates a new GraphBLAS unary operator $f_u = \langle \bold{D}({\sf d1}), 
\bold{D}({\sf d2}), {\sf unary\_func} \rangle$ and returns its identifier 
in {\sf unary\_op}.


%-----------------------------------------------------------------------------

\subsubsection{{\sf BinaryOp\_new}: Create a new GraphBLAS binary operator}

Initializes a new GraphBLAS binary operator with a specified user-defined 
function and its types (domains).

\paragraph{\syntax}

\begin{verbatim}
        GrB_info GrB_BinaryOp_new(GrB_BinaryOp *binary_op,
                                  GrB_Type      d1,
                                  GrB_Type      d2,
                                  GrB_Type      d3,
                                  void         *binary_func);
\end{verbatim}

\paragraph{Parameters}

\begin{itemize}[leftmargin=1.1in]
    \item[{\sf binary\_op}] ({\sf INOUT}) On successful return, contains the 
          identifier of the newly created GraphBLAS BinaryOp.
    \item[{\sf d1}]  ({\sf IN}) The {\sf GrB\_Type} of the left hand 
          argument of the binary operator being created. Should be one of the
          predefined GraphBLAS types in Table~\ref{Tab:PredefinedTypes}, or a
          user-defined GraphBLAS type.
    \item[{\sf d2}]  ({\sf IN}) The {\sf GrB\_Type} of the right hand 
          argument of the binary operator being created. Should be one of the
          predefined GraphBLAS types in Table~\ref{Tab:PredefinedTypes}, or a 
          user-defined GraphBLAS type.
    \item[{\sf d3}]  ({\sf IN}) The {\sf GrB\_Type} of the return
          value of the binary operator being created. Should be one of the
          predefined GraphBLAS types in Table~\ref{Tab:PredefinedTypes}, or a 
          user-defined GraphBLAS type.
    \item[{\sf binary\_func}] ({\sf IN}) A pointer to a user-defined function that 
          takes two input parameters of types consistent with {\sf d1} and 
          {\sf d2} and returns a value with the type consistent with {\sf d3}. \scott{DEFINE CONSISTENT}
\end{itemize}


\paragraph{Return Values}

\begin{itemize}[leftmargin=2.1in]
\item[{\sf GrB\_SUCCESS}]           operation completed successfully.
\item[{\sf GrB\_PANIC}]             unknown internal error.
\item[{\sf GrB\_OUTOFMEM}]          not enough memory available for operation.
\item[{\sf GrB\_NOOBJECT}]          the {\sf GrB\_Type} (for user-defined types)
                                    has not been initialized by a call to {\sf new}
\item[{\sf GrB\_INVALID\_VALUE}]    {\sf binary\_op} or {\sf binary\_func} pointer is {\sf NULL}.
\item[{\sf GrB\_INVALID\_VALUE}]    {\sf binary\_op} object is already initialized.
\item[{\sf GrB\_DOMAIN\_MISMATCH}]  the types in the function pointer signature are not   
                                    consistent with the {\sf GrB\_Type} parameters specified.
\end{itemize}

\paragraph{Description}

Creates a new GraphBLAS binary operator $f_b = \langle \bold{D}({\sf d1}), 
\bold{D}({\sf d2}), \bold{D}({\sf d3}), {\sf binary\_func} \rangle$ and returns its identifier in {\sf binary\_op}.


%-----------------------------------------------------------------------------

\subsubsection{{\sf Monoid\_new}: Create new GraphBLAS monoid}

Creates a new monoid with specified type, binary operator, and identity value.

\paragraph{\syntax}

\begin{verbatim}
        GrB_info GrB_Monoid_new(GrB_Monoid    *monoid,
                                GrB_Type       d1,
                                GrB_BinaryOp   binary_op,
                                <d1_type>      identity);
\end{verbatim}

\paragraph{Parameters}

\begin{itemize}[leftmargin=1.1in]
    \item[{\sf monoid}] ({\sf INOUT}) On successful return, contains the identifier 
    of the newly created GraphBLAS monoid.
    \item[{\sf d1}] ({\sf IN}) The {\sf GrB\_Type} the inputs and output of the 
    monoid being created. It should be one of the predefined GraphBLAS types in
    Table~\ref{Tab:PredefinedTypes}, or a user-defined GraphBLAS type.
    \item[{\sf binary\_op}] ({\sf IN}) An existing GraphBLAS binary operator with input types 
    consistent with {\sf d1}, and returns a value of type {\sf d1}. \scott{
    CONSISTENT or EQUAL?}
    \item[{\sf identity}]  ({\sf IN}) The value of the identity element of the 
    monoid. Must be the same type as the type corresponding to {\sf d1} according to
    Table~\ref{Tab:PredefinedTypes} or user-defined GraphBLAS types.
\end{itemize}

\paragraph{Return Values}

\begin{itemize}[leftmargin=2.1in]
\item[{\sf GrB\_SUCCESS}]           operation completed successfully.
\item[{\sf GrB\_PANIC}]             unknown internal error.
\item[{\sf GrB\_OUTOFMEM}]          not enough memory available for operation.
\item[{\sf GrB\_NOOBJECT}]          the {\sf GrB\_Type} (for user-defined types)
                                    or {\sf GrB\_BinaryOp} has not been
                                    initialized by a call to {\sf new}
\item[{\sf GrB\_INVALID\_VALUE}]    {\sf monoid} pointer is {\sf NULL}.
\item[{\sf GrB\_INVALID\_VALUE}]    {\sf monoid} object is already initialized.
\item[{\sf GrB\_DOMAIN\_MISMATCH}]  the types in the operator signature are not   
                                    consistent with the {\sf GrB\_Type} parameter specified.
\end{itemize}

\paragraph{Description}

Creates a new monoid $M = \langle \bold{D}({\sf d1}), 
{\sf binary\_op}, {\sf identity} \rangle$ and
returns its identifier in {\sf monoid}.


%-----------------------------------------------------------------------------
\subsubsection{{\sf Semiring\_new}: Create new GraphBLAS semiring}

Creates a new semiring with specified domain, operators, and elements.

\paragraph{\syntax}

\begin{verbatim}
        GrB_info GrB_Semiring_new(GrB_Semiring  *semiring,
                                  GrB_Monoid     add_op,
                                  GrB_BinaryOp   mul_op);
\end{verbatim}

\paragraph{Parameters}

\begin{itemize}[leftmargin=1.1in]
    \item[{\sf semiring}] ({\sf INOUT}) On successful return, contains the 
    identifier of the newly created GraphBLAS semiring.
    \item[{\sf add\_op}]  ({\sf IN}) An existing GraphBLAS monoid that specifies the addition operator and its identity.
    \item[{\sf mul\_op}]  ({\sf IN}) An existing GraphBLAS binary operator that specifies the semiring's multiplication operator.
\end{itemize}


\paragraph{Return Values}

\begin{itemize}[leftmargin=2.1in]
\item[{\sf GrB\_SUCCESS}]           operation completed successfully.
\item[{\sf GrB\_PANIC}]             unknown internal error.
\item[{\sf GrB\_OUTOFMEM}]          not enough memory available for this method to complete.
\item[{\sf GrB\_NOOBJECT}]          the {\sf add\_op} or {\sf mul\_op} object has
                                    not been initialized by a call to {\sf new}
\item[{\sf GrB\_INVALID\_VALUE}]    {\sf semiring} pointer is {\sf NULL}.
\item[{\sf GrB\_INVALID\_VALUE}]    {\sf semiring} object is already initialized.
\item[{\sf GrB\_DOMAIN\_MISMATCH}]  the output domain of {\sf mul\_op} does not
                                    match the domain of the {\sf add\_op} in the
                                    case where these domains ({\sf GrB\_Type}s)
                                    are user-defined types.
\end{itemize}

\paragraph{Description}

Creates a new semiring $S = \langle \bold{D_1}({\sf mul\_op}), \bold{D_2}({\sf mul\_op}), 
\bold{D}({\sf add\_op}), {\sf add\_op}, {\sf mul\_op}, \bold{0}({\sf add\_op})\rangle$ and 
returns its identifier in {\sf semiring}.  Note that $\bold{D_3}({\sf mul\_op})$ must be
the same as $\bold{D}({\sf add\_op})$.

