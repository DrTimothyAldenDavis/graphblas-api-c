%-----------------------------------------------------------------------------
\subsection{{\sf mxm}: Matrix-matrix multiply}

Multiplies a matrix with another matrix on a semiring. The result is a matrix.

\paragraph{\syntax}

\begin{verbatim}
        GrB_Info GrB_mxm(GrB_Matrix              *C,
                         const GrB_Matrix         Mask,
                         const GrB_BinaryOp       accum,
                         const GrB_Semiring       op,
                         const GrB_Matrix         A, 
                         const GrB_Matrix         B,
                         const GrB_Descriptor     desc);
\end{verbatim}

\paragraph{Parameters}

\begin{itemize}[leftmargin=1.1in]
    \item[{\sf C}]    ({\sf INOUT}) An existing GraphBLAS matrix. On
    input, the matrix provides values that may be accumulated with the
    result of the matrix product.   On output, the matrix holds the
    results of this operation.

    \item[{\sf Mask}] ({\sf IN}) An optional ``write'' mask that controls which
    results from this operation are stored into the output matrix
    ${\sf C}$.  If no mask is desired (\ie, all elements
    of result are copied into the output matrix), {\sf GrB\_NULL}
    should be specified. The mask dimensions must match those of the
    matrix {\sf C} and the domain of the {\sf Mask} matrix must be
    of type {\sf bool} or any of the predefined ``built-in'' types in
    Table~\ref{Tab:PredefinedTypes}.

    \item[{\sf accum}] ({\sf IN}) An optional operator used for accumulating
    entries into existing {\sf C} entries: ${\sf accum} = \langle D_x,
    D_y, D_z,\odot \rangle$. If assignment rather than accumulation is
    desired, {\sf GrB\_NULL} should be specified.

    \item[{\sf op}]   ({\sf IN}) Semiring used in the matrix-matrix
    multiply: ${\sf op}=\langle D_1,D_2,D_3,\oplus,\otimes,0 \rangle$.

    \item[{\sf A}]    ({\sf IN}) The GraphBLAS matrix holding the values
    for the left-hand matrix in the multiplication.

    \item[{\sf B}]    ({\sf IN}) The GraphBLAS matrix holding the values
    for the right-hand matrix in the multiplication.

    \item[{\sf desc}] ({\sf IN}) An optional operation descriptor. If
    a \emph{default} descriptor is desired, {\sf GrB\_NULL} should be
    used. Valid fields are as follows:  \\

    \begin{tabular}{lllp{2.5in}}
    Param   & Field           & Value               & Description \\ \hline
    {\sf C}    & {\sf GrB\_OUTP} & {\sf GrB\_REPLACE}  & Output matrix {\sf C} is cleared (all elements removed) before the result is stored in it. \\
    {\sf Mask} & {\sf GrB\_MASK} & {\sf GrB\_SCMP}     & Use the structural complement of {\sf Mask}. \\
    {\sf A}    & {\sf GrB\_INP0} & {\sf GrB\_TRAN}     & Use transpose of {\sf A} for operation. \\
    {\sf B}    & {\sf GrB\_INP1} & {\sf GrB\_TRAN}     & Use transpose of {\sf B} for operation. \\
    \end{tabular}
\end{itemize}

\paragraph{Return Values}

\begin{itemize}[leftmargin=2.1in]
	\item[{\sf GrB\_SUCCESS}]	      In blocking mode, operation
	completed successfully. In non-blocking mode, this indicates
	that the consistency tests on dimensions and domains for the
	input arguments passed successfully. Either way, output matrix
	{\sf C} is ready to be used in the next method of the sequence.

	\item[{\sf GrB\_PANIC}]		      Unknown internal error

    \item[{\sf GrB\_INVALID\_OBJECT}] This is returned in any execution mode 
    whenever one of the opaque GraphBLAS objects (input or output) is in an invalid 
    state caused by a previous execution error.  Call {GrB\_error()} to access 
    any error messages generated by the implementation.

	\item[{\sf GrB\_OUT\_OF\_MEMORY}]	      Not enough memory available
	for operation
    
    \item[{\sf GrB\_UNINITIALIZED\_OBJECT}]  One or more of the GraphBLAS 
    objects has not been initialized by a call to {\sf new} (or 
    {\sf Matrix\_dup} for the matrix parameters).
    
    \item[{\sf GrB\_NULL\_POINTER}]  The output matrix pointer is {\sf NULL}.

	\item[{\sf GrB\_DIMENSION\_MISMATCH}] Mask and/or matrix dimensions are
	incompatible.

	\item[{\sf GrB\_DOMAIN\_MISMATCH}]    The domains of the various
	matrices are incompatible with the corresponding domains of the
	accumulating operation, semiring, or mask.
\end{itemize}

\paragraph{Description}

{\sf GrB\_mxm} computes the matrix product ${\sf C} = {\sf
A} \otimes . \oplus {\sf B}$ or, if an optional binary accumulation
operator ($\odot$) is provided, ${\sf C} = {\sf C} \odot
\left({\sf A} \otimes . \oplus {\sf B}\right)$ (where matrices {\sf A}
and {\sf B} can be optionally transposed).  Logically, this operation
occurs in three steps:
\begin{enumerate}
\item[\bf Setup] The internal matrices and mask used in the computation are formed and their 
domains and dimensions are tested for consistency.
\item[\bf Compute] The indicated computations are carried out.
\item[\bf Output] The result is written into the output matrix, possibly under control of a mask.
\end{enumerate}

Up to four argument matrices are used in the {\sf GrB\_mxm} operation:
\begin{enumerate}
	\item ${\sf C} = \langle \bold{D}({\sf C}),\bold{nrows}({\sf C}),\bold{ncols}({\sf C}),\bold{L}({\sf C}) = \{(i,j,C_{ij}) \} \rangle$
	\item ${\sf Mask} = \langle \bold{D}({\sf Mask}),\bold{nrows}({\sf Mask}),\bold{ncols}({\sf Mask}),\bold{L}({\sf Mask}) = \{(i,j,M_{ij}) \} \rangle$ (optional)
	\item ${\sf A} = \langle \bold{D}({\sf A}),\bold{nrows}({\sf A}), \bold{ncols}({\sf A}),\bold{L}({\sf A}) = \{(i,j,A_{ij}) \} \rangle$
	\item ${\sf B} = \langle \bold{D}({\sf B}),\bold{nrows}({\sf B}), \bold{ncols}({\sf B}),\bold{L}({\sf B}) = \{(i,j,B_{ij}) \} \rangle$
\end{enumerate}

The argument matrices, the semiring, and the accumulation operator (if provided) are tested for domain consistency
as follows:
\begin{enumerate}
	\item The domain of {\sf Mask} (if not {\sf GrB\_NULL}) must be from one of the pre-defined types of Table~\ref{Tab:PredefinedTypes}.

	\item $\bold{D}({\sf A})$ must be compatible with $D_1$ of the semiring.

	\item $\bold{D}({\sf B})$ must be compatible with $D_2$ of the semiring.

	\item If {\sf accum} is {\sf GrB\_NULL}, then $\bold{D}({\sf C})$ must be compatible with $D_3$ of the semiring.

	\item If {\sf accum} is not {\sf GrB\_NULL}, then $\bold{D}({\sf C})$ must be compatible with $D_x$ and $D_z$ of the 
	accumulation operator and $D_3$ of the semiring must be compatible with $D_y$ of the accumulation operator.
\end{enumerate}
Two domains are compatible with each other if values from one domain can be cast to values in the other domain as per the rules of the C language.
In particular, domains from Table~\ref{Tab:PredefinedTypes} are all compatible with each other. A domain from a user-defined type is only compatible with itself.
If any consistency rule above is violated, execution of {\sf GrB\_mxm} ends and the domain mismatch error listed above is returned.

From the argument matrices, the internal matrices and mask used in the computation are formed ($\leftarrow$ denotes copy):
\begin{enumerate}

	\item Matrix $\matrix{\widetilde{C}} \leftarrow {\sf C}$.

	\item Two-dimensional mask $\matrix{\widetilde{M}}$ is computed from argument 
    {\sf Mask} as follows:
	\begin{enumerate}

		\item	If ${\sf Mask} = {\sf GrB\_NULL}$, then $\matrix{\widetilde{M}} = 
        \langle \bold{nrows}({\sf C}), \bold{ncols}({\sf C}), \{(i,j), 
        \forall i,j : 0 \leq i <  \bold{nrows}({\sf C}), 0 \leq j < \bold{ncols}({\sf C}) 
        \} \rangle$.

		\item	Otherwise, $\matrix{\widetilde{M}} = \langle \bold{nrows}({\sf Mask}), 
        \bold{ncols}({\sf Mask}), \{(i,j) : (i,j) \in \bold{ind}({\sf Mask}) \wedge 
        ({\sf bool}){\sf Mask}(i,j) = \true\} \rangle$.

		\item	If ${\sf desc[GrB\_MASK].GrB\_SCMP}$ is set, then 
        $\matrix{\widetilde{M}} \leftarrow \neg \matrix{\widetilde{M}}$.

	\end{enumerate}

	\item Matrix $\matrix{\widetilde{A}} \leftarrow {\sf desc[GrB\_INP0].GrB\_TRAN} \ ? \ {\sf A}^T : {\sf A}$.

	\item Matrix $\matrix{\widetilde{B}} \leftarrow {\sf desc[GrB\_INP1].GrB\_TRAN} \ ? \ {\sf B}^T : {\sf B}$.
\end{enumerate}

The internal matrices and masks are checked for shape consistency. The following 
conditions must hold:
\begin{enumerate}
	\item $\bold{nrows}(\matrix{\widetilde{C}}) = \bold{nrows}(\matrix{\widetilde{M}})$.

	\item $\bold{ncols}(\matrix{\widetilde{C}}) = \bold{ncols}(\matrix{\widetilde{M}})$.

	\item $\bold{nrows}(\matrix{\widetilde{C}}) = \bold{nrows}(\matrix{\widetilde{A}})$.

	\item $\bold{ncols}(\matrix{\widetilde{C}}) = \bold{ncols}(\matrix{\widetilde{B}})$.

	\item $\bold{ncols}(\matrix{\widetilde{A}}) = \bold{nrows}(\matrix{\widetilde{B}})$.
\end{enumerate}
If any consistency rule above is violated, execution of {\sf GrB\_mxm} ends and 
the dimension mismatch error listed above is returned.

From this point forward, in {\sf GrB\_NONBLOCKING} mode, the method can 
optionally exit with {\sf GrB\_SUCCESS} return code and defer any computation 
and/or execution error codes.

We are now ready to carry out the matrix multiplication and any additional 
associated operations.  We describe this in terms of two intermediate matrices:
\begin{itemize}
    \item $\matrix{\widetilde{T}}$: The matrix holding the product of matrices 
    $\matrix{\widetilde{A}}$ and $\matrix{\widetilde{B}}$.
    \item $\matrix{\widetilde{Z}}$: The matrix holding the result after 
    application of the (optional) accumulation operator.
\end{itemize}

The intermediate matrix $\matrix{\widetilde{T}} = \langle
D_3, \bold{nrows}(\matrix{\widetilde{A}}),
\bold{ncols}(\matrix{\widetilde{B}}), \bold{L}(\matrix{\widetilde{T}}) =
\{(i,j,T_{ij}) : \bold{ind}(\matrix{\widetilde{A}}(i,:)) \cap
\bold{ind}(\matrix{\widetilde{B}}(:,j)) \neq \emptyset \} \rangle$
is created.  The value of each of its elements is computed by \[T_{ij}
= \bigoplus_{k \in \bold{ind}(\matrix{\widetilde{A}}(i,:)) \cap
\bold{ind}(\matrix{\widetilde{B}}(:,j))} (\matrix{\widetilde{A}}(i,k)
\otimes \matrix{\widetilde{B}}(k,j)),\] where $\oplus$ and $\otimes$
are the additive and multiplicative operators of semiring {\sf op},
respectively.

The intermediate matrix $\matrix{\widetilde{Z}}$ is created as follows:
\begin{itemize}
    \item If ${\sf accum} = {\sf GrB\_NULL}$, then $\matrix{\widetilde{Z}} = \matrix{\widetilde{T}}$.

    \item If ${\sf accum} = \langle D_x, D_y, D_z, \odot \rangle$, then matrix $\matrix{\widetilde{Z}}$ is defined as 
        \[ \langle D_z, \bold{nrows}(\matrix{\widetilde{C}}), \bold{ncols}(\matrix{\widetilde{C}}),
        \bold{L}(\matrix{\widetilde{Z}}) 
		= \{(i,j,Z_{ij})  \forall (i,j) \in \bold{ind}(\matrix{\widetilde{C}}) \cup 
        \bold{ind}(\matrix{\widetilde{T}}) \} \rangle. \]

        The values of the elements of $\matrix{\widetilde{Z}}$ are computed based on the 
        relationships between the sets of indices in $\matrix{\widetilde{C}}$ and 
        $\matrix{\widetilde{T}}$.
\[
Z_{ij} = \matrix{\widetilde{C}}(i,j) \odot \matrix{\widetilde{T}}(i,j), \ \mbox{if}\  (i,j) \in  (\bold{ind}(\matrix{\widetilde{T}}) \cap \bold{ind}(\matrix{\widetilde{C}})),
\]
\[
Z_{ij} = \matrix{\widetilde{C}}(i,j) \ \mbox{if}\  (i,j) \in  (\bold{ind}(\matrix{\widetilde{C}}) - (\bold{ind}(\matrix{\widetilde{T}}) \cap \bold{ind}(\matrix{\widetilde{C}}))),
\]
\[
Z_{ij} = \matrix{\widetilde{T}}(i,j) \ \mbox{if}\  (i,j) \in  (\bold{ind}(\matrix{\widetilde{T}}) - (\bold{ind}(\matrix{\widetilde{T}}) \cap \bold{ind}(\matrix{\widetilde{C}}))).
\]
where the difference operator in the previous expressions refers to set difference.
\end{itemize}

Finally, the set of output values that make up the $\matrix{\widetilde{Z}}$ 
matrix are written into the final result matrix, {\sf C}. 
This is carried out under control of the mask which acts as a ``write mask''.
If {\sf desc[GrB\_OUTP].GrB\_REPLACE} is set, then any values in {\sf C} on 
input to {\sf GrB\_mxm()} are deleted and the contents of the new output matrix,
{\sf C}, is defined as,
\[ 
\bold{L}({\sf C}) = \{(i,j,Z_{ij}) : (i,j) \in (\bold{ind}(\matrix{\widetilde{Z}}) 
\cap \bold{ind}(\matrix{\widetilde{M}})) \}. 
\]
If {\sf desc[GrB\_OUTP].GrB\_REPLACE} is not set, the elements of 
$\matrix{\widetilde{Z}}$ indicated by the mask are copied into the result 
matrix, {\sf C}, and elements of {\sf C} that fall outside the set 
indicated by the mask are unchanged:
\[ 
\bold{L}({\sf C}) = \{(i,j,C_{ij}) : (i,j) \in (\bold{ind}(\matrix{\sf C}) \cap \bold{ind}(\neg \matrix{\widetilde{M}})) \} \cup \{(i,j,Z_{ij}) : (i,j) \in (\bold{ind}(\matrix{\widetilde{Z}}) \cap \bold{ind}(\matrix{\widetilde{M}})) \}. 
\]

In {\sf GrB\_BLOCKING} mode, the method exits with return value {\sf GrB\_SUCCESS} and the new content of matrix {\sf C} is as defined above and fully computed.
In {\sf GrB\_NONBLOCKING} mode, the method exits with return value {\sf GrB\_SUCCESS} and the new content of matrix {\sf C} is as defined above but may not be fully computed; however, it can be used in the next GraphBLAS 
method call in a sequence.

%-----------------------------------------------------------------------------

\subsection{{\sf vxm}: Vector-matrix multiply}

Multiplies a (row) vector with a matrix on an semiring. The result is a vector.

\paragraph{\syntax}

\begin{verbatim}
        GrB_Info GrB_vxm(GrB_Vector            *w,
                         const GrB_Vector       mask,
                         const GrB_BinaryOp     accum,
                         const GrB_Semiring     op,
                         const GrB_Vector       u, 
                         const GrB_Matrix       A,
                         const GrB_Descriptor   desc);
\end{verbatim}

\paragraph{Parameters}

\begin{itemize}[leftmargin=1.1in]
    \item[{\sf w}]    ({\sf INOUT}) An existing GraphBLAS vector.  On input,
    the vector provides values that may be accumulated with the result of the
    vector-matrix product.  On output, this vector holds the results of the
    operation.

    \item[{\sf mask}] ({\sf IN}) An optional ``write'' mask that controls which
    results from this operation are stored into the output vector
    ${\sf w}$.  If no mask is desired (\ie, all elements
    of result are copied into the output vector), {\sf GrB\_NULL}
    should be specified. The mask dimensions must match those of the
    vector {\sf w} and the domain of {\sf mask} must be
    of type {\sf bool} or any of the predefined ``built-in'' types in
    Table~\ref{Tab:PredefinedTypes}.

    \item[{\sf accum}] ({\sf IN}) An optional operator used for accumulating
    entries into existing {\sf w} entries: ${\sf accum} = \langle D_x,
    D_y, D_z,\odot \rangle$. If assignment rather than accumulation is
    desired, {\sf GrB\_NULL} should be specified.

    \item[{\sf op}]   ({\sf IN}) Semiring used in the vector-matrix
    multiply: ${\sf op}=\langle D_1,D_2,D_3,\oplus,\otimes,0 \rangle$.
    
    \item[{\sf u}]    ({\sf IN}) The GraphBLAS vector holding the values for
    the left-hand vector in the multiplication.
    
    \item[{\sf A}]    ({\sf IN}) The GraphBLAS matrix holding the values
    for the right-hand matrix in the multiplication.

    \item[{\sf desc}] ({\sf IN}) An optional operation descriptor.  If
    a \emph{default} descriptor is desired, {\sf GrB\_NULL} can be
    used.  Valid fields are as follows: \\
    
    \begin{tabular}{lllp{2.5in}}
    Param & Field  & Value & Description \\
    \hline
    {\sf w}    & {\sf GrB\_OUTP} & {\sf GrB\_REPLACE} & Output vector {\sf w} is cleared (all elements removed) before the result is stored in it.\\
    {\sf mask} & {\sf GrB\_MASK} & {\sf GrB\_SCMP}   & Use the structural complement of {\sf mask}. \\
    {\sf A}    & {\sf GrB\_INP1} & {\sf GrB\_TRAN}   & Use transpose of {\sf A} for operation. \\
    \end{tabular}
\end{itemize}

\paragraph{Return Values}

\begin{itemize}[leftmargin=2.1in]
    \item[{\sf GrB\_SUCCESS}]         In blocking mode, operation
	completed successfully. In non-blocking mode, this indicates
	that the consistency tests on dimensions and domains for the
	input arguments passed successfully. Either way, output vector
	{\sf w} is ready to be used in the next method of the sequence.

    \item[{\sf GrB\_PANIC}]           Unknown internal error.
    
    \item[{\sf GrB\_INVALID\_OBJECT}] This is returned in any execution mode 
    whenever one of the opaque GraphBLAS objects (input or output) is in an invalid 
    state caused by a previous execution error.  Call {GrB\_error()} to access 
    any error messages generated by the implementation.

    \item[{\sf GrB\_OUT\_OF\_MEMORY}]  Not enough memory available for operation.
    
    \item[{\sf GrB\_UNINITIALIZED\_OBJECT}] One or more of the GraphBLAS objects 
    has not been initialized by a call to {\sf new} (or {\sf dup} for matrix or
    vector parameters).
    
    \item[{\sf GrB\_NULL\_POINTER}]  The output vector pointer is {\sf NULL}.
    
    \item[{\sf GrB\_DIMENSION\_MISMATCH}] Mask, vector, and/or matrix 
    dimensions are incompatible.

	\item[{\sf GrB\_DOMAIN\_MISMATCH}]    The domains of the various
	vectors/matrices are incompatible with the corresponding domains of the
	accumulating operation, semiring, or mask.
\end{itemize}

\paragraph{Description}

{\sf GrB\_vxm} computes the vector-matrix product ${\sf w}^T = {\sf
u}^T \otimes . \oplus {\sf A}$, or, if an optional binary accumulation
operator ($\odot$) is provided, ${\sf w}^T = {\sf w}^T \odot
\left({\sf u}^T \otimes . \oplus {\sf A}\right)$ (where matrix {\sf A}
 can be optionally transposed).  Logically, this operation
occurs in three steps:
\begin{enumerate}[leftmargin=0.75in]
\item[\bf Setup] The internal vectors, matrices and mask used in the computation are formed and their domains/dimensions are tested for consistency.
\item[\bf Compute] The indicated computations are carried out.
\item[\bf Output] The result is written into the output vector, possibly under control of a mask.
\end{enumerate}

Up to four argument vectors or matrices are used in the {\sf GrB\_vxm} operation:
\begin{enumerate}
	\item ${\sf w} = \langle \bold{D}({\sf w}),\bold{size}({\sf w}),\bold{L}({\sf w}) = \{(i,w_i) \} \rangle$
	\item ${\sf mask} = \langle \bold{D}({\sf mask}),\bold{size}({\sf mask}),\bold{L}({\sf mask}) = \{(i,m_i) \} \rangle$ (optional)
	\item ${\sf u} = \langle \bold{D}({\sf u}),\bold{size}({\sf u}),\bold{L}({\sf u}) = \{(i,u_i) \} \rangle$
	\item ${\sf A} = \langle \bold{D}({\sf A}),\bold{nrows}({\sf A}), \bold{ncols}({\sf A}),\bold{L}({\sf A}) = \{(i,j,A_{ij}) \} \rangle$
\end{enumerate}

The argument matrices, vectors, the semiring, and the accumulation operator (if provided) 
are tested for domain consistency as follows:
\begin{enumerate}
	\item The domain of {\sf mask} (if not {\sf GrB\_NULL}) must be from one of the pre-defined types of Table~\ref{Tab:PredefinedTypes}.

	\item $\bold{D}({\sf u})$ must be compatible with $D_1$ of the semiring.

	\item $\bold{D}({\sf A})$ must be compatible with $D_2$ of the semiring.

	\item If {\sf accum} is {\sf GrB\_NULL}, then $\bold{D}({\sf w})$ must be compatible with $D_3$ of the semiring.

	\item If {\sf accum} is not {\sf GrB\_NULL}, then $\bold{D}({\sf w})$ must be compatible with $D_x$ and $D_z$ of the 
	accumulation operator and $D_3$ of the semiring must be compatible with $D_y$ of the accumulation operator.
\end{enumerate}
Two domains are compatible with each other if values from one domain can be cast 
to values in the other domain as per the rules of the C language.
In particular, domains from Table~\ref{Tab:PredefinedTypes} are all compatible 
with each other. A domain from a user-defined type is only compatible with itself.
If any consistency rule above is violated, execution of {\sf GrB\_vxm} ends and 
the domain mismatch error listed above is returned.

From the argument vectors and matrices, the internal matrices and mask used in 
the computation are formed ($\leftarrow$ denotes copy):
\begin{enumerate}
	\item Vector $\vector{\widetilde{w}} \leftarrow {\sf w}$.

	\item One-dimensional mask $\vector{\widetilde{m}}$ is computed from 
    argument {\sf mask} as follows:
	\begin{enumerate}
		\item	If ${\sf mask} = {\sf GrB\_NULL}$, then $\vector{\widetilde{m}} = 
        \langle \bold{size}({\sf w}), \{i, \forall i : 0 \leq i < 
        \bold{size}({\sf w}) \} \rangle$.

		\item	Otherwise, $\vector{\widetilde{m}} = 
        \langle \bold{size}({\sf mask}), \{i :  i \in \bold{ind}({\sf mask}) \wedge
        ({\sf bool}){\sf mask}(i) = \true \} \rangle$.

		\item	If ${\sf desc[GrB\_MASK].GrB\_SCMP}$ is set, then 
        $\vector{\widetilde{m}} \leftarrow \neg \vector{\widetilde{m}}$.
	\end{enumerate}

	\item Vector $\vector{\widetilde{u}} \leftarrow {\sf u}$.

	\item Matrix $\matrix{\widetilde{A}} \leftarrow {\sf desc[GrB\_INP1].GrB\_TRAN} \ ? \ {\sf A}^T : {\sf A}$.
\end{enumerate}

The internal matrices and masks are checked for shape consistency. The following 
conditions must hold:
\begin{enumerate}
	\item $\bold{size}(\vector{\widetilde{w}}) = \bold{size}(\vector{\widetilde{m}})$.

	\item $\bold{size}(\matrix{\widetilde{w}}) = \bold{ncols}(\matrix{\widetilde{A}})$.

	\item $\bold{size}(\matrix{\widetilde{u}}) = \bold{nrows}(\matrix{\widetilde{A}})$.
\end{enumerate}
If any consistency rule above is violated, execution of {\sf GrB\_vxm} ends and 
the dimension mismatch error listed above is returned.

From this point forward, in {\sf GrB\_NONBLOCKING} mode, the method can 
optionally exit with {\sf GrB\_SUCCESS} return code and defer any computation 
and/or execution error codes.

We are now ready to carry out the matrix multiplication and any additional 
associated operations.  We describe this in terms of two intermediate vectors:
\begin{itemize}
	\item $\vector{\widetilde{t}}$: The vector holding the product of vector
    $\vector{\widetilde{u}}^T$ and matrix $\matrix{\widetilde{A}}$.
	\item $\vector{\widetilde{z}}$: The vector holding the result after 
    application of the (optional) accumulation operator.
\end{itemize}

The intermediate vector $\vector{\widetilde{t}} = \langle
D_3, \bold{ncols}(\matrix{\widetilde{A}}),
\bold{L}(\vector{\widetilde{t}}) =
\{(j,t_j) : \bold{ind}(\vector{\widetilde{u}}) \cap
\bold{ind}(\matrix{\widetilde{A}}(:,j)) \neq \emptyset \} \rangle$
is created.  The value of each of its elements is computed by \[t_j
= \bigoplus_{k \in \bold{ind}(\vector{\widetilde{u}}) \cap
\bold{ind}(\matrix{\widetilde{A}}(:,j))} (\vector{\widetilde{u}}(k)
\otimes \matrix{\widetilde{A}}(k,j)),\] where $\oplus$ and $\otimes$
are the additive and multiplicative operators of semiring {\sf op},
respectively.

The intermediate vector $\vector{\widetilde{z}}$ is created as follows:
\begin{itemize}
    \item If ${\sf accum} = {\sf GrB\_NULL}$, then $\vector{\widetilde{z}} = \vector{\widetilde{t}}$. 
    
    \item If ${\sf accum} = \langle D_x, D_y, D_z, \odot \rangle$, then vector $\vector{\widetilde{z}}$ is defined as 
        \[ \langle D_z, \bold{size}(\vector{\widetilde{w}}), 
        \bold{L}(\vector{\widetilde{z}}) 
		= \{(i,z_{i})  \forall (i) \in \bold{ind}(\vector{\widetilde{w}}) \cup 
        \bold{ind}(\vector{\widetilde{t}}) \} \rangle.\]  

    The values of the elements of $\vector{\widetilde{z}}$ are computed based on the 
    relationships between the sets of indices in $\vector{\widetilde{w}}$ and 
    $\vector{\widetilde{t}}$.
\[
z_{i} = \vector{\widetilde{w}}(i) \odot \vector{\widetilde{t}}(i), \ \mbox{if}\  i \in  (\bold{ind}(\vector{\widetilde{t}}) \cap \bold{ind}(\vector{\widetilde{w}})),
\]
\[
z_{i} = \vector{\widetilde{w}}(i), \ \mbox{if}\  i \in  (\bold{ind}(\vector{\widetilde{w}}) - (\bold{ind}(\vector{\widetilde{t}}) \cap \bold{ind}(\vector{\widetilde{w}}))),
\]
\[
z_{i} = \vector{\widetilde{t}}(i), \ \mbox{if}\  i \in  (\bold{ind}(\vector{\widetilde{t}}) - (\bold{ind}(\vector{\widetilde{t}}) \cap \bold{ind}(\vector{\widetilde{w}}))).
\]
where the difference operator in the previous expressions refers to set difference.
\end{itemize}

Finally, the set of output values that make up the $\vector{\widetilde{z}}$ 
vector are written into the final result vector, {\sf w}. 
This is carried out under control of the mask which acts as a ``write mask''.
\begin{itemize}
\item If {\sf desc[GrB\_OUTP].GrB\_REPLACE} is set, then any values in {\sf w} 
on input to {\sf GrB\_vxm()} are deleted and the new output vector {\sf w} is,
\[ 
\bold{L}({\sf w}) = \{(i,z_{i}) : i \in (\bold{ind}(\vector{\widetilde{z}}) 
\cap \bold{ind}(\vector{\widetilde{m}})) \}. 
\]

\item If {\sf desc[GrB\_OUTP].GrB\_REPLACE} is not set, the elements of 
$\vector{\widetilde{z}}$ indicated by the mask are copied into the result 
vector, {\sf w}, and elements of {\sf w} that fall outside the set indicated by 
the mask are unchanged:
\[ 
\bold{L}({\sf w}) = \{(i,w_{i}) : i \in (\bold{ind}(\vector{\sf w}) 
\cap \bold{ind}(\neg \vector{\widetilde{m}})) \} \cup \{(i,z_{i}) : i \in 
(\bold{ind}(\vector{\widetilde{z}}) \cap \bold{ind}(\vector{\widetilde{m}})) \}. 
\]
\end{itemize}

In {\sf GrB\_BLOCKING} mode, the method exits with return value 
{\sf GrB\_SUCCESS} and the new content of vector {\sf w} is as defined above
and fully computed.  
In {\sf GrB\_NONBLOCKING} mode, the method exits with return value 
{\sf GrB\_SUCCESS} and the new content of vector {\sf w} is as defined above 
but may not be fully computed; however, it can be used in the next GraphBLAS 
method call in a sequence.


%-----------------------------------------------------------------------------

\subsection{{\sf mxv}: Matrix-vector multiply}

Multiplies a matrix by a vector on a semiring. The result is a vector.

\paragraph{\syntax}

\begin{verbatim}
        GrB_Info GrB_mxv(GrB_Vector            *w,
                         const GrB_Vector       mask,
                         const GrB_BinaryOp     accum,
                         const GrB_Semiring     op, 
                         const GrB_Matrix       A,
                         const GrB_Vector       u,
                         const GrB_Descriptor   desc);
\end{verbatim}

\paragraph{Parameters}

\begin{itemize}[leftmargin=1.1in]
    \item[{\sf w}]    ({\sf INOUT}) An existing GraphBLAS vector.  On input, the
    vector provides values that may be accumulated with the result of the
    matrix-vector product.  On output, this vector holds the results of the
    operation.
    
    \item[{\sf mask}] ({\sf IN}) An optional ``write'' mask that controls which
    results from this operation are stored into the output vector
    ${\sf w}$.  If no mask is desired (\ie, all elements
    of result are copied into the output vector), {\sf GrB\_NULL}
    should be specified. The mask dimensions must match those of the
    vector {\sf w} and the domain of {\sf mask} must be
    of type {\sf bool} or any of the predefined ``built-in'' types in
    Table~\ref{Tab:PredefinedTypes}.

	\item[{\sf accum}]  ({\sf IN}) An optional operator used for accumulating
    entries into existing {\sf w} entries: ${\sf accum} = \langle D_x,
    D_y, D_z,\odot \rangle$. If assignment rather than accumulation is
    desired, {\sf GrB\_NULL} should be specified.

    \item[{\sf op}]   ({\sf IN}) Semiring used in the vector-matrix
    multiply: ${\sf op}=\langle D_1,D_2,D_3,\oplus,\otimes,0 \rangle$.
    
    \item[{\sf A}]    ({\sf IN}) The GraphBLAS matrix holding the values
    for the left-hand matrix in the multiplication.
    
    \item[{\sf u}]    ({\sf IN}) The GraphBLAS vector holding the values for
    the right-hand vector in the multiplication.

    \item[{\sf desc}] ({\sf IN}) An optional operation descriptor.  If
    a \emph{default} descriptor is desired, {\sf GrB\_NULL} can be
    used.  Valid fields are as follows: \\
    
    \begin{tabular}{lllp{2.5in}}
    Param & Field  & Value & Description \\
    \hline
    {\sf w}    & {\sf GrB\_OUTP} & {\sf GrB\_REPLACE} & Output vector {\sf w} is cleared (all elements removed) before result is stored in it.\\
    {\sf mask} & {\sf GrB\_MASK} & {\sf GrB\_SCMP}   & Use the structural complement of {\sf mask}. \\
    {\sf A}    & {\sf GrB\_INP0} & {\sf GrB\_TRAN}   & Use transpose of {\sf A} for operation. \\
    \end{tabular}
\end{itemize}

\paragraph{Return Values}

\begin{itemize}[leftmargin=2.1in]
    \item[{\sf GrB\_SUCCESS}]         In blocking mode, operation
	completed successfully. In non-blocking mode, this indicates
	that the consistency tests on dimensions and domains for the
	input arguments passed successfully. Either way, output vector
	{\sf w} is ready to be used in the next method of the sequence.

    \item[{\sf GrB\_PANIC}]           Unknown internal error.
    
    \item[{\sf GrB\_INVALID\_OBJECT}] This is returned in any execution mode 
    whenever one of the opaque GraphBLAS objects (input or output) is in an invalid 
    state caused by a previous execution error.  Call {GrB\_error()} to access 
    any error messages generated by the implementation.

    \item[{\sf GrB\_OUT\_OF\_MEMORY}]  Not enough memory available for operation.
    
    \item[{\sf GrB\_UNINITIALIZED\_OBJECT}] One or more of the GraphBLAS objects 
    has not been initialized by a call to {\sf new} (or {\sf dup} for matrix or
    vector parameters).
    
    \item[{\sf GrB\_NULL\_POINTER}]  The output vector pointer is {\sf NULL}.
    
    \item[{\sf GrB\_DIMENSION\_MISMATCH}] Mask, vector, and/or matrix 
    dimensions are incompatible.

	\item[{\sf GrB\_DOMAIN\_MISMATCH}]    The domains of the various
	vectors/matrices are incompatible with the corresponding domains of the
	accumulating operation, semiring, or mask.
\end{itemize}

\paragraph{Description}

{\sf GrB\_mxv} computes the matrix-vector product ${\sf w} = {\sf A}
\otimes . \oplus {\sf u}$, or, if an optional binary accumulation
operator ($\odot$) is provided, ${\sf w} = {\sf w} \odot \left({\sf A}
\otimes . \oplus {\sf u}\right)$ (where matrix {\sf A}
 can be optionally transposed).  Logically, this operation
occurs in three steps:
\begin{enumerate}[leftmargin=0.85in]
\item[\bf Setup] The internal vectors, matrices and mask used in the computation are formed and their domains/dimensions are tested for consistency.
\item[\bf Compute] The indicated computations are carried out.
\item[\bf Output] The result is written into the output vector, possibly under control of a mask.
\end{enumerate}

Up to four argument vectors or matrices are used in the {\sf GrB\_mxv} operation:
\begin{enumerate}
	\item ${\sf w} = \langle \bold{D}({\sf w}),\bold{size}({\sf w}),\bold{L}({\sf w}) = \{(i,w_i) \} \rangle$
	\item ${\sf mask} = \langle \bold{D}({\sf mask}),\bold{size}({\sf mask}),\bold{L}({\sf mask}) = \{(i,m_i) \} \rangle$ (optional)
	\item ${\sf A} = \langle \bold{D}({\sf A}),\bold{nrows}({\sf A}), \bold{ncols}({\sf A}),\bold{L}({\sf A}) = \{(i,j,A_{ij}) \} \rangle$
	\item ${\sf u} = \langle \bold{D}({\sf u}),\bold{size}({\sf u}),\bold{L}({\sf u}) = \{(i,u_i) \} \rangle$
\end{enumerate}

The argument matrices, vectors, the semiring, and the accumulation operator (if provided) 
are tested for domain consistency as follows:
\begin{enumerate}
	\item The domain of {\sf mask} (if not {\sf GrB\_NULL}) must be from one of the pre-defined types of Table~\ref{Tab:PredefinedTypes}.

	\item $\bold{D}({\sf A})$ must be compatible with $D_1$ of the semiring.

	\item $\bold{D}({\sf u})$ must be compatible with $D_2$ of the semiring.

	\item If {\sf accum} is {\sf GrB\_NULL}, then $\bold{D}({\sf w})$ must be compatible with $D_3$ of the semiring.

	\item If {\sf accum} is not {\sf GrB\_NULL}, then $\bold{D}({\sf w})$ must be compatible with $D_x$ and $D_z$ of the 
	accumulation operator and $D_3$ of the semiring must be compatible with $D_y$ of the accumulation operator.
\end{enumerate}
Two domains are compatible with each other if values from one domain can be cast 
to values in the other domain as per the rules of the C language.
In particular, domains from Table~\ref{Tab:PredefinedTypes} are all compatible 
with each other. A domain from a user-defined type is only compatible with itself.
If any consistency rule above is violated, execution of {\sf GrB\_mxv} ends and 
the domain mismatch error listed above is returned.

From the argument vectors and matrices, the internal matrices and mask used in 
the computation are formed ($\leftarrow$ denotes copy):
\begin{enumerate}
	\item Vector $\vector{\widetilde{w}} \leftarrow {\sf w}$.

	\item One-dimensional mask $\vector{\widetilde{m}}$ is computed from 
    argument {\sf mask} as follows:
	\begin{enumerate}
		\item	If ${\sf mask} = {\sf GrB\_NULL}$, then $\vector{\widetilde{m}} = 
        \langle \bold{size}({\sf w}), \{i, \forall i : 0 \leq i < 
        \bold{size}({\sf w}) \} \rangle$.

		\item	Otherwise, $\vector{\widetilde{m}} = 
        \langle \bold{size}({\sf mask}), \{i :  i \in \bold{ind}({\sf mask}) \wedge
        ({\sf bool}){\sf mask}(i) = \true \} \rangle$.

		\item	If ${\sf desc[GrB\_MASK].GrB\_SCMP}$ is set, then 
        $\vector{\widetilde{m}} \leftarrow \neg \vector{\widetilde{m}}$.
	\end{enumerate}

	\item Matrix $\matrix{\widetilde{A}} \leftarrow {\sf desc[GrB\_INP0].GrB\_TRAN} \ ? \ {\sf A}^T : {\sf A}$.

	\item Vector $\vector{\widetilde{u}} \leftarrow {\sf u}$.
\end{enumerate}

The internal matrices and masks are checked for shape consistency. The following 
conditions must hold:
\begin{enumerate}
	\item $\bold{size}(\vector{\widetilde{w}}) = \bold{size}(\vector{\widetilde{m}})$.

	\item $\bold{size}(\vector{\widetilde{w}}) = \bold{nrows}(\matrix{\widetilde{A}})$.

	\item $\bold{size}(\vector{\widetilde{u}}) = \bold{ncols}(\matrix{\widetilde{A}})$.
\end{enumerate}
If any consistency rule above is violated, execution of {\sf GrB\_mxv} ends and 
the dimension mismatch error listed above is returned.

From this point forward, in {\sf GrB\_NONBLOCKING} mode, the method can 
optionally exit with {\sf GrB\_SUCCESS} return code and defer any computation 
and/or execution error codes.

We are now ready to carry out the matrix-vector multiplication and any additional 
associated operations.  We describe this in terms of two intermediate vectors:
\begin{itemize}
	\item $\vector{\widetilde{t}}$: The vector holding the product of matrix 
    $\matrix{\widetilde{A}}$ and vector $\vector{\widetilde{u}}$.
	\item $\vector{\widetilde{z}}$: The vector holding the result after 
    application of the (optional) accumulation operator.
\end{itemize}

The intermediate vector $\vector{\widetilde{t}} = \langle
D_3, \bold{nrows}(\matrix{\widetilde{A}}),
\bold{L}(\vector{\widetilde{t}}) =
\{(i,t_i) : \bold{ind}(\matrix{\widetilde{A}}(i,:)) \cap 
\bold{ind}(\vector{\widetilde{u}})
 \neq \emptyset \} \rangle$
is created.  The value of each of its elements is computed by 
\[t_i = \bigoplus_{k \in \bold{ind}(\matrix{\widetilde{A}}(i,:)) \cap
\bold{ind}(\vector{\widetilde{u}})} (\matrix{\widetilde{A}}(i,k)
\otimes \vector{\widetilde{u}}(k)),\] where $\oplus$ and $\otimes$
are the additive and multiplicative operators of semiring {\sf op},
respectively.

The intermediate vector $\vector{\widetilde{z}}$ is created as follows:
\begin{itemize}
    \item If ${\sf accum} = {\sf GrB\_NULL}$, then $\vector{\widetilde{z}} = \vector{\widetilde{t}}$.
    
    \item If ${\sf accum} = \langle D_x, D_y, D_z, \odot \rangle$, then vector $\vector{\widetilde{z}}$ is defined as 
        \[ \langle D_z, \bold{size}(\vector{\widetilde{w}}), 
        \bold{L}(\vector{\widetilde{z}}) 
		= \{(i,z_{i})  \forall (i) \in \bold{ind}(\vector{\widetilde{w}}) \cup 
        \bold{ind}(\vector{\widetilde{t}}) \} \rangle.\]  

    The values of the elements of $\vector{\widetilde{z}}$ are computed based on the 
    relationships between the sets of indices in $\vector{\widetilde{w}}$ and 
    $\vector{\widetilde{t}}$.
\[
z_{i} = \vector{\widetilde{w}}(i) \odot \vector{\widetilde{t}}(i), \ \mbox{if}\  i \in  (\bold{ind}(\vector{\widetilde{t}}) \cap \bold{ind}(\vector{\widetilde{w}})),
\]
\[
z_{i} = \vector{\widetilde{w}}(i), \ \mbox{if}\  i \in  (\bold{ind}(\vector{\widetilde{w}}) - (\bold{ind}(\vector{\widetilde{t}}) \cap \bold{ind}(\vector{\widetilde{w}}))),
\]
\[
z_{i} = \vector{\widetilde{t}}(i), \ \mbox{if}\  i \in  (\bold{ind}(\vector{\widetilde{t}}) - (\bold{ind}(\vector{\widetilde{t}}) \cap \bold{ind}(\vector{\widetilde{w}}))).
\]
where the difference operator in the previous expressions refers to set difference.
\end{itemize}

Finally, the set of output values that make up the $\vector{\widetilde{z}}$ 
vector are written into the final result vector, {\sf w}. 
This is carried out under control of the mask which acts as a ``write mask''.
\begin{itemize}
\item If {\sf desc[GrB\_OUTP].GrB\_REPLACE} is set, then any values in {\sf w} 
on input to {\sf GrB\_mxv()} are deleted and the new output vector {\sf w} is,
\[ 
\bold{L}({\sf w}) = \{(i,z_{i}) : i \in (\bold{ind}(\vector{\widetilde{z}}) 
\cap \bold{ind}(\vector{\widetilde{m}})) \}. 
\]

\item If {\sf desc[GrB\_OUTP].GrB\_REPLACE} is not set, the elements of 
$\vector{\widetilde{z}}$ indicated by the mask are copied into the result 
vector, {\sf w}, and elements of {\sf w} that fall outside the set indicated by 
the mask are unchanged:
\[ 
\bold{L}({\sf w}) = \{(i,w_{i}) : i \in (\bold{ind}(\vector{\sf w}) 
\cap \bold{ind}(\neg \vector{\widetilde{m}})) \} \cup \{(i,z_{i}) : i \in 
(\bold{ind}(\vector{\widetilde{z}}) \cap \bold{ind}(\vector{\widetilde{m}})) \}. 
\]
\end{itemize}

In {\sf GrB\_BLOCKING} mode, the method exits with return value 
{\sf GrB\_SUCCESS} and the new content of vector {\sf w} is as defined above
and fully computed.  
In {\sf GrB\_NONBLOCKING} mode, the method exits with return value 
{\sf GrB\_SUCCESS} and the new content of vector {\sf w} is as defined above 
but may not be fully computed; however, it can be used in the next GraphBLAS 
method call in a sequence.

