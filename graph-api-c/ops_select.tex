%-----------------------------------------------------------------------------

\subsection{{\sf select}:} Apply a select operator to the stored elements of an 
object to determine whether or not to keep them.
\label{Sec:Select}

%-----------------------------------------------------------------------------

\subsubsection{{\sf select}: Vector variant\scott{NEW CONTENT}} 

Apply a select operator (an index unary operator) to the elements of a vector.

\paragraph{\syntax}

\begin{verbatim}
        // scalar value variant
        GrB_Info GrB_select(GrB_Vector              w,
                            const GrB_Vector        mask,
                            const GrB_BinaryOp      accum,
                            const GrB_IndexUnaryOp  op,
                            const GrB_Vector        u,
                            <type>                  val,
                            const GrB_Descriptor    desc);

        // GraphBLAS scalar variant
        GrB_Info GrB_select(GrB_Vector              w,
                            const GrB_Vector        mask,
                            const GrB_BinaryOp      accum,
                            const GrB_IndexUnaryOp  op,
                            const GrB_Vector        u,
                            const GrB_Scalar        s,
                            const GrB_Descriptor    desc);

\end{verbatim}

\paragraph{Parameters}

\begin{itemize}[leftmargin=1.1in]
    \item[{\sf w}]    ({\sf INOUT}) An existing GraphBLAS vector.  On input,
    the vector provides values that may be accumulated with the result of the
    select operation.  On output, this vector holds the results of the
    operation.

    \item[{\sf mask}] ({\sf IN}) An optional ``write'' mask that controls which
    results from this operation are stored into the output vector {\sf w}. The 
    mask dimensions must match those of the vector {\sf w}. If the 
    {\sf GrB\_STRUCTURE} descriptor is {\em not} set for the mask, the domain of the
    {\sf mask} vector must be of type {\sf bool} or any of the predefined 
    ``built-in'' types in Table~\ref{Tab:PredefinedTypes}.  If the default
    mask is desired (\ie, a mask that is all {\sf true} with the dimensions of {\sf w}), 
    {\sf GrB\_NULL} should be specified.

    \item[{\sf accum}] ({\sf IN}) An optional binary operator used for accumulating
    entries into existing {\sf w} entries. If assignment rather than accumulation is
    desired, {\sf GrB\_NULL} should be specified.

{\color{red}
    \item[{\sf op}] ({\sf IN}) An index unary operator, 
    $F_i = \langle \Dout, \Din1, \mathbf{D}({\sf GrB\_Index}), \Din2, f_{i} \rangle$, 
    applied to each element stored in the input vector, {\sf u}. It is a function 
    of the stored element's value, its location index,
    and a user supplied scalar value (either {\sf s} or {\sf val}).
}
    \item[{\sf u}] ({\sf IN}) The GraphBLAS vector whose elements are passed 
    to the index unary operator.

    \item[{\sf val}] ({\sf IN}) An additional scalar value that is passed to the 
    index unary operator.

{\color{red}
    \item[{\sf s}] ({\sf IN}) An GraphBLAS scalar that is passed to the 
    index unary operator.  It must not be empty.
}
    
    \item[{\sf desc}] ({\sf IN}) An optional operation descriptor. If
    a \emph{default} descriptor is desired, {\sf GrB\_NULL} should be
    specified. Non-default field/value pairs are listed as follows:  \\

    \hspace*{-2em}\begin{tabular}{lllp{2.7in}}
        Param & Field  & Value & Description \\
        \hline
        {\sf w}    & {\sf GrB\_OUTP} & {\sf GrB\_REPLACE} & Output vector {\sf w}
        is cleared (all elements removed) before the result is stored in it.\\

        {\sf mask} & {\sf GrB\_MASK} & {\sf GrB\_STRUCTURE}   & The write mask is
        constructed from the structure (pattern of stored values) of the input
        {\sf mask} vector. The stored values are not examined.\\

        {\sf mask} & {\sf GrB\_MASK} & {\sf GrB\_COMP}   & Use the 
        complement of {\sf mask}. \\
    \end{tabular}
\end{itemize}

\paragraph{Return Values}

\begin{itemize}[leftmargin=2.2in]
    \item[{\sf GrB\_SUCCESS}]         In blocking mode, the operation completed
    successfully. In non-blocking mode, this indicates that the compatibility 
    tests on dimensions and domains for the input arguments passed successfully. 
    Either way, output vector {\sf w} is ready to be used in the next method of 
    the sequence.

    \item[{\sf GrB\_PANIC}]           Unknown internal error.

    \item[{\sf GrB\_INVALID\_OBJECT}] This is returned in any execution mode 
    whenever one of the opaque GraphBLAS objects (input or output) is in an invalid 
    state caused by a previous execution error.  Call {\sf GrB\_error()} to access 
    any error messages generated by the implementation.

    \item[{\sf GrB\_OUT\_OF\_MEMORY}] Not enough memory available for operation.

{\color{red}
    \item[{\sf GrB\_UNINITIALIZED\_OBJECT}] One or more of the GraphBLAS objects
    has not been initialized by a call to one of its constructors.
}

    \item[{\sf GrB\_DIMENSION\_MISMATCH}]  {\sf mask}, {\sf w} and/or {\sf u}
    dimensions are incompatible.

    \item[{\sf GrB\_DOMAIN\_MISMATCH}]    The domains of the various vectors are
    incompatible with the corresponding domains of the accumulation operator
    or index unary operator, or the mask's domain is not compatible with {\sf bool}
    (in the case where {\sf desc[GrB\_MASK].GrB\_STRUCTURE} is not set).

{\color{red}
    \item[{\sf GrB\_EMPTY\_OBJECT}] The {\sf GrB\_Scalar} {\sf s} used in the call
	is empty ($\mathbf{nvals}({\sf s}) = 0$) and therefore a value
	cannot be passed to the index unary operator.
}
\end{itemize}

\paragraph{Description}

This variant of {\sf GrB\_select} computes the result of applying a index unary operator
to select the elements of the input GraphBLAS vector.  The operator takes, as input,
the value of each stored element, along with the element's index and a scalar 
constant -- either {\sf val} or {\sf s}.  The corresponding element of the input vector is selected (kept)
if the function evaluates to {\sf true} when cast to {\sf bool}.  This acts like a
functional mask on the input vector as follows:
\begin{itemize}[leftmargin=2.1in]
\item[~] ${\sf w} = {\sf u}\langle f_i({\sf u}, \bold{ind}({\sf u}), 0, {\sf val})\rangle$,
\item[~] ${\sf w} = {\sf w} \odot {\sf u}\langle f_i({\sf u}, \bold{ind}({\sf u}), 0, {\sf val})\rangle$.  
\end{itemize}
{\color{red}
Correspondingly, if a {\sf GrB\_Scalar}, {\sf s}, is provided:
\begin{itemize}[leftmargin=2.1in]
\item[~] ${\sf w} = {\sf u}\langle f_i({\sf u}, \bold{ind}({\sf u}), 0, {\sf s})\rangle$,
\item[~] ${\sf w} = {\sf w} \odot {\sf u}\langle f_i({\sf u}, \bold{ind}({\sf u}), 0, {\sf s})\rangle$.  
\end{itemize}
}
%Note that if an index unary operator is provided that expects two indices {\it and} 
%accesses the second index the results are undefined.

Logically, this operation occurs in three steps:
\begin{enumerate}[leftmargin=0.75in]
\item[\bf Setup] The internal vectors and mask used in the computation are formed 
and their domains and dimensions are tested for compatibility.
\item[\bf Compute] The indicated computations are carried out.
\item[\bf Output] The result is written into the output vector, possibly under 
control of a mask.
\end{enumerate}

Up to three argument vectors are used in this {\sf GrB\_select} operation:
\begin{enumerate}
    \item ${\sf w} = \langle \bold{D}({\sf w}),\bold{size}({\sf w}),
    \bold{L}({\sf w}) = \{(i,w_i) \} \rangle$

    \item ${\sf mask} = \langle \bold{D}({\sf mask}),\bold{size}({\sf mask}),
    \bold{L}({\sf mask}) = \{(i,m_i) \} \rangle$ (optional)

    \item ${\sf u} = \langle \bold{D}({\sf u}),\bold{size}({\sf u}),
    \bold{L}({\sf u}) = \{(i,u_i) \} \rangle$
\end{enumerate}

The argument scalar, vectors, index unary operator and the accumulation 
operator (if provided) are tested for domain compatibility as follows:
\begin{enumerate}
    \item If {\sf mask} is not {\sf GrB\_NULL}, and ${\sf desc[GrB\_MASK].GrB\_STRUCTURE}$
    is not set, then $\bold{D}({\sf mask})$ must be from one of the pre-defined types of 
    Table~\ref{Tab:PredefinedTypes}.

    \item $\bold{D}({\sf w})$ must be compatible with $\bold{D}({\sf u})$.

    \item If {\sf accum} is not {\sf GrB\_NULL}, then $\bold{D}({\sf w})$ must be
    compatible with $\bDin1({\sf accum})$ and $\bDout({\sf accum})$ of the accumulation operator and 
    $\bold{D}({\sf u})$ must be compatible with $\bDin2({\sf accum})$ of the accumulation operator.

	\item $\bDout({\sf op})$ of the index unary operator must be from one of the pre-defined types of 
    Table~\ref{Tab:PredefinedTypes}; i.e., castable to {\sf bool}.

    \item $\bold{D}({\sf u})$ must be compatible with $\bDin1({\sf op})$ of the index unary operator.
    
{\color{red}
    \item $\bold{D}({\sf val})$ or $\bold{D}({\sf s})$, depending on the signature of the method,
    must be compatible with $\bDin2({\sf op})$ of the index unary operator.
}
\end{enumerate}

Two domains are compatible with each other if values from one domain can be cast 
to values in the other domain as per the rules of the C language.
In particular, domains from Table~\ref{Tab:PredefinedTypes} are all compatible 
with each other. A domain from a user-defined type is only compatible with itself.
If any compatibility rule above is violated, execution of {\sf GrB\_select} ends
and the domain mismatch error listed above is returned.

From the argument vectors, the internal vectors and mask used in 
the computation are formed ($\leftarrow$ denotes copy):
\begin{enumerate}
    \item Vector $\vector{\widetilde{w}} \leftarrow {\sf w}$.

    \item One-dimensional mask, $\vector{\widetilde{m}}$, is computed from 
    argument {\sf mask} as follows:
    \begin{enumerate}
        \item If ${\sf mask} = {\sf GrB\_NULL}$, then $\vector{\widetilde{m}} = 
        \langle \bold{size}({\sf w}), \{i, \ \forall \ i : 0 \leq i < 
        \bold{size}({\sf w}) \} \rangle$.

        \item If {\sf mask} $\ne$ {\sf GrB\_NULL},  
        \begin{enumerate}
            \item If ${\sf desc[GrB\_MASK].GrB\_STRUCTURE}$ is set, then
            $\vector{\widetilde{m}} = 
            \langle \bold{size}({\sf mask}), \{i : i \in \bold{ind}({\sf mask}) \} \rangle$,
            \item Otherwise, $\vector{\widetilde{m}} = 
            \langle \bold{size}({\sf mask}), \{i : i \in \bold{ind}({\sf mask}) \wedge
            ({\sf bool}){\sf mask}(i) = \true \} \rangle$.
        \end{enumerate}

        \item    If ${\sf desc[GrB\_MASK].GrB\_COMP}$ is set, then 
        $\vector{\widetilde{m}} \leftarrow \neg \vector{\widetilde{m}}$.
    \end{enumerate}

    \item Vector $\vector{\widetilde{u}} \leftarrow {\sf u}$.

{\color{red}
    \item Scalar $\widetilde{\sf s} \leftarrow {\sf s}$ ({\sf GrB\_Scalar} version only).
}
\end{enumerate}

The internal vectors and masks are checked for dimension compatibility. 
The following conditions must hold:
\begin{enumerate}
    \item $\bold{size}(\vector{\widetilde{w}}) = \bold{size}(\vector{\widetilde{m}})$
    \item $\bold{size}(\vector{\widetilde{u}}) = \bold{size}(\vector{\widetilde{w}})$.
\end{enumerate}
If any compatibility rule above is violated, execution of {\sf GrB\_select} ends and 
the dimension mismatch error listed above is returned.

From this point forward, in {\sf GrB\_NONBLOCKING} mode, the method can optionally exit 
with {\sf GrB\_SUCCESS} return code and defer any computation and/or execution error codes.

{\color{red}
If an empty {\sf GrB\_Scalar} $\widetilde{\sf s}$ is provided (\ie, $\mathbf{nvals}(\widetilde{\sf s}) = 0$),
the method returns with code {\sf GrB\_EMPTY\_OBJECT}. If a non-empty {\sf GrB\_Scalar}, 
$\widetilde{\sf s}$, is provided (\ie, $\mathbf{nvals}(\widetilde{\sf s}) = 1$), we then create an 
internal variable {\sf val} with the same domain as $\widetilde{\sf s}$ and set 
${\sf val} = \mathbf{val}(\widetilde{\sf s})$.
}

We are now ready to carry out the {\sf select} and any additional 
associated operations.  We describe this in terms of two intermediate vectors:
\begin{itemize}
    \item $\vector{\widetilde{t}}$: The vector holding the result from applying the index unary operator to the input vector
    $\vector{\widetilde{u}}$.
    \item $\vector{\widetilde{z}}$: The vector holding the result after 
    application of the (optional) accumulation operator.
\end{itemize}

{\color{red}
The intermediate vector, $\vector{\widetilde{t}}$, is created as follows:
\[
\begin{aligned}
\vector{\widetilde{t}} &= \langle
\bold{D}({\sf u}), \bold{size}(\vector{\widetilde{u}}),
\bold{L}(\vector{\widetilde{t}}) =
\{(i,\vector{\widetilde{u}}(i), : i \in \bold{ind}(\vector{\widetilde{u}}) \wedge ({\sf bool})f_i(\vector{\widetilde{u}}(i),i,0,{\sf val}) = \true \} \rangle, \\
\end{aligned}
\]
where $f_{i} = \mathbf{f}({\sf op})$.
}


The intermediate vector $\vector{\widetilde{z}}$ is created as follows, using what is called a \emph{standard vector accumulate}:
\begin{itemize}
    \item If ${\sf accum} = {\sf GrB\_NULL}$, then $\vector{\widetilde{z}} = \vector{\widetilde{t}}$.

    \item If ${\sf accum}$ is a binary operator, then $\vector{\widetilde{z}}$ is defined as
        \[ \vector{\widetilde{z}} = \langle \bDout({\sf accum}), \bold{size}(\vector{\widetilde{w}}),
        %\bold{L}(\vector{\widetilde{z}}) =
        \{(i,z_{i}) \ \forall \ i \in \bold{ind}(\vector{\widetilde{w}}) \cup 
        \bold{ind}(\vector{\widetilde{t}}) \} \rangle.\]

    The values of the elements of $\vector{\widetilde{z}}$ are computed based on the 
    relationships between the sets of indices in $\vector{\widetilde{w}}$ and 
    $\vector{\widetilde{t}}$.
\[
    z_{i} = \vector{\widetilde{w}}(i) \odot \vector{\widetilde{t}}(i), \ \mbox{if}\  
    i \in  (\bold{ind}(\vector{\widetilde{t}}) \cap \bold{ind}(\vector{\widetilde{w}})),
\]
\[
    z_{i} = \vector{\widetilde{w}}(i), \ \mbox{if}\  
    i \in (\bold{ind}(\vector{\widetilde{w}}) - (\bold{ind}(\vector{\widetilde{t}})
    \cap \bold{ind}(\vector{\widetilde{w}}))),
\]
\[
    z_{i} = \vector{\widetilde{t}}(i), \ \mbox{if}\  i \in  
    (\bold{ind}(\vector{\widetilde{t}}) - (\bold{ind}(\vector{\widetilde{t}})
    \cap \bold{ind}(\vector{\widetilde{w}}))),
\]
where $\odot  = \bigodot({\sf accum})$, and the difference operator refers to set difference.
\end{itemize}




Finally, the set of output values that make up vector $\vector{\widetilde{z}}$ 
are written into the final result vector {\sf w}, using
what is called a \emph{standard vector mask and replace}. 
This is carried out under control of the mask which acts as a ``write mask''.
\begin{itemize}
\item If {\sf desc[GrB\_OUTP].GrB\_REPLACE} is set, then any values in {\sf w} 
on input to this operation are deleted and the contents of the new output vector,
{\sf w}, is defined as,
\[ 
\bold{L}({\sf w}) = \{(i,z_{i}) : i \in (\bold{ind}(\vector{\widetilde{z}}) 
\cap \bold{ind}(\vector{\widetilde{m}})) \}. 
\]

\item If {\sf desc[GrB\_OUTP].GrB\_REPLACE} is not set, the elements of 
$\vector{\widetilde{z}}$ indicated by the mask are copied into the result 
vector, {\sf w}, and elements of {\sf w} that fall outside the set indicated by 
the mask are unchanged:
\[ 
\bold{L}({\sf w}) = \{(i,w_{i}) : i \in (\bold{ind}({\sf w}) 
\cap \bold{ind}(\neg \vector{\widetilde{m}})) \} \cup \{(i,z_{i}) : i \in 
(\bold{ind}(\vector{\widetilde{z}}) \cap \bold{ind}(\vector{\widetilde{m}})) \}. 
\]
\end{itemize}

In {\sf GrB\_BLOCKING} mode, the method exits with return value 
{\sf GrB\_SUCCESS} and the new content of vector {\sf w} is as defined above
and fully computed.  
In {\sf GrB\_NONBLOCKING} mode, the method exits with return value 
{\sf GrB\_SUCCESS} and the new content of vector {\sf w} is as defined above 
but may not be fully computed; however, it can be used in the next GraphBLAS 
method call in a sequence.


%-----------------------------------------------------------------------------

\subsubsection{{\sf select}: Matrix variant\scott{NEW CONTENT}}

Apply a select operator (an index unary operator) to the elements of a matrix.

\paragraph{\syntax}

\begin{verbatim}
        // scalar value variant
        GrB_Info GrB_select(GrB_Matrix              C,
                            const GrB_Matrix        Mask,
                            const GrB_BinaryOp      accum,
                            const GrB_IndexUnaryOp  op,
                            const GrB_Matrix        A,
                            <type>                  val,
                            const GrB_Descriptor    desc);

        // GraphBLAS scalar variant
        GrB_Info GrB_select(GrB_Matrix              C,
                            const GrB_Matrix        Mask,
                            const GrB_BinaryOp      accum,
                            const GrB_IndexUnaryOp  op,
                            const GrB_Matrix        A,
                            const GrB_Scalar        s,
                            const GrB_Descriptor    desc);
\end{verbatim}

\paragraph{Parameters}

\begin{itemize}[leftmargin=1.1in]
    \item[{\sf C}]    ({\sf INOUT}) An existing GraphBLAS matrix. On input,
    the matrix provides values that may be accumulated with the result of the
    select operation.  On output, the matrix holds the results of the
    operation.

    \item[{\sf Mask}] ({\sf IN}) An optional ``write'' mask that controls which
    results from this operation are stored into the output matrix {\sf C}. The 
    mask dimensions must match those of the matrix {\sf C}. If the 
    {\sf GrB\_STRUCTURE} descriptor is {\em not} set for the mask, the domain of the 
    {\sf Mask} matrix must be of type {\sf bool} or any of the predefined 
    ``built-in'' types in Table~\ref{Tab:PredefinedTypes}.  If the default
    mask is desired (\ie, a mask that is all {\sf true} with the dimensions of {\sf C}), 
    {\sf GrB\_NULL} should be specified.

    \item[{\sf accum}] ({\sf IN}) An optional binary operator used for accumulating
    entries into existing {\sf C} entries. If assignment rather than accumulation is
    desired, {\sf GrB\_NULL} should be specified.

{\color{red}
   \item[{\sf op}] ({\sf IN}) An index unary operator, 
    $F_i = \langle \Dout, \Din1, \mathbf{D}({\sf GrB\_Index}), \Din2, f_{i} \rangle$, 
    applied to each element stored in the input matrix, {\sf A}. It is a function 
    of the stored element's value, its row and column indices,
    and a user supplied scalar value (either {\sf s} or {\sf val}).
}

    \item[{\sf A}] ({\sf IN}) The GraphBLAS matrix whose elements are passed 
    to the index unary operator.

    \item[{\sf val}] ({\sf IN}) An additional scalar value that is passed to the 
    index unary operator.

{\color{red}
    \item[{\sf s}] ({\sf IN}) An GraphBLAS scalar that is passed to the 
    index unary operator.  It must not be empty.
}

    \item[{\sf desc}] ({\sf IN}) An optional operation descriptor. If
    a \emph{default} descriptor is desired, {\sf GrB\_NULL} should be
    specified. Non-default field/value pairs are listed as follows:  \\

    \hspace*{-2em}\begin{tabular}{lllp{2.7in}}
        Param & Field  & Value & Description \\
        \hline
        {\sf C}    & {\sf GrB\_OUTP} & {\sf GrB\_REPLACE} & Output matrix {\sf C}
        is cleared (all elements removed) before the result is stored in it.\\

        {\sf Mask} & {\sf GrB\_MASK} & {\sf GrB\_STRUCTURE}   & The write mask is
        constructed from the structure (pattern of stored values) of the input
        {\sf Mask} matrix. The stored values are not examined.\\

        {\sf Mask} & {\sf GrB\_MASK} & {\sf GrB\_COMP}   & Use the 
        complement of {\sf Mask}. \\

        {\sf A}    & {\sf GrB\_INP0} & {\sf GrB\_TRAN}   & Use transpose of {\sf A}
        for the operation. \\
    \end{tabular}
\end{itemize}


\paragraph{Return Values}

\begin{itemize}[leftmargin=2.1in]
    \item[{\sf GrB\_SUCCESS}]         In blocking mode, the operation completed
    successfully. In non-blocking mode, this indicates that the compatibility 
    tests on dimensions and domains for the input arguments passed successfully. 
    Either way, output mattrix {\sf C} is ready to be used in the next method of 
    the sequence.

    \item[{\sf GrB\_PANIC}]           Unknown internal error.

    \item[{\sf GrB\_INVALID\_OBJECT}] This is returned in any execution mode 
    whenever one of the opaque GraphBLAS objects (input or output) is in an invalid 
    state caused by a previous execution error.  Call {\sf GrB\_error()} to access 
    any error messages generated by the implementation.

    \item[{\sf GrB\_OUT\_OF\_MEMORY}] Not enough memory available for operation.

{\color{red}
    \item[{\sf GrB\_UNINITIALIZED\_OBJECT}] One or more of the GraphBLAS objects
    has not been initialized by a call to one of its constructors.
}

    \item[{\sf GrB\_DIMENSION\_MISMATCH}]  {\sf Mask}, {\sf C} and/or {\sf A}
    dimensions are incompatible.

    \item[{\sf GrB\_DOMAIN\_MISMATCH}]    The domains of the various matrices are
    incompatible with the corresponding domains of the accumulation operator
    or index unary operator, or the mask's domain is not compatible with {\sf bool}
    (in the case where {\sf desc[GrB\_MASK].GrB\_STRUCTURE} is not set).

{\color{red}
    \item[{\sf GrB\_EMPTY\_OBJECT}] The {\sf GrB\_Scalar} {\sf s} used in the call
	is empty ($\mathbf{nvals}({\sf s}) = 0$) and therefore a value
	cannot be passed to the index unary operator.
}
\end{itemize}

\paragraph{Description}

This variant of {\sf GrB\_select} computes the result of applying a index unary operator
to select the elements of the input GraphBLAS matrix.  The operator takes, as input,
the value of each stored element, along with the element's row and column indices and a scalar 
constant -- from either {\sf val} or {\sf s}.  The corresponding element of the input matrix is selected (kept)
if the function evaluates to {\sf true} when cast to {\sf bool}.  This acts like a
functional mask on the input matrix as follows when specifying a transparent scalar value:
\begin{itemize}[leftmargin=2.1in]
    \item[~] ${\sf C} =               {\sf A}\langle f_{i}({\sf A}, \bold{row}(\bold{ind}({\sf A})), \bold{col}(\bold{ind}({\sf A})), {\sf val})\rangle$, or
    \item[~] ${\sf C} = {\sf C} \odot {\sf A}\langle f_{i}({\sf A}, \bold{row}(\bold{ind}({\sf A})), \bold{col}(\bold{ind}({\sf A})), {\sf val})\rangle$.  
\end{itemize}
{\color{red}
Correspondingly, if a {\sf GrB\_Scalar}, {\sf s}, is provided:
\begin{itemize}[leftmargin=2.1in]
\item[~] ${\sf C} =               {\sf A}\langle f_{i}({\sf A}, \bold{row}(\bold{ind}({\sf A})), \bold{col}(\bold{ind}({\sf A})), {\sf s})\rangle$, or
\item[~] ${\sf C} = {\sf C} \odot {\sf A}\langle f_{i}({\sf A}, \bold{row}(\bold{ind}({\sf A})), \bold{col}(\bold{ind}({\sf A})), {\sf s})\rangle$.  
\end{itemize}
Where the $\bold{row}$ and $\bold{col}$ functions extract the row and column indices from a list of two-dimensional indices, respectively.
}

Logically, this operation occurs in three steps:
\begin{enumerate}[leftmargin=0.85in]
\item[\bf Setup] The internal matrices and mask used in the computation are formed 
and their domains and dimensions are tested for compatibility.
\item[\bf Compute] The indicated computations are carried out.
\item[\bf Output] The result is written into the output matrix, possibly under 
control of a mask.
\end{enumerate}

Up to three argument matrices are used in the {\sf GrB\_select} operation:
\begin{enumerate}
    \item ${\sf C} = \langle \bold{D}({\sf C}),\bold{nrows}({\sf C}),
    \bold{ncols}({\sf C}),\bold{L}({\sf C}) = \{(i,j,C_{ij}) \} \rangle$

    \item ${\sf Mask} = \langle \bold{D}({\sf Mask}),\bold{nrows}({\sf Mask}),
    \bold{ncols}({\sf Mask}),\bold{L}({\sf Mask}) = \{(i,j,M_{ij}) \} \rangle$ (optional)

    \item ${\sf A} = \langle \bold{D}({\sf A}),\bold{nrows}({\sf A}),
    \bold{ncols}({\sf A}),\bold{L}({\sf A}) = \{(i,j,A_{ij}) \} \rangle$
\end{enumerate}

The argument scalar, matrices, index unary operator and the accumulation 
operator (if provided) are tested for domain compatibility as follows:
\begin{enumerate}
    \item If {\sf Mask} is not {\sf GrB\_NULL}, and ${\sf desc[GrB\_MASK].GrB\_STRUCTURE}$
    is not set, then $\bold{D}({\sf Mask})$ must be from one of the pre-defined types of 
    Table~\ref{Tab:PredefinedTypes}.

    \item $\bold{D}({\sf C})$ must be compatible with $\bold{D}({\sf A})$.

    \item If {\sf accum} is not {\sf GrB\_NULL}, then $\bold{D}({\sf C})$ must be
    compatible with $\bDin1({\sf accum})$ and $\bDout({\sf accum})$ of the accumulation operator and 
    $\bold{D}({\sf A})$ must be compatible with $\bDin2({\sf accum})$ of the accumulation operator.

	\item $\bDout({\sf op})$ of the index unary operator must be from one of the pre-defined types of 
    Table~\ref{Tab:PredefinedTypes}; i.e., castable to {\sf bool}.

    \item $\bold{D}({\sf A})$ must be compatible with $\bDin1({\sf op})$ of the index unary operator.
    
{\color{red}
    \item $\bold{D}({\sf val})$ or $\bold{D}({\sf s})$, depending on the signature of the method,
    must be compatible with $\bDin2({\sf op})$ of the index unary operator.
}
\end{enumerate}

Two domains are compatible with each other if values from one domain can be cast 
to values in the other domain as per the rules of the C language.
In particular, domains from Table~\ref{Tab:PredefinedTypes} are all compatible 
with each other. A domain from a user-defined type is only compatible with itself.
If any compatibility rule above is violated, execution of {\sf GrB\_select} ends and 
the domain mismatch error listed above is returned.

From the argument matrices, the internal matrices, mask, and index arrays used in 
the computation are formed ($\leftarrow$ denotes copy):
\begin{enumerate}
    \item Matrix $\matrix{\widetilde{C}} \leftarrow {\sf C}$.

    \item Two-dimensional mask, $\matrix{\widetilde{M}}$, is computed from 
    argument {\sf Mask} as follows:
    \begin{enumerate}
        \item If ${\sf Mask} = {\sf GrB\_NULL}$, then $\matrix{\widetilde{M}} = 
        \langle \bold{nrows}({\sf C}), \bold{ncols}({\sf C}), \{(i,j), 
        \forall i,j : 0 \leq i <  \bold{nrows}({\sf C}), 0 \leq j < 
        \bold{ncols}({\sf C}) \} \rangle$.

        \item If {\sf Mask} $\ne$ {\sf GrB\_NULL},
        \begin{enumerate}
            \item If ${\sf desc[GrB\_MASK].GrB\_STRUCTURE}$ is set, then 
            $\matrix{\widetilde{M}} = \langle \bold{nrows}({\sf Mask}), 
            \bold{ncols}({\sf Mask}), \{(i,j) : (i,j) \in \bold{ind}({\sf Mask}) \} \rangle$,
            \item Otherwise, $\matrix{\widetilde{M}} = \langle \bold{nrows}({\sf Mask}), 
            \bold{ncols}({\sf Mask}), \\ \{(i,j) : (i,j) \in \bold{ind}({\sf Mask}) \wedge 
            ({\sf bool}){\sf Mask}(i,j) = \true\} \rangle$.
        \end{enumerate}

        \item    If ${\sf desc[GrB\_MASK].GrB\_COMP}$ is set, then 
        $\matrix{\widetilde{M}} \leftarrow \neg \matrix{\widetilde{M}}$.
    \end{enumerate}

    \item Matrix $\matrix{\widetilde{A}}$ is computed from argument {\sf A} as 
    follows:  $\matrix{\widetilde{A}}\leftarrow {\sf desc[GrB\_INP0].GrB\_TRAN} \ ? \ {\sf A}^T : {\sf A}$

{\color{red}
    \item Scalar $\widetilde{\sf s} \leftarrow {\sf s}$ ({\sf GrB\_Scalar} version only).
}
\end{enumerate}

The internal matrices and mask are checked for dimension compatibility. 
The following conditions must hold:
\begin{enumerate}
    \item $\bold{nrows}(\matrix{\widetilde{C}}) = \bold{nrows}(\matrix{\widetilde{M}})$.

    \item $\bold{ncols}(\matrix{\widetilde{C}}) = \bold{ncols}(\matrix{\widetilde{M}})$.

    \item $\bold{nrows}(\matrix{\widetilde{C}}) = \bold{nrows}(\matrix{\widetilde{A}})$.

    \item $\bold{ncols}(\matrix{\widetilde{C}}) = \bold{ncols}(\matrix{\widetilde{A}})$.
\end{enumerate}
If any compatibility rule above is violated, execution of {\sf GrB\_select} ends and 
the dimension mismatch error listed above is returned.

From this point forward, in {\sf GrB\_NONBLOCKING} mode, the method can optionally exit
with {\sf GrB\_SUCCESS} return code and defer any computation and/or execution error codes.

{\color{red}
If an empty {\sf GrB\_Scalar} $\widetilde{\sf s}$ is provided (\ie, $\mathbf{nvals}(\widetilde{\sf s}) = 0$),
the method returns with code {\sf GrB\_EMPTY\_OBJECT}. If a non-empty {\sf GrB\_Scalar}, 
$\widetilde{\sf s}$, is provided (\ie, $\mathbf{nvals}(\widetilde{\sf s}) = 1$), we then create an 
internal variable {\sf val} with the same domain as $\widetilde{\sf s}$ and set 
${\sf val} = \mathbf{val}(\widetilde{\sf s})$.
}

We are now ready to carry out the {\sf select} and any additional 
associated operations.  We describe this in terms of two intermediate matrices:
\begin{itemize}
    \item $\matrix{\widetilde{T}}$: The matrix holding the result from applying the index unary operator to the input matrix
    $\matrix{\widetilde{A}}$.

    \item $\matrix{\widetilde{Z}}$: The matrix holding the result after 
    application of the (optional) accumulation operator.
\end{itemize}

{\color{red}
The intermediate matrix, $\matrix{\widetilde{T}}$, is created as follows:
\[
\begin{aligned}
\matrix{\widetilde{T}} = \langle & \bold{D}({\sf A}),
                           \bold{nrows}(\matrix{\widetilde{A}}), 
                           \bold{ncols}(\matrix{\widetilde{A}}), \\
						  & \bold{L}(\matrix{\widetilde{T}}) =
    \{(i,j,\matrix{\widetilde{A}}(i,j) : {i,j} \in \bold{ind}(\matrix{\widetilde{A}}) 
	\wedge
({\sf bool})f_i(\matrix{\widetilde{A}}(i,j),i,j,{\sf val}) = \true \} \rangle, 
\end{aligned}
\]

where $f_i = \mathbf{f}({\sf op})$.
}


The intermediate matrix $\matrix{\widetilde{Z}}$ is created as follows, using what is called a \emph{standard matrix accumulate}:
\begin{itemize}
    \item If ${\sf accum} = {\sf GrB\_NULL}$, then $\matrix{\widetilde{Z}} = \matrix{\widetilde{T}}$.

    \item If ${\sf accum}$ is a binary operator, then $\matrix{\widetilde{Z}}$ is defined as
        \[ \langle \bDout({\sf accum}), \bold{nrows}(\matrix{\widetilde{C}}), \bold{ncols}(\matrix{\widetilde{C}}),
        %\bold{L}(\matrix{\widetilde{Z}}) =
        \{(i,j,Z_{ij})  \forall (i,j) \in \bold{ind}(\matrix{\widetilde{C}}) \cup 
        \bold{ind}(\matrix{\widetilde{T}}) \} \rangle.\]

    The values of the elements of $\matrix{\widetilde{Z}}$ are computed based on the
    relationships between the sets of indices in $\matrix{\widetilde{C}}$ and 
    $\matrix{\widetilde{T}}$.
\[
    Z_{ij} = \matrix{\widetilde{C}}(i,j) \odot \matrix{\widetilde{T}}(i,j), \ \mbox{if}\  
    (i,j) \in  (\bold{ind}(\matrix{\widetilde{T}}) \cap \bold{ind}(\matrix{\widetilde{C}})),
\]
\[
    Z_{ij} = \matrix{\widetilde{C}}(i,j), \ \mbox{if}\  
    (i,j) \in (\bold{ind}(\matrix{\widetilde{C}}) - (\bold{ind}(\matrix{\widetilde{T}})
    \cap \bold{ind}(\matrix{\widetilde{C}}))),
\]
\[
    Z_{ij} = \matrix{\widetilde{T}}(i,j), \ \mbox{if}\  (i,j) \in  
    (\bold{ind}(\matrix{\widetilde{T}}) - (\bold{ind}(\matrix{\widetilde{T}})
    \cap \bold{ind}(\matrix{\widetilde{C}}))),
\]
where $\odot  = \bigodot({\sf accum})$, and the difference operator refers to set difference.
\end{itemize}


%\emph{Standard Matrix Mask and Replace Options}

Finally, the set of output values that make up the $\matrix{\widetilde{Z}}$ 
matrix are written into the final result matrix, {\sf C}. 
This is carried out under control of the mask which acts as a ``write mask''.
\begin{itemize}
\item If {\sf desc[GrB\_OUTP].GrB\_REPLACE} is set, then any values in {\sf C} on
input to this operation are deleted and the contents of the new output matrix,
{\sf C}, is defined as,
\[
\bold{L}({\sf C}) = \{(i,j,Z_{ij}) : (i,j) \in (\bold{ind}(\matrix{\widetilde{Z}}) 
\cap \bold{ind}(\matrix{\widetilde{M}})) \}. 
\]

\item If {\sf desc[GrB\_OUTP].GrB\_REPLACE} is not set, the elements of 
$\matrix{\widetilde{Z}}$ indicated by the mask are copied into the result 
matrix, {\sf C}, and elements of {\sf C} that fall outside the set 
indicated by the mask are unchanged:
\[
\bold{L}({\sf C}) = \{(i,j,C_{ij}) : (i,j) \in (\bold{ind}({\sf C}) 
\cap \bold{ind}(\neg \matrix{\widetilde{M}})) \} \cup \{(i,j,Z_{ij}) : (i,j) \in 
(\bold{ind}(\matrix{\widetilde{Z}}) \cap \bold{ind}(\matrix{\widetilde{M}})) \}.
\]
\end{itemize}

In {\sf GrB\_BLOCKING} mode, the method exits with return value 
{\sf GrB\_SUCCESS} and the new content of matrix {\sf C} is as defined above
and fully computed.
In {\sf GrB\_NONBLOCKING} mode, the method exits with return value 
{\sf GrB\_SUCCESS} and the new content of matrix {\sf C} is as defined above
but may not be fully computed; however, it can be used in the next GraphBLAS 
method call in a sequence.

