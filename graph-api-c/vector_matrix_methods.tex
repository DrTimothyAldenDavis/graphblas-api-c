\subsection{Vector Methods}

%-----------------------------------------------------------------------------
\subsubsection{{\sf Vector\_new}: Create new vector}

Creates a new vector with specified domain and size.

\paragraph{\syntax}

\begin{verbatim}
        GrB_Info GrB_Vector_new(GrB_Vector *v,
                                GrB_Type    d,
                                GrB_Index   nsize);
\end{verbatim}

\paragraph{Parameters}

\begin{itemize}[leftmargin=1.1in]
    \item[{\sf v}] ({\sf INOUT}) On successful return, contains a handle
                                 to the newly created GraphBLAS vector.
    \item[{\sf d}] ({\sf IN})    The type corresponding to the domain of the 
                                 vector being created.  Can be one of the 
                                 predefined GraphBLAS types in 
                                 Table~\ref{Tab:PredefinedTypes}, or an existing 
                                 user-defined GraphBLAS type.
    \item[{\sf nsize}] ({\sf IN}) The size of the vector being created.
\end{itemize}

\paragraph{Return Values}

\begin{itemize}[leftmargin=2.1in]
    \item[{\sf GrB\_SUCCESS}]         In blocking mode, the operation completed
    successfully. In non-blocking mode, this indicates that the API checks 
    for the input arguments passed successfully. Either way, output vector 
    {\sf v} is ready to be used in the next method of the sequence.

    \item[{\sf GrB\_PANIC}]           Unknown internal error.
    
    \item[{\sf GrB\_INVALID\_OBJECT}] This is returned in any execution mode 
    whenever one of the opaque GraphBLAS objects (input or output) is in an invalid 
    state caused by a previous execution error.  Call {\sf GrB\_error()} to access 
    any error messages generated by the implementation.

    \item[{\sf GrB\_OUT\_OF\_MEMORY}] Not enough memory available for operation.
    
    \item[{\sf GrB\_UNINITIALIZED\_OBJECT}]  The {\sf GrB\_Type} object has not 
    been initialized by a call to {\sf GrB\_Type\_new} (needed for user-defined types).
    
    \item[{\sf GrB\_NULL\_POINTER}]  The {\sf v} pointer is {\sf NULL}.
    
    \item[{\sf GrB\_INVALID\_VALUE}] {\sf nsize} is zero.
\end{itemize}

\paragraph{Description}

Creates a new vector $\vector{v}$ of domain $\mathbf{D}({\sf d})$, size {\sf nsize}, 
and empty $\mathbf{L}(\vector{v})$. The method returns a handle to the new vector in {\sf v}.

It is not an error to call this method more than once on the same variable;  
however, the handle to the previously created object will be overwritten. 

%\scott{In the context of non-blocking can this operation be deferred?}
%\aydin{why not?}

%-----------------------------------------------------------------------------
\subsubsection{{\sf Vector\_dup}: Create a copy of a GraphBLAS vector}

Creates a new vector with the same domain, size, and contents as another vector.

\paragraph{\syntax}

\begin{verbatim}
        GrB_Info GrB_Vector_dup(GrB_Vector       *w,
                                const GrB_Vector  u);
\end{verbatim}

\paragraph{Parameters}

\begin{itemize}[leftmargin=1.1in]
    \item[{\sf w}]  ({\sf INOUT}) On successful return, contains a handle
                                  to the newly created GraphBLAS vector.
    \item[{\sf u}]  ({\sf IN})    The GraphBLAS vector to be duplicated.
\end{itemize}

\paragraph{Return Values}

\begin{itemize}[leftmargin=2.1in]
    \item[{\sf GrB\_SUCCESS}]         In blocking mode, the operation completed
    successfully. In non-blocking mode, this indicates that the API checks 
    for the input arguments passed successfully. Either way, output vector 
    {\sf w} is ready to be used in the next method of the sequence.

    \item[{\sf GrB\_PANIC}]           Unknown internal error.
    
    \item[{\sf GrB\_INVALID\_OBJECT}] This is returned in any execution mode 
    whenever one of the opaque GraphBLAS objects (input or output) is in an invalid 
    state caused by a previous execution error.  Call {\sf GrB\_error()} to access 
    any error messages generated by the implementation.

    \item[{\sf GrB\_OUT\_OF\_MEMORY}] Not enough memory available for operation.
    
    \item[{\sf GrB\_UNINITIALIZED\_OBJECT}]  The GraphBLAS vector, {\sf u}, has 
    not been initialized by a call to {\sf Vector\_new} or {\sf Vector\_dup}.
    
    \item[{\sf GrB\_NULL\_POINTER}]  The {\sf w} pointer is {\sf NULL}.
\end{itemize}

\paragraph{Description}

Creates a new vector $\vector{w}$ of domain $\mathbf{D}({\sf u})$, size 
$\mathbf{size}({\sf u})$, and contents $\mathbf{L}({\sf u})$. The method returns a 
handle to the new vector in {\sf w}.

It is not an error to call this method more than once on the same variable;  
however, the handle to the previously created object will be overwritten. 

%-----------------------------------------------------------------------------
\subsubsection{{\sf Vector\_clear}: Clear a vector}

Removes all the elements (tuples) from a vector.

\paragraph{\syntax}

\begin{verbatim}
        GrB_Info GrB_Vector_clear(GrB_Vector v);
\end{verbatim}

\paragraph{Parameters}

\begin{itemize}[leftmargin=1.1in]
    \item[{\sf v}] ({\sf INOUT}) An existing GraphBLAS vector to clear.
\end{itemize}

\paragraph{Return Values}

\begin{itemize}[leftmargin=2.1in]
    \item[{\sf GrB\_SUCCESS}]         In blocking mode, the operation completed
    successfully. In non-blocking mode, this indicates that the API checks 
    for the input arguments passed successfully. Either way, output vector 
    {\sf v} is ready to be used in the next method of the sequence.

    \item[{\sf GrB\_PANIC}]           Unknown internal error.
    
    \item[{\sf GrB\_INVALID\_OBJECT}] This is returned in any execution mode 
    whenever one of the opaque GraphBLAS objects (input or output) is in an invalid 
    state caused by a previous execution error.  Call {\sf GrB\_error()} to access 
    any error messages generated by the implementation.

    \item[{\sf GrB\_OUT\_OF\_MEMORY}] Not enough memory available for operation.
    
    \item[{\sf GrB\_UNINITIALIZED\_OBJECT}]  The GraphBLAS vector, {\sf v}, has 
    not been initialized by a call to {\sf Vector\_new} or {\sf Vector\_dup}.
    
\end{itemize}

\paragraph{Description}

Removes all elements (tuples) from an existing vector. After the call to
{\sf GrB\_Vector\_clear(v)}, 
$\mathbf{L}(\vector{v}) = \emptyset$. The size of the vector does not change. 


%-----------------------------------------------------------------------------
\subsubsection{{\sf Vector\_type}: scalar type of a vector\scott{PLEASE REVIEW}}

Creates a copy of the GraphBLAS type that was specified in the the creation of
the specified vector.

\paragraph{\syntax}

\begin{verbatim}
        GrB_Info GrB_Vector_type(GrB_Type         *type,
                                 const GrB_Vector  v);
\end{verbatim}

\paragraph{Parameters}

\begin{itemize}[leftmargin=1.1in]
    \item[{\sf type}] ({\sf OUT}) On successful return, contains a handle
                                  to the newly created GraphBLAS type.
    \item[{\sf v}]    ({\sf IN})  The GraphBLAS vector to be queried.
\end{itemize}

\paragraph{Return Values}

\begin{itemize}[leftmargin=2.1in]
    \item[{\sf GrB\_SUCCESS}]         In blocking mode, the operation completed
    successfully. In non-blocking mode, this indicates that the API checks 
    for the input arguments passed successfully. Either way, the output,
    {\sf type}, is ready to be used in the next method of the sequence.

    \item[{\sf GrB\_PANIC}]           Unknown internal error.
    
    \item[{\sf GrB\_INVALID\_OBJECT}] This is returned in any execution mode 
    if the input vector, {\sf v}, is in an invalid 
    state caused by a previous execution error.  Call {\sf GrB\_error()} to access 
    any error messages generated by the implementation.

    \item[{\sf GrB\_OUT\_OF\_MEMORY}] Not enough memory available for operation.
    
    \item[{\sf GrB\_UNINITIALIZED\_OBJECT}]  The GraphBLAS vector, {\sf v}, has 
    not been initialized by a call to {\sf Vector\_new} or {\sf Vector\_dup}.
    
    \item[{\sf GrB\_NULL\_POINTER}]  The {\sf type} pointer is {\sf NULL}.
\end{itemize}

\paragraph{Description}

This method creates a new {\sf GrB\_Type} object that is a copy of the GraphBLAS 
type used in the creation of the provided vector.

%-----------------------------------------------------------------------------
\subsubsection{{\sf Vector\_size}: Size of a vector}

Retrieve the size of a vector.

\paragraph{\syntax}

\begin{verbatim}
        GrB_Info GrB_Vector_size(GrB_Index        *nsize,
                                 const GrB_Vector  v);
\end{verbatim}

\paragraph{Parameters}

\begin{itemize}[leftmargin=1.1in]
    \item[{\sf nsize}] ({\sf OUT}) On successful return, is set to the size 
                                   of the vector.
    \item[{\sf v}]     ({\sf IN})  An existing GraphBLAS vector being queried.
\end{itemize}

\paragraph{Return Values}

\begin{itemize}[leftmargin=2.1in]
    \item[{\sf GrB\_SUCCESS}]   In blocking or non-blocking mode, the operation 
    completed successfully and the value of {\sf nsize} has been set.

    \item[{\sf GrB\_PANIC}]     Unknown internal error.
    
    \item[{\sf GrB\_INVALID\_OBJECT}] This is returned in any execution mode 
    whenever one of the opaque GraphBLAS objects (input or output) is in an invalid 
    state caused by a previous execution error.  Call {\sf GrB\_error()} to access 
    any error messages generated by the implementation.

    \item[{\sf GrB\_UNINITIALIZED\_OBJECT}]  The GraphBLAS vector, {\sf v}, has 
    not been initialized by a call to {\sf Vector\_new} or {\sf Vector\_dup}.
    
    \item[{\sf GrB\_NULL\_POINTER}]  {\sf nsize} pointer is {\sf NULL}.
\end{itemize}

\paragraph{Description}

Return $\mathbf{size}({\sf v})$ in {\sf nsize}.

%-----------------------------------------------------------------------------
\subsubsection{{\sf Vector\_nvals}: Number of stored elements in a vector}
\label{Sec:Vector_nvals}

Retrieve the number of stored elements (tuples) in a vector.

\paragraph{\syntax}

\begin{verbatim}
        GrB_Info GrB_Vector_nvals(GrB_Index        *nvals,
                                  const GrB_Vector  v);
\end{verbatim}

\paragraph{Parameters}

\begin{itemize}[leftmargin=1.1in]
    \item[{\sf nvals}] ({\sf OUT}) On successful return, this is set to the number of 
                                   stored elements (tuples) in the vector.
    \item[{\sf v}]     ({\sf IN})  An existing GraphBLAS vector being queried.
\end{itemize}


\paragraph{Return Values}

\begin{itemize}[leftmargin=2.1in]
    \item[{\sf GrB\_SUCCESS}]  In blocking or non-blocking mode, the operation 
    completed successfully and the value of {\sf nvals} has been set. 

    \item[{\sf GrB\_PANIC}]    Unknown internal error.
    
    \item[{\sf GrB\_INVALID\_OBJECT}] This is returned in any execution mode 
    whenever one of the opaque GraphBLAS objects (input or output) is in an invalid 
    state caused by a previous execution error.  Call {\sf GrB\_error()} to access 
    any error messages generated by the implementation.

    \item[{\sf GrB\_OUT\_OF\_MEMORY}] Not enough memory available for operation.
    
    \item[{\sf GrB\_UNINITIALIZED\_OBJECT}]  The GraphBLAS vector, {\sf v}, has 
    not been initialized by a call to {\sf Vector\_new} or {\sf Vector\_dup}.
    
    \item[{\sf GrB\_NULL\_POINTER}]  The {\sf nvals} pointer is {\sf NULL}.
\end{itemize}

\paragraph{Description}


Return $\mathbf{nvals}({\sf v})$ in {\sf nvals}. This is the number of stored 
elements in vector {\sf v}, which is the size of $\mathbf{L}(\vector{v})$ (see 
Section~\ref{Sec:Vectors}).

%-----------------------------------------------------------------------------

\subsubsection{{\sf Vector\_build}: Store elements from tuples into a vector}
\label{Sec:Vector_build}

\paragraph{\syntax}

\begin{verbatim}
        GrB_Info GrB_Vector_build(GrB_Vector             w,
                                  const GrB_Index       *indices,
                                  const <type>          *values,
                                  GrB_Index              n,
                                  const GrB_BinaryOp     dup);
\end{verbatim}

\paragraph{Parameters}

\begin{itemize}[leftmargin=1.1in]
    \item[{\sf w}]       ({\sf INOUT}) An existing Vector object to store the result.
    \item[{\sf indices}] ({\sf IN}) Pointer to an array of indices. 
    \item[{\sf values}]  ({\sf IN}) Pointer to an array of scalars of a type that
                                     is compatible with the domain of vector {\sf w}.
    \item[{\sf n}]   ({\sf IN}) The number of entries contained in each array (the same for \arg{indices} and \arg{values}).
    \item[{\sf dup}]     ({\sf IN}) An associative and commutative binary operator to apply when duplicate values for
	    the same location are present in the input arrays. All three domains of {\sf dup} must be the same; hence
	    $dup=\langle D_{dup},D_{dup},D_{dup},\oplus \rangle$.
\end{itemize}

\paragraph{Return Values}

\begin{itemize}[leftmargin=2.1in]
    \item[{\sf GrB\_SUCCESS}]         In blocking mode, the operation completed
    successfully. In non-blocking mode, this indicates that the API checks 
    for the input arguments passed successfully. Either way, output vector 
    {\sf w} is ready to be used in the next method of the sequence.

    \item[{\sf GrB\_PANIC}]           Unknown internal error.
    
    \item[{\sf GrB\_INVALID\_OBJECT}] This is returned in any execution mode 
    whenever one of the opaque GraphBLAS objects (input or output) is in an invalid 
    state caused by a previous execution error.  Call {\sf GrB\_error()} to access 
    any error messages generated by the implementation.

    \item[{\sf GrB\_OUT\_OF\_MEMORY}] Not enough memory available for operation.
    
    \item[{\sf GrB\_UNINITIALIZED\_OBJECT}]  Either {\sf w} has not been 
    initialized by a call to {\sf by GrB\_Vector\_new} or 
    {\sf by GrB\_Vector\_dup}, or
    {\sf dup} has not been initialized by a call to {\sf by GrB\_BinaryOp\_new}.
    
    \item[{\sf GrB\_NULL\_POINTER}]  {\sf indices} or {\sf values} 
    pointer is {\sf NULL}.

    \item[{\sf GrB\_INDEX\_OUT\_OF\_BOUNDS}] A value in {\sf indices} is outside 
    the allowed range for {\sf w}.
    
	\item[{\sf GrB\_DOMAIN\_MISMATCH}]    Either the domains of the GraphBLAS 
    binary operator {\sf dup} are not all the same, or the domains of 
    {\sf values} and {\sf w} are incompatible with each other or $D_{dup}$.
	
	\item[{\sf GrB\_OUTPUT\_NOT\_EMPTY}]    Output vector {\sf w} already contains valid tuples (elements).
	In other words, {\sf GrB\_Vector\_nvals(C)} returns a positive value.
\end{itemize}

\paragraph{Description}

An internal vector  $\vector{\widetilde{w}} = \langle D_{dup},\mathbf{size}({\sf w}),\emptyset \rangle$ is created, which only differs from ${\sf w}$ in its domain.

Each tuple $\{ {\sf indices[k]}, {\sf values[k]}\}$, where $0\leq k < {\sf n}$, is a contribution to the output in the form of 

$$\vector{\widetilde{w}}({\sf indices[k]}) = (D_{dup})\, {\sf values[k]}.$$

If multiple values for the same location are present in the input arrays, the 
{\sf dup} binary operand is used to reduce them before assignment into $\vector{\widetilde{w}}$ as follows: 

\[
\vector{\widetilde{w}}_{i}
= \bigoplus_{k:\, {\sf indices[k]} = i}  (D_{dup})\, {\sf values[k]}
,\] 

where $\oplus$ is the {\sf dup} binary operator. Finally, the resulting 
$\vector{\widetilde{w}}$ is copied into ${\sf w}$ via typecasting its values to 
$\mathbf{D}({\sf w})$ if necessary.  If $\oplus$ is not associative or not 
commutative, the result is undefined.  

The nonopaque input arrays, {\sf indices} and {\sf values}, must be at least as
large as {\sf n}. 

It is an error to call this function on an output object with existing elements. In other words, 
{\sf GrB\_Vector\_nvals(w)} should evaluate to zero prior to calling this function.

After {\sf GrB\_Vector\_build} returns, it is safe for a programmer to 
modify or delete the arrays {\sf indices} or {\sf values}.


%-----------------------------------------------------------------------------
\subsubsection{{\sf Vector\_setElement}: Set a single element in a vector}

Set one element of a vector to a given value.

\paragraph{\syntax}

\begin{verbatim}
        GrB_Info GrB_Vector_setElement(GrB_Vector   w,
                                       <type>       val,
                                       GrB_Index    index);
\end{verbatim}

\paragraph{Parameters}

\begin{itemize}[leftmargin=1.1in]
    \item[{\sf w}]   ({\sf INOUT}) An existing GraphBLAS vector for which an 
    element is to be assigned.

    \item[{\sf val}]   ({\sf IN}) Scalar value to assign.  The type must
    be compatible with the domain of {\sf w}.

    \item[{\sf index}] ({\sf IN}) The location of the element to be assigned.
\end{itemize}

\paragraph{Return Values}

\begin{itemize}[leftmargin=2.1in]
    \item[{\sf GrB\_SUCCESS}]         In blocking mode, the operation completed
    successfully. In non-blocking mode, this indicates that the compatibility 
    tests on index/dimensions and domains for the input arguments passed successfully. 
    Either way, the output vector {\sf w} is ready to be used in the next method of 
    the sequence.

    \item[{\sf GrB\_PANIC}]   Unknown internal error.
    
    \item[{\sf GrB\_INVALID\_OBJECT}] This is returned in any execution mode 
    whenever one of the opaque GraphBLAS objects (input or output) is in an invalid 
    state caused by a previous execution error.  Call {\sf GrB\_error()} to access 
    any error messages generated by the implementation.

    \item[{\sf GrB\_OUT\_OF\_MEMORY}]  Not enough memory available for operation.
    
    \item[{\sf GrB\_UNINITIALIZED\_OBJECT}]  The GraphBLAS vector, {\sf w}, has 
    not been initialized by a call to {\sf Vector\_new} or {\sf Vector\_dup}.
    
    \item[{\sf GrB\_INVALID\_INDEX}]  {\sf index} specifies a location 
    that is outside the dimensions of {\sf w}.

    \item[{\sf GrB\_DOMAIN\_MISMATCH}]     The domains of {\sf w} and {\sf val}
    are incompatible.
\end{itemize}

\paragraph{Description}

First, the scalar and output vector are tested for domain compatibility as follows:
$\mathbf{D}({\sf val})$ must be compatible with $\mathbf{D}({\sf w})$. Two domains 
are compatible with each other if values from one domain can be cast to values 
in the other domain as per the rules of the C language. In particular, domains 
from Table~\ref{Tab:PredefinedTypes} are all compatible with each other. A domain 
from a user-defined type is only compatible with itself. If any compatibility 
rule above is violated, execution of {\sf GrB\_Vector\_setElement} ends and 
the domain mismatch error listed above is returned.

Then, the {\sf index} parameter is checked for a valid value where the following
condition must hold:
\[
	0\ \leq\ {\sf index}\ <\ \mathbf{size}({\sf w})
\]
If this condition is violated, execution of {\sf GrB\_Vector\_extractElement} 
ends and the invalid index error listed above is returned.

We are now ready to carry out the assignment {\sf val}; that is:
\[
    {\sf w}({\sf index}) = {\sf val}
\]
If a value existed at this location in {\sf w}, it will be overwritten; otherwise,
and new value is stored in {\sf w}.

In {\sf GrB\_BLOCKING} mode, the method exits with return value 
{\sf GrB\_SUCCESS} and the new contents of {\sf w} is as defined above
and fully computed.  
In {\sf GrB\_NONBLOCKING} mode, the method exits with return value 
{\sf GrB\_SUCCESS} and the new content of vector {\sf w} is as defined above 
but may not be fully computed; however, it can be used in the next GraphBLAS 
method call in a sequence.


%-----------------------------------------------------------------------------

\subsubsection{{\sf Vector\_extractElement}: Extract a single element from a vector.}
\label{Sec:extract_single_element_vec}

Extract one element of a vector into a scalar. 

\paragraph{\syntax}

\begin{verbatim}
        GrB_Info GrB_Vector_extractElement(<type>           *val,
                                           const GrB_Vector  u,
                                           GrB_Index         index); 
\end{verbatim}

\paragraph{Parameters}

\begin{itemize}[leftmargin=1in]
    \item[{\sf val}]   ({\sf INOUT}) Pointer to a scalar of type that is 
    compatible with the domain of vector {\sf w}. On successful return, this scalar 
    holds the result of the operation. Any previous value in {\sf val} is 
    overwritten.

    \item[{\sf u}]     ({\sf IN}) The GraphBLAS vector from which an element
    is extracted.
    
    \item[{\sf index}] ({\sf IN}) The location in {\sf u} to extract.
\end{itemize}

\paragraph{Return Values}

\begin{itemize}[leftmargin=2.1in]
    \item[{\sf GrB\_SUCCESS}]  In blocking or non-blocking mode, the operation 
    completed successfully. This indicates that the compatibility tests on 
    dimensions and domains for the input arguments passed successfully, and
    the output scalar, {\sf val}, has been computed and is ready to be used in 
    the next method of the sequence.

    \item[{\sf GrB\_PANIC}]   Unknown internal error.
    
    \item[{\sf GrB\_INVALID\_OBJECT}] This is returned in any execution mode 
    whenever one of the opaque GraphBLAS objects (input or output) is in an invalid 
    state caused by a previous execution error.  Call {\sf GrB\_error()} to access 
    any error messages generated by the implementation.

    \item[{\sf GrB\_OUT\_OF\_MEMORY}]  Not enough memory available for operation.
   % \scott{Is this error possible?}
    %\aydin{I think it might be possible. We don't know the internal "extract" algorithm so it might not be an "in-place" algorithm. Better safe than sorry}
    
    \item[{\sf GrB\_UNINITIALIZED\_OBJECT}]  The GraphBLAS vector, {\sf u}, has 
    not been initialized by a call to {\sf Vector\_new} or {\sf Vector\_dup}.
    
    \item[{\sf GrB\_NULL\_POINTER}]    {\sf val} pointer is {\sf NULL}.

    \item[{\sf GrB\_NO\_VALUE}]  There is no stored value at specified location.
    
    \item[{\sf GrB\_INVALID\_INDEX}]  {\sf index} specifies a location 
    that is outside the dimensions of {\sf w}.

    \item[{\sf GrB\_DOMAIN\_MISMATCH}]     The domains of the vector or scalar
    are incompatible.
\end{itemize}

\paragraph{Description}

First, the scalar and input vector are tested for domain compatibility as follows:
$\mathbf{D}({\sf val})$ must be compatible with $\mathbf{D}({\sf u})$. Two domains 
are compatible with each other if values from one domain can be cast to values 
in the other domain as per the rules of the C language. In particular, domains 
from Table~\ref{Tab:PredefinedTypes} are all compatible with each other. A domain 
from a user-defined type is only compatible with itself. If any compatibility 
rule above is violated, execution of {\sf GrB\_Vector\_extractElement} ends and 
the domain mismatch error listed above is returned.

Then, the {\sf index} parameter is checked for a valid value where the following
condition must hold:
\[
	0\ \leq\ {\sf index}\ <\ \mathbf{size}({\sf u})
\]
If this condition is violated, execution of {\sf GrB\_Vector\_extractElement} 
ends and the invalid index error listed above is returned.

We are now ready to carry out the extract into the output argument, {\sf val};  
that is:
\[
    {\sf val} = {\sf u}({\sf index})
\]
where the following condition must be true:
\[
    {\sf index} \in \mathbf{ind}({\sf u})
\]
If this condition is violated, execution of {\sf GrB\_Vector\_extractElement} 
ends and the "no value" error listed above is returned.

In both {\sf GrB\_BLOCKING} mode {\sf GrB\_NONBLOCKING} mode
if the method exits with return value {\sf GrB\_SUCCESS}, the  new 
contents of {\sf val} are as defined above.  In other words, the method
does not return until any operations required to fully compute 
the GraphBLAS vector {\sf u} have completed. 

In {\sf GrB\_NONBLOCKING} mode, if the return value is 
not {\sf GrB\_SUCCESS}, an error in a method occurring earlier in the sequence
may have occurred that prevents completion of the GraphBLAS vector {\sf u}.
The {\sf GrB\_error()} method should be called for additional information 
about these errors.


%-----------------------------------------------------------------------------

\subsubsection{{\sf Vector\_extractTuples}: Extract tuples from a vector}
\label{Sec:Vector_extractTuples}

Extract the contents of a GraphBLAS vector into non-opaque data structures.

\paragraph{\syntax}

\begin{verbatim}
        GrB_Info GrB_Vector_extractTuples(GrB_Index            *indices,
                                          <type>               *values,
                                          GrB_Index            *n, 
                                          const GrB_Vector      v);

\end{verbatim}

\begin{itemize}[leftmargin=1.1in]
    \item[{\sf indices}] ({\sf OUT}) Pointer to an array of indices that is
                        large enough to hold all of the stored values' indices.
    \item[{\sf values}] ({\sf OUT}) Pointer to an array of scalars of a type 
                        that is large enough to hold all of the stored values
                        whose type is compatible with $\mathbf{D}(\vector{v})$.
    \item[{\sf n}] ({\sf INOUT}) Pointer to a value indicating (on input) the number of
                        elements the {\sf values} and
                        {\sf indices} arrays can hold. Upon return, it will contain the
                        number of values written to the arrays.
    \item[{\sf v}]      ({\sf IN})  An existing GraphBLAS vector.
\end{itemize}

\paragraph{Return Values}

\begin{itemize}[leftmargin=2.1in]
    \item[{\sf GrB\_SUCCESS}]  In blocking or non-blocking mode, the operation 
    completed successfully. This indicates that the compatibility tests on 
    the input argument passed successfully, and the output arrays, {\sf indices}
    and {\sf values}, have been computed.

    \item[{\sf GrB\_PANIC}]   Unknown internal error.
    
    \item[{\sf GrB\_INVALID\_OBJECT}] This is returned in any execution mode 
    whenever one of the opaque GraphBLAS objects (input or output) is in an invalid 
    state caused by a previous execution error.  Call {\sf GrB\_error()} to access 
    any error messages generated by the implementation.

    \item[{\sf GrB\_OUT\_OF\_MEMORY}]  Not enough memory available for operation.
    %\scott{Is this error possible?}
    %\aydin{I think here it is really possible. We don't know the internal "extracttuples" algorithm so it might not be an "in-place" algorithm. Better safe than sorry}

    \item[{\sf GrB\_INSUFFICIENT\_SPACE}]  Not enough space in {\sf indices} and 
    {\sf values} (as indicated by the {\sf n} parameter) to hold all of the 
    tuples that will be extacted.
    
    \item[{\sf GrB\_UNINITIALIZED\_OBJECT}]  The GraphBLAS vector, {\sf v}, has 
    not been initialized by a call to {\sf Vector\_new} or {\sf Vector\_dup}.
    
    \item[{\sf GrB\_NULL\_POINTER}] {\sf indices}, {\sf values}, or {\sf n}
    pointer is {\sf NULL}.
     
    \item[{\sf GrB\_DOMAIN\_MISMATCH}] The domains of the {\sf v} vector or 
    {\sf values} array are incompatible with one another.
\end{itemize}


\paragraph{Description}


This method will extract all the tuples from the GraphBLAS vector {\sf v}.  
The values associated with those tuples are placed in the
{\sf values} array and the indices are placed in the {\sf indices} array. 
Both {\sf indices} and {\sf values} must be pre-allocated by the user to have enough
space to hold at least {\sf GrB\_Vector\_nvals(v)} elements before calling
this function. 

Upon return of this function, {\sf n} will be set to the number of values (and 
indices) copied.  Also, the entries of {\sf indices} are unique, but not 
necessarily sorted.  Each tuple $(i,v_i)$ in {\sf v} is unzipped and copied 
into a distinct $k$th location in output vectors:

$$ \{{\sf indices[k]}, {\sf values[k]}\} \leftarrow (i,v_i),$$

where $0 \leq k < {\sf GrB\_Vector\_nvals(v)}$. No gaps in
output vectors are allowed; that is, if {\sf indices[k]} and {\sf values[k]} 
exist upon return, so does
{\sf indices[j]} and {\sf values[j]} for all $j$ such that $0 \leq j < k$.

Note that if the value in {\sf n} on input is less than the number of values
contained in the vector {\sf v}, then a {\sf GrB\_INSUFFICIENT\_SPACE} error 
is returned because it is undefined which subset of values would
be extracted otherwise.

In both {\sf GrB\_BLOCKING} mode {\sf GrB\_NONBLOCKING} mode
if the method exits with return value {\sf GrB\_SUCCESS}, the  new 
contents of the arrays {\sf indices} and {\sf values} are as defined above.  In 
other words, the method does not return until any operations required to fully 
compute the GraphBLAS vector {\sf v} have completed. 

In {\sf GrB\_NONBLOCKING} mode, if the return value is 
not {\sf GrB\_SUCCESS}, an error in a method occurring earlier in the sequence
may have occurred that prevents completion of the GraphBLAS vector {\sf v}.
The {\sf GrB\_error()} method should be called for additional information 
about these errors.



%==============================================================================
\subsection{Matrix Methods}

%-----------------------------------------------------------------------------
\subsubsection{{\sf Matrix\_new}: Create new matrix}

Creates a new matrix with specified domain and dimensions.

\paragraph{\syntax}

\begin{verbatim}
        GrB_Info GrB_Matrix_new(GrB_Matrix *A,
                                GrB_Type    d,
                                GrB_Index   nrows,
                                GrB_Index   ncols);
\end{verbatim}

\paragraph{Parameters}

\begin{itemize}[leftmargin=1.1in]
    \item[{\sf A}] ({\sf INOUT}) On successful return, contains a handle to 
                                 the newly created GraphBLAS matrix.
    \item[{\sf d}] ({\sf IN})    The type corresponding to the domain of the matrix 
                                 being created. Can be one of the predefined
                                 GraphBLAS types in Table~\ref{Tab:PredefinedTypes}, 
                                 or an existing user-defined GraphBLAS type.
    \item[{\sf nrows}] ({\sf IN}) The number of rows of the matrix being created.
    \item[{\sf ncols}] ({\sf IN}) The number of columns of the matrix being created.
\end{itemize}


\paragraph{Return Values}

\begin{itemize}[leftmargin=2.1in]
    \item[{\sf GrB\_SUCCESS}]         In blocking mode, the operation completed
    successfully. In non-blocking mode, this indicates that the API checks 
    for the input arguments passed successfully. Either way, output matrix 
    {\sf A} is ready to be used in the next method of the sequence.

    \item[{\sf GrB\_PANIC}]           Unknown internal error.
    
    \item[{\sf GrB\_INVALID\_OBJECT}] This is returned in any execution mode 
    whenever one of the opaque GraphBLAS objects (input or output) is in an invalid 
    state caused by a previous execution error.  Call {\sf GrB\_error()} to access 
    any error messages generated by the implementation.

    \item[{\sf GrB\_OUT\_OF\_MEMORY}] Not enough memory available for operation.
    
    \item[{\sf GrB\_UNINITIALIZED\_OBJECT}]  The {\sf GrB\_Type} object has not 
    been initialized by a call to {\sf GrB\_Type\_new} (needed for user-defined types).
    
    \item[{\sf GrB\_NULL\_POINTER}]  The {\sf A} pointer is {\sf NULL}.
    
    \item[{\sf GrB\_INVALID\_VALUE}] {\sf nrows} or {\sf ncols} is zero.
\end{itemize}

\paragraph{Description}

Creates a new matrix $\matrix{A}$ of domain $\mathbf{D}({\sf d})$, size 
{\sf nrows $\times$ ncols}, and empty $\mathbf{L}(\matrix{A})$. The method returns a
handle to the new matrix in {\sf A}.

It is not an error to call this method more than once on the same variable;  
however, the handle to the previously created object will be overwritten. 

%-----------------------------------------------------------------------------
\subsubsection{{\sf Matrix\_dup}: Create a copy of a GraphBLAS matrix}

Creates a new matrix with the same domain, dimensions, and contents as 
another matrix.

\paragraph{\syntax}

\begin{verbatim}
        GrB_Info GrB_Matrix_dup(GrB_Matrix       *C,
                                const GrB_Matrix  A);
\end{verbatim}

\paragraph{Parameters}

\begin{itemize}[leftmargin=1.1in]
    \item[{\sf C}] ({\sf INOUT}) On successful return, contains a handle to 
                                 the newly created GraphBLAS matrix.
    \item[{\sf A}] ({\sf IN})    The GraphBLAS matrix to be duplicated.
\end{itemize}


\paragraph{Return Values}

\begin{itemize}[leftmargin=2.1in]
    \item[{\sf GrB\_SUCCESS}]         In blocking mode, the operation completed
    successfully. In non-blocking mode, this indicates that the API checks 
    for the input arguments passed successfully. Either way, output matrix 
    {\sf C} is ready to be used in the next method of the sequence.

    \item[{\sf GrB\_PANIC}]           Unknown internal error.
    
    \item[{\sf GrB\_INVALID\_OBJECT}] This is returned in any execution mode 
    whenever one of the opaque GraphBLAS objects (input or output) is in an invalid 
    state caused by a previous execution error.  Call {\sf GrB\_error()} to access 
    any error messages generated by the implementation.

    \item[{\sf GrB\_OUT\_OF\_MEMORY}] Not enough memory available for operation.
    
    \item[{\sf GrB\_UNINITIALIZED\_OBJECT}]  The GraphBLAS matrix, {\sf A}, has 
    not been initialized by a call to {\sf Matrix\_new} or {\sf Matrix\_dup}.
    
    \item[{\sf GrB\_NULL\_POINTER}]   The {\sf C} pointer is {\sf NULL}.
\end{itemize}

\paragraph{Description}

Creates a new matrix $\matrix{C}$ of domain $\mathbf{D}({\sf A})$, size 
$\mathbf{nrows}({\sf A}) \times \mathbf{ncols}({\sf A})$, and contents 
$\mathbf{L}({\sf A})$. It returns a handle to it in {\sf C}.

It is not an error to call this method more than once on the same variable;  
however, the handle to the previously created object will be overwritten. 

%-----------------------------------------------------------------------------
\subsubsection{{\sf Matrix\_clear}: Clear a matrix}

Removes all elements (tuples) from a matrix.

\paragraph{\syntax}

\begin{verbatim}
        GrB_Info GrB_Matrix_clear(GrB_Matrix A);
\end{verbatim}

\paragraph{Parameters}

\begin{itemize}[leftmargin=1.1in]
    \item[{\sf A}] ({\sf IN}) An exising GraphBLAS matrix to clear.
\end{itemize}

\paragraph{Return Values}

\begin{itemize}[leftmargin=2.1in]
    \item[{\sf GrB\_SUCCESS}]         In blocking mode, the operation completed
    successfully. In non-blocking mode, this indicates that the API checks 
    for the input arguments passed successfully. Either way, output matrix 
    {\sf A} is ready to be used in the next method of the sequence.

    \item[{\sf GrB\_PANIC}]           Unknown internal error.
    
    \item[{\sf GrB\_INVALID\_OBJECT}] This is returned in any execution mode 
    whenever one of the opaque GraphBLAS objects (input or output) is in an invalid 
    state caused by a previous execution error.  Call {\sf GrB\_error()} to access 
    any error messages generated by the implementation.

    \item[{\sf GrB\_OUT\_OF\_MEMORY}] Not enough memory available for operation.
    
    \item[{\sf GrB\_UNINITIALIZED\_OBJECT}]  The GraphBLAS matrix, {\sf *A}, has 
    not been initialized by a call to {\sf Matrix\_new} or {\sf Matrix\_dup}.
    
\end{itemize}

\paragraph{Description}

Removes all elements (tuples) from an existing matrix. After the call to
{\sf GrB\_Matrix\_clear(A)},
$\mathbf{L}(\matrix{A}) = \emptyset$. The dimensions of the matrix do not change.

%-----------------------------------------------------------------------------
\subsubsection{{\sf Matrix\_type}: scalar type of a matrix\scott{PLEASE REVIEW}}

Creates a copy of the GraphBLAS type that was specified in the the creation of
the specified matrix.

\paragraph{\syntax}

\begin{verbatim}
        GrB_Info GrB_Matrix_type(GrB_Type         *type,
                                 const GrB_matrix  A);
\end{verbatim}

\paragraph{Parameters}

\begin{itemize}[leftmargin=1.1in]
    \item[{\sf type}] ({\sf OUT}) On successful return, contains a handle
                                  to the newly created GraphBLAS type.
    \item[{\sf A}]    ({\sf IN})  The GraphBLAS matrix to be queried.
\end{itemize}

\paragraph{Return Values}

\begin{itemize}[leftmargin=2.1in]
    \item[{\sf GrB\_SUCCESS}]         In blocking mode, the operation completed
    successfully. In non-blocking mode, this indicates that the API checks 
    for the input arguments passed successfully. Either way, the output,
    {\sf type}, is ready to be used in the next method of the sequence.

    \item[{\sf GrB\_PANIC}]           Unknown internal error.
    
    \item[{\sf GrB\_INVALID\_OBJECT}] This is returned in any execution mode 
    if the input matrix, {\sf A}, is in an invalid 
    state caused by a previous execution error.  Call {\sf GrB\_error()} to access 
    any error messages generated by the implementation.

    \item[{\sf GrB\_OUT\_OF\_MEMORY}] Not enough memory available for operation.
    
    \item[{\sf GrB\_UNINITIALIZED\_OBJECT}]  The GraphBLAS matrix, {\sf A}, has 
    not been initialized by a call to {\sf Matrix\_new} or {\sf Matrix\_dup}.
    
    \item[{\sf GrB\_NULL\_POINTER}]  The {\sf type} pointer is {\sf NULL}.
\end{itemize}

\paragraph{Description}

This method creates a new {\sf GrB\_Type} object that is a copy of the GraphBLAS
type used in the creation of the provided matrix.

%-----------------------------------------------------------------------------
\subsubsection{{\sf Matrix\_nrows}: Number of rows in a matrix}

Retrieve the number of rows in a matrix.

\paragraph{\syntax}

\begin{verbatim}
        GrB_Info GrB_Matrix_nrows(GrB_Index        *nrows,
                                  const GrB_Matrix  A);
\end{verbatim}

\paragraph{Parameters}

\begin{itemize}[leftmargin=1.1in]
    \item[{\sf nrows}] ({\sf OUT}) On successful return, contains the number of rows in the matrix.
    \item[{\sf A}] ({\sf IN}) An existing GraphBLAS matrix being queried.
\end{itemize}


\paragraph{Return Values}

\begin{itemize}[leftmargin=2.1in]
    \item[{\sf GrB\_SUCCESS}]   In blocking or non-blocking mode, the operation 
    completed successfully and the value of {\sf nrows} has been set.

    \item[{\sf GrB\_PANIC}]     Unknown internal error.
    
    \item[{\sf GrB\_INVALID\_OBJECT}] This is returned in any execution mode 
    whenever one of the opaque GraphBLAS objects (input or output) is in an invalid 
    state caused by a previous execution error.  Call {\sf GrB\_error()} to access 
    any error messages generated by the implementation.

    \item[{\sf GrB\_UNINITIALIZED\_OBJECT}]  The GraphBLAS matrix, {\sf A}, has 
    not been initialized by a call to {\sf Matrix\_new} or {\sf Matrix\_dup}.
    
    \item[{\sf GrB\_NULL\_POINTER}]  {\sf nrows} pointer is {\sf NULL}.
\end{itemize}

\paragraph{Description}

Return $\mathbf{nrows}({\sf A})$ in {\sf nrows} (the number of rows).

%-----------------------------------------------------------------------------
\subsubsection{{\sf Matrix\_ncols}: Number of columns in a matrix}

Retrieve the number of columns in a matrix.

\paragraph{\syntax}

\begin{verbatim}
        GrB_Info GrB_Matrix_ncols(GrB_Index        *ncols,
                                  const GrB_Matrix  A);
\end{verbatim}

\paragraph{Parameters}

\begin{itemize}[leftmargin=1.1in]
    \item[{\sf ncols}] ({\sf OUT}) On successful return, contains the number of columns in the matrix.
    \item[{\sf A}] ({\sf IN}) An existing GraphBLAS matrix being queried.
\end{itemize}

\paragraph{Return Values}

\begin{itemize}[leftmargin=2.1in]
    \item[{\sf GrB\_SUCCESS}]   In blocking or non-blocking mode, the operation 
    completed successfully and the value of {\sf ncols} has been set.

    \item[{\sf GrB\_PANIC}]     Unknown internal error.
    
    \item[{\sf GrB\_INVALID\_OBJECT}] This is returned in any execution mode 
    whenever one of the opaque GraphBLAS objects (input or output) is in an invalid 
    state caused by a previous execution error.  Call {\sf GrB\_error()} to access 
    any error messages generated by the implementation.

    \item[{\sf GrB\_UNINITIALIZED\_OBJECT}]  The GraphBLAS matrix, {\sf A}, has 
    not been initialized by a call to {\sf Matrix\_new} or {\sf Matrix\_dup}.
    
    \item[{\sf GrB\_NULL\_POINTER}]  {\sf ncols} pointer is {\sf NULL}.
\end{itemize}

\paragraph{Description}

Return $\mathbf{ncols}({\sf A})$ in {\sf ncols} (the number of columns).

%-----------------------------------------------------------------------------
\subsubsection{{\sf Matrix\_nvals}: Number of stored elements in a matrix}
\label{Sec:Matrix_nvals}

Retrieve the number of stored elements (tuples) in a matrix.

\paragraph{\syntax}

\begin{verbatim}
        GrB_Info GrB_Matrix_nvals(GrB_Index        *nvals,
                                  const GrB_Matrix  A);
\end{verbatim}

\paragraph{Parameters}

\begin{itemize}[leftmargin=1.1in]
    \item[{\sf nvals}] ({\sf OUT}) On successful return, contains the number of 
    stored elements (tuples) in the matrix.
    \item[{\sf A}] ({\sf IN}) An existing GraphBLAS matrix being queried.
\end{itemize}

\paragraph{Return Values}

\begin{itemize}[leftmargin=2.1in]
    \item[{\sf GrB\_SUCCESS}]  In blocking or non-blocking mode, the operation 
    completed successfully and the value of {\sf nvals} has been set. 

    \item[{\sf GrB\_PANIC}]    Unknown internal error.
    
    \item[{\sf GrB\_INVALID\_OBJECT}] This is returned in any execution mode 
    whenever one of the opaque GraphBLAS objects (input or output) is in an invalid 
    state caused by a previous execution error.  Call {\sf GrB\_error()} to access 
    any error messages generated by the implementation.

    \item[{\sf GrB\_OUT\_OF\_MEMORY}] Not enough memory available for operation.
    
    \item[{\sf GrB\_UNINITIALIZED\_OBJECT}]  The GraphBLAS matrix, {\sf A}, has 
    not been initialized by a call to {\sf Matrix\_new} or {\sf Matrix\_dup}.
    
    \item[{\sf GrB\_NULL\_POINTER}]  The {\sf nvals} pointer is {\sf NULL}.
\end{itemize}

\paragraph{Description}

Return $\mathbf{nvals}({\sf A})$ in {\sf nvals}.  This is the number of tuples 
stored in matrix {\sf A}, which is the size of $\mathbf{L}(\matrix{A})$
(see Section~\ref{Sec:Matrices}).

%-----------------------------------------------------------------------------

\subsubsection{{\sf Matrix\_build}: Store elements from tuples into a matrix}
\label{Sec:Matrix_build}

\paragraph{\syntax}

% AYDIN: Avoid page break due to preceding table
\begin{Verbatim}[samepage=true]    
        GrB_Info GrB_Matrix_build(GrB_Matrix             C,
                                  const GrB_Index       *row_indices,
                                  const GrB_Index       *col_indices, 
                                  const <type>          *values,
                                  GrB_Index              n,
                                  const GrB_BinaryOp     dup);
\end{Verbatim}

\paragraph{Parameters}

\begin{itemize}[leftmargin=1.1in]
    \item[{\sf C}]      ({\sf INOUT}) An existing Matrix object to store the result.
    \item[{\sf row\_indices}] ({\sf IN}) Pointer to an array of row indices. 
    \item[{\sf col\_indices}] ({\sf IN}) Pointer to an array of column indices. 
    \item[{\sf values}] ({\sf IN}) Pointer to an array of scalars of a type that
                                   is compatible with the domain of matrix, {\sf C}.
    \item[{\sf n}]  ({\sf IN}) The number of entries contained in each array (the same for \arg{row\_indices}, \arg{col\_indices}, and \arg{values}).
    \item[{\sf dup}]    ({\sf IN}) An associative and commutative binary function to apply when duplicate values 
                        for the same location are present in the input arrays.  
			All three domains of {\sf dup} must be the same; hence
	    $dup=\langle D_{dup},D_{dup},D_{dup},\oplus \rangle$.
\end{itemize}
                        %\scott{Is {\sf GrB\_NULL} allowed?} \aydin{No, it doesn't make sense}

\paragraph{Return Values}

\begin{itemize}[leftmargin=2.1in]
    \item[{\sf GrB\_SUCCESS}]         In blocking mode, the operation completed
    successfully. In non-blocking mode, this indicates that the API checks 
    for the input arguments passed successfully. Either way, output matrix 
    {\sf C} is ready to be used in the next method of the sequence.

    \item[{\sf GrB\_PANIC}]           Unknown internal error.
    
    \item[{\sf GrB\_INVALID\_OBJECT}] This is returned in any execution mode 
    whenever one of the opaque GraphBLAS objects (input or output) is in an invalid 
    state caused by a previous execution error.  Call {\sf GrB\_error()} to access 
    any error messages generated by the implementation.

    \item[{\sf GrB\_OUT\_OF\_MEMORY}] Not enough memory available for operation.
    
    \item[{\sf GrB\_UNINITIALIZED\_OBJECT}]  Either {\sf C} has not been 
    initialized by a call to {\sf by GrB\_Matrix\_new} or 
    {\sf by GrB\_Matrix\_dup}, or
    {\sf dup} has not been initialized by a call to {\sf by GrB\_BinaryOp\_new}.
    
    \item[{\sf GrB\_NULL\_POINTER}]  {\sf row\_indices}, 
    {\sf col\_indices} or {\sf values} pointer is {\sf NULL}.

    \item[{\sf GrB\_INDEX\_OUT\_OF\_BOUNDS}] A value in {\sf row\_indices} or
    {\sf col\_indices} is outside the allowed range for {\sf C}.

	\item[{\sf GrB\_DOMAIN\_MISMATCH}]    Either the domains of the GraphBLAS 
    binary operator {\sf dup} are not all the same, or the domains of 
    {\sf values} and {\sf C} are incompatible with each other or $D_{dup}$.
	
	\item[{\sf GrB\_OUTPUT\_NOT\_EMPTY}]    Output matrix {\sf C} already contains valid tuples (elements).
	In other words, {\sf GrB\_Matrix\_nvals(C)} returns a positive value.
\end{itemize}

\paragraph{Description}

An internal matrix $\matrix{\widetilde{C}} = \langle D_{dup},
    \mathbf{nrows}({\sf C}),
    \mathbf{ncols}({\sf C}),\emptyset \rangle$ is created, which only differs from ${\sf C}$ in its domain.

Each tuple $\{ {\sf row\_indices[k]}, {\sf col\_indices[k]}, {\sf values[k]}\}$, where $0\leq k < {\sf n}$, is a contribution to the output in the form of 

$$\matrix{\widetilde{C}}({\sf row\_indices[k]}, {\sf col\_indices[k]}) =  (D_{dup})\, {\sf values[k]}.$$

If multiple values for the same location are present in the input arrays, the 
{\sf dup} binary operand is used to reduce them before assignment into $\matrix{\widetilde{C}}$ as follows:

\[
\matrix{\widetilde{C}}_{ij}
= \bigoplus_{k:\, {\sf row\_indices[k]} = i\, \land\, {\sf col\_indices[k]} = j}   (D_{dup})\,{\sf values[k]}
,\] 

where $\oplus$ is the {\sf dup} binary operator. Finally, the resulting 
$\matrix{\widetilde{C}}$ is copied into ${\sf C}$ via typecasting its values to 
$\mathbf{D}({\sf C})$ if necessary.  If $\oplus$ is not associative or not 
commutative, the result is undefined.  

The nonopaque input arrays {\sf row\_indices}, {\sf col\_indices}, and {\sf values} must be at least as large as {\sf n}. 

It is an error to call this function on an output object with existing elements. In other words, 
{\sf GrB\_Matrix\_nvals(C)} should evaluate to zero prior to calling this function.

After {\sf GrB\_Matrix\_build} returns, it is safe for a programmer to 
modify or delete the arrays {\sf row\_indices}, {\sf col\_indices}, or {\sf values}.



%-----------------------------------------------------------------------------
\subsubsection{{\sf Matrix\_setElement}: Set a single element in matrix}

Set one element of a matrix to a given value.

\paragraph{\syntax}

\begin{verbatim}
        GrB_Info GrB_Matrix_setElement(GrB_Matrix   C,
                                       <type>       val,
                                       GrB_Index    row_index,
                                       GrB_Index    col_index); 
\end{verbatim}

\paragraph{Parameters}

\begin{itemize}[leftmargin=1.1in]
    \item[{\sf C}]   ({\sf INOUT}) An existing GraphBLAS matrix for which an 
    element is to be assigned.

    \item[{\sf val}]   ({\sf IN})  Scalar value to assign.  The type must
    be compatible with the domain of {\sf C}.
    
    \item[{\sf row\_index}] ({\sf IN}) Row index of element to be assigned
    \item[{\sf col\_index}] ({\sf IN}) Column index of element to be assigned
\end{itemize}

\paragraph{Return Values}

\begin{itemize}[leftmargin=2.1in]
    \item[{\sf GrB\_SUCCESS}]         In blocking mode, the operation completed
    successfully. In non-blocking mode, this indicates that the compatibility 
    tests on index/dimensions and domains for the input arguments passed successfully. 
    Either way, the output matrix {\sf C} is ready to be used in the next method of 
    the sequence.

    \item[{\sf GrB\_PANIC}]   Unknown internal error.
    
    \item[{\sf GrB\_INVALID\_OBJECT}] This is returned in any execution mode 
    whenever one of the opaque GraphBLAS objects (input or output) is in an invalid 
    state caused by a previous execution error.  Call {\sf GrB\_error()} to access 
    any error messages generated by the implementation.

    \item[{\sf GrB\_OUT\_OF\_MEMORY}]  Not enough memory available for operation.
    
    \item[{\sf GrB\_UNINITIALIZED\_OBJECT}]  The GraphBLAS matrix, {\sf C}, has 
    not been initialized by a call to {\sf Matrix\_new} or {\sf Matrix\_dup}.

    \item[{\sf GrB\_INVALID\_INDEX}]  {\sf row\_index} or {\sf col\_index} is 
    outside the allowable range (i.e., not less than $\mathbf{nrows}({\sf C})$ or
    $\mathbf{ncols}({\sf C})$, respectively).

    \item[{\sf GrB\_DOMAIN\_MISMATCH}]     The domains of {\sf C} and {\sf val}
    are incompatible.
\end{itemize}

\paragraph{Description}

First, the scalar and output matrix are tested for domain compatibility as follows:  
$\mathbf{D}({\sf val})$ must be compatible with $\mathbf{D}({\sf C})$. Two domains 
are compatible with each other if values from one domain can be cast to values 
in the other domain as per the rules of the C language.  In particular, domains 
from Table~\ref{Tab:PredefinedTypes} are all compatible with each other. A domain 
from a user-defined type is only compatible with itself.  If any compatibility 
rule above is violated, execution of {\sf GrB\_Matrix\_extractElement} ends and
the domain mismatch error listed above is returned.

Then, both index parameters are checked for valid values where following
conditions must hold:
\[
\begin{aligned}
    0\ \leq\ {\sf row\_index} & \ <\ \mathbf{nrows}({\sf C}), \\
    0\ \leq\ {\sf col\_index} & \ <\ \mathbf{ncols}({\sf C})
\end{aligned}
\]
If either of these conditions is violated, execution of 
{\sf GrB\_Matrix\_extractElement} ends and the invalid 
index error listed above is returned. 

We are now ready to carry out the assignment of {\sf val}; that is,
\[
{\sf C}({\sf row\_index},{\sf col\_index}) = {\sf val} 
\]
If a value existed at this location in {\sf C}, it will be overwritten; otherwise,
and new value is stored in {\sf C}.

In {\sf GrB\_BLOCKING} mode, the method exits with return value 
{\sf GrB\_SUCCESS} and the new contents of {\sf C} is as defined above
and fully computed.  
In {\sf GrB\_NONBLOCKING} mode, the method exits with return value 
{\sf GrB\_SUCCESS} and the new content of vector {\sf C} is as defined above 
but may not be fully computed; however, it can be used in the next GraphBLAS 
method call in a sequence.


%-----------------------------------------------------------------------------

\subsubsection{{\sf Matrix\_extractElement}: Extract a single element from a matrix}
\label{Sec:extract_single_element_mat}

%\scott{Is OUTOFMEMORY error possible (perhaps only in non-blocking)? }
%\scott{still need to deal with notation for type of first parameter.}

Extract one element of a matrix into a scalar. 

\paragraph{\syntax}

\begin{verbatim}
        GrB_Info GrB_Matrix_extractElement(<type>           *val,
                                           const GrB_Matrix  A,
                                           GrB_Index         row_index,
                                           GrB_Index         col_index); 

\end{verbatim}

\paragraph{Parameters}

\begin{itemize}[leftmargin=1in]
    \item[{\sf val}]   ({\sf OUT}) Pointer to a scalar of type that is 
    compatible with the domain of matrix {\sf A}. On successful return, this scalar 
    holds the result of the operation.  Any previous value in {\sf val} is 
    overwritten.

    \item[{\sf A}]     ({\sf IN}) The GraphBLAS matrix from which an element is
    extracted.
    
    \item[{\sf row\_index}] ({\sf IN}) The row index of location in {\sf A} 
    to extract.

    \item[{\sf col\_index}] ({\sf IN}) The column index of location in {\sf A} 
    to extract.
\end{itemize}

\paragraph{Return Values}

\begin{itemize}[leftmargin=2.1in]
    \item[{\sf GrB\_SUCCESS}]  In blocking or non-blocking mode, the operation 
    completed successfully. This indicates that the compatibility tests on 
    dimensions and domains for the input arguments passed successfully, and
    the output scalar, {\sf val}, has been computed and is ready to be used in 
    the next method of the sequence.

    \item[{\sf GrB\_PANIC}]   Unknown internal error.
    
    \item[{\sf GrB\_INVALID\_OBJECT}] This is returned in any execution mode 
    whenever one of the opaque GraphBLAS objects (input or output) is in an invalid 
    state caused by a previous execution error.  Call {\sf GrB\_error()} to access 
    any error messages generated by the implementation.

    \item[{\sf GrB\_OUT\_OF\_MEMORY}]  Not enough memory available for operation.
    
    \item[{\sf GrB\_UNINITIALIZED\_OBJECT}]  The GraphBLAS matrix, {\sf A}, has 
    not been initialized by a call to {\sf Matrix\_new} or {\sf Matrix\_dup}.
    
    \item[{\sf GrB\_NULL\_POINTER}]    {\sf val} pointer is {\sf NULL}.

    \item[{\sf GrB\_NO\_VALUE}]  There is no stored value at specified location.
    
    \item[{\sf GrB\_INVALID\_INDEX}]  {\sf row\_index} or {\sf col\_index} is 
    outside the allowable range (i.e. less than zero or greater than or equal to  $\mathbf{nrows}({\sf A})$ or
    $\mathbf{ncols}({\sf A})$, respectively).

    \item[{\sf GrB\_DOMAIN\_MISMATCH}]     The domains of the matrix and scalar
    are incompatible.
\end{itemize}

\paragraph{Description}

First, the scalar and input matrix are tested for domain compatibility as follows:  
$\mathbf{D}({\sf val})$ must be compatible with $\mathbf{D}({\sf A})$. Two domains 
are compatible with each other if values from one domain can be cast to values 
in the other domain as per the rules of the C language.  In particular, domains 
from Table~\ref{Tab:PredefinedTypes} are all compatible with each other. A domain 
from a user-defined type is only compatible with itself.  If any compatibility 
rule above is violated, execution of {\sf GrB\_Matrix\_extractElement} ends and
the domain mismatch error listed above is returned.

Then, both index parameters are checked for valid values where following
conditions must hold:
\[
\begin{aligned}
    0\ \leq\ {\sf row\_index} & \ <\ \mathbf{nrows}({\sf A}), \\
    0\ \leq\ {\sf col\_index} & \ <\ \mathbf{ncols}({\sf A})
\end{aligned}
\]
If either of these conditions is violated, execution of 
{\sf GrB\_Matrix\_extractElement} ends and the invalid 
index error listed above is returned. 

We are now ready to carry out the extract into the output argument, {\sf val}; 
that is,
\[
{\sf val} = {\sf A}({\sf row\_index},{\sf col\_index})
\]
where the following condition must be true:
\[
    ({\sf row\_index},{\sf col\_index}) \ \in \ \mathbf{ind}({\sf A})
\]
If this condition is violated, execution of {\sf GrB\_Matrix\_extractElement} 
ends and the "no value" error listed above is returned.


In both {\sf GrB\_BLOCKING} mode {\sf GrB\_NONBLOCKING} mode
if the method exits with return value {\sf GrB\_SUCCESS}, the  new 
contents of  {\sf val} are as defined above.  In other words, the method
does not return until any operations required to fully compute 
the GraphBLAS matrix {\sf A} have completed. 

In {\sf GrB\_NONBLOCKING} mode, if the return value is 
other than  {\sf GrB\_SUCCESS}, an error in a method occurring earlier in the sequence
may have occurred that prevents completion of the GraphBLAS matrix {\sf A}.
The {\sf GrB\_error()} method should be called for additional information 
about such errors.


%-----------------------------------------------------------------------------

\subsubsection{{\sf Matrix\_extractTuples}: Extract tuples from a matrix}
\label{Sec:Matrix_extractTuples}

Extract the contents of a GraphBLAS matrix into non-opaque data structures.

\paragraph{\syntax}

\begin{verbatim}
        GrB_Info GrB_Matrix_extractTuples(GrB_Index            *row_indices,
                                          GrB_Index            *col_indices,
                                          <type>               *values, 
                                          GrB_Index            *n, 
                                          const GrB_Matrix      A);
\end{verbatim}

\paragraph{Parameters}

\begin{itemize}[leftmargin=1.1in]
    \item[{\sf row\_indices}] ({\sf OUT}) Pointer to an array of row indices
                        that is large enough to hold all of the row indices.
    \item[{\sf col\_indices}] ({\sf OUT}) Pointer to an array of column indices
                        that is large enough to hold all of the column indices. 
    \item[{\sf values}] ({\sf OUT}) Pointer to an array of scalars of a type
                        that is large enough to hold all of the stored values whose
                        type is compatible with $\mathbf{D}(\matrix{A})$.
    \item[{\sf n}] ({\sf INOUT}) Pointer to a value indicating (in input) the number of
                        elements the {\sf values}, {\sf row\_indices}, and
                        {\sf col\_indices} arrays can hold. Upon return, it will contain the
                        number of values written to the arrays.
    \item[{\sf A}]      ({\sf IN}) An existing GraphBLAS matrix.
\end{itemize}

\paragraph{Return Values}

\begin{itemize}[leftmargin=2.1in]
    \item[{\sf GrB\_SUCCESS}]  In blocking or non-blocking mode, the operation 
    completed successfully. This indicates that the compatibility tests on 
    the input argument passed successfully, and the output arrays, {\sf indices}
    and {\sf values}, have been computed.

    \item[{\sf GrB\_PANIC}]   Unknown internal error.
    
    \item[{\sf GrB\_INVALID\_OBJECT}] This is returned in any execution mode 
    whenever one of the opaque GraphBLAS objects (input or output) is in an invalid 
    state caused by a previous execution error.  Call {\sf GrB\_error()} to access 
    any error messages generated by the implementation.

    \item[{\sf GrB\_OUT\_OF\_MEMORY}]  Not enough memory available for operation.
	 %   \scott{Is this error possible?} \jose{out-of-memory is always possible -- we don't restrict implementation}

    \item[{\sf GrB\_INSUFFICIENT\_SPACE}]  Not enough space in {\sf row\_indices}, 
    {\sf col\_indices}, and {\sf values} (as indicated by the {\sf n} parameter) 
    to hold all of the tuples that will be extacted.
    
    \item[{\sf GrB\_UNINITIALIZED\_OBJECT}]  The GraphBLAS matrix, {\sf A}, has 
    not been initialized by a call to {\sf Matrix\_new} or {\sf Matrix\_dup}.
    
    \item[{\sf GrB\_NULL\_POINTER}]  {\sf row\_indices}, {\sf col\_indices}, 
    {\sf values} or {\sf n} pointer is {\sf NULL}.
    
    \item[\sf GrB\_DOMAIN\_MISMATCH] The domains of the {\sf A} matrix and 
    {\sf values} array are incompatible with one another.
\end{itemize}

\paragraph{Description}


This method will extract all the tuples from the GraphBLAS matrix {\sf A}.  
The values associated with those tuples are placed in the
{\sf values} array, the column indices are placed in the {\sf col\_indices} array, 
and the row indices are placed in the {\sf row\_indices} array. 
These output arrays are pre-allocated by the user before calling
this function such that each output array has enough
space to hold at least {\sf GrB\_Matrix\_nvals(A)} elements. 

Upon return of this function, a pair of $\{{\sf row\_indices[k], col\_indices[k]}\}$ are unique for every valid $k$, 
but they are not required to be sorted in any particular order.
Each tuple $(i,j,A_{ij})$ in {\sf A} is unzipped and copied into a distinct $k$th location in output vectors:  

$$\{{\sf row\_indices[k]}, {\sf col\_indices[k]}, {\sf values[k]}\} \leftarrow (i,j,A_{ij}),$$

where $0 \leq k < {\sf GrB\_Matrix\_nvals(v)}$. 
No gaps in output vectors are allowed; that is, if {\sf row\_indices[k]},  {\sf col\_indices[k]}  and {\sf values[k]} exist upon return, 
so does {\sf row\_indices[j]}, {\sf col\_indices[j]} and {\sf values[j]} for all $j$ such that $0 \leq j < k$.

Note that if the value in {\sf n} on input is less than the number of values
contained in the matrix {\sf A}, then a {\sf GrB\_INSUFFICIENT\_SPACE} error 
is returned since it is undefined which subset of values would
be extracted.

In both {\sf GrB\_BLOCKING} mode {\sf GrB\_NONBLOCKING} mode
if the method exits with return value {\sf GrB\_SUCCESS}, the  new 
contents of the arrays {\sf row\_indices}, {\sf col\_indices} and {\sf values} are as defined above.  In other words, the method
does not return until any operations required to fully compute 
the GraphBLAS vector {\sf A} have completed. 

In {\sf GrB\_NONBLOCKING} mode, if the return value is 
not {\sf GrB\_SUCCESS}, an error in a method occurring earlier in the sequence
may have occurred that prevents completion of the GraphBLAS vector {\sf A}.
The {\sf GrB\_error()} method should be called for additional information 
about these errors.


