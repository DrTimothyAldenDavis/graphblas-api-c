\subsection{Vector Methods}

%-----------------------------------------------------------------------------
\subsubsection{{\sf Vector\_new}: Create new vector}

Creates a new vector with specified domain and size.

\paragraph{\syntax}

\begin{verbatim}
        GrB_Info GrB_Vector_new(GrB_Vector *v,
                                GrB_Type    d,
                                GrB_Index   nsize);
\end{verbatim}

\paragraph{Parameters}

\begin{itemize}[leftmargin=1.1in]
    \item[{\sf v}] ({\sf INOUT}) On successful return, contains a handle
                                 to the newly created GraphBLAS vector.
    \item[{\sf d}] ({\sf IN})    The type corresponding to the domain of the 
                                 vector being created.  Can be one of the 
                                 predefined GraphBLAS types in 
                                 Table~\ref{Tab:PredefinedTypes}, or an existing 
                                 user-defined GraphBLAS type.
    \item[{\sf nsize}] ({\sf IN}) The size of the vector being created.
\end{itemize}

\paragraph{Return Values}

\begin{itemize}[leftmargin=2.1in]
    \item[{\sf GrB\_SUCCESS}]         In blocking mode, operation completed
    successfully. In non-blocking mode, this indicates that the API checks 
    for the input arguments passed successfully. Either way, output vector 
    {\sf v} is ready to be used in the next method of the sequence.

    \item[{\sf GrB\_PANIC}]           Unknown internal error..
    
    \item[{\sf GrB\_INVALID\_OBJECT}] This is returned in any execution mode 
    whenever one of the opaque GraphBLAS objects (input or output) is in an invalid 
    state caused by a previous execution error.  Call {GrB\_error()} to access 
    any error messages generated by the implementation.

    \item[{\sf GrB\_OUT\_OF\_MEMORY}] Not enough memory available for operation.
    
    \item[{\sf GrB\_UNINITIALIZED\_OBJECT}]  The {\sf GrB\_Type} object has not 
    been initialized by a call to {\sf new} (needed for user-defined types).
    
    \item[{\sf GrB\_NULL\_POINTER}]  The {\sf v} pointer is {\sf NULL}.
    
    \item[{\sf GrB\_INVALID\_VALUE}] {\sf nsize} is zero
\end{itemize}

\paragraph{Description}

Creates a new vector $\vector{v}$ of domain $\bold{D}({\sf d})$, size {\sf nsize}, 
and empty $\bold{L}(\vector{v})$. It returns a handle to it in {\sf v}.

It is not an error to call this method more than once on the same variable;  
however, the handle to the previously created objects will be overwritten. 

\scott{In the context of non-blocking can this operation be deferred?}

%-----------------------------------------------------------------------------
\subsubsection{{\sf Vector\_dup}: Create a copy of a GraphBLAS vector}

Creates a new vector with the same domain, size, and contents as another vector.

\paragraph{\syntax}

\begin{verbatim}
        GrB_Info GrB_Vector_dup(GrB_Vector       *w,
                                const GrB_Vector  u);
\end{verbatim}

\paragraph{Parameters}

\begin{itemize}[leftmargin=1.1in]
    \item[{\sf w}]  ({\sf INOUT}) On successful return, contains a handle
                                  to the newly created GraphBLAS vector.
    \item[{\sf u}]  ({\sf IN})    The GraphBLAS vector to be duplicated.
\end{itemize}

\paragraph{Return Values}

\begin{itemize}[leftmargin=2.1in]
    \item[{\sf GrB\_SUCCESS}]         In blocking mode, operation completed
    successfully. In non-blocking mode, this indicates that the API checks 
    for the input arguments passed successfully. Either way, output vector 
    {\sf w} is ready to be used in the next method of the sequence.

    \item[{\sf GrB\_PANIC}]           Unknown internal error.
    
    \item[{\sf GrB\_INVALID\_OBJECT}] This is returned in any execution mode 
    whenever one of the opaque GraphBLAS objects (input or output) is in an invalid 
    state caused by a previous execution error.  Call {GrB\_error()} to access 
    any error messages generated by the implementation.

    \item[{\sf GrB\_OUT\_OF\_MEMORY}] Not enough memory available for operation.
    
    \item[{\sf GrB\_UNINITIALIZED\_OBJECT}]  The GraphBLAS vector, {\sf u}, has 
    not been initialized by a call to {\sf Vector\_new} or {\sf Vector\_dup}.
    
    \item[{\sf GrB\_NULL\_POINTER}]  The {\sf w} pointer is {\sf NULL}.
\end{itemize}

\paragraph{Description}

Creates a new vector $\vector{w}$ of domain $\bold{D}({\sf u})$, size 
$\bold{size}({\sf u})$, and contents $\bold{L}({\sf u})$. It returns a 
handle to it in {\sf w}.

It is not an error to call this method more than once on the same variable;  
however, the handle to the previously created objects will be overwritten. 

%-----------------------------------------------------------------------------
\subsubsection{{\sf Vector\_clear}: Clear a vector}

Removes all the elements from a vector.

\paragraph{\syntax}

\begin{verbatim}
        GrB_Info GrB_Vector_clear(GrB_Vector *v);
\end{verbatim}

\paragraph{Parameters}

\begin{itemize}[leftmargin=1.1in]
    \item[{\sf v}] ({\sf INOUT}) An existing GraphBLAS vector to clear.
\end{itemize}

\paragraph{Return Values}

\begin{itemize}[leftmargin=2.1in]
    \item[{\sf GrB\_SUCCESS}]         In blocking mode, operation completed
    successfully. In non-blocking mode, this indicates that the API checks 
    for the input arguments passed successfully. Either way, output vector 
    {\sf v} is ready to be used in the next method of the sequence.

    \item[{\sf GrB\_PANIC}]           Unknown internal error.
    
    \item[{\sf GrB\_INVALID\_OBJECT}] This is returned in any execution mode 
    whenever one of the opaque GraphBLAS objects (input or output) is in an invalid 
    state caused by a previous execution error.  Call {GrB\_error()} to access 
    any error messages generated by the implementation.

    \item[{\sf GrB\_OUT\_OF\_MEMORY}] Not enough memory available for operation.
    
    \item[{\sf GrB\_UNINITIALIZED\_OBJECT}]  The GraphBLAS vector, {\sf *v}, has 
    not been initialized by a call to {\sf Vector\_new} or {\sf Vector\_dup}.
    
    \item[{\sf GrB\_NULL\_POINTER}]  The {\sf v} pointer is {\sf NULL}.
\end{itemize}

\paragraph{Description}

Removes all tuples from an existing vector.

%-----------------------------------------------------------------------------
\subsubsection{{\sf Vector\_size}: Size of a vector}

Retrieve the size of a vector.

\paragraph{\syntax}

\begin{verbatim}
        GrB_Info GrB_Vector_size(GrB_Index        *nsize,
                                 const GrB_Vector  v);
\end{verbatim}

\paragraph{Parameters}

\begin{itemize}[leftmargin=1.1in]
    \item[{\sf nsize}] ({\sf OUT}) On successful return, is set to the size 
                                   of the vector.
    \item[{\sf v}]     ({\sf IN})  An existing GraphBLAS vector being queried.
\end{itemize}

\paragraph{Return Values}

\begin{itemize}[leftmargin=2.1in]
    \item[{\sf GrB\_SUCCESS}]   In blocking or non-blocking mode, the operation 
    completed successfully and the value of {\sf nsize} has been set.

    \item[{\sf GrB\_PANIC}]     Unknown internal error.
    
    \item[{\sf GrB\_INVALID\_OBJECT}] This is returned in any execution mode 
    whenever one of the opaque GraphBLAS objects (input or output) is in an invalid 
    state caused by a previous execution error.  Call {GrB\_error()} to access 
    any error messages generated by the implementation.

    \item[{\sf GrB\_UNINITIALIZED\_OBJECT}]  The GraphBLAS vector, {\sf v}, has 
    not been initialized by a call to {\sf Vector\_new} or {\sf Vector\_dup}.
    
    \item[{\sf GrB\_NULL\_POINTER}]  {\sf nsize} pointer is {\sf NULL}.
\end{itemize}

\paragraph{Description}

Return $\bold{size}({\sf v})$ in {\sf nsize}.

%-----------------------------------------------------------------------------
\subsubsection{{\sf Vector\_nvals}: Number of stored elements in a vector}
\label{Sec:Vector_nvals}

Retrieve the number of stored elements (tuples) in a vector.

\paragraph{\syntax}

\begin{verbatim}
        GrB_Info GrB_Vector_nvals(GrB_Index        *nvals,
                                  const GrB_Vector  v);
\end{verbatim}

\paragraph{Parameters}

\begin{itemize}[leftmargin=1.1in]
    \item[{\sf nvals}] ({\sf OUT}) On successful return, is set to the number of 
                                   stored elements (tuples) in the vector.
    \item[{\sf v}]     ({\sf IN})  An existing GraphBLAS vector being queried.
\end{itemize}


\paragraph{Return Values}

\begin{itemize}[leftmargin=2.1in]
    \item[{\sf GrB\_SUCCESS}]  In blocking or non-blocking mode, the operation 
    completed successfully and the value of {\sf nvals} has been set. 

    \item[{\sf GrB\_PANIC}]    Unknown internal error.
    
    \item[{\sf GrB\_INVALID\_OBJECT}] This is returned in any execution mode 
    whenever one of the opaque GraphBLAS objects (input or output) is in an invalid 
    state caused by a previous execution error.  Call {GrB\_error()} to access 
    any error messages generated by the implementation.

    \item[{\sf GrB\_OUT\_OF\_MEMORY}] Not enough memory available for operation.
    
    \item[{\sf GrB\_UNINITIALIZED\_OBJECT}]  The GraphBLAS vector, {\sf v}, has 
    not been initialized by a call to {\sf Vector\_new} or {\sf Vector\_dup}.
    
    \item[{\sf GrB\_NULL\_POINTER}]  The {\sf nvals} pointer is {\sf NULL}.
\end{itemize}

\paragraph{Description}


Return $\bold{nvals}({\sf v})$ in {\sf nvals}. This is the number of stored 
elements in vector {\sf v} (the size of $\bold{L}(\vector{v})$ in 
Section~\ref{Sec:Vectors}).

%-----------------------------------------------------------------------------

\subsubsection{{\sf Vector\_build}: Store elements from tuples into a vector}
\label{Sec:Vector_build}

\paragraph{\syntax}

\begin{verbatim}
        GrB_Info GrB_Vector_build(GrB_Vector            *w,
                                  const GrB_Index       *indices,
                                  const <type>          *values,
                                  GrB_Index              nvals,
                                  const GrB_BinaryOp     dup);
\end{verbatim}

\paragraph{Parameters}

\begin{itemize}[leftmargin=1.1in]
    \item[{\sf w}]       ({\sf INOUT}) An existing Vector object to store the result.
    \item[{\sf indices}] ({\sf IN}) Pointer to an array of indices. 
    \item[{\sf values}]  ({\sf IN}) Pointer to an array of scalars of a type that
                                     is compatible with the domain of vector {\sf w}.
    \item[{\sf nvals}]   ({\sf IN}) The number of entries contained in each array (the same for \arg{indices} and \arg{values}.
    \item[{\sf dup}]     ({\sf IN}) A binary function to apply when duplicate values for
                         the same location are present in the input arrays.
\end{itemize}

\paragraph{Return Values}

\begin{itemize}[leftmargin=2.1in]
    \item[{\sf GrB\_SUCCESS}]         In blocking mode, operation completed
    successfully. In non-blocking mode, this indicates that the API checks 
    for the input arguments passed successfully. Either way, output vector 
    {\sf w} is ready to be used in the next method of the sequence.

    \item[{\sf GrB\_PANIC}]           Unknown internal error.
    
    \item[{\sf GrB\_INVALID\_OBJECT}] This is returned in any execution mode 
    whenever one of the opaque GraphBLAS objects (input or output) is in an invalid 
    state caused by a previous execution error.  Call {GrB\_error()} to access 
    any error messages generated by the implementation.

    \item[{\sf GrB\_OUT\_OF\_MEMORY}] Not enough memory available for operation.
    
    \item[{\sf GrB\_UNINITIALIZED\_OBJECT}]  Either GraphBLAS object, {\sf w} or
    {\sf dup}, has not been initialized by a call to its respective {\sf new} (or
    {\sf Vector\_dup} for {\sf w}).
    
    \item[{\sf GrB\_NULL\_POINTER}]  {\sf w}, {\sf indices} or {\sf values} 
    pointer is {\sf NULL}.

    \item[{\sf GrB\_INDEX\_OUT\_OF\_BOUNDS}] A value in {\sf indices} is outside 
    the allowed range for \arg{w}.
    
	\item[{\sf GrB\_DOMAIN\_MISMATCH}]    The domains of {\sf values} and {\sf w}
	are incompatible with each other or binary operator ({\sf dup}).
\end{itemize}

\paragraph{Description}

\scott{Does this replace any existing content?}

\scott{Add statement that input arrays are available to be modified by the user
on return from this method even in non-blocking mode.}

For $i = 0,\ldots,\arg{nvals}-1$, do the following:
\begin{enumerate}
    \item If $\arg{indices}[i] \notin \bold{i}(\arg{w})$, then $\bold{L}(\arg{w}) \leftarrow \bold{L}(\arg{w}) \cup (\arg{indices}[i], \arg{values}[i])$.
    \item If $\arg{indices}[i] \in \bold{i}(\arg{w})$, then replace the tuple $(\arg{indices}[i], v_{\arg{indices}[i]}) \in \bold{L}(\arg{u})$ with the tuple \\ $(\arg{indices}[i], \arg{dup}(v_{\arg{indices}[i]},\arg{values}[i]))$.
\end{enumerate}

\scott{questionable {\sf dup} behaviour.}

\scott{I don't believe the following statement should be true?}

After a call to {\sf GrB\_Vector\_build}, the program should perform a 
{\sf GrB\_wait} on vector \arg{u} before
modifying or deleting arrays \arg{indices} and \arg{values}.


%-----------------------------------------------------------------------------
\subsubsection{{\sf Vector\_setElement}: Set a single element in a vector}

Set one element of a vector to a given value.

\paragraph{\syntax}

\begin{verbatim}
        GrB_Info GrB_Vector_setElement(GrB_Vector  *w,
                                       <type>       val,
                                       GrB_Index    index);
\end{verbatim}

\paragraph{Parameters}

\begin{itemize}[leftmargin=1.1in]
    \item[{\sf w}]   ({\sf INOUT}) An existing GraphBLAS vector for which an 
    element is to be assigned.

    \item[{\sf val}]   ({\sf IN}) Value to assign.  The type must
    be compatible with the domain of {\sf w}.

    \item[{\sf index}] ({\sf IN}) The location of the element to be assigned.
\end{itemize}

\paragraph{Return Values}

\begin{itemize}[leftmargin=2.1in]
    \item[{\sf GrB\_SUCCESS}]         In blocking mode, the operation completed
    successfully. In non-blocking mode, this indicates that the consistency 
    tests on index/dimensions and domains for the input arguments passed successfully. 
    Either way, output vector {\sf w} is ready to be used in the next method of 
    the sequence.

    \item[{\sf GrB\_PANIC}]   Unknown internal error.
    
    \item[{\sf GrB\_INVALID\_OBJECT}] This is returned in any execution mode 
    whenever one of the opaque GraphBLAS objects (input or output) is in an invalid 
    state caused by a previous execution error.  Call {GrB\_error()} to access 
    any error messages generated by the implementation.

    \item[{\sf GrB\_OUT\_OF\_MEMORY}]  Not enough memory available for operation.
    
    \item[{\sf GrB\_UNINITIALIZED\_OBJECT}]  The GraphBLAS vector, {\sf w}, has 
    not been initialized by a call to {\sf Vector\_new} or {\sf Vector\_dup}.
    
    \item[{\sf GrB\_NULL\_POINTER}]    {\sf w} pointer is {\sf NULL}.

    \item[{\sf GrB\_INVALID\_INDEX}]  {\sf index} specifies a location 
    that is outside the dimensions of {\sf w}.

    \item[{\sf GrB\_DOMAIN\_MISMATCH}]     The domains of the vector or scalar
    are incompatible.
\end{itemize}

\paragraph{Description}

First, the scalar and output vector are tested for domain consistency as follows:
$\bold{D}({\sf val})$ must be compatible with $\bold{D}({\sf w})$. Two domains 
are compatible with each other if values from one domain can be cast to values 
in the other domain as per the rules of the C language. In particular, domains 
from Table~\ref{Tab:PredefinedTypes} are all compatible with each other. A domain 
from a user-defined type is only compatible with itself. If any consistency 
rule above is violated, execution of {\sf GrB\_Vector\_setElement} ends and 
the domain mismatch error listed above is returned.

Then, the {\sf index} parameter is checked for a valid value where the following
condition must hold:
\[
	0\ \leq\ {\sf index}\ <\ \bold{size}({\sf w})
\]
If this condition is violated, execution of {\sf GrB\_Vector\_extractElement} 
ends and the invalid index error listed above is returned.

We are now ready to carry out the assignment {\sf val}; that is:
\[
    {\sf w}({\sf index}) = {\sf val}
\]
If a value existed at this location in {\sf w}, it will be overwritten; otherwise,
and new value is stored in {\sf w}.

In {\sf GrB\_BLOCKING} mode, the method exits with return value 
{\sf GrB\_SUCCESS} and the new contents of {\sf w} is as defined above
and fully computed.  
In {\sf GrB\_NONBLOCKING} mode, the method exits with return value 
{\sf GrB\_SUCCESS} and the new content of vector {\sf w} is as defined above 
but may not be fully computed; however, it can be used in the next GraphBLAS 
method call in a sequence.


%-----------------------------------------------------------------------------

\subsubsection{{\sf Vector\_extractElement}: Extract a single element from a vector.}
\label{Sec:extract_single_element_vec}

\scott{Is OUTOFMEMORY error possible (perhaps only in non-blocking)? }

\scott{still need to deal with notation for type of first parameter.}

Extract one element of a vector into a scalar. 

\paragraph{\syntax}

\begin{verbatim}
        GrB_Info GrB_Vector_extractElement(<type>           *val,
                                           const GrB_Vector  u,
                                           GrB_Index         index); 
\end{verbatim}

\paragraph{Parameters}

\begin{itemize}[leftmargin=1in]
    \item[{\sf val}]   ({\sf INOUT}) Pointer to a scalar. On successful return, 
    this scalar holds the result of the operation. Any previous value is 
    overwritten.

    \item[{\sf u}]     ({\sf IN}) The GraphBLAS vector from which an element
    is extracted.
    
    \item[{\sf index}] ({\sf IN}) The location in {\sf u} to extract.
\end{itemize}

\paragraph{Return Values}

\begin{itemize}[leftmargin=2.1in]
    \item[{\sf GrB\_SUCCESS}]  In blocking or non-blocking mode, the operation 
    completed successfully. This indicates that the consistency tests on 
    dimensions and domains for the input arguments passed successfully, and
    the output scalar, {\sf val}, has been computed and is ready to be used in 
    the next method of the sequence.

    \item[{\sf GrB\_PANIC}]   Unknown internal error.
    
    \item[{\sf GrB\_INVALID\_OBJECT}] This is returned in any execution mode 
    whenever one of the opaque GraphBLAS objects (input or output) is in an invalid 
    state caused by a previous execution error.  Call {GrB\_error()} to access 
    any error messages generated by the implementation.

    \item[{\sf GrB\_OUT\_OF\_MEMORY}]  Not enough memory available for operation.
    \scott{Is this error possible?}
    
    \item[{\sf GrB\_UNINITIALIZED\_OBJECT}]  The GraphBLAS vector, {\sf u}, has 
    not been initialized by a call to {\sf Vector\_new} or {\sf Vector\_dup}.
    
    \item[{\sf GrB\_NULL\_POINTER}]    {\sf val} pointer is {\sf NULL}.

    \item[{\sf GrB\_NO\_VALUE}]  There is no stored value at specified location.
    
    \item[{\sf GrB\_INVALID\_INDEX}]  {\sf index} is outside the allowable range
    (i.e., not less than $\bold{size}({\sf u})$).

    \item[{\sf GrB\_DOMAIN\_MISMATCH}]     The domains of the vector or scalar
    are incompatible.
\end{itemize}

\paragraph{Description}

First, the scalar and input vector are tested for domain consistency as follows:
$\bold{D}({\sf val})$ must be compatible with $\bold{D}({\sf u})$. Two domains 
are compatible with each other if values from one domain can be cast to values 
in the other domain as per the rules of the C language. In particular, domains 
from Table~\ref{Tab:PredefinedTypes} are all compatible with each other. A domain 
from a user-defined type is only compatible with itself. If any consistency 
rule above is violated, execution of {\sf GrB\_Vector\_extractElement} ends and 
the domain mismatch error listed above is returned.

Then, the {\sf index} parameter is checked for a valid value where the following
condition must hold:
\[
	0\ \leq\ {\sf index}\ <\ \bold{size}({\sf u})
\]
If this condition is violated, execution of {\sf GrB\_Vector\_extractElement} 
ends and the invalid index error listed above is returned.

We are now ready to carry out the extract into the output argument, {\sf val};  
that is:
\[
    {\sf val} = {\sf u}({\sf index})
\]
where the following condition must be true:
\[
    {\sf index} \in \bold{ind}({\sf u})
\]
If this condition is violated, execution of {\sf GrB\_Vector\_extractElement} 
ends and the "no value" error listed above is returned.

In {\sf GrB\_BLOCKING} mode, the method exits with return value 
{\sf GrB\_SUCCESS} and the new content of {\sf val} is as defined above
and fully computed.  
In {\sf GrB\_NONBLOCKING} mode, the method exits with return value 
{\sf GrB\_SUCCESS} and the new content of {\sf val} is as defined above 
and is also fully computed because this is a sequence terminating method.

\scott{Where and how do we say that this method is sequence terminating? E.g. 
This method is sequence terminating because we are extracting the contents
of an opaque data structure into a non-opaque data structure; therefore, we 
require that in non-blocking mode that enough computation be performed to 
compute the entire vector prior to returnin from this method.  
The GrB\_error() method should be called in non-blocking mode if more 
error information about 'upstream' operations.}

%-----------------------------------------------------------------------------

\subsubsection{{\sf Vector\_extractTuples}: Extract tuples from a vector}
\label{Sec:Vector_extractTuples}

Extract the contents of a GraphBLAS vector into non-opaque data structures.

\paragraph{\syntax}

\begin{verbatim}
        GrB_Info GrB_Vector_extractTuples(GrB_Index            *indices,
                                          <type>               *values, 
                                          const GrB_Vector      v);

\end{verbatim}

\begin{itemize}[leftmargin=1.1in]
    \item[{\sf indices}] ({\sf OUT}) Pointer to an array of indices that is sufficient to
                        hold all of the stored values' indices (no checking is performed).
    \item[{\sf values}] ({\sf OUT}) Pointer to an array of scalars of a type that is sufficient to
                        hold all of the stored values (no checking is performed) whose
                        type is compatible with $\bold{D}(\vector{v})$.
    \item[{\sf v}]      ({\sf IN})  An existing GraphBLAS vector.
\end{itemize}

\paragraph{Return Values}

\begin{itemize}[leftmargin=2.1in]
    \item[{\sf GrB\_SUCCESS}]  In blocking or non-blocking mode, the operation 
    completed successfully. This indicates that the consistency tests on 
    the input argument passed successfully, and the output arrays, {\sf indices}
    and {\sf values}, have been computed.

    \item[{\sf GrB\_PANIC}]   Unknown internal error.
    
    \item[{\sf GrB\_INVALID\_OBJECT}] This is returned in any execution mode 
    whenever one of the opaque GraphBLAS objects (input or output) is in an invalid 
    state caused by a previous execution error.  Call {GrB\_error()} to access 
    any error messages generated by the implementation.

    \item[{\sf GrB\_OUT\_OF\_MEMORY}]  Not enough memory available for operation.
    \scott{Is this error possible?}
    
    \item[{\sf GrB\_UNINITIALIZED\_OBJECT}]  The GraphBLAS vector, {\sf v}, has 
    not been initialized by a call to {\sf Vector\_new} or {\sf Vector\_dup}.
    
    \item[{\sf GrB\_NULL\_POINTER}] {\sf indices} or {\sf values} pointer is {\sf NULL}.
     
    \item[{\sf GrB\_DOMAIN\_MISMATCH}] The domains of the {\sf v} vector or 
    {\sf values} array are incompatible with one another.
\end{itemize}


\paragraph{Description}
\scott{DESCRIPTION MISSING}


\scott{Does allocation of non-opaque arrays occur within function -- then we need OUT\_OF\_MEMORY error.  As currently written, it is assumed that the user
is expected to call *\_nvals() function on the matrix to
determine how much memory to allocate and pass pointers to the pre 
allocated memory.  In this case NULL\_POINTER is returned (as 
shown above) if NULL pointers are passed.}

%==============================================================================
\subsection{Matrix Methods}

%-----------------------------------------------------------------------------
\subsubsection{{\sf Matrix\_new}: Create new matrix}

Creates a new matrix with specified domain and dimensions.

\paragraph{\syntax}

\begin{verbatim}
        GrB_Info GrB_Matrix_new(GrB_Matrix *A,
                                GrB_Type    d,
                                GrB_Index   nrows,
                                GrB_Index   ncols);
\end{verbatim}

\paragraph{Parameters}

\begin{itemize}[leftmargin=1.1in]
    \item[{\sf A}] ({\sf INOUT}) On successful return, contains a handle to 
                                 the newly created GraphBLAS matrix.
    \item[{\sf d}] ({\sf IN})    The type corresponding to the domain of the matrix 
                                 being created. Can be one of the predefined
                                 GraphBLAS types in Table~\ref{Tab:PredefinedTypes}, 
                                 or an existing user-defined GraphBLAS type.
    \item[{\sf nrows}] ({\sf IN}) The number of rows of the matrix being created.
    \item[{\sf ncols}] ({\sf IN}) The number of columns of the matrix being created.
\end{itemize}


\paragraph{Return Values}

\begin{itemize}[leftmargin=2.1in]
    \item[{\sf GrB\_SUCCESS}]         In blocking mode, operation completed
    successfully. In non-blocking mode, this indicates that the API checks 
    for the input arguments passed successfully. Either way, output matrix 
    {\sf A} is ready to be used in the next method of the sequence.

    \item[{\sf GrB\_PANIC}]           Unknown internal error.
    
    \item[{\sf GrB\_INVALID\_OBJECT}] This is returned in any execution mode 
    whenever one of the opaque GraphBLAS objects (input or output) is in an invalid 
    state caused by a previous execution error.  Call {GrB\_error()} to access 
    any error messages generated by the implementation.

    \item[{\sf GrB\_OUT\_OF\_MEMORY}] Not enough memory available for operation.
    
    \item[{\sf GrB\_UNINITIALIZED\_OBJECT}]  The {\sf GrB\_Type} object has not 
    been initialized by a call to {\sf new} (needed for user-defined types).
    
    \item[{\sf GrB\_NULL\_POINTER}]  The {\sf A} pointer is {\sf NULL}.
    
    \item[{\sf GrB\_INVALID\_VALUE}] {\sf nrows} or {\sf ncols} is zero.
\end{itemize}

\paragraph{Description}

Creates a new matrix $\matrix{A}$ of domain $\bold{D}({\sf d})$, size 
{\sf nrows $\times$ ncols}, and empty $\bold{L}(\matrix{A})$. It returns a
handle to it in {\sf A}.

It is not an error to call this method more than once on the same variable;  
however, the handle to the previously created objects will be overwritten. 

%-----------------------------------------------------------------------------
\subsubsection{{\sf Matrix\_dup}: Create a copy of a GraphBLAS matrix}

Creates a new matrix with the same domain, dimensions, and contents as 
another matrix.

\paragraph{\syntax}

\begin{verbatim}
        GrB_Info GrB_Matrix_dup(GrB_Matrix       *C,
                                const GrB_Matrix  A);
\end{verbatim}

\paragraph{Parameters}

\begin{itemize}[leftmargin=1.1in]
    \item[{\sf C}] ({\sf INOUT}) On successful return, contains a handle to 
                                 the newly created GraphBLAS matrix.
    \item[{\sf A}] ({\sf IN})    The GraphBLAS matrix to be duplicated.
\end{itemize}


\paragraph{Return Values}

\begin{itemize}[leftmargin=2.1in]
    \item[{\sf GrB\_SUCCESS}]         In blocking mode, operation completed
    successfully. In non-blocking mode, this indicates that the API checks 
    for the input arguments passed successfully. Either way, output matrix 
    {\sf C} is ready to be used in the next method of the sequence.

    \item[{\sf GrB\_PANIC}]           Unknown internal error.
    
    \item[{\sf GrB\_INVALID\_OBJECT}] This is returned in any execution mode 
    whenever one of the opaque GraphBLAS objects (input or output) is in an invalid 
    state caused by a previous execution error.  Call {GrB\_error()} to access 
    any error messages generated by the implementation.

    \item[{\sf GrB\_OUT\_OF\_MEMORY}] Not enough memory available for operation.
    
    \item[{\sf GrB\_UNINITIALIZED\_OBJECT}]  The GraphBLAS matrix, {\sf A}, has 
    not been initialized by a call to {\sf Matrix\_new} or {\sf Matrix\_dup}.
    
    \item[{\sf GrB\_NULL\_POINTER}]   The {\sf C} pointer is {\sf NULL}.
\end{itemize}

\paragraph{Description}

Creates a new matrix $\matrix{C}$ of domain $\bold{D}({\sf A})$, size 
$\bold{nrows}({\sf A}) \times \bold{ncols}({\sf A})$, and contents 
$\bold{L}({\sf A})$. It returns a handle to it in {\sf C}.

It is not an error to call this method more than once on the same variable;  
however, the handle to the previously created objects will be overwritten. 

%-----------------------------------------------------------------------------
\subsubsection{{\sf Matrix\_clear}: Clear a matrix}

Removes all elements from a matrix.

\paragraph{\syntax}

\begin{verbatim}
        GrB_Info GrB_Matrix_clear(GrB_Matrix *A);
\end{verbatim}

\paragraph{Parameters}

\begin{itemize}[leftmargin=1.1in]
    \item[{\sf A}] ({\sf IN}) An exising GraphBLAS matrix to clear.
\end{itemize}

\paragraph{Return Values}

\begin{itemize}[leftmargin=2.1in]
    \item[{\sf GrB\_SUCCESS}]         In blocking mode, operation completed
    successfully. In non-blocking mode, this indicates that the API checks 
    for the input arguments passed successfully. Either way, output matrix 
    {\sf A} is ready to be used in the next method of the sequence.

    \item[{\sf GrB\_PANIC}]           Unknown internal error.
    
    \item[{\sf GrB\_INVALID\_OBJECT}] This is returned in any execution mode 
    whenever one of the opaque GraphBLAS objects (input or output) is in an invalid 
    state caused by a previous execution error.  Call {GrB\_error()} to access 
    any error messages generated by the implementation.

    \item[{\sf GrB\_OUT\_OF\_MEMORY}] Not enough memory available for operation.
    
    \item[{\sf GrB\_UNINITIALIZED\_OBJECT}]  The GraphBLAS matrix, {\sf *A}, has 
    not been initialized by a call to {\sf Matrix\_new} or {\sf Matrix\_dup}.
    
    \item[{\sf GrB\_NULL\_POINTER}]  The {\sf A} pointer is {\sf NULL}.
\end{itemize}

\paragraph{Description}

Removes all elements (tuples) from an existing matrix.

%-----------------------------------------------------------------------------
\subsubsection{{\sf Matrix\_nrows}: Number of rows in a matrix}

Retrieve the number of rows in a matrix.

\paragraph{\syntax}

\begin{verbatim}
        GrB_Info GrB_Matrix_nrows(GrB_Index        *nrows,
                                  const GrB_Matrix  A);
\end{verbatim}

\paragraph{Parameters}

\begin{itemize}[leftmargin=1.1in]
    \item[{\sf nrows}] ({\sf OUT}) On successful return, contains the number of rows in the matrix.
    \item[{\sf A}] ({\sf IN}) An existing GraphBLAS matrix being queried.
\end{itemize}


\paragraph{Return Values}

\begin{itemize}[leftmargin=2.1in]
    \item[{\sf GrB\_SUCCESS}]   In blocking or non-blocking mode, the operation 
    completed successfully and the value of {\sf nrows} has been set.

    \item[{\sf GrB\_PANIC}]     Unknown internal error.
    
    \item[{\sf GrB\_INVALID\_OBJECT}] This is returned in any execution mode 
    whenever one of the opaque GraphBLAS objects (input or output) is in an invalid 
    state caused by a previous execution error.  Call {GrB\_error()} to access 
    any error messages generated by the implementation.

    \item[{\sf GrB\_UNINITIALIZED\_OBJECT}]  The GraphBLAS matrix, {\sf A}, has 
    not been initialized by a call to {\sf Matrix\_new} or {\sf Matrix\_dup}.
    
    \item[{\sf GrB\_NULL\_POINTER}]  {\sf nrows} pointer is {\sf NULL}.
\end{itemize}

\paragraph{Description}

Return $\bold{nrows}({\sf A})$ in {\sf nrows} (the number of rows).

%-----------------------------------------------------------------------------
\subsubsection{{\sf Matrix\_ncols}: Number of columns in a matrix}

Retrieve the number of columns in a matrix.

\paragraph{\syntax}

\begin{verbatim}
        GrB_Info GrB_Matrix_ncols(GrB_Index        *ncols,
                                  const GrB_Matrix  A);
\end{verbatim}

\paragraph{Parameters}

\begin{itemize}[leftmargin=1.1in]
    \item[{\sf ncols}] ({\sf OUT}) On successful return, contains the number of columns in the matrix.
    \item[{\sf A}] ({\sf IN}) An existing GraphBLAS matrix being queried.
\end{itemize}

\paragraph{Return Values}

\begin{itemize}[leftmargin=2.1in]
    \item[{\sf GrB\_SUCCESS}]   In blocking or non-blocking mode, the operation 
    completed successfully and the value of {\sf ncols} has been set.

    \item[{\sf GrB\_PANIC}]     Unknown internal error.
    
    \item[{\sf GrB\_INVALID\_OBJECT}] This is returned in any execution mode 
    whenever one of the opaque GraphBLAS objects (input or output) is in an invalid 
    state caused by a previous execution error.  Call {GrB\_error()} to access 
    any error messages generated by the implementation.

    \item[{\sf GrB\_UNINITIALIZED\_OBJECT}]  The GraphBLAS matrix, {\sf A}, has 
    not been initialized by a call to {\sf Matrix\_new} or {\sf Matrix\_dup}.
    
    \item[{\sf GrB\_NULL\_POINTER}]  {\sf ncols} pointer is {\sf NULL}.
\end{itemize}

\paragraph{Description}

Return $\bold{ncols}({\sf A})$ in {\sf ncols} (the number of columns).

%-----------------------------------------------------------------------------
\subsubsection{{\sf Matrix\_nvals}: Number of stored elements in a matrix}
\label{Sec:Matrix_nvals}

Retrieve the number of stored elements (tuples) in a matrix.

\paragraph{\syntax}

\begin{verbatim}
        GrB_Info GrB_Matrix_nvals(GrB_Index        *nvals,
                                  const GrB_Matrix  A);
\end{verbatim}

\paragraph{Parameters}

\begin{itemize}[leftmargin=1.1in]
    \item[{\sf nvals}] ({\sf OUT}) On successful return, contains the number of 
    stored elements (tuples) in the matrix.
    \item[{\sf A}] ({\sf IN}) An existing GraphBLAS matrix being queried.
\end{itemize}

\paragraph{Return Values}

\begin{itemize}[leftmargin=2.1in]
    \item[{\sf GrB\_SUCCESS}]  In blocking or non-blocking mode, the operation 
    completed successfully and the value of {\sf nvals} has been set. 

    \item[{\sf GrB\_PANIC}]    Unknown internal error.
    
    \item[{\sf GrB\_INVALID\_OBJECT}] This is returned in any execution mode 
    whenever one of the opaque GraphBLAS objects (input or output) is in an invalid 
    state caused by a previous execution error.  Call {GrB\_error()} to access 
    any error messages generated by the implementation.

    \item[{\sf GrB\_OUT\_OF\_MEMORY}] Not enough memory available for operation.
    
    \item[{\sf GrB\_UNINITIALIZED\_OBJECT}]  The GraphBLAS matrix, {\sf A}, has 
    not been initialized by a call to {\sf Matrix\_new} or {\sf Matrix\_dup}.
    
    \item[{\sf GrB\_NULL\_POINTER}]  The {\sf nvals} pointer is {\sf NULL}.
\end{itemize}

\paragraph{Description}

Return in {\sf nvals} the number of tuples (the size of $\bold{L}(\matrix{A})$
in Section~\ref{Sec:Matrices}) stored in matrix {\sf A}.

%-----------------------------------------------------------------------------

\subsubsection{{\sf Matrix\_build}: Store elements from tuples into a matrix}
\label{Sec:Matrix_build}

\paragraph{\syntax}

% AYDIN: Avoid page break due to preceding table
\begin{Verbatim}[samepage=true]    
        GrB_Info GrB_Matrix_build(GrB_Matrix            *C,
                                  const GrB_Index       *row_indices,
                                  const GrB_Index       *col_indices, 
                                  const <type>          *values,
                                  GrB_Index              nvals,
                                  const GrB_BinaryOp     dup);
\end{Verbatim}

\paragraph{Parameters}

\begin{itemize}[leftmargin=1.1in]
    \item[{\sf C}]      ({\sf INOUT}) An existing Matrix object to store the result.
    \item[{\sf row\_indices}] ({\sf IN}) Pointer to an array of row indices. 
    \item[{\sf col\_indices}] ({\sf IN}) Pointer to an array of column indices. 
    \item[{\sf values}] ({\sf IN}) Pointer to an array of scalars of a type that
                                   is compatible with the domain of matrix, {\sf C}.
    \item[{\sf nvals}]  ({\sf IN}) The number of values contained in each array.
    \item[{\sf dup}]    ({\sf IN}) A binary function to apply when duplicate values 
                        for the same location are present in the input arrays. \scott{Is {\sf GrB\_NULL} allowed?}
\end{itemize}

\paragraph{Return Values}

\begin{itemize}[leftmargin=2.1in]
    \item[{\sf GrB\_SUCCESS}]         In blocking mode, operation completed
    successfully. In non-blocking mode, this indicates that the API checks 
    for the input arguments passed successfully. Either way, output matrix 
    {\sf C} is ready to be used in the next method of the sequence.

    \item[{\sf GrB\_PANIC}]           Unknown internal error.
    
    \item[{\sf GrB\_INVALID\_OBJECT}] This is returned in any execution mode 
    whenever one of the opaque GraphBLAS objects (input or output) is in an invalid 
    state caused by a previous execution error.  Call {GrB\_error()} to access 
    any error messages generated by the implementation.

    \item[{\sf GrB\_OUT\_OF\_MEMORY}] Not enough memory available for operation.
    
    \item[{\sf GrB\_UNINITIALIZED\_OBJECT}]  Either GraphBLAS object, {\sf C} or
    {\sf dup}, has not been initialized by a call to its respective {\sf new} (or
    {\sf Matrix\_dup} for {\sf C}).
    
    \item[{\sf GrB\_NULL\_POINTER}]  {\sf C}, {\sf row\_indices}, 
    {\sf col\_indices} or {\sf values} pointer is {\sf NULL}.

    \item[{\sf GrB\_INDEX\_OUT\_OF\_BOUNDS}] A value in {\sf indices} is outside 
    the allowed range for \arg{w}.
    
	\item[{\sf GrB\_DOMAIN\_MISMATCH}]    The domains of {\sf values} and {\sf C}
	are incompatible with each other or binary operator ({\sf dup}).
\end{itemize}

\paragraph{Description}

\scott{Does this replace any existing content?}

\scott{Add statement that input arrays are available to be modified by the user
on return from this method even in non-blocking mode.}

Each tuple $\{ {\sf row\_indices[i]}, {\sf col\_indices[i]}, {\sf values[i]}\}$ is a contribution to the output in the form of 

$$\matrix{C}[{\sf row\_indices[i]}, {\sf col\_indices[i]}] = {\sf values[i]}.$$

If multiple values for the same location are present in the input arrays, the 
{\sf dup} binary operand is used to reduce them before assignment n into {\sf C}.
 
{\sf row\_indices}, {\sf col\_indices}, and {\sf values} should be of the same length. 


%-----------------------------------------------------------------------------
\subsubsection{{\sf Matrix\_setElement}: Set a single element in matrix}

Set one element of a matrix to a given value.

\paragraph{\syntax}

\begin{verbatim}
        GrB_Info GrB_Matrix_setElement(GrB_Matrix  *C,
                                       <type>       val,
                                       GrB_Index    row_index,
                                       GrB_Index    col_index); 
\end{verbatim}

\paragraph{Parameters}

\begin{itemize}[leftmargin=1.1in]
    \item[{\sf C}]   ({\sf INOUT}) An existing GraphBLAS matrix for which an 
    element is to be assigned.

    \item[{\sf val}]   ({\sf IN})  Value to assign.  The type must
    be compatible with the domain of {\sf C}.
    
    \item[{\sf row\_index}] ({\sf IN}) Row index of element to be assigned
    \item[{\sf col\_index}] ({\sf IN}) Column index of element to be assigned
\end{itemize}

\paragraph{Return Values}

\begin{itemize}[leftmargin=2.1in]
    \item[{\sf GrB\_SUCCESS}]         In blocking mode, the operation completed
    successfully. In non-blocking mode, this indicates that the consistency 
    tests on index/dimensions and domains for the input arguments passed successfully. 
    Either way, output matrix {\sf C} is ready to be used in the next method of 
    the sequence.

    \item[{\sf GrB\_PANIC}]   Unknown internal error.
    
    \item[{\sf GrB\_INVALID\_OBJECT}] This is returned in any execution mode 
    whenever one of the opaque GraphBLAS objects (input or output) is in an invalid 
    state caused by a previous execution error.  Call {GrB\_error()} to access 
    any error messages generated by the implementation.

    \item[{\sf GrB\_OUT\_OF\_MEMORY}]  Not enough memory available for operation.
    
    \item[{\sf GrB\_UNINITIALIZED\_OBJECT}]  The GraphBLAS matrix, {\sf C}, has 
    not been initialized by a call to {\sf Matrix\_new} or {\sf Matrix\_dup}.
    
    \item[{\sf GrB\_NULL\_POINTER}]    {\sf C} pointer is {\sf NULL}.

    \item[{\sf GrB\_INVALID\_INDEX}]  The ({\sf row\_index, col\_index}) tuple
    specifies a position that outside the dimensions of {\sf C}.
    
    \item[{\sf GrB\_DOMAIN\_MISMATCH}]     The domains of the matrix or scalar
    are incompatible.
\end{itemize}

\paragraph{Description}

First, the scalar and output matrix are tested for domain consistency as follows:  
$\bold{D}({\sf val})$ must be compatible with $\bold{D}({\sf C})$. Two domains 
are compatible with each other if values from one domain can be cast to values 
in the other domain as per the rules of the C language.  In particular, domains 
from Table~\ref{Tab:PredefinedTypes} are all compatible with each other. A domain 
from a user-defined type is only compatible with itself.  If any consistency 
rule above is violated, execution of {\sf GrB\_Matrix\_extractElement} ends and
the domain mismatch error listed above is returned.

Then, both index parameters are checked for valid values where following
conditions must hold:
\[
\begin{aligned}
    0\ \leq\ {\sf row\_index} & \ <\ \bold{nrows}({\sf C}), \\
    0\ \leq\ {\sf col\_index} & \ <\ \bold{ncols}({\sf C})
\end{aligned}
\]
If either of these conditions is violated, execution of 
{\sf GrB\_Matrix\_extractElement} ends and the invalid 
index error listed above is returned. 

We are now ready to carry out the assignment of {\sf val}; that is,
\[
{\sf C}({\sf row\_index},{\sf col\_index}) = {\sf val} 
\]
If a value existed at this location in {\sf C}, it will be overwritten; otherwise,
and new value is stored in {\sf C}.

In {\sf GrB\_BLOCKING} mode, the method exits with return value 
{\sf GrB\_SUCCESS} and the new contents of {\sf C} is as defined above
and fully computed.  
In {\sf GrB\_NONBLOCKING} mode, the method exits with return value 
{\sf GrB\_SUCCESS} and the new content of vector {\sf C} is as defined above 
but may not be fully computed; however, it can be used in the next GraphBLAS 
method call in a sequence.


%-----------------------------------------------------------------------------

\subsubsection{{\sf Matrix\_extractElement}: Extract a single element from a matrix}
\label{Sec:extract_single_element_mat}

\scott{Is OUTOFMEMORY error possible (perhaps only in non-blocking)? }

\scott{still need to deal with notation for type of first parameter.}

Extract one element of a matrix into a scalar. 

\paragraph{\syntax}

\begin{verbatim}
        GrB_Info GrB_Matrix_extractElement(<type>           *val,
                                           const GrB_Matrix  A,
                                           GrB_Index         row_index,
                                           GrB_Index         col_index); 

\end{verbatim}

\paragraph{Parameters}

\begin{itemize}[leftmargin=1in]
    \item[{\sf val}]   ({\sf OUT}) Pointer to a scalar. On successful return, 
    this scalar holds the result of the operation.  Any previous value is overwritten.

    \item[{\sf A}]     ({\sf IN}) The GraphBLAS matrix from which an element is
    extracted.
    
    \item[{\sf row\_index}] ({\sf IN}) The row index of location in {\sf A} 
    to extract.

    \item[{\sf col\_index}] ({\sf IN}) The column index of location in {\sf A} 
    to extract.
\end{itemize}

\paragraph{Return Values}

\begin{itemize}[leftmargin=2.1in]
    \item[{\sf GrB\_SUCCESS}]  In blocking or non-blocking mode, the operation 
    completed successfully. This indicates that the consistency tests on 
    dimensions and domains for the input arguments passed successfully, and
    the output scalar, {\sf val}, has been computed and is ready to be used in 
    the next method of the sequence.

    \item[{\sf GrB\_PANIC}]   Unknown internal error.
    
    \item[{\sf GrB\_INVALID\_OBJECT}] This is returned in any execution mode 
    whenever one of the opaque GraphBLAS objects (input or output) is in an invalid 
    state caused by a previous execution error.  Call {GrB\_error()} to access 
    any error messages generated by the implementation.

    \item[{\sf GrB\_OUT\_OF\_MEMORY}]  Not enough memory available for operation.
    \scott{Is this error possible?}
    
    \item[{\sf GrB\_UNINITIALIZED\_OBJECT}]  The GraphBLAS matrix, {\sf A}, has 
    not been initialized by a call to {\sf Matrix\_new} or {\sf Matrix\_dup}.
    
    \item[{\sf GrB\_NULL\_POINTER}]    {\sf val} pointer is {\sf NULL}.

    \item[{\sf GrB\_NO\_VALUE}]  There is no stored value at specified location.
    
    \item[{\sf GrB\_INVALID\_INDEX}]  {\sf row\_index} or {\sf col\_index} is 
    outside the allowable range (i.e., not less than $\bold{nrows}({\sf A})$ or
    $\bold{ncols}({\sf A})$, respectively).

    \item[{\sf GrB\_DOMAIN\_MISMATCH}]     The domains of the matrix and scalar
    are incompatible.
\end{itemize}

\paragraph{Description}

First, the scalar and input matrix are tested for domain consistency as follows:  
$\bold{D}({\sf val})$ must be compatible with $\bold{D}({\sf A})$. Two domains 
are compatible with each other if values from one domain can be cast to values 
in the other domain as per the rules of the C language.  In particular, domains 
from Table~\ref{Tab:PredefinedTypes} are all compatible with each other. A domain 
from a user-defined type is only compatible with itself.  If any consistency 
rule above is violated, execution of {\sf GrB\_Matrix\_extractElement} ends and
the domain mismatch error listed above is returned.

Then, both index parameters are checked for valid values where following
conditions must hold:
\[
\begin{aligned}
    0\ \leq\ {\sf row\_index} & \ <\ \bold{nrows}({\sf A}), \\
    0\ \leq\ {\sf col\_index} & \ <\ \bold{ncols}({\sf A})
\end{aligned}
\]
If either of these conditions is violated, execution of 
{\sf GrB\_Matrix\_extractElement} ends and the invalid 
index error listed above is returned. 

We are now ready to carry out the extract into the output argument, {\sf val}; 
that is,
\[
{\sf val} = {\sf A}({\sf row\_index},{\sf col\_index})
\]
where the following condition must be true:
\[
    ({\sf row\_index},{\sf col\_index}) \ \in \ \bold{ind}({\sf A})
\]
If this condition is violated, execution of {\sf GrB\_Matrix\_extractElement} 
ends and the "no value" error listed above is returned.

In {\sf GrB\_BLOCKING} mode, the method exits with return value 
{\sf GrB\_SUCCESS} and the new content of {\sf val} is as defined above
and fully computed.  
In {\sf GrB\_NONBLOCKING} mode, the method exits with return value 
{\sf GrB\_SUCCESS} and the new content of {\sf val} is as defined above 
and is also fully computed because this is a sequence terminating method.

\scott{Where and how do we say that this method is sequence terminating? E.g. 
This method is sequence terminating because we are extracting the contents
of an opaque data structure into a non-opaque data structure; therefore, we 
require that in non-blocking mode that enough computation be performed to 
compute the entire vector prior to returnin from this method.  
The GrB\_error() method should be called in non-blocking mode if more 
error information about 'upstream' operations.}


%-----------------------------------------------------------------------------

\subsubsection{{\sf Matrix\_extractTuples}: Extract tuples from a matrix}
\label{Sec:Matrix_extractTuples}

Extract the contents of a GraphBLAS matrix into non-opaque data structures.

\paragraph{\syntax}

\begin{verbatim}
        GrB_Info GrB_Matrix_extractTuples(GrB_Index            *row_indices,
                                          GrB_Index            *col_indices,
                                          <type>               *values, 
                                          const GrB_Matrix      A);
\end{verbatim}

\paragraph{Parameters}

\begin{itemize}[leftmargin=1.1in]
    \item[{\sf row\_indices}] ({\sf OUT}) Pointer to an array of row indices that is sufficient to
                        hold all of the row indices (no checking is performed).
    \item[{\sf col\_indices}] ({\sf OUT}) Pointer to an array of column indices that is sufficient to
                        hold all of the column indices (no checking is performed). 
    \item[{\sf values}] ({\sf OUT}) Pointer to an array of scalars of a type that is sufficient to
                        hold all of the stored values (no checking is performed) whose
                        type is compatible with $\bold{D}(\matrix{A})$.
    \item[{\sf A}]      ({\sf IN}) An existing GraphBLAS matrix.
\end{itemize}

\paragraph{Return Values}

\begin{itemize}[leftmargin=2.1in]
    \item[{\sf GrB\_SUCCESS}]  In blocking or non-blocking mode, the operation 
    completed successfully. This indicates that the consistency tests on 
    the input argument passed successfully, and the output arrays, {\sf indices}
    and {\sf values}, have been computed.

    \item[{\sf GrB\_PANIC}]   Unknown internal error.
    
    \item[{\sf GrB\_INVALID\_OBJECT}] This is returned in any execution mode 
    whenever one of the opaque GraphBLAS objects (input or output) is in an invalid 
    state caused by a previous execution error.  Call {GrB\_error()} to access 
    any error messages generated by the implementation.

    \item[{\sf GrB\_OUT\_OF\_MEMORY}]  Not enough memory available for operation.
    \scott{Is this error possible?}
    
    \item[{\sf GrB\_UNINITIALIZED\_OBJECT}]  The GraphBLAS matrix, {\sf A}, has 
    not been initialized by a call to {\sf Matrix\_new} or {\sf Matrix\_dup}.
    
    \item[{\sf GrB\_NULL\_POINTER}]  {\sf row\_indices}, {\sf col\_indices} or 
    {\sf values} pointer is {\sf NULL}.
    
    \item[\sf GrB\_DOMAIN\_MISMATCH] The domains of the {\sf A} matrix and 
    {\sf values} array are incompatible with one another.
\end{itemize}

\paragraph{Description}
\scott{DESCRIPTION MISSING}

\scott{Does allocation of non-opaque arrays occur within function -- then we need OUT\_OF\_MEMORY error.  As currently written, it is assumed that the user
is expected to call *\_nvals() function on the matrix to
determine how much memory to allocate and pass pointers to the pre 
allocated memory.  In this case NULL\_POINTER is returned (as 
shown above) if NULL pointers are passed.}
