\subsection{Vector Methods}

%All methods can be defined in use programs by including the {\tt GraphBLAS.h} header file.

%\scott{As with all *\_new operations, what happens when I new an object a second time?}

%-----------------------------------------------------------------------------
\subsubsection{{\sf Vector\_new}: Create new vector}

Creates a new vector with specified domain and size.

\paragraph{\syntax}

\begin{verbatim}
        GrB_info GrB_Vector_new(GrB_Vector *v,
                                GrB_Type    d,
                                GrB_Index   nsize);
\end{verbatim}

\paragraph{Parameters}

\begin{itemize}[leftmargin=1.1in]
    \item[{\sf v}] ({\sf INOUT}) On successful return, contains the identifier 
                                 of the newly created GraphBLAS vector.
    \item[{\sf d}] ({\sf IN})    The type corresponding to the domain of the 
                                 vector being created.  Can be one of the 
                                 predefined GraphBLAS types in 
                                 Table~\ref{Tab:PredefinedTypes}, or an existing 
                                 user-defined GraphBLAS type.
    \item[{\sf nsize}] ({\sf IN}) The size of the vector being created.
\end{itemize}

\paragraph{Return Values}

\begin{itemize}[leftmargin=2.1in]
\item[{\sf GrB\_SUCCESS}]    operation completed successfully.
\item[{\sf GrB\_PANIC}]      unknown internal error.
\item[{\sf GrB\_OUTOFMEM}]   not enough memory available for operation.
\item[{\sf GrB\_NOOBJECT}]   the {\sf GrB\_Type} parameter (for user-defined
                             types) has not been initialized by a
                             call to {\sf new}.
\item[{\sf GrB\_INVALID\_VALUE}]    {\sf v} pointer is {\sf NULL}.
\item[{\sf GrB\_INVALID\_VALUE}]    {\sf nsize} is zero.
\item[{\sf GrB\_INVALID\_VALUE}]    {\sf v} object is already initialized.
\end{itemize}

\paragraph{Description}

Creates a new vector $\vector{v}$ of domain $\bold{D}({\sf d})$, size {\sf nsize}, 
and empty $\bold{L}(\vector{v})$. It returns in {\sf v} this vector $\vector{v}$.

%-----------------------------------------------------------------------------
\subsubsection{{\sf Vector\_clear}: Clear a vector}

Removes all the elements from a vector.

\paragraph{\syntax}

\begin{verbatim}
        GrB_info GrB_Vector_clear(GrB_Vector *v);
\end{verbatim}

\paragraph{Parameters}

\begin{itemize}[leftmargin=1.1in]
    \item[{\sf v}] ({\sf IN}) An existing GraphBLAS vector to clear.
\end{itemize}

\paragraph{Return Values}

\begin{itemize}[leftmargin=2.1in]
\item[{\sf GrB\_SUCCESS}]   operation completed successfully.
\item[{\sf GrB\_PANIC}]     unknown internal error.
\item[{\sf GrB\_NOOBJECT}]  the vector has not been initialized with a call to new.
\item[{\sf GrB\_INVALID\_VALUE}]    {\sf v} pointer is {\sf NULL}.
\end{itemize}

\paragraph{Description}

Removes all tuples from an existing vector.

%-----------------------------------------------------------------------------
\subsubsection{{\sf Vector\_size}: Size of a vector}

Retrieve the size of a vector.

\paragraph{\syntax}

\begin{verbatim}
        GrB_info GrB_Vector_size(GrB_Index        *nsize,
                                 const GrB_Vector  v);
\end{verbatim}

\paragraph{Parameters}

\begin{itemize}[leftmargin=1.1in]
    \item[{\sf nsize}] ({\sf OUT}) On successful return, is set to the size ($N$) 
                                   of the vector.
    \item[{\sf v}]     ({\sf IN})  An existing GraphBLAS vector being queried.
\end{itemize}

\paragraph{Return Values}

\begin{itemize}[leftmargin=2.1in]
\item[{\sf GrB\_SUCCESS}]   operation completed successfully.
\item[{\sf GrB\_PANIC}]     unknown internal error.
\item[{\sf GrB\_NOOBJECT}]  vector has not been initialized with a call to {\sf new}.
\item[{\sf GrB\_INVALID\_VALUE}]    {\sf nsize} pointer is {\sf NULL}.
\end{itemize}

\paragraph{Description}

Return in {\sf nsize} the size (parameter $N$ in Section~\ref{Sec:Vectors}) in vector $\vector{v}$.

%-----------------------------------------------------------------------------
\subsubsection{{\sf Vector\_nvals}: Number of stored elements in a vector}

Retrieve the number of stored elements (tuples) in a vector.

\paragraph{\syntax}

\begin{verbatim}
        GrB_info GrB_Vector_nvals(GrB_Index        *nvals,
                                  const GrB_Vector  v);
\end{verbatim}

\paragraph{Parameters}

\begin{itemize}[leftmargin=1.1in]
    \item[{\sf nvals}] ({\sf OUT}) On successful return, is set to the number of 
                                   stored elements (tuples) in the vector.
    \item[{\sf v}]     ({\sf IN})  An existing GraphBLAS vector being queried.
\end{itemize}


\paragraph{Return Values}

\begin{itemize}[leftmargin=2.1in]
\item[{\sf GrB\_SUCCESS}]   operation completed successfully.
\item[{\sf GrB\_PANIC}]     unknown internal error.
\item[{\sf GrB\_NOOBJECT}]  vector has not been initialized with a call to {\sf new}.
\item[{\sf GrB\_INVALID\_VALUE}]    {\sf nvals} pointer is {\sf NULL}.
\end{itemize}

\paragraph{Description}

Return in {\sf nvals} the number of stored elements (the size of $\bold{L}(\vector{v})$
in Section~\ref{Sec:Vectors}) in vector {\sf v}.


%==============================================================================================
\subsection{Matrix Methods}

%-----------------------------------------------------------------------------
\subsubsection{{\sf Matrix\_new}: Create new matrix}

Creates a new matrix with specified domain and dimensions.

\paragraph{\syntax}

\begin{verbatim}
        GrB_info GrB_Matrix_new(GrB_Matrix *A,
                                GrB_Type    d,
                                GrB_Index   nrows,
                                GrB_Index   ncols);
\end{verbatim}

\paragraph{Parameters}

\begin{itemize}[leftmargin=1.1in]
    \item[{\sf A}] ({\sf INOUT}) On successful return, contains the identifier of 
                                 the newly created GraphBLAS matrix.
    \item[{\sf d}] ({\sf IN})    The type corresponding to the domain of the matrix 
                                 being created. Can be one of the predefined
                                 GraphBLAS types in Table~\ref{Tab:PredefinedTypes}, 
                                 or an existing user-defined GraphBLAS type.
    \item[{\sf nrows}] ({\sf IN}) The number of rows of the matrix being created.
    \item[{\sf ncols}] ({\sf IN}) The number of columns of the matrix being created.
\end{itemize}


\paragraph{Return Values}

\begin{itemize}[leftmargin=2.1in]
\item[{\sf GrB\_SUCCESS}]   operation completed successfully.
\item[{\sf GrB\_PANIC}]     unknown internal error.
\item[{\sf GrB\_OUTOFMEM}]  not enough memory available for operation.
\item[{\sf GrB\_NOOBJECT}]   the {\sf GrB\_Type} parameter (for user-defined
                             types) has not been initialized by a
                             call to {\sf new}.
\item[{\sf GrB\_INVALID\_VALUE}]    {\sf nrows} or {\sf ncols} is zero.
\item[{\sf GrB\_INVALID\_VALUE}]    {\sf A} pointer is {\sf NULL}.
\item[{\sf GrB\_INVALID\_VALUE}]    {\sf A} object is already initialized.
\end{itemize}

\paragraph{Description}

Creates a new matrix $\matrix{A}$ of domain $\bold{D}({\sf d})$, size {\sf nrows $\times$ ncols}, and
empty $\bold{L}(\matrix{A})$. It returns, in {\sf A}, this matrix $\matrix{A}$.

%-----------------------------------------------------------------------------
\subsubsection{{\sf Matrix\_clear}: Clear a matrix}

Removes all elements from a matrix.

\paragraph{\syntax}

\begin{verbatim}
        GrB_info GrB_Matrix_clear(GrB_Matrix *A);
\end{verbatim}

\paragraph{Parameters}

\begin{itemize}[leftmargin=1.1in]
    \item[{\sf A}] ({\sf IN}) An exising GraphBLAS matrix to clear.
\end{itemize}

\paragraph{Return Values}

\begin{itemize}[leftmargin=2.1in]
\item[{\sf GrB\_SUCCESS}]   operation completed successfully.
\item[{\sf GrB\_PANIC}]     unknown internal error.
\item[{\sf GrB\_NOOBJECT}]  the matrix has not been initialized with a call to new.
\item[{\sf GrB\_INVALID\_VALUE}]    {\sf A} pointer is {\sf NULL}.
\end{itemize}

\paragraph{Description}

Removes all elements (tuples) from an existing matrix.

%-----------------------------------------------------------------------------
\subsubsection{{\sf Matrix\_nrows}: Number of rows in a matrix}

Retrieve the number of rows in a matrix.

\paragraph{\syntax}

\begin{verbatim}
        GrB_info GrB_Matrix_nrows(GrB_Index        *nrows,
                                  const GrB_Matrix  A);
\end{verbatim}

\paragraph{Parameters}

\begin{itemize}[leftmargin=1.1in]
    \item[{\sf nrows}] ({\sf OUT}) On successful return, contains the number of rows in the matrix.
    \item[{\sf A}] ({\sf IN}) An existing GraphBLAS matrix being queried.
\end{itemize}


\paragraph{Return Values}

\begin{itemize}[leftmargin=2.1in]
\item[{\sf GrB\_SUCCESS}]   operation completed successfully.
\item[{\sf GrB\_PANIC}]     unknown internal error.
\item[{\sf GrB\_NOOBJECT}]  matrix has not been initialized with a call to {\sf new}.
\item[{\sf GrB\_INVALID\_VALUE}]    {\sf nrows} pointer is {\sf NULL}.
\end{itemize}

\paragraph{Description}

Return in {\sf nrows} the number of rows (parameter $M$ in Section~\ref{Sec:Matrices}) in matrix {\sf A}.

%-----------------------------------------------------------------------------
\subsubsection{{\sf Matrix\_ncols}: Number of columns in a matrix}

Retrieve the number of columns in a matrix.

\paragraph{\syntax}

\begin{verbatim}
        GrB_info GrB_Matrix_ncols(GrB_Index        *ncols,
                                  const GrB_Matrix  A);
\end{verbatim}

\paragraph{Parameters}

\begin{itemize}[leftmargin=1.1in]
    \item[{\sf ncols}] ({\sf OUT}) On successful return, contains the number of columns in the matrix.
    \item[{\sf A}] ({\sf IN}) An existing GraphBLAS matrix being queried.
\end{itemize}

\paragraph{Return Values}

\begin{itemize}[leftmargin=2.1in]
\item[{\sf GrB\_SUCCESS}]   operation completed successfully.
\item[{\sf GrB\_PANIC}]     unknown internal error.
\item[{\sf GrB\_NOOBJECT}]  matrix has not been initialized with a call to {\sf new}.
\item[{\sf GrB\_INVALID\_VALUE}]    {\sf ncols} pointer is {\sf NULL}.
\end{itemize}

\paragraph{Description}

Return in {\sf ncols} the number of columns (parameter $N$ in Section~\ref{Sec:Matrices}) in matrix {\sf A}.

%-----------------------------------------------------------------------------
\subsubsection{{\sf Matrix\_nvals}: Number of stored elements in a matrix}

Retrieve the number of stored elements (tuples) in a matrix.

\paragraph{\syntax}

\begin{verbatim}
        GrB_info GrB_Matrix_nvals(GrB_Index        *nvals,
                                  const GrB_Matrix  A);
\end{verbatim}

\paragraph{Parameters}

\begin{itemize}[leftmargin=1.1in]
    \item[{\sf nvals}] ({\sf OUT}) On successful return, contains the number of 
    stored elements (tuples) in the matrix.
    \item[{\sf A}] ({\sf IN}) An existing GraphBLAS matrix being queried.
\end{itemize}

\paragraph{Return Values}

\begin{itemize}[leftmargin=2.1in]
\item[{\sf GrB\_SUCCESS}]   operation completed successfully.
\item[{\sf GrB\_PANIC}]     unknown internal error.
\item[{\sf GrB\_NOOBJECT}]  matrix has not been initialized with a call to {\sf new}.
\item[{\sf GrB\_INVALID\_VALUE}]    {\sf nvals} pointer is {\sf NULL}.
\end{itemize}

\paragraph{Description}

Return in {\sf nvals} the number of tuples (the size of $\bold{L}(\matrix{A})$
in Section~\ref{Sec:Matrices}) stored in matrix {\sf A}.
