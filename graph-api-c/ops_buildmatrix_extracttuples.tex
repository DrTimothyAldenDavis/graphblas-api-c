\subsection{{\sf buildMatrix}: Store elements from tuples into a matrix}
\label{Sec:buildMatrix}

\paragraph{C99 Syntax}

% AYDIN: Avoid page break due to preceding table
\begin{Verbatim}[samepage=true]    
        GrB_info GrB_buildMatrix(GrB_Matrix               *C,
                                 const GrB_Matrix          Mask,
                                 const GrB_BinaryOp  accum,
                                 const GrB_Index          *rowIDs,
                                 const GrB_Index          *colIDs, 
                                 const <type>             *values,
                                 GrB_Index                 n,
                                 const GrB_BinaryOp  dup,
                                 const GrB_Descriptor      desc);
\end{Verbatim}

\paragraph{Parameters}

\begin{itemize}[leftmargin=1.1in]
    \item[{\sf C}]      ({\sf OUTP}) An existing Matrix object to store the result.
    \item[{\sf Mask}]   ({\sf MASK}) Output mask specifies which locations in
                        {\sf C} can be modified.  If no mask is desired,
                        {\sf GrB\_NULL} should be specified.
    \item[{\sf accum}]  Operator used for accumulating entries into existing
                        \matrix{C} entries. If no accumulation is desired,
                        {\sf GrB\_NULL} should be specified.
    \item[{\sf rowIDs}] ({\sf ARG0}) Pointer to an array of row indices. 
    \item[{\sf colIDs}] ({\sf ARG1}) Pointer to an array of column indices. 
    \item[{\sf values}] ({\sf ARG2}) Pointer to an array of scalars of a type that
                                     is compatible with the domain of matrix, {\sf C}.
    \item[{\sf n}]      ({\sf ARG3}) The number of values contained in each array.
    \item[{\sf dup}]    A binary function to apply when duplicate values for
                        the same location are present in the input arrays.
                                     
    \item[{\sf desc}]   Operation descriptor. If a
    \emph{default} descriptor is desired, {\sf GrB\_NULL} can be
    used.  Valid fields and values are as follows: \\
    \begin{tabular}{lll}
    Field  & Value & Description \\
    \hline
    {\sf MASK} & {\sf GrB\_SCMP}   & Use the structural complement of the mask, . \\
    {\sf MASK} & {\sf GrB\_NOCAST} & Prohibit casting from $\bold{D}({\sf Mask})$ to {\sf bool} domain. \\
    {\sf ARG2} & {\sf GrB\_NOCAST} & Prohibit casting from $\bold{D}({\sf values})$ to $\bold{D}(\matrix{C})$. \\
    \end{tabular}
\end{itemize}

\paragraph{Return Values}

\begin{itemize}[leftmargin=2.1in]
\item[{\sf GrB\_SUCCESS}]     operation completed successfully.
\item[{\sf GrB\_PANIC}]        unknown internal error.
\item[{\sf GrB\_OUTOFMEM}]    not enough memory available for operation.
\item[{\sf GrB\_NOMATRIX}]  C does not existing
\item[{\sf GrB\_INDEX\_OUTOFBOUNDS}]
        A value in i references a nonexistent row in C, or
        the value in j references a nonexistent column in C (matrix version).
\item[\sf GrB\_DOMAIN\_MISMATCH]  
       mismatch between value type and matrix domain (and descriptor did not allow cast.
\end{itemize}


\paragraph{Description}
Each tuple $\{ {\sf rowIDs[i]}, {\sf colIDs[i]}, {\sf values[i]}\}$ is a contribution to the output in the form of 

$$\matrix{C}[{\sf rowIDs[i]},{\sf colIDs[i]}] = {\sf accum}({\sf values[i]}, \matrix{C}[{\sf rowIDs[i]}, {\sf colIDs[i]}]).$$

If {\sf accum} parameter is not provided, then the contribution is of the form 

$$\matrix{C}[{\sf rowIDs[i]}, {\sf colIDs[i]}] = {\sf values[i]}.$$

If multiple values for the same location are present in the input arrays, the 
dup function is used to reduce them before assignment or accumulation into {\sf C}.

If a mask is used, then only locations in {\sf C} specified by the mask can be assigned or
accumulated into.
 
{\sf rowIDs}, {\sf colIDs}, and {\sf values} should be of the same length. 

%============================================================================
\subsection{{\sf buildVector}: Store elements from tuples into a vector}
\label{Sec:buildVector}

\paragraph{C99 Syntax}

\begin{verbatim}
        GrB_info GrB_buildVector(GrB_Vector               *u,
                                 const GrB_Vector          mask,
                                 const GrB_BinaryOp  accum,
                                 const GrB_Index          *indices,
                                 const <type>             *values,
                                 GrB_Index                 n,
                                 const GrB_Descriptor      desc);
\end{verbatim}

\paragraph{Parameters}

\begin{itemize}[leftmargin=1.1in]
        \item[{\sf u}]      ({\sf OUTP}) An existing Vector object to store the result.
        \item[{\sf mask}]   ({\sf MASK}) Output mask specifies which locations in
                        {\sf u} can be modified.  If no mask is desired,
                        {\sf GrB\_NULL} should be specified.
    \item[{\sf accum}]  Operator used for accumulating entries into existing
                        {\sf u} entries. If no accumulation is desired,
                        {\sf GrB\_NULL} should be specified.
    \item[{\sf indices}] ({\sf ARG0}) Pointer to an array of indices. 
    \item[{\sf values}] ({\sf ARG1}) Pointer to an array of scalars of a type that
                                     is compatible with the domain of vector {\sf u}.
        \item[{\sf n}]      The number of entries contained in each array (must be the same for \arg{indices} and \arg{values}.
        \item[{\sf desc}]   Operation descriptor. If a
    \emph{default} descriptor is desired, {\sf GrB\_NULL} can be
    used.  Valid fields and values are as follows: \\
        \begin{tabular}{lll}
            Field  & Value & Description \\
            \hline
            {\sf MASK} & {\sf GrB\_SCMP}   & Use the structural complement of the mask. \\
            {\sf MASK} & {\sf GrB\_NOCAST} & Prohibit casting from $\bold{D}({\sf mask})$ to {\sf bool} domain. \\
            {\sf ARG1} & {\sf GrB\_NOCAST} & Prohibit casting from the type of {\sf values} to $\bold{D}(\arg{u})$. \\
        \end{tabular}
\end{itemize}

\paragraph{Return Values}

\begin{itemize}[leftmargin=2.1in]
    \item[{\sf GrB\_SUCCESS}]     Operation completed successfully.
    \item[{\sf GrB\_PANIC}]            Unknown internal error.
    \item[{\sf GrB\_OUTOFMEM}]    Not enough memory available for operation.
    \item[{\sf GrB\_NOMATRIX}]      {\sf u} does not exist
    \item[{\sf GrB\_INDEX\_OUTOFBOUNDS}]
                            A value in {\sf indices} is outside the allowed range for \arg{u}.
    \item[\sf GrB\_DOMAIN\_MISMATCH]  
                       Mismatch between value type and vector domain, or mask domain and {\sf bool},  and descriptor did not allow cast.
\end{itemize}

\paragraph{Description}

Compute mask vector $\vector{m}$ from input parameter \arg{mask} based on descriptor \arg{desc}.
Return {\sf GrB\_DOMAIN\_MISMATCH} if $\bold{D}({\sf mask})$ is not {\sf bool} and descriptor does not allow casting.
Then, for $i = 0,\ldots,\arg{n}-1$, do the following:
\begin{enumerate}
    \item If $\arg{indices}[i] \notin \bold{L}(\vector{m})$, continue to next $i$.
    \item If $\arg{indices}[i] \notin \bold{i}(\arg{u})$, then $\bold{L}(\arg{u}) \leftarrow \bold{L}(\arg{u}) \cup (\arg{indices}[i], \arg{values}[i])$.
    \item If $\arg{indices}[i] \in \bold{i}(\arg{u})$, then replace the tuple $(\arg{indices}[i], v_{\arg{indices}[i]}) \in \bold{L}(\arg{u})$ with the tuple \\ $(\arg{indices}[i], \arg{accum}(v_{\arg{indices}[i]},\arg{values}[i]))$.
\end{enumerate}

After a call to {\sf GrB\_buildVector}, the program should perform a {\sf GrB\_wait} on vector \arg{u} before
modifying or deleting arrays \arg{indices} and \arg{values}.

%============================================================================

\subsection{{\sf extractTuples}: Extract tuples from a matrix}
\label{Sec:extractTuples}

Placeholder

%\aydin{Aydin to fill} 

\paragraph{C99 Syntax}

\begin{verbatim}
        GrB_info GrB_extractTuples(GrB_Index            *rowIDs,
                                   GrB_Index            *colIDs,
                                   <type>               *values, 
                                   const GrB_Matrix      A,
                                   const GrB_Matrix      Mask,
                                   const GrB_Descriptor  desc,
                                   char                 *err);
\end{verbatim}

\paragraph{Parameters}

\begin{itemize}[leftmargin=1.1in]
    \item[{\sf rowIDs}] Pointer to an array of row indices that is sufficient to
                        hold all of the row indices (no checking is performed).
    \item[{\sf colIDs}] Pointer to an array of column indices that is sufficient to
                        hold all of the column indices (no checking is performed). 
    \item[{\sf values}] Pointer to an array of scalars of a type that is sufficient to
                        hold all of the stored values (no checking is performed) whose
                        type is compatible with $\bold{D}(\matrix{A})$.
    \item[{\sf A}]      ({\sf ARG0}) An existing Matrix object to store the result.
    \item[{\sf Mask}]   ({\sf MASK}) Input mask specifies which locations in
                        {\sf A} that can be extracted.  If no mask is desired,
                        {\sf GrB\_NULL} should be specified.
    \item[{\sf desc}]   Operation descriptor. If a
    \emph{default} descriptor is desired, {\sf GrB\_NULL} can be
    used.  Valid fields and values are as follows: \\
    \begin{tabular}{lll}
    Field  & Value & Description \\
    \hline
    {\sf MASK} & {\sf GrB\_NOCAST} & Prohibit casting from $\bold{D}({\sf Mask})$ to {\sf bool} domain. \\
    {\sf MASK} & {\sf GrB\_SCMP}   & Use the structural complement of the mask, . \\
    {\sf ARG0} & {\sf GrB\_NOCAST} & Prohibit casting from $\bold{D}(\matrix{A})$ to $\bold{D}({\sf values})$. \\
    \end{tabular}
    \item[{\sf err}]     A null terminated string containing additional error information.
\end{itemize}

\paragraph{Return Values}
%\scott{TODO}
%
\paragraph{Description}
\scott{TODO}


%============================================================================

\subsection{{\sf extractTuples}: Extract tuples from a vector}
\label{Sec:extractTuples}

Placeholder

%\aydin{Aydin to fill} 

\paragraph{C99 Syntax}

\begin{verbatim}
        GrB_info GrB_extractTuples(GrB_Index            *indices,
                                   <type>               *values, 
                                   const GrB_Vector      v,
                                   const GrB_Vector      mask,
                                   const GrB_Descriptor  desc,
                                   char                 *err);

\end{verbatim}

\paragraph{Parameters}
\begin{itemize}
 \item[{\sf err}]     A null terminated string containing additional error information.
\end{itemize}

\paragraph{Return Values}
%\scott{TODO}
%
\paragraph{Description}
\scott{TODO}
