\subsection{Descriptor Methods}

%-----------------------------------------------------------------------------
\subsubsection{{\sf Descriptor\_new}: Create new descriptor}

Creates a new (empty) descriptor.

\paragraph{\syntax}

\begin{verbatim}
        GrB_info GrB_Descriptor_new(GrB_Descriptor *desc);
\end{verbatim}

\paragraph{Parameters}

\begin{itemize}[leftmargin=1.1in]
    \item[{\sf desc}] Identifier of new descriptor created.
\end{itemize}

\paragraph{Return Value}

\begin{itemize}[leftmargin=2.1in]
\item[{\sf GrB\_SUCCESS}]           operation completed successfully.
\item[{\sf GrB\_PANIC}]             unknown internal error.
\item[{\sf GrB\_OUTOFMEM}]          not enough memory available for this method to complete.
\end{itemize}

\paragraph{Description}

Returns in {\sf desc} the identifier of a newly created empty descriptor.
A newly created descriptor can be populated with calls to {\sf Descriptor\_set}.


%-----------------------------------------------------------------------------
\subsubsection{{\sf Descriptor\_set}: Set content of descriptor}

Sets the content (details of an operation) for a field of an existing
descriptor.

\paragraph{\syntax}

\begin{verbatim}
        GrB_info GrB_Descriptor_set(GrB_Descriptor desc,
                                    GrB_Field      field,
                                    GrB_Value      val);
\end{verbatim}

\paragraph{Parameters}

\begin{itemize}[leftmargin=1.1in]
    \item[{\sf desc}]  The descriptor being modified by this method.
    \item[{\sf field}] The descriptor field being set.
    \item[{\sf val}]   New value for the field being set.
\end{itemize}

\paragraph{Return Values}

\begin{itemize}[leftmargin=2.1in]
\item[{\sf GrB\_SUCCESS}]           operation completed successfully.
\item[{\sf GrB\_PANIC}]             unknown internal error.
\item[{\sf GrB\_OUTOFMEM}]          not enough memory available for operation.
\item[{\sf GrB\_INVALID\_VALUE}]    invalid value set on the field.
\end{itemize}

\paragraph{Description}

The fields of a descriptor include: {\sf GrB\_OUTP} for the output parameter 
(result) of a method; {\sf GrB\_MASK} for the mask argument to a method; 
{\sf GrB\_ARG0} through {\sf GrB\_ARGn} for the other input parameters 
(from first to last) of a method not including the operators used.

Valid values for a field of a descriptor (depending on the field) are as follows:

\begin{itemize}[leftmargin=1.5in]
\item[{\sf GrB\_SCMP}]   compute the structural complement of the corresponding mask
                         (GrB\_MASK) parameter
\item[{\sf GrB\_TRAN}]   compute the transpose of the corresponding parameter (valid
                         for input matrices only).
\item[{\sf GrB\_NOCAST}] do not allow casting of values from input parameters to input domains 
                         of operation or from output domain of operation to output 
                         parameter. If specified, mismatching domains will cause a 
                         run-time error. Default is to allow casting.
\end{itemize}

It is possible to specify a combination of values for a field. For 
example, if a matrix is to be both transposed and allow casting, one would use the field value
${\sf GrB\_TRAN} \mid {\sf GrB\_NOCAST}$.  

%\scott{It is TBD whether or not any other combinations
%needs to be supported: only masks can be complemented and we may not need to support
%transpose on them.}

%\scott{I removed GrB\_NOP and ACCUM/GrB\_ACC options from the list of possible fields/values.}


