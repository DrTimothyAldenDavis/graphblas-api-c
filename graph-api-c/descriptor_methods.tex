\subsection{Descriptor Methods}

%-----------------------------------------------------------------------------
\subsubsection{{\sf Descriptor\_new}: Create new descriptor}

Creates a new (empty) descriptor.

\paragraph{\syntax}

\begin{verbatim}
        GrB_info GrB_Descriptor_new(GrB_Descriptor *desc);
\end{verbatim}

\paragraph{Parameters}

\begin{itemize}[leftmargin=1.1in]
    \item[{\sf desc}] Identifier of new descriptor created.
\end{itemize}

\paragraph{Return Value}

\begin{itemize}[leftmargin=2.1in]
\item[{\sf GrB\_SUCCESS}]           operation completed successfully.
\item[{\sf GrB\_PANIC}]             unknown internal error.
\item[{\sf GrB\_OUTOFMEM}]          not enough memory available for this method to complete.
\end{itemize}

\paragraph{Description}

Returns in {\sf desc} the identifier of a newly created empty descriptor.
A newly created descriptor can be populated with calls to {\sf Descriptor\_set}.


%-----------------------------------------------------------------------------
\subsubsection{{\sf Descriptor\_set}: Set content of descriptor}

Sets the content (details of an operation) for a field of an existing
descriptor.

\paragraph{\syntax}

\begin{verbatim}
        GrB_info GrB_Descriptor_set(GrB_Descriptor desc,
                                    GrB_Field      field,
                                    GrB_Value      val);
\end{verbatim}

\scott{GrB\_Value is not a specified type.  Given the or operation for setting multiple values, this should be some integer type.}

\paragraph{Parameters}

\begin{itemize}[leftmargin=1.1in]
    \item[{\sf desc}]  The descriptor being modified by this method.
    \item[{\sf field}] The descriptor field being set.
    \item[{\sf val}]   New value for the field being set.
\end{itemize}

\paragraph{Return Values}

\begin{itemize}[leftmargin=2.1in]
\item[{\sf GrB\_SUCCESS}]           operation completed successfully.
\item[{\sf GrB\_PANIC}]             unknown internal error.
\item[{\sf GrB\_OUTOFMEM}]          not enough memory available for operation.
\item[{\sf GrB\_INVALID\_VALUE}]    invalid value set on the field.
\end{itemize}

\paragraph{Description}

Valid values for the {\sf field} paramater include the following:

\begin{itemize}[leftmargin=1.5in]
\item[{\sf GrB\_OUTP}]   refers to the output parameter (result) of the operation.
\item[{\sf GrB\_MASK}]   refers to the mask parameter of the operation.
\item[{\sf GrB\_INP}$x$] refers to the input parameters of the operation (matrices and vectors), 
                         where $x$ is '0' for the first input argument, '1' is for the second, and so on.
\end{itemize}

Valid values for the {\sf val} parameter are built from one or
more of the following together:

\begin{itemize}[leftmargin=1.5in]
\item[{\sf GrB\_SCMP}]   Use the structural complement of the corresponding mask
                         (GrB\_MASK) parameter
\item[{\sf GrB\_TRAN}]   compute the transpose of the corresponding parameter (valid
                         for input matrice parameters only).
\end{itemize}

When multiple modifiers need to be specified for a given field, the value parameters 
should be OR-ed together and a single call to set is used.  
%For example, if a matrix is to 
%be both transposed and prohibit casting, the {\sf val} parameter should be set to
%${\sf GrB\_TRAN} \mid {\sf GrB\_NOCAST}$.  

\scott{Is there any reason to prohibit the mask from being transposed?}


