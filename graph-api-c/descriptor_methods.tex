\subsection{Descriptor Methods}

%-----------------------------------------------------------------------------
\subsubsection{{\sf Descriptor\_new}: Create new descriptor}

Creates a new (empty) descriptor.

\paragraph{\syntax}

\begin{verbatim}
        GrB_info GrB_Descriptor_new(GrB_Descriptor *desc);
\end{verbatim}

\paragraph{Parameters}

\begin{itemize}[leftmargin=1.1in]
    \item[{\sf desc}] ({\sf INOUT}) On successful return, contains the 
    identifier of the newly created GraphBLAS descriptor.
\end{itemize}

\paragraph{Return Value}

\begin{itemize}[leftmargin=2.1in]
\item[{\sf GrB\_SUCCESS}]           operation completed successfully.
\item[{\sf GrB\_PANIC}]             unknown internal error.
\item[{\sf GrB\_OUTOFMEM}]          not enough memory available for operation.
\item[{\sf GrB\_INVALID\_VALUE}]    {\sf desc} pointer is {\sf NULL}.
\item[{\sf GrB\_INVALID\_VALUE}]    {\sf desc} object is already initialized.
\end{itemize}

\paragraph{Description}

Returns in {\sf desc} the identifier of a newly created empty descriptor.
A newly created descriptor can be populated by calls to {\sf Descriptor\_set}.


%-----------------------------------------------------------------------------
\subsubsection{{\sf Descriptor\_set}: Set content of descriptor}

Sets the content (details of an operation) for a field of an existing
descriptor.

\paragraph{\syntax}

\begin{verbatim}
        GrB_info GrB_Descriptor_set(GrB_Descriptor desc,
                                    GrB_Field      field,
                                    GrB_Value      val);
\end{verbatim}

\paragraph{Parameters}

\begin{itemize}[leftmargin=1.1in]
    \item[{\sf desc}]  ({\sf IN}) An existing GraphBLAS descriptor to be modified.
    \item[{\sf field}] ({\sf IN}) The field being set.
    \item[{\sf val}]   ({\sf IN}) New value for the field being set.
\end{itemize}

\paragraph{Return Values}

\begin{itemize}[leftmargin=2.1in]
\item[{\sf GrB\_SUCCESS}]           operation completed successfully.
\item[{\sf GrB\_PANIC}]             unknown internal error.
\item[{\sf GrB\_OUTOFMEM}]          not enough memory available for operation.
\item[{\sf GrB\_NOOBJECT}]          the {\sf desc} parameter has not been
                                    initialized by a call to {\sf new}.
\item[{\sf GrB\_INVALID\_VALUE}]    invalid value set on the field, or invalid field.
\end{itemize}

\paragraph{Description}

Valid values for the {\sf field} paramater include the following:

\begin{itemize}[leftmargin=1.5in]
\item[{\sf GrB\_OUTP}]   refers to the output parameter (result) of the operation.
\item[{\sf GrB\_MASK}]   refers to the mask parameter of the operation.
\item[{\sf GrB\_INP}$x$] refers to the input parameters of the operation 
                         (matrices and vectors), where $x$ is '0' for the first input argument, and '1' is for the second.
\end{itemize}

Valid values for the {\sf val} parameter are built from one or
more of the following together:

\begin{itemize}[leftmargin=1.5in]
\item[{\sf GrB\_SCMP}]    Use the structural complement of the corresponding mask
                          (GrB\_MASK) parameter.
\item[{\sf GrB\_TRAN}]    Use the transpose of the corresponding matrix parameter
                          (valid for input matrix parameters only).
\item[{\sf GrB\_REPLACE}] When assigning the masked values to the output matrix
                          or vector, clear the matrix first (or clear the
                          non-masked entries).  The default behavior is to leave
                          non-masked locations unchanged.  Valid for the
                          {\sf GrB\_OUTP} parameter only.
\end{itemize}

\scott{Should we allow the mask to be transposed? If
allowed the following text could be optionally added; otherwise, we have no need
for OR-ing multiple values because no field has more than one valid value:
"When multiple modifiers need to be specified for a given field, the value 
parameters should be OR-ed together and a single call to set is used."}
 



