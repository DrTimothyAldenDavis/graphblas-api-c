\subsection{Descriptor Methods}

\subsubsection{Create new descriptor ({\sf Descriptor\_new})}

Creates a new (empty) descriptor.

\paragraph{C99 Syntax}

\begin{verbatim}
#include "GraphBLAS.h"
GrB_info GrB_Descriptor_new(GrB_Descriptor *d)
\end{verbatim}

\paragraph{Output Parameters}

\begin{itemize}
	\item[{\sf d}] Identifier of new descriptor created.
\end{itemize}

\paragraph{Return Value}

\begin{tabular}{rl} 
{\sf GrB\_SUCCESS} 	& operation completed successfully \\
{\sf GrB\_PANIC}	& unknown internal error \\
{\sf GrB\_OUTOFMEM}	& not enough memory available for operation \\
{\sf GrB\_MISMATCH}	& mismatch between field and new value
\end{tabular}

\paragraph{Description}

Returns in {\sf d} the identifier of a newly created empty descriptor.
A newly created descriptor can be populated with calls to {\sf
Descriptor\_set}.

\subsubsection{Set content of descriptor ({\sf Descriptor\_set})}

\comment{
\scott{Naming nit: I propose {\sf Descriptor\_set}.  "adding" implies
accumulation (OR) of flags across many calls.  Allowing only set which
overwrites any existing values is simpler.} \jose{Agreed and modified.}  \scott{OK TO REMOVE}
}

Sets the content (details of an operation) for a field of an existing
descriptor.

\paragraph{C99 Syntax}

\begin{verbatim}
#include "GraphBLAS.h"
GrB_info GrB_Descriptor_set(GrB_Descriptor d,GrB_Field f,GrB_Value v)
\end{verbatim}

\paragraph{Input Parameters}

\begin{itemize}
	\item[{\sf d}] The descriptor being modified by this method.
	\item[{\sf f}] The descriptor field being set.
	\item[{\sf v}] New value for the field being set.
\end{itemize}

\paragraph{Return Value}

\begin{tabular}{rl} 
{\sf GrB\_SUCCESS} 	& operation completed successfully \\
{\sf GrB\_PANIC}	& unknown internal error \\
{\sf GrB\_OUTOFMEM}	& not enough memory available for operation \\
{\sf GrB\_MISMATCH}	& mismatch between field and new value
\end{tabular}

\paragraph{Description}

The fields of a descriptor include: {\sf GrB\_OUTP} for the 
output parameter (result) of a method; {\sf GrB\_MASK} for the mask
argument to a method; {\sf GrB\_ARG0} through {\sf GrB\_ARG9} for
the input parameters (from first to last) of a method.

Valid values for a field of a descriptor are as follows:

\begin{tabular}{rl} 
{\sf GrB\_NOP} 	& no operation to be performed for the corresponding parameter \\
{\sf GrB\_LNOT}	& \parbox[t]{5in}{compute the logical inverse \scott{structural complement?} of the corresponding parameter}  \\
{\sf GrB\_TRAN}	& compute the transpose of the corresponding parameter (for matrices) \\
{\sf GrB\_ACC}  & accumulate result of operation to current values in destination (for output parameter) \\
{\sf GrB\_CAST} & \parbox[t]{5in}{allow casting of values from input parameters to input domains of operation
                  or from output domain of operation to output parameter. (Otherwise, mismatching domains will cause a run-time error.)}
\end{tabular}

\scott{GrB\_LNOT clashes with operator in Table 2}

\ajy{GrB\_LNOT: logical inverse of non-mask parameters can be implemented by modifying the operators; therefore consider restricting this to masks only.}
\jose{I am OK with restricting {\sf GrB\_LNOT} to masks only.}

It is possible to specify a combination of values for a field. For 
example, if a matrix is to be both transposed and logically inverted
(element by element), one would use the field value
${\sf GrB\_TRAN} \mid {\sf GrB\_LNOT}$. 

